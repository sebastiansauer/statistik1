% Options for packages loaded elsewhere
\PassOptionsToPackage{unicode}{hyperref}
\PassOptionsToPackage{hyphens}{url}
%
\documentclass[
  letterpaper,
]{scrbook}

\usepackage{amsmath,amssymb}
\usepackage{iftex}
\ifPDFTeX
  \usepackage[T1]{fontenc}
  \usepackage[utf8]{inputenc}
  \usepackage{textcomp} % provide euro and other symbols
\else % if luatex or xetex
  \usepackage{unicode-math}
  \defaultfontfeatures{Scale=MatchLowercase}
  \defaultfontfeatures[\rmfamily]{Ligatures=TeX,Scale=1}
\fi
\usepackage{lmodern}
\ifPDFTeX\else  
    % xetex/luatex font selection
    \setmainfont[]{Times New Roman}
    \setsansfont[]{Merriweather Sans}
    \setmonofont[]{Roboto}
\fi
% Use upquote if available, for straight quotes in verbatim environments
\IfFileExists{upquote.sty}{\usepackage{upquote}}{}
\IfFileExists{microtype.sty}{% use microtype if available
  \usepackage[]{microtype}
  \UseMicrotypeSet[protrusion]{basicmath} % disable protrusion for tt fonts
}{}
\makeatletter
\@ifundefined{KOMAClassName}{% if non-KOMA class
  \IfFileExists{parskip.sty}{%
    \usepackage{parskip}
  }{% else
    \setlength{\parindent}{0pt}
    \setlength{\parskip}{6pt plus 2pt minus 1pt}}
}{% if KOMA class
  \KOMAoptions{parskip=half}}
\makeatother
\usepackage{xcolor}
\usepackage[paperwidth=6in,paperheight=9in]{geometry}
\ifLuaTeX
  \usepackage{luacolor}
  \usepackage[soul]{lua-ul}
\else
  \usepackage{soul}
  
\fi
\setlength{\emergencystretch}{3em} % prevent overfull lines
\setcounter{secnumdepth}{5}
% Make \paragraph and \subparagraph free-standing
\makeatletter
\ifx\paragraph\undefined\else
  \let\oldparagraph\paragraph
  \renewcommand{\paragraph}{
    \@ifstar
      \xxxParagraphStar
      \xxxParagraphNoStar
  }
  \newcommand{\xxxParagraphStar}[1]{\oldparagraph*{#1}\mbox{}}
  \newcommand{\xxxParagraphNoStar}[1]{\oldparagraph{#1}\mbox{}}
\fi
\ifx\subparagraph\undefined\else
  \let\oldsubparagraph\subparagraph
  \renewcommand{\subparagraph}{
    \@ifstar
      \xxxSubParagraphStar
      \xxxSubParagraphNoStar
  }
  \newcommand{\xxxSubParagraphStar}[1]{\oldsubparagraph*{#1}\mbox{}}
  \newcommand{\xxxSubParagraphNoStar}[1]{\oldsubparagraph{#1}\mbox{}}
\fi
\makeatother

\usepackage{color}
\usepackage{fancyvrb}
\newcommand{\VerbBar}{|}
\newcommand{\VERB}{\Verb[commandchars=\\\{\}]}
\DefineVerbatimEnvironment{Highlighting}{Verbatim}{commandchars=\\\{\}}
% Add ',fontsize=\small' for more characters per line
\usepackage{framed}
\definecolor{shadecolor}{RGB}{241,243,245}
\newenvironment{Shaded}{\begin{snugshade}}{\end{snugshade}}
\newcommand{\AlertTok}[1]{\textcolor[rgb]{0.68,0.00,0.00}{#1}}
\newcommand{\AnnotationTok}[1]{\textcolor[rgb]{0.37,0.37,0.37}{#1}}
\newcommand{\AttributeTok}[1]{\textcolor[rgb]{0.40,0.45,0.13}{#1}}
\newcommand{\BaseNTok}[1]{\textcolor[rgb]{0.68,0.00,0.00}{#1}}
\newcommand{\BuiltInTok}[1]{\textcolor[rgb]{0.00,0.23,0.31}{#1}}
\newcommand{\CharTok}[1]{\textcolor[rgb]{0.13,0.47,0.30}{#1}}
\newcommand{\CommentTok}[1]{\textcolor[rgb]{0.37,0.37,0.37}{#1}}
\newcommand{\CommentVarTok}[1]{\textcolor[rgb]{0.37,0.37,0.37}{\textit{#1}}}
\newcommand{\ConstantTok}[1]{\textcolor[rgb]{0.56,0.35,0.01}{#1}}
\newcommand{\ControlFlowTok}[1]{\textcolor[rgb]{0.00,0.23,0.31}{\textbf{#1}}}
\newcommand{\DataTypeTok}[1]{\textcolor[rgb]{0.68,0.00,0.00}{#1}}
\newcommand{\DecValTok}[1]{\textcolor[rgb]{0.68,0.00,0.00}{#1}}
\newcommand{\DocumentationTok}[1]{\textcolor[rgb]{0.37,0.37,0.37}{\textit{#1}}}
\newcommand{\ErrorTok}[1]{\textcolor[rgb]{0.68,0.00,0.00}{#1}}
\newcommand{\ExtensionTok}[1]{\textcolor[rgb]{0.00,0.23,0.31}{#1}}
\newcommand{\FloatTok}[1]{\textcolor[rgb]{0.68,0.00,0.00}{#1}}
\newcommand{\FunctionTok}[1]{\textcolor[rgb]{0.28,0.35,0.67}{#1}}
\newcommand{\ImportTok}[1]{\textcolor[rgb]{0.00,0.46,0.62}{#1}}
\newcommand{\InformationTok}[1]{\textcolor[rgb]{0.37,0.37,0.37}{#1}}
\newcommand{\KeywordTok}[1]{\textcolor[rgb]{0.00,0.23,0.31}{\textbf{#1}}}
\newcommand{\NormalTok}[1]{\textcolor[rgb]{0.00,0.23,0.31}{#1}}
\newcommand{\OperatorTok}[1]{\textcolor[rgb]{0.37,0.37,0.37}{#1}}
\newcommand{\OtherTok}[1]{\textcolor[rgb]{0.00,0.23,0.31}{#1}}
\newcommand{\PreprocessorTok}[1]{\textcolor[rgb]{0.68,0.00,0.00}{#1}}
\newcommand{\RegionMarkerTok}[1]{\textcolor[rgb]{0.00,0.23,0.31}{#1}}
\newcommand{\SpecialCharTok}[1]{\textcolor[rgb]{0.37,0.37,0.37}{#1}}
\newcommand{\SpecialStringTok}[1]{\textcolor[rgb]{0.13,0.47,0.30}{#1}}
\newcommand{\StringTok}[1]{\textcolor[rgb]{0.13,0.47,0.30}{#1}}
\newcommand{\VariableTok}[1]{\textcolor[rgb]{0.07,0.07,0.07}{#1}}
\newcommand{\VerbatimStringTok}[1]{\textcolor[rgb]{0.13,0.47,0.30}{#1}}
\newcommand{\WarningTok}[1]{\textcolor[rgb]{0.37,0.37,0.37}{\textit{#1}}}

\providecommand{\tightlist}{%
  \setlength{\itemsep}{0pt}\setlength{\parskip}{0pt}}\usepackage{longtable,booktabs,array}
\usepackage{calc} % for calculating minipage widths
% Correct order of tables after \paragraph or \subparagraph
\usepackage{etoolbox}
\makeatletter
\patchcmd\longtable{\par}{\if@noskipsec\mbox{}\fi\par}{}{}
\makeatother
% Allow footnotes in longtable head/foot
\IfFileExists{footnotehyper.sty}{\usepackage{footnotehyper}}{\usepackage{footnote}}
\makesavenoteenv{longtable}
\usepackage{graphicx}
\makeatletter
\newsavebox\pandoc@box
\newcommand*\pandocbounded[1]{% scales image to fit in text height/width
  \sbox\pandoc@box{#1}%
  \Gscale@div\@tempa{\textheight}{\dimexpr\ht\pandoc@box+\dp\pandoc@box\relax}%
  \Gscale@div\@tempb{\linewidth}{\wd\pandoc@box}%
  \ifdim\@tempb\p@<\@tempa\p@\let\@tempa\@tempb\fi% select the smaller of both
  \ifdim\@tempa\p@<\p@\scalebox{\@tempa}{\usebox\pandoc@box}%
  \else\usebox{\pandoc@box}%
  \fi%
}
% Set default figure placement to htbp
\def\fps@figure{htbp}
\makeatother
% definitions for citeproc citations
\NewDocumentCommand\citeproctext{}{}
\NewDocumentCommand\citeproc{mm}{%
  \begingroup\def\citeproctext{#2}\cite{#1}\endgroup}
\makeatletter
 % allow citations to break across lines
 \let\@cite@ofmt\@firstofone
 % avoid brackets around text for \cite:
 \def\@biblabel#1{}
 \def\@cite#1#2{{#1\if@tempswa , #2\fi}}
\makeatother
\newlength{\cslhangindent}
\setlength{\cslhangindent}{1.5em}
\newlength{\csllabelwidth}
\setlength{\csllabelwidth}{3em}
\newenvironment{CSLReferences}[2] % #1 hanging-indent, #2 entry-spacing
 {\begin{list}{}{%
  \setlength{\itemindent}{0pt}
  \setlength{\leftmargin}{0pt}
  \setlength{\parsep}{0pt}
  % turn on hanging indent if param 1 is 1
  \ifodd #1
   \setlength{\leftmargin}{\cslhangindent}
   \setlength{\itemindent}{-1\cslhangindent}
  \fi
  % set entry spacing
  \setlength{\itemsep}{#2\baselineskip}}}
 {\end{list}}
\usepackage{calc}
\newcommand{\CSLBlock}[1]{\hfill\break\parbox[t]{\linewidth}{\strut\ignorespaces#1\strut}}
\newcommand{\CSLLeftMargin}[1]{\parbox[t]{\csllabelwidth}{\strut#1\strut}}
\newcommand{\CSLRightInline}[1]{\parbox[t]{\linewidth - \csllabelwidth}{\strut#1\strut}}
\newcommand{\CSLIndent}[1]{\hspace{\cslhangindent}#1}

% load packages
%\usepackage{multicol}
\usepackage{fontspec}
\usepackage{emoji}
\usepackage{xltxtra}
%\usepackage{xcolor}
\usepackage{listings}
\usepackage{fvextra}
\usepackage[german]{csquotes}


\definecolor{ycol}{RGB}{230,159,0}
\definecolor{modelcol}{RGB}{86,180,233}
\definecolor{errorcol}{RGB}{0,158,115}
\definecolor{beta0col}{RGB}{213,94,0}
\definecolor{beta1col}{RGB}{0,114,178}
\definecolor{xcol}{RGB}{204,121,167}


\lstset{
  breaklines=true
}

\DefineVerbatimEnvironment{Highlighting}{Verbatim}{breaklines,commandchars=\\\{\}}
\DefineVerbatimEnvironment{OutputCode}{Verbatim}{breaklines,commandchars=\\\{\}}
\usepackage{booktabs}
\usepackage{caption}
\usepackage{longtable}
\usepackage{colortbl}
\usepackage{array}
\usepackage{anyfontsize}
\usepackage{multirow}
\makeatletter
\@ifpackageloaded{tcolorbox}{}{\usepackage[skins,breakable]{tcolorbox}}
\@ifpackageloaded{fontawesome5}{}{\usepackage{fontawesome5}}
\definecolor{quarto-callout-color}{HTML}{909090}
\definecolor{quarto-callout-note-color}{HTML}{0758E5}
\definecolor{quarto-callout-important-color}{HTML}{CC1914}
\definecolor{quarto-callout-warning-color}{HTML}{EB9113}
\definecolor{quarto-callout-tip-color}{HTML}{00A047}
\definecolor{quarto-callout-caution-color}{HTML}{FC5300}
\definecolor{quarto-callout-color-frame}{HTML}{acacac}
\definecolor{quarto-callout-note-color-frame}{HTML}{4582ec}
\definecolor{quarto-callout-important-color-frame}{HTML}{d9534f}
\definecolor{quarto-callout-warning-color-frame}{HTML}{f0ad4e}
\definecolor{quarto-callout-tip-color-frame}{HTML}{02b875}
\definecolor{quarto-callout-caution-color-frame}{HTML}{fd7e14}
\makeatother
\makeatletter
\@ifpackageloaded{bookmark}{}{\usepackage{bookmark}}
\makeatother
\makeatletter
\@ifpackageloaded{caption}{}{\usepackage{caption}}
\AtBeginDocument{%
\ifdefined\contentsname
  \renewcommand*\contentsname{Inhaltsverzeichnis}
\else
  \newcommand\contentsname{Inhaltsverzeichnis}
\fi
\ifdefined\listfigurename
  \renewcommand*\listfigurename{Abbildungsverzeichnis}
\else
  \newcommand\listfigurename{Abbildungsverzeichnis}
\fi
\ifdefined\listtablename
  \renewcommand*\listtablename{Tabellenverzeichnis}
\else
  \newcommand\listtablename{Tabellenverzeichnis}
\fi
\ifdefined\figurename
  \renewcommand*\figurename{Abbildung}
\else
  \newcommand\figurename{Abbildung}
\fi
\ifdefined\tablename
  \renewcommand*\tablename{Tabelle}
\else
  \newcommand\tablename{Tabelle}
\fi
}
\@ifpackageloaded{float}{}{\usepackage{float}}
\floatstyle{ruled}
\@ifundefined{c@chapter}{\newfloat{codelisting}{h}{lop}}{\newfloat{codelisting}{h}{lop}[chapter]}
\floatname{codelisting}{Listing}
\newcommand*\listoflistings{\listof{codelisting}{Listingverzeichnis}}
\usepackage{amsthm}
\theoremstyle{definition}
\newtheorem{exercise}{Übungsaufgabe}[chapter]
\theoremstyle{definition}
\newtheorem{example}{Beispiel}[chapter]
\theoremstyle{definition}
\newtheorem{definition}{Definition}[chapter]
\theoremstyle{remark}
\AtBeginDocument{\renewcommand*{\proofname}{Beweis}}
\newtheorem*{remark}{Anmerkung}
\newtheorem*{solution}{Lösung}
\newtheorem{refremark}{Anmerkung}[chapter]
\newtheorem{refsolution}{Lösung}[chapter]
\makeatother
\makeatletter
\makeatother
\makeatletter
\@ifpackageloaded{caption}{}{\usepackage{caption}}
\@ifpackageloaded{subcaption}{}{\usepackage{subcaption}}
\makeatother
\makeatletter
\@ifpackageloaded{fontawesome5}{}{\usepackage{fontawesome5}}
\makeatother

\ifLuaTeX
\usepackage[bidi=basic]{babel}
\else
\usepackage[bidi=default]{babel}
\fi
\babelprovide[main,import]{ngerman}
\ifPDFTeX
\else
\babelfont{rm}[]{Times New Roman}
\fi
% get rid of language-specific shorthands (see #6817):
\let\LanguageShortHands\languageshorthands
\def\languageshorthands#1{}
\ifLuaTeX
  \usepackage[german]{selnolig} % disable illegal ligatures
\fi
\usepackage{csquotes}
\usepackage{bookmark}

\IfFileExists{xurl.sty}{\usepackage{xurl}}{} % add URL line breaks if available
\urlstyle{same} % disable monospaced font for URLs
\hypersetup{
  pdftitle={Statistik1},
  pdfauthor={Sebastian Sauer},
  pdflang={de-DE},
  pdfkeywords={Statistik, Prognose, Modellierung, R, Datenanalyse, Regression},
  hidelinks,
  pdfcreator={LaTeX via pandoc}}


\title{Statistik1}
\author{Sebastian Sauer}
\date{2025-01-10}

\begin{document}
\frontmatter
\maketitle

\renewcommand*\contentsname{Inhaltsverzeichnis}
{
\setcounter{tocdepth}{2}
\tableofcontents
}

\mainmatter
\bookmarksetup{startatroot}

\chapter*{Vorwort}\label{vorwort}
\addcontentsline{toc}{chapter}{Vorwort}

\markboth{Vorwort}{Vorwort}

\emph{Willkommen bei Statistik1!}

Dieses Buch führt Sie in die Grundlagen der Statistik ein mit einem
Schwerpunkt auf Vorhersagen. Es ist ein angewandtes Buch für Anfänger.
Anders gesagt: Sie lernen, Daten aufzubereiten und mit Hilfe einfacher
Modelle Vorhersagen abzuleiten.

Die Online-Version dieses Buches ist frei verfügbar und unter der
\href{https://creativecommons.org/licenses/by-nc-nd/4.0/deed.de}{CC-BY-NC-ND-4.0-Lizenz}
publiziert.

Dieses Buch führt in die Statistik ein; es soll Freude am Lernen
bereiten und hat nur ein Thema: Vorhersagen mittels moderner
statistischen Methoden. Alle Inhalte dieses Buch erklären einen Aspekt
der statistischen (Vorhersage-)Modellierung. Es wendet sich an
Studierende ohne Vorkenntnisse in Statistik. Viele Statistikbücher gibt
es schon auf dieser Welt, braucht es da noch eines? Ja, es gibt viele
Statistikbücher, aber (meines Wissens) in deutscher Sprache keines, das
Freude beim Lernen vermittelt, sich auf statistische
Vohersage-Modellierung konzentriert und moderne Werkzeuge einsetzt.
Diese Lücke soll dieses Buch schließen. Freude am Lernen, beim
Angstgegner Statistik, wie soll das gehen? Viele
Verständnisschwierigkeiten rühren daher, dass Lehrbücher kompliziert
geschrieben sind. Solcher Schreibweise liegt wohl die Überlegung
zugrunde, dass die Konzepte präzise und nuanciert erläutert sein
müssten. Meiner Ansicht nach wird da das Ziel mit dem Weg verwechselt:
Am Anfang darf eine Erklärung ruhig etwas grober sein. Überblicken die
Leser und Leserinnen die Materie einigermaßen, können sie sich im
nächsten Schritt mit den Details vertraut machen, was Präzision und
Tiefe verlangt. Darüber hinaus verwendet dieses Buch eine lockere
Sprache für einen entspannten Lesefluss. Für einigen Komfort beim Lesen
wurde gesorgt: Lernziele, Definitionen, Beispiele, Übungen, Hinweise,
Fehlerquellen, Tipps, Literatur, Querverweise, QR-Codes zu externen
Medien und mehr werden im Buch verwendet; an Erklärbildern wurde nicht
gespart.

Der Inhalt des Buches ist ganz auf statistische Modelle zur Vorhersage
ausgerichtet. \enquote{Statistische Modelle} ist ein sperriger Begriff,
aber er sagt nur, dass es darum geht, fachliche Fragen in statistisch
greifbare Bausteine zu gießen. Ein Beispiel: Studentin Anna fragt sich,
ob sie die Prüfung besteht, wenn Sie 42 Stunden büffelt? Student Bert
meint, dass motivierte Studis am meisten vom Lernen profitieren.
Studentin Carla ist hingegen überzeugt, dass Lernen nix bringt, sondern
dass die Intelligenz allein für den Prüfungserfolg verantwortlich sei.
Damit haben wir drei (noch recht unpräzise wissenschaftliche) Modelle.
Die Statistik hat nun die Aufgabe, möglichst präzise Antworten zu
liefern; dafür sind Zahlen hilfreich. Wenn Anna, Bert und Carla ihre
Überlegungen fachlich schärfen und dann in statistische Sprache
übersetzen, können wir mit Antworten rechnen, manchmal sogar mit
präzisen. Was nicht heißt, dass diese Antworten immer richtig oder
nützlich sind. Tja, das Leben ist nicht leicht.

Mit Blick auf den Spagat zwischen Theorie und Anwendung irrt das Buch
(bzw. sein Autor) zugunsten der Seite der Anwendung. Ich wollte lieber
befähigen, praktische Probleme zu lösen, als tiefen theoretischen
Einblick zu vermitteln. Meine Hoffnung ist, dass die Freude am Können
beflügelt, sich im nächsten Schritt tiefer mit der Materie zu
beschäftigen. Ist es nicht auch so im Alltag? Was Freude macht, wo sich
Erfolge einstellen, dort arbeiten wir gerne weiter.

Da sich das Buch auf ein Thema, Modellierung, konzentriert, bleiben
andere Themen außen vor, vor allem Inferenzstatistik. Vielleicht freut
sich die eine oder der andere, von diesem Thema verschont zu sein. Ich
denke, dass Modellierung für die Forschung und für die Praxis ein
zentraler Gedanke ist; für zwei große Themen erscheint mir dieses Buch
zu eng. Wenn Sie Fragen oder Feedback haben, bin ich für Ihre Hinweise
dankbar. Stellen Sie sie gerne hier ein:
\url{https://github.com/sebastiansauer/statistik1/issues}.

Ich wünsche Ihnen viel Freude und Erfolg beim Statistik lernen!

Ihr

Sebastian Sauer

\bookmarksetup{startatroot}

\chapter*{Vorwort}\label{vorwort-1}
\addcontentsline{toc}{chapter}{Vorwort}

\markboth{Vorwort}{Vorwort}

Dieses Buch führt in die Statistik ein; es soll Freude am Lernen
bereiten und hat nur ein Thema: Vorhersagen mittels moderner
statistischen Methoden. Alle Inhalte dieses Buch erklären einen Aspekt
der statistischen (Vorhersage-)Modellierung. Es wendet sich an
Studierende ohne Vorkenntnisse in Statistik. Viele Statistikbücher gibt
es schon auf dieser Welt, braucht es da noch eines? Ja, es gibt viele
Statistikbücher, aber (meines Wissens) in deutscher Sprache keines, das
Freude beim Lernen vermittelt, sich auf statistische
Vohersage-Modellierung konzentriert und moderne Werkzeuge einsetzt.
Diese Lücke soll dieses Buch schließen. Freude am Lernen, beim
Angstgegner Statistik, wie soll das gehen? Viele
Verständnisschwierigkeiten rühren daher, dass Lehrbücher kompliziert
geschrieben sind. Solcher Schreibweise liegt wohl die Überlegung
zugrunde, dass die Konzepte präzise und nuanciert erläutert sein
müssten. Meiner Ansicht nach wird da das Ziel mit dem Weg verwechselt:
Am Anfang darf eine Erklärung ruhig etwas grober sein. Überblicken die
Leser und Leserinnen die Materie einigermaßen, können sie sich im
nächsten Schritt mit den Details vertraut machen, was Präzision und
Tiefe verlangt. Darüber hinaus verwendet dieses Buch eine lockere
Sprache für einen entspannten Lesefluss. Für einigen Komfort beim Lesen
wurde gesorgt: Lernziele, Definitionen, Beispiele, Übungen, Hinweise,
Fehlerquellen, Tipps, Literatur, Querverweise, QR-Codes zu externen
Medien und mehr werden im Buch verwendet; an Erklärbildern wurde nicht
gespart.

Der Inhalt des Buches ist ganz auf statistische Modelle zur Vorhersage
ausgerichtet. \enquote{Statistische Modelle} ist ein sperriger Begriff,
aber er sagt nur, dass es darum geht, fachliche Fragen in statistisch
greifbare Bausteine zu gießen. Ein Beispiel: Studentin Anna fragt sich,
ob sie die Prüfung besteht, wenn Sie 42 Stunden büffelt? Student Bert
meint, dass motivierte Studis am meisten vom Lernen profitieren.
Studentin Carla ist hingegen überzeugt, dass Lernen nix bringt, sondern
dass die Intelligenz allein für den Prüfungserfolg verantwortlich sei.
Damit haben wir drei (noch recht unpräzise wissenschaftliche) Modelle.
Die Statistik hat nun die Aufgabe, möglichst präzise Antworten zu
liefern; dafür sind Zahlen hilfreich. Wenn Anna, Bert und Carla ihre
Überlegungen fachlich schärfen und dann in statistische Sprache
übersetzen, können wir mit Antworten rechnen, manchmal sogar mit
präzisen. Was nicht heißt, dass diese Antworten immer richtig oder
nützlich sind. Tja, das Leben ist nicht leicht.

Mit Blick auf den Spagat zwischen Theorie und Anwendung irrt das Buch
(bzw. sein Autor) zugunsten der Seite der Anwendung. Ich wollte lieber
befähigen, praktische Probleme zu lösen, als tiefen theoretischen
Einblick zu vermitteln. Meine Hoffnung ist, dass die Freude am Können
beflügelt, sich im nächsten Schritt tiefer mit der Materie zu
beschäftigen. Ist es nicht auch so im Alltag? Was Freude macht, wo sich
Erfolge einstellen, dort arbeiten wir gerne weiter.

Da sich das Buch auf ein Thema, Modellierung, konzentriert, bleiben
andere Themen außen vor, vor allem Inferenzstatistik. Vielleicht freut
sich die eine oder der andere, von diesem Thema verschont zu sein. Ich
denke, dass Modellierung für die Forschung und für die Praxis ein
zentraler Gedanke ist; für zwei große Themen erscheint mir dieses Buch
zu eng. Wenn Sie Fragen oder Feedback haben, bin ich für Ihre Hinweise
dankbar. Stellen Sie sie gerne hier ein:
\url{https://github.com/sebastiansauer/statistik1/issues}.

Ich wünsche Ihnen viel Freude und Erfolg beim Statistik lernen!

Ihr

Sebastian Sauer

\bookmarksetup{startatroot}

\chapter{Organisatorisches}\label{organisatorisches}

\begin{figure}[H]

{\centering \includegraphics[width=0.33\linewidth,height=\textheight,keepaspectratio]{005-orga_files/figure-pdf/statistik-und-du-guter-fit-1.pdf}

}

\caption{Statistik und Du: Passt!}

\end{figure}%

\section{Es geht um Ihren Lernerfolg}\label{es-geht-um-ihren-lernerfolg}

Meister Yoda rät: Lesen Sie die folgenden Hinweise, s.
Abbildung~\ref{fig-yoda}.

\begin{figure}[H]

\centering{

\includegraphics[width=0.5\linewidth,height=\textheight,keepaspectratio]{img/yoda.jpg}

}

\caption{\label{fig-yoda}Lesen Sie die folgenden Hinweise im eigenen
Interesse (imgflip, 2024b)}

\end{figure}%

\subsection{Lernziele}\label{lernziele}

\begin{itemize}
\item
  Die Studentis sind mit wesentlichen Methoden der explorativen
  Datenanalyse vertraut und können diese selbständig anwenden.
\item
  Die Studentis können gängige Forschungsfragen in lineare Modelle
  übersetzen, diese auf echte Datensätze anwenden und die Ergebnisse
  interpretieren.
\end{itemize}

\subsection{Was lerne ich hier und wozu ist das
gut?}\label{was-lerne-ich-hier-und-wozu-ist-das-gut}

\emph{Was lerne ich hier?}

Sie lernen das \emph{Handwerk der Datenanalyse} mit einem Schwerpunkt
auf Vorhersage. Anders gesagt: Sie lernen, \emph{Daten aufzubereiten}
und aus Daten \emph{Vorhersagen} abzuleiten. Zum Beispiel: Kommt ein
Student zu Ihnen und sagt \enquote{Ich habe 42 Stunden für die Klausur
gelernt, welche Note kann ich in der Klausur erwarten?}. Darauf Ihre
Antwort: \enquote{Auf Basis meiner Daten und meines Modells müsstest du
eine 2.7 schreiben!} Außerdem lernen Sie, wie man die Güte einer
Vorhersage auf Stichhaltigkeit prüft. Denn Vorhersagen kann man ja in
jeder Eckkneipe oder beim Wahrsager bekommen. Wir wollen aber belastbare
Vorhersagen und zumindest wissen, wie gut die Vorhersagen bisher waren.

\emph{Warum ist das wichtig?}

Wir wollen nicht auf Leuten vertrauen, die behaupten, sie wüssten, was
für uns richtig und gut ist. Wir wollen selber die Fakten beurteilen
können.

\emph{Wozu brauche ich das im Job?}

Datenanalyse spielt bereits heute in vielen Berufen eine Rolle. Tendenz
stark zunehmend.

\emph{Wozu brauche ich das im weiterem Studium?}

In Forschungsarbeiten (wie in empirischen Forschungsprojekten, etwa in
der Abschlussarbeit) ist es üblich, statistische Ergebnisse hinsichtlich
quantitativ zu analysieren.

\emph{Ist Statistik nicht sehr abstrakt?}

Der Schwerpunkt dieses Kurses liegt auf Anwenden und Tun; ähnlich dem
Erlernen eines Handwerks. Theorien und Abstraktionen stehen nur am Rand.

\emph{Gibt es auch gute Jobs, wenn man sich mit Daten auskennt?}

Das Forum (2020) berichtet zu den \enquote{Top 20 job roles in
increasing and decreasing demand across industries} (S. 30, Abb. 22):

\begin{enumerate}
\def\labelenumi{\arabic{enumi}.}
\tightlist
\item
  Data Analysts und Scientists
\item
  AI and Machine Learning Specialists
\item
  Big Data Specialists
\end{enumerate}

\subsection{Was ist hier das
Erfolgsgeheimnis?}\label{was-ist-hier-das-erfolgsgeheimnis}

Das Lesen einer Schwimmfibel nutzt wenig, wenn Sie Freischwimmer werden
wollen. Es hilft nichts: Rein in die Fluten! Wenn das Wasser nicht tief
ist, man jederzeit Pause machen kann und die Erfolge sich schnell
einstellen, steht Ihrem Fortschritt beim Lernen nichts im Weg. Ich gebe
zu, der Vergleich ist nicht gerade subtil. Aber es ist so: Sie lernen
durch Tun (Lovett \& Greenhouse, 2000). Dieses Buch bietet dafür
reichhaltige Gelegenheit. Nutzen Sie sie. Jedes Kapitel führt am Ende
eine Reihe von Aufgaben auf, alle mit Lösungen. So können Sie Ihren
Lernfortschritt testen. Das Schwierigkeiten auftreten, wenn man etwas
Neues lernt, ist normal. Das geht fast allen so. Ihren Lernerfolg kann
nur eine Sache gefährden: Wenn Sie aufgeben. Bleiben Sie dran, und der
Erfolg wird sich einstellen! Abbildung~\ref{fig-lernen} zeigt Daten von
\(N=1646\) Studentis, die zeigen, dass regelmäßiges Üben und Dranbleiben
der Schlüssel zum Erfolg ist (Sauer, 2017).

\begin{figure}[H]

\centering{

\includegraphics[width=1\linewidth,height=\textheight,keepaspectratio]{005-orga_files/figure-pdf/fig-lernen-1.pdf}

}

\caption{\label{fig-lernen}Der Zusammenhang von Lernzeit (1: gering bis
5:hoch) von Klausurerfolg}

\end{figure}%

\begin{tcolorbox}[enhanced jigsaw, bottomtitle=1mm, leftrule=.75mm, breakable, title=\textcolor{quarto-callout-important-color}{\faExclamation}\hspace{0.5em}{Wichtig}, bottomrule=.15mm, titlerule=0mm, left=2mm, opacityback=0, colframe=quarto-callout-important-color-frame, rightrule=.15mm, colback=white, coltitle=black, toprule=.15mm, toptitle=1mm, colbacktitle=quarto-callout-important-color!10!white, arc=.35mm, opacitybacktitle=0.6]

\emph{Dran bleiben} ist der Schlüssel zum Erfolg. Üben Sie regelmäßig.
Geben Sie bei Schwierigkeiten nicht auf.

\emoji{person-lifting-weights} \emoji{clockwise-vertical-arrows}
\emoji{key} \emoji{glowing-star} \(\square\)

\end{tcolorbox}

\subsection{Motivieren Sie mich!}\label{motivieren-sie-mich}

\begin{figure}

\begin{minipage}{0.80\linewidth}
Schauen Sie sich das Video mit einer
\href{https://youtu.be/jtNlzpcPr5Y}{Ansprache zur Motivation}
an.\end{minipage}%
%
\begin{minipage}{0.20\linewidth}

\begin{center}
\includegraphics[width=0.75\linewidth,height=\textheight,keepaspectratio]{005-orga_files/figure-pdf/unnamed-chunk-2-1.pdf}
\end{center}

\end{minipage}%

\end{figure}%

\subsection{Voraussetzungen}\label{voraussetzungen}

Um von diesem Kurs am besten zu profitieren, sollten Sie Folgendes
mitbringen:

\begin{itemize}
\tightlist
\item
  Bereitschaft, Neues zu lernen
\item
  Bereitschaft, nicht gleich aufzugeben
\item
  Kenntnis grundlegender Methoden wissenschaftlichen Arbeitens
\end{itemize}

Was Sie \emph{nicht} brauchen, sind besondere Mathe- oder
Statistik-Vorkenntnisse.

\subsection{Überblick über das
Buch}\label{uxfcberblick-uxfcber-das-buch}

Abb. Abbildung~\ref{fig-ueberblick} gibt einen Überblick über den
Verlauf und die Inhalte des Buches. Das Diagramm hilft Ihnen zu
verorten, wo welches Thema im Gesamtzusammenhang steht.

\begin{figure}[H]

\centering{

\includegraphics[width=4in,height=3.1in]{005-orga_files/figure-latex/mermaid-figure-1.png}

}

\caption{\label{fig-ueberblick}Überblick über den Inhalt und Verlauf des
Buches}

\end{figure}%

Das Diagramm zeigt auch den Ablauf einer typischen Datenanalyse.
Natürlich kann man sich auch andere sinnvolle Darstellungen dieses
Ablaufs vorstellen.

\section{Lernhilfen}\label{lernhilfen}

\begin{figure}

\begin{minipage}{0.80\linewidth}
Auf der Webseite
\href{https://sebastiansauer.github.io/Datenwerk/}{\enquote{Datenwerk}}
wird eine \emph{große Zahl an Aufgaben }bereitgestellt. Am Ende jedes
Kapitels dieses Buchs finden Sie eine Auswahl an Aufgabennamen, die Sie
im Datenwerk lösen können. Beachten Sie die
\href{https://sebastiansauer.github.io/Datenwerk/hinweise}{Hinweise zu
den Aufgaben}.\end{minipage}%
%
\begin{minipage}{0.20\linewidth}

\begin{center}
\includegraphics[width=0.75\linewidth,height=\textheight,keepaspectratio]{005-orga_files/figure-pdf/unnamed-chunk-4-1.pdf}
\end{center}

\end{minipage}%

\end{figure}%

Außerdem tauchen \emph{im Verlauf jedes Kapitels Übungsaufgaben} an
verschiedenen Stellen auf, so dass Sie den jeweiligen Stoff sofort üben
und Ihr Verständnis prüfen können.

\begin{figure}

\begin{minipage}{0.80\linewidth}
Schauen Sie sich mal den YouTube-Kanal
\texttt{@sebastiansauerstatistics} an und dort die
\href{https://www.youtube.com/playlist?list=PLRR4REmBgpIEaIyeNBgNGPgmhQJ_T1y8_}{Playlist
\enquote{R}}. Dort finden Sie \emph{Videos zum Thema dieses
Buches}.\end{minipage}%
%
\begin{minipage}{0.20\linewidth}

\begin{center}
\includegraphics[width=0.75\linewidth,height=\textheight,keepaspectratio]{005-orga_files/figure-pdf/unnamed-chunk-5-1.pdf}
\end{center}

\end{minipage}%

\end{figure}%

Im Buch finden sich mehrere Arten von \emph{Hervorhebungen}, wie
Beispiele, Fehlerquellen, Definitionen und Hinweise (und verlinkt), so
dass Sie schnell finden können.

Das Buch verweißt auf eine Reihe von \emph{Online-Materialien}. So ist
der gesamte R-Code für dieses Buch auf dem Github-Repo dieses Buches zu
finden: \url{https://github.com/sebastiansauer/statistik1}.

\section{Software: R}\label{software-r}

Sie benötigen R, RStudio und einige R-Pakete für diesen Kurs. Dieses
Buch enthält \enquote{mittel} viel R. Auf fortgeschrittene R-Techniken
wurde aber komplett verzichtet. Dem einen Anfänger oder der anderen
Anfängerin mag es dennoch \enquote{viel Code} erscheinen. Es wäre ja
auch möglich gewesen, auf R zu verzichten und stattdessen eine
\enquote{Klick-Software} zu verwenden.
\href{https://jasp-stats.org/}{JASP} oder
\href{https://www.jamovi.org/}{Jamovi} sind Beispiele für tolle Software
aus dieser Kategorie. Ich glaube aber, der Verzicht auf eine
Skriptsprache (R) wäre ein schlechter Dienst an den Studentis. Mit Blick
auf eine \enquote{High-Tech-Zukunft} sollte man zumindest mit etwas
Computer-Code vertraut sein. Auf Computercode zu verzichten erschiene
mir daher fahrlässig für die \enquote{Zukunftsfestigkeit} der
Ausbildung.

Sie finden den R-Code für jedes Kapitel
\href{https://github.com/sebastiansauer/statistik1/tree/main/R-code-for-all-chapters}{auf
Github}.\footnote{\url{https://github.com/sebastiansauer/statistik1/tree/main/R-code-for-all-chapters}}

\section{Griechische Buchstaben}\label{sec-greek}

In diesem Buch werden ein paar (wenige) griechische Buchstaben
verwendet, die in der Statistik üblich sind. Häufig werden
\emph{griechische} Buchstaben verwendet, um eine Grundgesamtheit
(Population) zu beschreiben (die meistens unbekannt ist). Lateinische
(\enquote{normale}) Buchstaben werden demgegenüber verwendet, um eine
Stichprobe (Datensatz, vorliegende Daten) zu beschreiben.
Tabelle~\ref{tbl-griech} stellt diese Buchstaben zusammen mit ihrer
Aussprache und Bedeutung vor.

\begin{longtable}[]{@{}lllr@{}}
\caption{Griechische Buchstaben, die in diesem Buch verwendet
werden.}\label{tbl-griech}\tabularnewline
\toprule\noalign{}
Zeichen & Aussprache & Buchstabe & Bedeutung in der Statistik \\
\midrule\noalign{}
\endfirsthead
\toprule\noalign{}
Zeichen & Aussprache & Buchstabe & Bedeutung in der Statistik \\
\midrule\noalign{}
\endhead
\bottomrule\noalign{}
\endlastfoot
\(\beta\) & beta & b & Regressionskoeffizent \\
\(\mu\) & mü & m & Mittelwert \\
\(\sigma\) & sigma & s & Streuung \\
\(\Sigma\) & Sigma & S & Summenzeichen \\
\(\rho\) & rho & r & Korrelation (nach Pearson) \\
\end{longtable}

Mehr griechische Buchstaben finden sich
\href{https://de.wikipedia.org/wiki/Griechisches_Alphabet}{z.B. in
Wikipedia}.\footnote{\url{https://de.wikipedia.org/wiki/Griechisches_Alphabet}}

\part{Vorbereiten}

\chapter{Prüfung}\label{pruxfcfung-1}

Die folgenden Hinweise sind dem
\href{https://hinweisbuch.netlify.app/}{Hinweisbuch} des Autors
entnommen. Lesen Sie auch die übrigen Hinweise dort.\footnote{\url{https://hinweisbuch.netlify.app/}}

\section{Prüfungleistung}\label{pruxfcfungleistung-1}

Die Prüfungsleistung besteht aus einer Hauptleistung (keine
Bonusleistung).

Die Hauptleistung besteht aus einer Projektarbeit im Form eines
\emph{Prognosewettbewerbs}.

\section{Zum Prognosewettbewerb}\label{zum-prognosewettbewerb-1}

\href{https://hinweisbuch.netlify.app/}{Im Hinweisbuch} finden Sie
\href{https://hinweisbuch.netlify.app/080-hinweise-pruefung-prognosewettbewerb-frame}{Hinweise
zur Prüfung}.\footnote{\url{https://hinweisbuch.netlify.app/080-hinweise-pruefung-prognosewettbewerb-frame}}

\section{Prüfungsrelevanter Stoff}\label{pruxfcfungsrelevanter-stoff-1}

Beachten Sie die
\href{https://hinweisbuch.netlify.app/010-hinweise-pruefung-allgemein-frame\#pr\%C3\%BCfungsrelevanter-stoff}{Hinweise
zum prüfungsrelevanten Stoff}.\footnote{\url{https://hinweisbuch.netlify.app/010-hinweise-pruefung-allgemein-frame\#pr\%C3\%BCfungsrelevanter-stoff}}

\section{Wie kann ich mich auf die Prüfung
vorbereiten?}\label{wie-kann-ich-mich-auf-die-pruxfcfung-vorbereiten-1}

\href{https://hinweisbuch.netlify.app/150-hinweise-pruefungsvorbereitung-frame}{Hier}
finden Sie Hinweise zur Prüfungsvorbereitung.\footnote{\url{https://hinweisbuch.netlify.app/150-hinweise-pruefungsvorbereitung-frame}}

\section{Allgemeine
Prüfungsheinweise}\label{allgemeine-pruxfcfungsheinweise}

Die folgenden Hinweise gelten grundsätzlich, d.h. soweit nicht anders in
der jeweiligen Prüfung bzw. der jeweiligen Aufgabe angegeben.
Nichtbeachten von Prüfungshinweisen kann zu Punkteabzug oder
Nichtbestehen führen. Lesen Sie sich diese Hinweise im eigenen Interesse
sorgfältig durch. Kenntnis dieser Hinweise wird bei der Begutachtung
vorausgesetzt.

Für eine einfachere Kommunikation kontaktieren Sie mich per E-Mail bei
Fragen, die nur Sie betreffen. Bei Fragen von allgemeinem Interesse
(z.B. \enquote{Bis wann müssen wir die Arbeit abgeben?}) nutzen Sie
bitte (sofern verfügbar) das Kursforum, damit die Kommilitonen auch von
dem Austausch profitieren.

Beachten Sie die
\href{https://hinweisbuch.netlify.app/010-hinweise-pruefung-allgemein-frame}{allgemeinen
Prüfungshinweise}.\footnote{\url{https://hinweisbuch.netlify.app/010-hinweise-pruefung-allgemein-frame}}

\section{Lieblingsfehler}\label{lieblingsfehler-1}

Vermeiden Sie diese
\href{https://hinweisbuch.netlify.app/170-beispiele-fehler-prognosewettbewerb-frame}{häufigen
Fehler im Prognosewettbewerb}.\footnote{\url{https://hinweisbuch.netlify.app/170-beispiele-fehler-prognosewettbewerb-frame}}

\section{Fazit}\label{fazit-1}

\emoji{four-leaf-clover}\emoji{four-leaf-clover}\emoji{four-leaf-clover}VIEL
ERFOLG!\emoji{four-leaf-clover}\emoji{four-leaf-clover}\emoji{four-leaf-clover}

\chapter{Rahmen}\label{rahmen}

\[
\definecolor{ycol}{RGB}{230,159,0}
\definecolor{modelcol}{RGB}{86,180,233}
\definecolor{errorcol}{RGB}{0,158,115}
\definecolor{beta0col}{RGB}{213,94,0}
\definecolor{beta1col}{RGB}{0,114,178}
\definecolor{xcol}{RGB}{204,121,167}
\]

\section{Lernsteuerung}\label{lernsteuerung}

Abbildung~\ref{fig-ueberblick} zeigt den Standort dieses Kapitels im
Lernpfad und gibt damit einen Überblick über das Thema dieses Kapitels
im Kontext aller Kapitel. Abbildung~\ref{fig-tidy5} zeigt, dass unser
Vorgehen in diesem Buch einem Fließband gleicht: Schritt für Schritt, in
der richtigen Reihenfolge, vom Anfang bis Ende, erarbeiten wir unser
\enquote{Datenprodukt}.

\begin{figure}[H]

\centering{

\includegraphics[width=0.75\linewidth,height=\textheight,keepaspectratio]{img/tidydata_5.jpg}

}

\caption{\label{fig-tidy5}Datenanalyse als eine Abfolge am Fließband
(Horst, 2024)}

\end{figure}%

\subsection{Lernziele}\label{lernziele-1}

\begin{itemize}
\tightlist
\item
  Sie können eine Definition von Statistik wiedergeben.
\item
  Sie können eine Definition von Daten wiedergeben.
\item
  Sie können den Begriff Tidy-Daten erläutern.
\item
  Sie können Beispiele für verschiedene Skalenniveaus nennen.
\end{itemize}

\subsection{Erfolsgrezept}\label{erfolsgrezept}

Ihren Lernerfolg kann man als von drei Faktoren abhängig betrachten: 1)
Ihrer Lehrkraft, 2) Ihrer Mitarbeit im Unterricht und 3) Ihrem
Eigenstudium zuhause (Vor- bzw. Nachbereitung des Unterrichts), s.
Abbildung~\ref{fig-erfolgsrezept}.

\begin{figure}[H]

\centering{

\includegraphics[width=4in,height=1.02in]{010-rahmen_files/figure-latex/mermaid-figure-1.png}

}

\caption{\label{fig-erfolgsrezept}Ihr Lernerfolg besteht aus drei
Komponenten: Der Lehrkraft, Ihrer Mitarbeit im Unterricht und Ihrem
Eigenstudium, d.h. Ihrer Vor- bzw. Nachbereitung zuhause.}

\end{figure}%

Eine gute Lehrkraft ist wie der Funke, der eine (Lern-)Flamme entzündet.
Aber es braucht Brennmaterial, einen Holzscheit, das ist Ihre
Konzentration im Unterricht. Ein Holzscheit alleine ist aber nicht
nachhaltig, es braucht viel mehr Brennmaterial, um das Feuer am Leben zu
halten: Ihr Eigenstudium, das stetige, selbständige Beschäftigen mit der
Materie.

\section{Was ist Statistik und wozu ist sie
gut?}\label{was-ist-statistik-und-wozu-ist-sie-gut}

In diesem Buch werden die Begriffe \emph{Statistik}, \emph{Datenanalyse}
und \emph{Data Science} synonym verwendet.

\begin{exercise}[Nutzen von
Statistik]\protect\hypertarget{exr-nutzen-statistik}{}\label{exr-nutzen-statistik}

Fragen Sie einen Chatbot (wie ChatGPT), was der Nutzen von Statistik
ist. Diskutieren Sie die Antwort. \(\square\)

\end{exercise}

Hm. Vielleicht fragen wir mal ChatGPT, was es unter Statistik versteht:

\begin{quote}
{\emoji{student}} Was ist Statistik? Bitte definiere verständlich.
\end{quote}

\begin{quote}
{\emoji{robot}} Statistik ist eine Wissenschaft, die sich mit der
Sammlung, Analyse, Interpretation, Präsentation und Übertragung von
Daten beschäftigt. Ziel ist es, Muster und Trends in den Daten zu
erkennen und verlässliche Schlussfolgerungen über eine größere Gruppe
(z.B. Bevölkerung) auf der Basis einer Stichprobe (Teilmenge) zu
treffen. Die Statistik hilft uns bei der Beantwortung von Fragen, indem
sie uns dabei hilft, die Unsicherheit in den Daten zu messen und zu
verstehen\footnote{Release 2023-Jan}.
\end{quote}

Es gibt mehrere Definition von Statistik; hier ist eine, s.
Definition~\ref{def-statistik}.

\begin{definition}[Statistik]\protect\hypertarget{def-statistik}{}\label{def-statistik}

Statistik fasst Daten zusammen, um wesentliche Informationen den Daten
zu entnehmen und beschreibt die Ungewissheit unserer Schlüsse (Kaplan,
2009; R. A. Poldrack, 2023) .\(\square\)

\end{definition}

Betrachten wir die zwei Bestimmungsstücke einer Definition von Statistik
genauer:

\begin{enumerate}
\def\labelenumi{\arabic{enumi}.}
\tightlist
\item
  Daten zusammenfassen
\item
  Ungewissheit beschreiben
\end{enumerate}

\subsection{Daten zusammenfassen}\label{daten-zusammenfassen}

Abbildung~\ref{fig-zsmnfassen} verdeutlicht das Prinzip des
Zusammenfassens von Daten. Anschaulich gesprochen: Eine Menge von Zahlen
wird zu einer einzelnen Zahl \enquote{zusammengedampft}. Eine einzelne
Zahl ist wesentlich besser zu verstehen als eine große Menge von Zahlen.
Bei vielen Zahlen würde man den Überblick verlieren.

\begin{figure}[H]

\begin{minipage}{0.45\linewidth}

\centering{

\includegraphics[width=1\linewidth,height=\textheight,keepaspectratio]{010-rahmen_files/figure-pdf/fig-zsmnfassen-1.pdf}

}

\subcaption{\label{fig-zsmnfassen-1}Zusammenfassen einer Variable zu
einem Punktwert, hier zum Mittelwert}

\end{minipage}%
%
\begin{minipage}{0.10\linewidth}
~\end{minipage}%
%
\begin{minipage}{0.45\linewidth}

\centering{

\includegraphics[width=1\linewidth,height=\textheight,keepaspectratio]{010-rahmen_files/figure-pdf/fig-zsmnfassen-2.pdf}

}

\subcaption{\label{fig-zsmnfassen-2}Zusammenfassen zweier Variablen zu
einer Geraden}

\end{minipage}%

\caption{\label{fig-zsmnfassen}Daten zusammenfassen}

\end{figure}%

\subsection{Unterschiedlichkeit
messen}\label{unterschiedlichkeit-messen}

Eine allgegenwärtige Tatsache ist, dass die Dinge der Welt sich
unterscheiden, etwa, dass Exemplare einer Gattung sich unterscheiden. So
sind nicht alle Menschen gleich groß, nicht alle Bücher gleich lang oder
nicht alle Tage gleich warm.

Ein zentrales Vorgehen bei statistischen Analysen ist es, die
\emph{Unterschiedlichkeit der Dinge} zu beschreiben, präziser gesagt:
die \emph{Variation zu quantifizieren}. Betrachten wir dazu das Beispiel
in s. Abbildung~\ref{fig-groesse}.

\begin{figure}[H]

\centering{

\includegraphics[width=1\linewidth,height=\textheight,keepaspectratio]{010-rahmen_files/figure-pdf/fig-groesse-1.png}

}

\caption{\label{fig-groesse}Wenig Variation in der Körpergröße bei den
Basketballern. Alles lange Kerle. Viel Variation bei den Schachspielern:
Manche sind klein, ander groß.}

\end{figure}%

Bei den Basketballern gibt es \emph{geringe} Variation in der
Körpergröße - alle sind groß, ähnlich groß. Bei den Schachspielern gibt
es (im Verhältnis) \emph{hohe} Variation: Einige Personen sind groß,
andere klein.

Eine \emph{Abweichung} (auch \emph{Residuum}) genannt, zeigt hier die
Differenz von Mittelwert und dem Wert der Körpergröße bei der jeweiligen
Person. Wenn wir allgemein von einer Person \(i\) sprechen, Das Merkmal
\emph{Körpergröße} mit \(X\) bezeichnen und den Mittelwert der
Körpergröße als \(\bar{x}\) (\enquote{x quer}), dann können wir knapp
und präzise das Residuum der \(i\)-ten Person mit \(r_i\) bezeichnen und
entsprechend definieren.

\begin{definition}[Residuum]\protect\hypertarget{def-residuum}{}\label{def-residuum}

Das Residuum des Merkmals \(X\) der \(i\)-ten Beobachtung ist definiert
als die Differenz vom Wert \(x_i\) und einem Referenzwert, etwa dem
Mittelwert, \(\bar{x}\):

\(r_i = x_i - \bar{x}\). \(\square\)

\end{definition}

\section{Was ist das Ziel Ihrer
Analyse?}\label{was-ist-das-ziel-ihrer-analyse}

\subsection{Arten von Zielen}\label{arten-von-zielen}

\begin{figure}[H]

\centering{

\includegraphics[width=4in,height=0.97in]{010-rahmen_files/figure-latex/mermaid-figure-6.png}

}

\caption{\label{fig-ziele}Zielarten einer Datenanalyse}

\end{figure}%

Beispiele für die einzelnen Zielarten der Datenanalyse:

\begin{itemize}
\tightlist
\item
  \emph{Beschreiben}: Wie groß ist der Gender-Paygap in der Branche X im
  Zeitraum Y?
\item
  \emph{Vorhersagen}: Wenn ich 100 Stunden auf die Statistikklausur
  lernen, welche Note kann ich dann erwarten?
\item
  \emph{Erklären}: Wie viel bringt mir das Lernen auf die
  Statistikklausur?
\end{itemize}

\begin{exercise}[]\protect\hypertarget{exr-ziele-stat}{}\label{exr-ziele-stat}

Benennen Sie Beispiele für die die drei Zielarten von Datenanalysen!
\(\square\)

\end{exercise}

\subsection{Forschungsfrage}\label{forschungsfrage}

Eine Forschungsfrage ist die Leitfrage Ihrer Analyse. Sie definiert, was
Sie herausfinden wollen. Häufig sind Forschungsfragen so aufgebaut:

\begin{quote}
Hat X einen Einfluss auf Y?
\end{quote}

Eine Forschungsfrage weist häufig folgende Struktur auf, s.
Abbildung~\ref{fig-fo-struktur}.

\begin{figure}[H]

\centering{

\includegraphics[width=4in,height=0.8in]{010-rahmen_files/figure-latex/mermaid-figure-5.png}

}

\caption{\label{fig-fo-struktur}Struktur eine Forschungsfrage}

\end{figure}%

\begin{example}[Forschungsfrage 1 - Lernen und
Prüfungserfolg]\protect\hypertarget{exm-fofrage1}{}\label{exm-fofrage1}

~

\begin{quote}
Hat Lernen (X) einen Einfluss auf den Prüfungserfolg (Y)? Verringert
Joggen (X) die Menge des Hüftgolds (Y)? Um welchen Betrag erhöht sich
der Umsatz (Y), wenn wir 1000€ mehr Werbung ausgeben? (X)\(\square\)
\end{quote}

\end{example}

\begin{example}[Forschungsfrage 2 - Handynutzung und
Konzentration]\protect\hypertarget{exm-braindrain}{}\label{exm-braindrain}

Eine Forschungsfrage könnte lauten zum Thema Handynutzung:

\begin{quote}
Verringert intensive Handynutzung die Konzentrationsfähigkeit?
\(\square\)
\end{quote}

\end{example}

\begin{example}[Forschungsfrage 3 - Produktmerkmale und
Verkaufserlös]\protect\hypertarget{exm-fofrage2}{}\label{exm-fofrage2}

Nach dem Studium haben Sie bei einem großen Online-Auktionshaus
angeheuert. Da Sie angaben, sich im Studium \st{intensiv} etwas mit
Statistik beschäftigt zu haben, hat man Sie in die Abteilung für
Forschung und Entwicklung (F\&E) gesteckt. Heute ist es Ihre Aufgabe,
Auktionen zur Spielekonsole Wii zu analysieren, genauer gesagt geht es
um das Spiel Mariokart. Ihre Forschungsfrage lautet:

\begin{quote}
Welche Produktmerkmale stehen mit einem hohen Verkaufserlös in
Zusammenhang?\(\square\)
\end{quote}

\end{example}

\subsection{Aus der Forschung:
Smartphone-Brain-Drain}\label{aus-der-forschung-smartphone-brain-drain-1}

Ward et al. (2017) untersuchten die Forschungsfrage, ob die bloße
Gegenwart eines Handies (z.B. wenn es vor Ihnen auf dem Tisch liegt)
dazu führt, dass man abgelenkt wird und daher schlechtere kognitive
Leistungen zeigt.

Leider schreiben die Autoren Ihre Hypothese nicht glasklar, aber
implizit ist obige Hypothese herauszulesen (S. 142):

\begin{quote}
First, smartphones may redirect the orientation of conscious attention
away from the focal task and toward thoughts or behaviors associated
with one's phone. Prior research provides ample evidence that \ldots{}
this digital distraction adversely affects both performance \ldots{} and
enjoyment.
\end{quote}

Später formulieren Sie Ihre Hypothese noch genauer (S. 143):

\begin{quote}
In two experiments, we test the hypothesis that the mere presence of
one's own smartphone reduces available cognitive capacity.
\end{quote}

Die Ergebnisse unterstützen Ihre Hypothese, s.
Abbildung~\ref{fig-braindrain}. Im Diagramm ist ersichtlich, dass die
kognitive Leistung (Y-Achse) sowohl in der Kapazität des
Arbeitsgedächtnisses (links) als auch in der fluiden Intelligenz
(rechts) am geringsten ist, wenn das Handy auf dem Schreibtisch (Desk)
liegt. Am besten ist die kognitive Leistung, wenn das Handy nicht im
Raum ist.\(\square\)

\begin{figure}[H]

\centering{

\pandocbounded{\includegraphics[keepaspectratio]{img/braindrain1.jpg}}

}

\caption{\label{fig-braindrain}Handy in Sichtweite verringert die
kognitiven Ressourcen, Ward et al. (2017), S. 145}

\end{figure}%

\begin{exercise}[]\protect\hypertarget{exr-braindrain-chatgpt}{}\label{exr-braindrain-chatgpt}

Fragen Sie einen Bot (z.B. ChatGPT) zum Stand der Forschung hinsichtlich
der Braindrain-Forschungsfrage. Diskutieren Sie die Antwort, auch in
ihren Grenzen. \(\square\)

\end{exercise}

\begin{tcolorbox}[enhanced jigsaw, bottomtitle=1mm, leftrule=.75mm, breakable, title=\textcolor{quarto-callout-caution-color}{\faFire}\hspace{0.5em}{Vorsicht}, bottomrule=.15mm, titlerule=0mm, left=2mm, opacityback=0, colframe=quarto-callout-caution-color-frame, rightrule=.15mm, colback=white, coltitle=black, toprule=.15mm, toptitle=1mm, colbacktitle=quarto-callout-caution-color!10!white, arc=.35mm, opacitybacktitle=0.6]

Es ist ein häufiger Fehler, in der Forschungsfrage zu formulieren
\enquote{X führt zu Y}, aber in der Analyse keine Methode zu verwenden,
die geeignet ist, kausale Zusammenhänge aufzudecken. Es reicht nicht,
dass man z.B. einen (negativen) Zusammenhang zwischen der Häufigkeit von
Smartphone-Nutzung und Konzentrationsfähigkeit findet (Schwaiger \&
Tahir, 2022), um zu sagen: \enquote{Daddeln macht dumm!}. Es könnte ja
z.B. auch umgekehrt sein. Platt gesagt: \enquote{Dummheit führt zu
Daddeln}. Weitere Erklärungen sind möglich. Vorsicht also mit
(vor)schnellen Aussagen zu kausalen Abhängigkeiten.

\end{tcolorbox}

\subsection{Der Prozess der
Datenanalyse}\label{der-prozess-der-datenanalyse}

Datenanalyse ist eine Art des Problemlösens. Anders gesagt, man macht es
nicht zum Spaß (jedenfalls nicht alle von uns), sondern um ein Ziel zu
erreichen, d.h. ein Problem zu lösen. Daher analysiert man nicht gleich
zu Anfang wild drauf los. Zunächst 1) klärt man das Problem und das
Ziel. Dann 2) plant man das Vorgehen, z.B. welche Daten man erheben
möchte. Als nächstes 3) erhebt man die Daten und bereitet sie auf.
Schließlich kann man sie 4) endlich analysieren. Aber Daten sprechen
nicht für sich, man muss sie 5) interpretieren und Schlüsse daraus
ziehen. Dazu gehört auch, dass man die Schwächen der eigenen Analyse
kritisch beleuchtet, vgl. Abbildung~\ref{fig-ppdac}. Diesen Ablauf nennt
man auch das PPDAC-Modell (MacKay \& Oldford, 2000):

\begin{itemize}
\tightlist
\item
  P: \emph{Problem} (Problem und Ziel und Sachgegenstand verstehen)
\item
  P: \emph{Plan} (Vorgehen planen)
\item
  D: \emph{Data} (Daten erheben und aufbereiten)
\item
  A: \emph{Analysis} (Daten analysieren)
\item
  C: \emph{Conclusions} (Schlussfolgerungen ziehen)
\end{itemize}

\begin{figure}[H]

\centering{

\includegraphics[width=4in,height=0.54in]{010-rahmen_files/figure-latex/mermaid-figure-4.png}

}

\caption{\label{fig-ppdac}Datenanalyse als Prozess: Das PPDAC-Modell}

\end{figure}%

\section{Was sind Daten?}\label{was-sind-daten}

\begin{definition}[Daten]\protect\hypertarget{def-daten}{}\label{def-daten}

Daten kann man als eine geordnete Folge von Zeichen
definieren.\(\square\)

\end{definition}

Daten kommen häufig in Tabellenform vor; so sind sie (oft) am besten zu
untersuchen, s. Tabelle~\ref{tbl-daten}. Die erste Spalte \texttt{id}
ist nur eine laufende Nummer. Sie dient dazu, die einzelnen
Beobachtungen (hier Studentis) identifizieren zu können und birgt
ansonsten keine Information. Beispiele für ID-Variablen sind z.B.
Matrikulationsnummer, Personalausweisnummern oder Bestellnummern.

\begin{table}

\caption{\label{tbl-daten}So sehen Daten in Form einer Tabelle aus.}

\centering{

\fontsize{12.0pt}{14.4pt}\selectfont
\begin{tabular*}{\linewidth}{@{\extracolsep{\fill}}rlr}
\toprule
id & name & note \\ 
\midrule\addlinespace[2.5pt]
1 & Anna & 1.3 \\ 
2 & Berta & 2.3 \\ 
3 & Carla & 3.0 \\ 
\bottomrule
\end{tabular*}

}

\end{table}%

\begin{example}[Daten zur Forschungsfrage
2]\protect\hypertarget{exm-daten}{}\label{exm-daten}

Hier ist ein Auszug der Daten zur Tabelle \texttt{mariokart}, s.
Tabelle~\ref{tbl-mariokart}.

\begin{longtable}[]{@{}rrrr@{}}

\caption{\label{tbl-mariokart}Auszug aus der Tabelle mariokart}

\tabularnewline

\toprule\noalign{}
n\_bids & start\_pr & total\_pr & wheels \\
\midrule\noalign{}
\endhead
\bottomrule\noalign{}
\endlastfoot
20 & 0.99 & 52 & 1 \\
13 & 0.99 & 37 & 1 \\
16 & 0.99 & 46 & 1 \\
18 & 0.99 & 44 & 1 \\
20 & 0.01 & 71 & 2 \\
19 & 0.99 & 45 & 0 \\

\end{longtable}

Eine Erklärung (Data-Dictionary) aller Variablen des Datensatzes
\texttt{mariokart} findet sich
\href{https://www.openintro.org/data/index.php?data=mariokart}{hier}.\footnote{\url{https://www.openintro.org/data/index.php?data=mariokart}}
\(\square\)

\end{example}

\begin{definition}[Data-Dictionary]\protect\hypertarget{def-datadict}{}\label{def-datadict}

Eine Erklärung, was die Namen einer Datentabelle bedeuten, nennt man
\emph{Codebook} or \emph{Data-Dictionary}.\(\square\)

\end{definition}

\subsection{Was ist eine Variable?}\label{was-ist-eine-variable}

\begin{definition}[Variable]\protect\hypertarget{def-var}{}\label{def-var}

Eine Variable ist ein Platzhalter, der für ein Merkmal steht, das
verschiedene Werte annehmen kann.\(\square\)

\end{definition}

Man kann sich eine Variable wie einen Behälter vorstellen, auf dem mit
einem Stift geschrieben steht, was für eine Art Inhalt darin ist, s.
Abbildung~\ref{fig-var-zuweisen}.

\begin{figure}[H]

\centering{

\includegraphics[width=0.25\linewidth,height=\textheight,keepaspectratio]{img/Variablen_zuweisen.png}

}

\caption{\label{fig-var-zuweisen}Wir definieren eine Variable
\enquote{temp} mit dem Inhalt \enquote{9}}

\end{figure}%

\subsection{Beobachtungseinheit}\label{beobachtungseinheit}

\begin{definition}[Beobachtungseinheit]\protect\hypertarget{def-beobeinheit}{}\label{def-beobeinheit}

Beobachtungseinheiten sind die Dinge, die wir untersuchen (beobachten).
Beobachtungseinheiten sind die Träger von Variablen.\(\square\)

\end{definition}

In Tabelle~\ref{tbl-daten} gibt es drei Variablen: \texttt{id},
\texttt{Name} und \texttt{Note}. Es gibt auch drei
Beobachtungseinheiten: \emph{Anna}, \emph{Berta} und \emph{Carla.}

\subsection{Wert}\label{wert}

\begin{definition}[Wert]\protect\hypertarget{def-wert}{}\label{def-wert}

Ein \emph{Wert} ist der Inhalt einer Variablen.\(\square\)

\end{definition}

In Abbildung~\ref{fig-var-zuweisen} ist der Wert von \texttt{temp} 9. In
Tabelle~\ref{tbl-daten} hat die Variable \texttt{name} drei Elemente:
Anna, Berta, Carla. Der Wert des 2. Elements ist Berta.

\begin{definition}[Ausprägung]\protect\hypertarget{def-auspraegung}{}\label{def-auspraegung}

Als \emph{Ausprägungen} bezeichnet man die verschiedenen Werte einer
Variablen. \(\square\)

\end{definition}

\begin{example}[]\protect\hypertarget{exm-geschlecht}{}\label{exm-geschlecht}

In einer Studie wurden zehn Probanden untersucht. Die Variable
\texttt{geschlecht} dokumentiert die Geschlechter der Personen:

\begin{Shaded}
\begin{Highlighting}[]
\NormalTok{geschlecht }\OtherTok{\textless{}{-}} \FunctionTok{c}\NormalTok{(}\StringTok{"Mann"}\NormalTok{, }\StringTok{"Frau"}\NormalTok{, }\StringTok{"Frau"}\NormalTok{, }\StringTok{"Frau"}\NormalTok{, }\StringTok{"Mann"}\NormalTok{,}
                \StringTok{"Frau"}\NormalTok{, }\StringTok{"Mann"}\NormalTok{, }\StringTok{"Mann"}\NormalTok{, }\StringTok{"divers"}\NormalTok{, }\StringTok{"Frau"}\NormalTok{)}
\NormalTok{geschlecht}
\DocumentationTok{\#\#  [1] "Mann"   "Frau"   "Frau"   "Frau"   "Mann"   "Frau"   "Mann"   "Mann"  }
\DocumentationTok{\#\#  [9] "divers" "Frau"}
\end{Highlighting}
\end{Shaded}

In dieser Variable (die aus 10 Werten besteht) finden sich drei
Ausprägungen: divers, Frau, Mann.\(\square\)

\end{example}

\begin{tcolorbox}[enhanced jigsaw, bottomtitle=1mm, leftrule=.75mm, breakable, title=\textcolor{quarto-callout-tip-color}{\faLightbulb}\hspace{0.5em}{Tipp}, bottomrule=.15mm, titlerule=0mm, left=2mm, opacityback=0, colframe=quarto-callout-tip-color-frame, rightrule=.15mm, colback=white, coltitle=black, toprule=.15mm, toptitle=1mm, colbacktitle=quarto-callout-tip-color!10!white, arc=.35mm, opacitybacktitle=0.6]

Gerade haben Sie etwas Computer-Syntax gesehen, genauer gesagt, Befehle
aus der Programmiersprache \emph{R}. Bisher haben wir diese Befehle
nicht kennengelernt. Sie verstehen Sie vermutlich (nicht ganz).
Ignorieren Sie diese Befehle einfach erstmal.

\end{tcolorbox}

\subsection{Tidy Data}\label{tidy-data}

\begin{definition}[Tidy
Data]\protect\hypertarget{def-tidy}{}\label{def-tidy}

Unter \emph{Tidy-Data} (tidy data, \enquote{Normalform}) versteht man
eine Tabelle, in der jede Zeile eine Beobachtungseinheit darstellt, jede
Spalte eine Variable und jede Zelle der Tabelle einen Wert, s.
Abbildung~\ref{fig-tidy1}. (Zusätzlich ist noch eine \enquote{Kopfzeile}
erlaubt, in der die Namen der Variablen stehen.)\(\square\)

\end{definition}

Tabelle~\ref{tbl-daten} ist ein Beispiel für Tidy-Data.
Abbildung~\ref{fig-tidy1} zeigt ein Sinnbild für Tidy-Data (Wickham \&
Grolemund, 2018). Und

\begin{figure}[H]

\centering{

\includegraphics[width=0.75\linewidth,height=\textheight,keepaspectratio]{img/tidy-1.png}

}

\caption{\label{fig-tidy1}Tidy-Data-Sinnbild (Wickham, 2023)}

\end{figure}%

\begin{tcolorbox}[enhanced jigsaw, bottomtitle=1mm, leftrule=.75mm, breakable, title=\textcolor{quarto-callout-important-color}{\faExclamation}\hspace{0.5em}{Wichtig}, bottomrule=.15mm, titlerule=0mm, left=2mm, opacityback=0, colframe=quarto-callout-important-color-frame, rightrule=.15mm, colback=white, coltitle=black, toprule=.15mm, toptitle=1mm, colbacktitle=quarto-callout-important-color!10!white, arc=.35mm, opacitybacktitle=0.6]

Für eine statistische Analyse ist es oft sinnvoll, dass die Daten im
Tidy-Format vorliegen.

\end{tcolorbox}

Der Vorteil des Tidy-Formats ist es, dass man weiß, wie die Daten
aufgebaut sind. Außerdem können Statistikprogramme oft mit dieser Form
am besten umgehen, s. Abbildung~\ref{fig-tidy3}.

\begin{figure}[H]

\centering{

\includegraphics[width=0.75\linewidth,height=\textheight,keepaspectratio]{img/tidydata_3.jpg}

}

\caption{\label{fig-tidy3}Immer schön Ordnung halten\ldots{} (Horst,
2023)}

\end{figure}%

\begin{example}[]\protect\hypertarget{exm-widelong}{}\label{exm-widelong}

Ihre Firmat produziert zwei Produkte: Hämmer und Nägel. Im Folgenden
sind zwei Tabellen dargestellt, die die gleichen Informationen
darstellen: den Umsatz für die Jahre 2021 und 2022. Einmal ist dazu eine
Nicht-Tidy-Tabelle (Tabelle~\ref{tbl-untidy1}) und einmal eine
Tidy-Tabelle (Tabelle~\ref{tbl-tidy1}) verwendet. \(\square\)

\end{example}

\begin{longtable}[]{@{}lrrr@{}}

\caption{\label{tbl-untidy1}Beispiel für eine NICHT-Tidy-Tabelle
(Breitformat)}

\tabularnewline

\toprule\noalign{}
Produkt & Umsatz\_2021 & Umsatz\_2022 & Umsatz\_2023 \\
\midrule\noalign{}
\endhead
\bottomrule\noalign{}
\endlastfoot
Hämmer & 10 & 11 & 12 \\
Nägel & 15 & 10 & 5 \\

\end{longtable}

\begin{longtable}[]{@{}llr@{}}

\caption{\label{tbl-tidy1}Beispiel für eine Tidy-Tabelle (Langformat)}

\tabularnewline

\toprule\noalign{}
Produkt & Jahr & Umsatz \\
\midrule\noalign{}
\endhead
\bottomrule\noalign{}
\endlastfoot
Hämmer & 2021 & 10 \\
Hämmer & 2022 & 11 \\
Hämmer & 2023 & 12 \\
Nägel & 2021 & 15 \\
Nägel & 2022 & 10 \\
Nägel & 2023 & 5 \\

\end{longtable}

\begin{exercise}[]\protect\hypertarget{exr-widelong}{}\label{exr-widelong}

Suchen Sie ein Beispiel für eine Konfiguration einer Tabelle im Long-
vs.~Wide-Format. \(\square\)

\end{exercise}

\begin{quote}
{\emoji{student}} Wozu braucht man das Tidy-Format?
\end{quote}

\begin{quote}
{\emoji{woman-teacher}} In vielen Software-Programmen der Datenanalyse
weißt man z.B. der X- oder Y-Variable eine Spalte einer Tabelle zu.
Möchte man etwa die Veränderung des Umsatzes im Verlauf der Jahre
visualisieren oder analysieren, so braucht es die Spalten
\enquote*{Jahr} und \enquote*{Umsatz}, also ein Tidy-Format,
Tabelle~\ref{tbl-untidy1} bzw. Tabelle~\ref{tbl-tidy1}.
\end{quote}

Abbildung~\ref{fig-tidy} stellt auf Basis einer \enquote{Tidy-Tabelle}
(Tabelle~\ref{tbl-tidy1}) ein Diagramm dar. Ohne Tidy-Daten wäre dieses
Diagramm nicht (so einfach) zu erstellen gewesen.

\begin{figure}[H]

\centering{

\includegraphics[width=1\linewidth,height=\textheight,keepaspectratio]{010-rahmen_files/figure-pdf/fig-tidy-1.pdf}

}

\caption{\label{fig-tidy}Beispiel für eine Visualisierung auf Basis
einer Tidy-Tabelle, vgl. Tabelle~\ref{tbl-tidy1}}

\end{figure}%

\subsection{Je mehr, desto besser (?)}\label{je-mehr-desto-besser}

Was Daten betrifft, könnte man behaupten: \enquote{Viel hilft viel} oder
\enquote{Je mehr, desto besser}. Natürlich unter sonst gleichen
Umständen.\footnote{Ceteris paribus, auf Latein, hört sich gleich viel
  schlauer an.} Viel Datenmüll ist natürlich nicht besser als ein paar
knappe, wasserdichte Fakten!

\begin{example}[]\protect\hypertarget{exm-samplesize}{}\label{exm-samplesize}

Um Ihre eigene Lehraktivität zu organisieren, wollen Sie sich ein Bild
machen, wie viel Ihre Nebensitzerinner und Nebensitzer im Hörsaal so
lernen. Sie blicken nach links und fragen \enquote{wie viel lernst du
so?}. Sie blicken nach recht und wiederholen die Frage gerichtet an den
Kommilitonen, der rechts neben Ihnen sitzt. Dann addieren Sie die zwei
Zahlen (unter der Annahme, dass Sie zwei Zahlen bekommen haben), und
teilen durch zwei, um den Mittelwert zu erhalten.\(\square\)

\end{example}

Ein kritischer Geist könnte anmerken, dass Sie besser die Untersuchung
nicht gemacht hätten (auch wenn Sie, vielleicht ohne zu wollen, eine
statistische Untersuchung angestellt haben). Denn bei so wenig befragten
Personen ist die Ungenauigkeit Ihrer Schätzung der typischen Lernzeit
bei Studentis einfach zu hoch.

Abbildung~\ref{fig-sample-estimate} veranschaulicht, dass man einen
Mittelwert genauer schätzen kann, wenn man auf eine größere Stichprobe
zurückgreift. Das Teilbild links zeigt den Mittelwert einer Stichprobe
mit \(n=20\) Beobachtungen. Das Teilbild rechts zeigt den Mittelwert
einer Stichprobe mit \(n=200\) Beobachtungen (jeweils aus der gleichen
Grundgesamtheit). Wie man sieht, ist im linken Teilbild die Streuung
(Variation) höher als im rechten Teilbild:

\begin{figure}[H]

\centering{

\includegraphics[width=1\linewidth,height=\textheight,keepaspectratio]{010-rahmen_files/figure-pdf/fig-sample-estimate-1.pdf}

}

\caption{\label{fig-sample-estimate}Schätzgenauigkeit als Funktion der
Stichprobengröße. Jeder Punkt stellt eine Stichprobe dar, entweder mit
n=20 (links) oder mit n=200 (rechts). Kleine Stichproben (links) haben
im Schnitt eine größere Abweichung vom wahren Mittelwert als größere
Stichproben (rechts).}

\end{figure}%

\begin{tcolorbox}[enhanced jigsaw, bottomtitle=1mm, leftrule=.75mm, breakable, title=\textcolor{quarto-callout-important-color}{\faExclamation}\hspace{0.5em}{Wichtig}, bottomrule=.15mm, titlerule=0mm, left=2mm, opacityback=0, colframe=quarto-callout-important-color-frame, rightrule=.15mm, colback=white, coltitle=black, toprule=.15mm, toptitle=1mm, colbacktitle=quarto-callout-important-color!10!white, arc=.35mm, opacitybacktitle=0.6]

Mehr Daten = genauere Ergebnisse (unter sonst gleichen Umständen)
\(\square\)

\end{tcolorbox}

\begin{exercise}[Live-Experiment zum Effekt der
Stichprobengröße]\protect\hypertarget{exr-kleine-grosse-stipro}{}\label{exr-kleine-grosse-stipro}

In diesem Live-Experiment untersuchen wir den Effekt der
\emph{Stichprobengröße} auf die Streuung des Mittelwerts in der
\emph{Stichprobe.} Streuen die Ergebnisse mehr in kleinen Stichproben
als in großen? Probieren wir es aus!

In diesem Experiment werfen Sie (in kleinen Gruppen) eine Münze (auf
faire Art und Weise) und notieren das Ergebnis (Kopf oder Zahl). Uns
interessiert dabei die Frage, ob die Ergebnisse bei kleinen Stichproben
(n=5 Münzwürfe) anders streuen als in großen Stichproben (n=20
Münzwürfe).

\begin{figure}

\begin{minipage}{0.80\linewidth}
Sie brauchen nur experimentierfreudige Partner (Kleingruppen mit 2-4
Personen), eine faire Münze und dann kann's los gehen!
\href{https://docs.google.com/forms/d/e/1FAIpQLSeAwqNyZtyQwttq5JrQdQ2AO7w5vzcVDXjiejKnyFNxiWtEag/viewform?usp=sf_link}{Klicken
Sie hier, um mit dem Experiment zu starten}.\end{minipage}%
%
\begin{minipage}{0.20\linewidth}

\begin{center}
\includegraphics[width=0.75\linewidth,height=\textheight,keepaspectratio]{010-rahmen_files/figure-pdf/unnamed-chunk-16-1.pdf}
\end{center}

\end{minipage}%

\end{figure}%

Die Daten aller Versuche können Sie
\href{https://docs.google.com/spreadsheets/d/11mKFFpr-Y1CMPpq4dGA-JA_Z9jRkPbXolo54Y0G_2gE/edit?usp=sharing}{hier}
einsehen.\footnote{\url{https://tinyurl.com/3w8ke2n2}} \(\square\)

\end{exercise}

\begin{example}[Dorfschulen machen die schlauesten
Schüler!]\protect\hypertarget{exm-schule-samplesize}{}\label{exm-schule-samplesize}

In einer Pressemitteilung sei zu lesen, dass die besten Schüler in den
Dorfschulen zu finden seien (Das ist eine fiktive Geschichte). Mit etwas
Recherche finden Sie heraus, dass diese Aussage für belastbaren Daten
beruht: Tatsächlich sind die Notendurchschnitte auf den kleinen
Dorfschulen deutlich besser als in den großen Schulen in der Stadt. Also
stimmt die Behauptung der Pressemitteilung? Die gute Landluft lässt das
Hirn wachsen? Sie recherchieren noch etwas weiter in den Daten. Dann
fällt Ihnen auf: Die \emph{schlechtesten} Schüler kommen auch aus den
Dorfschulen! Eine statistische Erklärung bietet sich an: In den
Dorfschulen gibt es nur wenig Kinder und kleine Klassen -- die
Stichproben sind also klein. Bei kleinen Stichproben gibt es viel
Variation um den Mittelwert herum, s.
Abbildung~\ref{fig-sample-estimate}, und zwar nach oben (guter
Notenschnitt) und nach unten (schlechter Notenschnitt). \(\square\)

\end{example}

\section{Arten von Variablen}\label{sec-arten-variablen}

\subsection{Nach Position in der
Forschungsfrage}\label{nach-position-in-der-forschungsfrage}

Angenommen, Ihre Forschungsfrage lautet:

\begin{quote}
Hat Lernen einen Einfluss auf den Prüfungserfolg?
\end{quote}

In dem Fall gilt:

\begin{itemize}
\tightlist
\item
  \emph{Lernen} ist die Input-Variable, X-Variable, Ursache, unabhängig
  Variable (UV)
\item
  \emph{Prüfungserfolg} ist die Output-Variable, Y-Variable, Wirkung,
  abhängige Variable (AV)
\end{itemize}

Abbildung~\ref{fig-ueberblick-fragen} stellt diese beiden
\enquote{Positionen} einer Variable dar. Die erste Position ist vor dem
Pfeil. Die zweite Position ist nach dem Pfeil.

\begin{figure}[H]

\centering{

\includegraphics[width=4in,height=1.37in]{010-rahmen_files/figure-latex/mermaid-figure-3.png}

}

\caption{\label{fig-ueberblick-fragen}X und Y als synonyme Bezeichnungen
für Input- und Output-Variablen einer Forschungsfrage}

\end{figure}%

\begin{exercise}[]\protect\hypertarget{exr-uvav}{}\label{exr-uvav}

Überlegen Sie sich eine Forschungsfrage, die eine UV und eine AV
enthält. Nennen Sie einer anderen Person diese Forschungsfrage und
fragen Sie, was die UV und die AV ist. Bei richtiger Antwort belohnen
Sie großzügig. \(\square\)

\end{exercise}

\subsection{Nach dem Skalenniveau}\label{nach-dem-skalenniveau}

\begin{definition}[Skalenniveau]\protect\hypertarget{def-skalenniveau}{}\label{def-skalenniveau}

Der Begriff \emph{Skalenniveau} wird verwendet, um die Art und Menge der
Information, die in Variablen enthalten ist, zu benennen. Diese
Klassifikation basiert auf den Eigenschaften der Daten und den
mathematischen Operationen, die sinnvoll auf diese Daten angewendet
werden können. \(\square\)

\end{definition}

Abbildung~\ref{fig-skalenniveau} gibt einen Überblick über typisch
verwendete Skalenniveaus.

\begin{figure}[H]

\centering{

\includegraphics[width=4in,height=1.51in]{010-rahmen_files/figure-latex/mermaid-figure-2.png}

}

\caption{\label{fig-skalenniveau}Skalenniveaus}

\end{figure}%

\subsection{Beispiele für
Skalenniveaus}\label{beispiele-fuxfcr-skalenniveaus}

Beispiele zu den Skalenniveaus sind in Tabelle~\ref{tbl-skalen-bsps}
aufgeführt. \(\square\)

\begin{longtable}[]{@{}ll@{}}

\caption{\label{tbl-skalen-bsps}Beispiele für Skalenniveaus}

\tabularnewline

\toprule\noalign{}
Variable & Skalenniveau \\
\midrule\noalign{}
\endhead
\bottomrule\noalign{}
\endlastfoot
Haarfarbe & Nominalskala \\
Augenfarbe & Nominalskala \\
Geschlecht & Nominalskala \\
Automarke & Nominalskala \\
Partei & Nominalskala \\
Lieblingsessen & Ordinalskala \\
Medaillen beim 100-Meter-Lauf & Ordinalskala \\
Uniranking & Ordinalskala \\
IQ & Intervallskala \\
Extraversion & Intervallskala \\
Temperatur in Celcius & Intervallskala \\
Temperatur in Fahrenheit & Intervallskala \\
Temperatur in Kelvin & Verhältnisskala \\
Körpergröße & Verhältnisskala \\
Geschwindigkeit & Verhältnisskala \\
Länge & Verhältnisskala \\

\end{longtable}

Jenachdem über welches Skalenniveau eine Variable verfügt, sind
verschiedenen Rechenoperationen erlaubt, s.
{Tabelle~\ref{tbl-skalenniveaus-pdf}}.

\begin{longtable}[]{@{}
  >{\raggedright\arraybackslash}p{(\linewidth - 10\tabcolsep) * \real{0.2237}}
  >{\raggedright\arraybackslash}p{(\linewidth - 10\tabcolsep) * \real{0.1579}}
  >{\raggedright\arraybackslash}p{(\linewidth - 10\tabcolsep) * \real{0.1447}}
  >{\raggedright\arraybackslash}p{(\linewidth - 10\tabcolsep) * \real{0.1579}}
  >{\raggedright\arraybackslash}p{(\linewidth - 10\tabcolsep) * \real{0.1184}}
  >{\raggedright\arraybackslash}p{(\linewidth - 10\tabcolsep) * \real{0.1974}}@{}}

\caption{\label{tbl-skalenniveaus-pdf}Erlaubte Rechenoperationen nach
Skalenniveau}

\tabularnewline

\toprule\noalign{}
\begin{minipage}[b]{\linewidth}\raggedright
Skalenniveau
\end{minipage} & \begin{minipage}[b]{\linewidth}\raggedright
Quantitativ
\end{minipage} & \begin{minipage}[b]{\linewidth}\raggedright
Gleichheit
\end{minipage} & \begin{minipage}[b]{\linewidth}\raggedright
Reihenfolge
\end{minipage} & \begin{minipage}[b]{\linewidth}\raggedright
Addition
\end{minipage} & \begin{minipage}[b]{\linewidth}\raggedright
Multiplikation
\end{minipage} \\
\midrule\noalign{}
\endhead
\bottomrule\noalign{}
\endlastfoot
Nominalniveau & nein & ja & nein & nein & nein \\
Ordinalniveau & nein & ja & ja & nein & nein \\
Intervallniveau & ja & ja & ja & ja & nein \\
Verhältnisniveau & ja & ja & ja & ja & ja \\

\end{longtable}

Was soll das bedeuten, \enquote{Rechenoperationen}? Schauen wir uns für
jedes Skalenniveau ein \enquote{Rechenbeispiel} an.

\emph{Nominalskala}: Die Variable \emph{Geschlecht} ist nominalskaliert.
Das bedeutet, dass ihre Ausprägungen \emph{Frau} und \emph{Mann} z.B.
nicht (sinnvoll) addiert oder sonstwie \enquote{verrechnet} werden
können. Man könnte, z.B. um das Eintippen zu erleichtern, Frauen mit
\texttt{1} kodieren und Männer mit \texttt{2}. Damit darf man aber nicht
rechnen! Nicht addieren, nicht multiplizieren, etc. Es macht keinen Sinn
zu sagen: \enquote{Ich habe eine Frau und einen Mann in meiner Tabelle,
das ist im Schnitt ein diverses Geschlecht, weil der Mittelwert von 1
und 2 ist 1,5!} Die \emph{einzige} \enquote{Rechenoperation}, die man
auf der Nominalskala machen darf, ist die Prüfung auf \emph{Gleichheit}:
Mann kann feststellen, ob ein Objekt gleich zu einem anderen ist oder
unterschiedlich. Also ob zwei Personen das gleiche Geschlecht haben oder
von unterschiedlichem Geschlecht sind. Anders ausgedrückt:

\begin{itemize}
\tightlist
\item
  FRAU \(\ne\) MANN
\item
  FRAU \(=\) FRAU
\item
  MANN \(=\) MANN
\end{itemize}

\emph{Ordinalskala}: Diese Skala entspricht einer Rangordnung. Eine
Rangordnung ist etwa die geordnete Abfolge Ihres Leibgerichte (1. Pizza,
2. Spagetthi, 3. Schnitzel). Etwas \enquote{formaler} ausgedrückt:

\begin{itemize}
\tightlist
\item
  \(\text{Pizza} \succ \text{Spagetthi} \succ \text{Schnitzel}\)
\end{itemize}

Das komische Zeichen \(\succ\) soll heißen: \enquote{Ist auf meiner
Liste von Leibgerichten weiter oben, mag ich lieber}. Man kann aber
\emph{nicht} sagen, \enquote{Ich mag aber Pizza um 42\% mehr als die
Spagetthi und die wieder um 73\% mehr als ein Schnitzel!}. Zumindest
kann man das nicht ohne weitere Informationen und Annahmen. Es gibt also
Dinge auf der Welt, die man leicht in eine Rangordnung bringen kann,
aber die man nur schwer in der Größe der Unterschiede bemessen kann. Das
ist die Ordinalskala.

\begin{tcolorbox}[enhanced jigsaw, bottomtitle=1mm, leftrule=.75mm, breakable, title=\textcolor{quarto-callout-important-color}{\faExclamation}\hspace{0.5em}{Wichtig}, bottomrule=.15mm, titlerule=0mm, left=2mm, opacityback=0, colframe=quarto-callout-important-color-frame, rightrule=.15mm, colback=white, coltitle=black, toprule=.15mm, toptitle=1mm, colbacktitle=quarto-callout-important-color!10!white, arc=.35mm, opacitybacktitle=0.6]

Die Ordinalskale erlaubt, Objekte zu ordnen (hinsichtlich eines
Merkmals). Die Abstände zwischen den Objekten können nicht quantifiziert
werden. \(\square\)

\end{tcolorbox}

\emph{Intervallskala}: Das ist vielleicht eine Überraschung für Sie:
Wenn es heute 10°C hat und morgen 5°C -- dann ist es heute \emph{nicht}
doppelt so warm wie morgen. Ja, 10 ist das Doppelte von 5. Aber
\emph{10° Celcius} ist \emph{nicht} doppelt so warm wie 20° Celcius.
Wenn Sie das verwundert: Das ist normal, so geht es vielen Leuten, wenn
sie das zum ersten Mal hören. Der Grund, dass es nicht erlaubt ist,
Verhältnisse (wie doppelt/halb so viel etc.) auf der Celcius-Skala zu
bilden, ist, dass der Nullpunkt der Skala, 0° C, kein echter,
physikalischer Nullpunkt ist. Bei 0° C liegt eben nicht Null
Wärmeenergie vor. Stattdessen wurde eine Wärmenergiemenge gewählt, die
für uns Menschen ganz praktisch, da augenfällig ist: der Gefrierpunkt
von Wasser. Was bei der Intervallskala erlaubt ist, ist das Addieren
(und Subtrahieren): heute 10°C, morgen 5°C, das ist ein Unterschied von
5°C. Oder: Im Schnitt waren es 7,5°C, das ist genau in der Mitte von 5
und 10°C. Abbildung~\ref{fig-intervall} versinnbildlicht die
Intervallskala.

\begin{figure}[H]

\centering{

\includegraphics[width=1\linewidth,height=\textheight,keepaspectratio]{010-rahmen_files/figure-pdf/fig-intervall-1.png}

}

\caption{\label{fig-intervall}Ein Metermaß steckt im trüben Wasser. Auf
dem Metermaß können wir die aufgedruckten Zahlen ablesen. Aber wir
wissen nicht, ob der Metermaß auf dem Boden steht. Wir wissen demnach
nicht, ob der vom Metermaß angegebene Nullpunkt der wahre Nullpunkt
(Meeresboden) ist.}

\end{figure}%

\emph{Verhältnisskala}: Eine Verhältnisskala ist das, was man sich
gemeinhin unter einer metrische Variable vorstellt: Man kann
\enquote{normal} rechnen, alle Rechenoperationen sind erlaubt. Zuzüglich
zu denen, die auch in anderen, \enquote{niedrigeren} Skalenniveaus
erlaubt sind, ist das das Bilden von Verhältnissen -- Multiplizieren, s.
Abbildung~\ref{fig-verhaeltnis}.

\begin{figure}[H]

\centering{

\includegraphics[width=1\linewidth,height=\textheight,keepaspectratio]{010-rahmen_files/figure-pdf/fig-verhaeltnis-1.png}

}

\caption{\label{fig-verhaeltnis}Puh! Der rote Flitzer ist 10 Mal so
teuer wie die blaue Möhre. Kohlen zusammenkratzen.}

\end{figure}%

\begin{figure}

\begin{minipage}{0.80\linewidth}
In \href{https://www.youtube.com/watch?v=_mN3kFe56ng}{diesem Video} gibt
es noch ausführlichere Erklärung zum Thema Skalenniveaus.\end{minipage}%
%
\begin{minipage}{0.20\linewidth}

\begin{center}
\includegraphics[width=0.75\linewidth,height=\textheight,keepaspectratio]{010-rahmen_files/figure-pdf/qr-youtube-skalenniveaus-1.pdf}
\end{center}

\end{minipage}%

\end{figure}%

Außerdem können quantitative Variablen untergliedert werden in:

\begin{itemize}
\tightlist
\item
  \emph{stetige} Variablen, das sind Variablen, bei denen man zwischen
  zwei Ausprägungen immer noch eine weitere quetschen kann. So gibt es
  einen Wert für die Köpergröße zwischen 1.60\,m und 1.61\,m. Und einen
  Wert zwischen 1.601\,m und 1.602\,m, etc.
\item
  \emph{diskrete} Variablen, das sind metrische Variablen, die nur
  bestimmte Ausprägungen haben, häufig sind das die natürlichen Zahlen:
  \(1,2,...\). Ein Beispiel wäre die Anzahl der Kinder in einer Familie.
\end{itemize}

\begin{tcolorbox}[enhanced jigsaw, bottomtitle=1mm, leftrule=.75mm, breakable, title=\textcolor{quarto-callout-tip-color}{\faLightbulb}\hspace{0.5em}{Tipp}, bottomrule=.15mm, titlerule=0mm, left=2mm, opacityback=0, colframe=quarto-callout-tip-color-frame, rightrule=.15mm, colback=white, coltitle=black, toprule=.15mm, toptitle=1mm, colbacktitle=quarto-callout-tip-color!10!white, arc=.35mm, opacitybacktitle=0.6]

Fragen nach Skalenniveaus gehören zu den Lieblingsprüfungsfragen in
diesem Themenbereich. Sie sind gut beraten, sich gerade mit dieser Frage
intensiver zu beschäftigen. Auch in thematisch angrenzenden Fächern wird
immer wieder die Frage nach dem Skalennvieau aufgeworfen. Das zeigt
natürlich auch die hohe Relevanz des Themas.

\end{tcolorbox}

\begin{exercise}[]\protect\hypertarget{exr-skalenniveaus}{}\label{exr-skalenniveaus}

Überlegen Sie sich für einige Variablen die Skalenniveaus und befragen
Sie dann eine:n Kommilitonen dazu. \(\square\)

\end{exercise}

\section{Modelle}\label{modelle}

Woran denken Sie beim Wort \enquote{Modell}? Vielleicht an
Spielzeugautos, s. Abbildung~\ref{fig-matchbox}.

\begin{figure}[H]

\centering{

\includegraphics[width=0.25\linewidth,height=\textheight,keepaspectratio]{img/matchbox.jpg}

}

\caption{\label{fig-matchbox}Matchbox-Autos sind Modelle für Autos,
Spurzem (2017)}

\end{figure}%

\begin{definition}[Modelle]\protect\hypertarget{def-modelle}{}\label{def-modelle}

Modelle sind ein vereinfachtes Abbild der Realität, eine
\emph{Repräsentation} (Kaplan, 2009).\(\square\)

\end{definition}

\begin{example}[Beispiele für
Modelle]\protect\hypertarget{exm-Modelle}{}\label{exm-Modelle}

Puppen sind Modelle für Babies, Landkarten für Landstriche und das
Atommodell von Nils Bohrist ein Modell für Atome. \(\square\)

\end{example}

Auch in der Statistik nutzen wir Modelle. Helfen Sie Prof.~Weiss-Ois: Er
blickt nicht durch, s. Abbildung~\ref{fig-prof}. Gerne würde er wissen,
wie viele Stunden seine Studentis auf die Prüfung lernen. Aber mit so
vielen Zahlen kann er nicht umgehen \ldots{} Geben Sie ihm ein Modell:
Sagen Sie ihm, wie lang die Studis typischerweise lernen (sagen Sie ihm
ein einfach den \emph{Mittelwert} der Lernzeiten, 9.6 Stunden).

\begin{figure}[H]

\begin{minipage}{0.46\linewidth}

\includegraphics[width=0.25\linewidth,height=\textheight,keepaspectratio]{img/teacher.png}

\subcaption{\label{}Vorher: 12, 8, 10, 11, 10, 9, 13, 9, 14, 9, 12, 14,
7, 9, 9, 11, 9, 4, 5, 12, 9, 6, 9, 12, 13, 9, 9, 6, 10\ldots{} Oh je, so
viele Zahlen! Ich check nix! Wie viel lernen denn jetzt meine Studis?!
(flaticon, 2024)}
\end{minipage}%
%
\begin{minipage}{0.09\linewidth}
~\end{minipage}%
%
\begin{minipage}{0.46\linewidth}

\includegraphics[width=0.25\linewidth,height=\textheight,keepaspectratio]{img/teacher.png}

\subcaption{\label{}Nachher: Ah, 9.6 Stunden! Yeah, jetzt weiß ich, wie
viel die Studis so typischerweise lernen. Viel zu wenig natürlich!
(flaticon, 2024)}
\end{minipage}%

\caption{\label{fig-prof}Prof.~I. Ch. Weiss-Ois hat den Mittelwert
verstanden.}

\end{figure}%

Der Nutzen von Modellen ist, dass sie komplexe Sachverhalte vereinfachen
und damit oft überhaupt erst dem Verständnis oder einer Untersuchung
zugänglich machen: Modelle ermöglichen Verständnis. In der Datenanalyse
bzw. Statistik (die beiden Begriffe werden hier weitgehend synonym
gebraucht) fassen sie oft viele Daten prägnant zusammen, z.B. zu einer
einzelnen Kennzahl. Das Verrückte an Modellen ist, dass man
Informationen wegwirft, um eine (andere, hoffentlich nützlichere)
Information zu bekommen (Stigler, 2016). Weniger ist mehr?!

\section{Praxisbezug}\label{praxisbezug}

Wir leben im Datenzeitalter; Daten durchdringen alle Bereiche des
beruflichen, gesellschaftlichen und privaten Lebens. Die Datenanalyse
hat sich in den letzten Jahren massiv verändert, da Datenmengen und
Methoden einen regelrechten Boom erlebt haben.

Diese Entwicklung ist durchaus auch kritisch zu betrachten; viele
Menschen betrachten die Entwicklung im Datenzeitalter -- Stichwort
künstliche Intelligenz -- mit Sorge. Egal ob man Daten als Segen oder
Fluch betrachtet, in beiden Fällen ist es wichtig, mit Daten umgehen zu
können. Mit der wachsenden Bedeutung von Daten wächst in gleichem Maße
die Bedeutung von Datenanalyse. Denn Daten ohne Sinn sind nutzlos. Aus
diesem Grund kann man sagen, dass Datenanalyse (und damit auch Statistik
als eine spezielle Art von Datenanalyse) zu stark nachgefragten Jobs
gehören.

\section{Wie man mit Statistik
lügt}\label{wie-man-mit-statistik-luxfcgt}

Das \emph{File-Drawer-Problem}: Sie haben ein tolles Experiment
durchgeführt, viel Arbeit, viel Stress, endlich geschafft, puh. Von den
20 Variablen (als AV, s. Kapitel~\ref{sec-arten-variablen}), die Sie
untersucht haben, zeigt nur 1 einen interessanten Effekt, leider. 1 von
20, das hört sich nicht so toll an. Wäre es da nicht \enquote{elegant},
die 19 Variablen ohne schönen Effekt einfach in der Schublade liegen zu
lassen bis zum Sankt-Nimmerleins-Tag? Dann könnten Sie stattdessen als
Ergebnis nur die eine Variable mit schönen Ergebnis präsentieren, ganz
ohne widersprechende Befunde.

Dieser Versuchung nicht zu erliegen, kann schwer sein. Es ist aber
gefährlich, missliebige Ergebnisse zu verschweigen: Die anderen Menschen
bekommen dann ein falsches Bild der Ergebnislage; man spricht von
\href{https://de.wikipedia.org/wiki/Publikationsbias}{Publikationsbias}
(Marks‐Anglin \& Chen, 2020). Wer Ergebnisse verschweig, verzerrt die
insgesamte Befundlage (Rothstein, 2014) -- klarer Fall von
wissenschaftlichem Fehlverhalten.

\section{Fazit}\label{fazit-2}

Die Aufgabe von Statistik ist es, durch Zusammenfassen von Daten Modelle
zu bilden, die es uns einfacher machen, schwierige Sachverhalte zu
verstehen. Zentral ist dabei, die Analyse von Variabilität der Daten.
Daten kommen in verschiedenen Varianten vor, typischerweise in
Tabellenform, möglichst im Tidy-Format.

\section{Aufgaben}\label{aufgaben}

Die Webseite \href{https://datenwerk.netlify.app}{datenwerk.netlify.app}
stellt eine Reihe von einschlägigen Übungsaufgaben bereit. Sie können
die Suchfunktion der Webseite nutzen, um die Aufgaben mit den folgenden
Namen zu suchen:

\begin{enumerate}
\def\labelenumi{\arabic{enumi}.}
\tightlist
\item
  \href{https://sebastiansauer.github.io/Datenwerk/posts/variation01/variation01.html}{variation01}
\item
  \href{https://sebastiansauer.github.io/Datenwerk/posts/def-statistik01/def-statistik01}{Def-Statistik01}
\item
  \href{https://sebastiansauer.github.io/Datenwerk/posts/tidy1/tidy1.html}{tidy1}
\item
  \href{https://sebastiansauer.github.io/Datenwerk/posts/skalenniveau1a/skalenniveau1a}{Skalenniveau1a}
\item
  \href{https://sebastiansauer.github.io/Datenwerk/posts/ziele-statistik/ziele-statistik}{Ziele-Statistik}
\item
  \href{https://sebastiansauer.github.io/Datenwerk/posts/variation02/variation02.html}{variation02}
\item
  \href{https://sebastiansauer.github.io/Datenwerk/posts/skalenniveau1b/skalenniveau1b}{Skalenniveau1b}
\item
  \href{https://sebastiansauer.github.io/Datenwerk/posts/tidydata1/tidydata1.html}{tidydata1}
\end{enumerate}

\section{Vertiefung}\label{vertiefung}

\subsection{Excel für Könner}\label{excel-fuxfcr-kuxf6nner}

In vielen Organisationen werden Exceltabellen für bestimmte Zwecke der
Datenverarbeitung verwendet. Excel (und ähnliche Programme) hat
bestimmte Stärken und Vorteile, aber auch gewisse Nachteile und
Schwäche; das liegt z.T. daran, dass Excel für bestimmte Aufgaben besser
und für andere weniger gut geeignet ist. Wenn man mit Excel arbeitet,
wiederholen sich erfahrungsgemäß immer wieder die gleichen Fehler bzw.
suboptimalen Vorgehensweise zum Aufbau einer Exceltabelle.

\href{https://www.tandfonline.com/doi/full/10.1080/00031305.2017.1375989}{Dieser
Artikel} von Broman \& Woo (2018) zeigt anhand einiger praktischer
Tipps, wie man Exceltabellen so aufbaut, dass Fehler minimiert werden.

\begin{exercise}[Fassen Sie den Artikel von Broman \& Woo (2018)
zusammen]\protect\hypertarget{exr-xls-paper}{}\label{exr-xls-paper}

Die Lehrkraft teilt Sie dazu in Gruppen ein und weist jeder Gruppe einen
Abschnitt des Artikels zu. Fassen Sie das \emph{Wesentliche} (und nur
das Wesentliche) an einem geeigneten Ort zusammen (z.B. auf einem
Miro-Board). \(\square\)

\end{exercise}

\subsection{Sind wir süchtig nach dem
Handy?}\label{sind-wir-suxfcchtig-nach-dem-handy}

\begin{figure}

\begin{minipage}{0.80\linewidth}
Sind Sie süchtig nach Ihrem Handy? Lassen Sie uns eine kleine Studie
dazu (ggf. live im Hörsaal) durchführen. Füllen Sie
\href{https://forms.gle/PP8yb6Ubqq3JU78F9}{diese Umfrage} zum Thema
Smartphonse-Sucht aus (anonym und kein Muss).\end{minipage}%
%
\begin{minipage}{0.20\linewidth}

\begin{center}
\includegraphics[width=0.75\linewidth,height=\textheight,keepaspectratio]{010-rahmen_files/figure-pdf/qr-google-forms-handysucht-1.pdf}
\end{center}

\end{minipage}%

\end{figure}%

Kernstück der Umfrage ist die Smartphone-Sucht-Skala (Kwon et al.,
2013). Eine Studie fand, dass ca. ein Siebtel der Studierenden süchtig
nach ihrem Smartphone sind (Haug et al., 2015); demnach könnte dem Thema
eine hohe Bedeutsamkeit zukommen. Wir werden die Daten im weiteren
Verlauf auswerten.

\section{Literaturhinweise}\label{literaturhinweise}

Einen Einblick in die Fundamente statistischer Analyse bietet Stigler
(2016). Cetinkaya-Rundel \& Hardin (2021), stellen grundlegende Konzepte
der Analyse von Daten im Kapitel 1, \enquote{Hello data}, vor. Downey
(2023) illustriert statistische Überraschungsmoment auf unterhaltsame,
und vor allem: sofataugliche Art.

\chapter{Daten einlesen}\label{daten-einlesen}

\section{Lernsteuerung}\label{lernsteuerung-1}

Abb. Abbildung~\ref{fig-ueberblick} den Standort dieses Kapitels im
Lernpfad und gibt damit einen Überblick über das Thema dieses Kapitels
im Kontext aller Kapitel.

\subsection{Lernziele}\label{lernziele-2}

\begin{itemize}
\tightlist
\item
  Sie können R und RStudio starten.
\item
  Sie können R-Pakete installieren und starten.
\item
  Sie können Variablen in R zuweisen und auslesen.
\item
  Sie können Daten in R importieren.
\item
  Sie können den Begriff \emph{Reproduzierbarkeit} definieren.
\end{itemize}

\subsection{Ab diesem Kapitel benötigen Sie
R}\label{ab-diesem-kapitel-benuxf6tigen-sie-r}

Bitte stellen Sie sicher, dass Sie R rechtzeitig einsatzbereit haben.
Weiter unten in diesem Kapitel finden Sie Installationshinweise
(Kapitel~\ref{sec-install-r}). Falls Sie dieses Kapitel zum ersten Mal
bzw. sich noch nicht mit R auskennen, werden Sie vielleicht einigen
Inhalten begegnen, die Sie noch nicht gleich verstehen. Keine Sorge, das
ist normal. Mit etwas Übung wird Ihnen bald alles schnell von der Hand
gehen.

\section{Errrstkontakt}\label{errrstkontakt}

\subsection{Warum R?}\label{warum-r}

Gründe, die für den Einsatz von R sprechen:

\begin{enumerate}
\def\labelenumi{\arabic{enumi}.}
\item
  R ist kostenlos, andere Softwarepakete für Datenanalyse sind teuer.
\item
  R und R-Befehle sind quelloffen, d.h. man kann sich die
  zugrundeliegenden Computerbefehle anschauen. Jeder kann prüfen, ob R
  vernünftig arbeitet. Alle können beitragen.
\item
  R hat die neuesten Methoden.
\item
  R hat eine große Community.
\item
  R ist maßgeschneidert für Datenanalyse.
\end{enumerate}

Allerdings gibt es auch abweichende Meinungen, s.
Abbildung~\ref{fig-bill-excel}.

\begin{figure}[H]

\centering{

\includegraphics[width=0.5\linewidth,height=\textheight,keepaspectratio]{img/bill-gates-excel.jpg}

}

\caption{\label{fig-bill-excel}Manche finden Excel cooler als R, nicht
wahr, Bill Gates? (imgflip, 2024a)}

\end{figure}%

\subsection{R und Reproduzierbarkeit}\label{r-und-reproduzierbarkeit}

\begin{definition}[Reproduzierbarkeit]\protect\hypertarget{def-repro}{}\label{def-repro}

Ein (wissenschaftlicher) Befunde ist reproduzierbar, wenn andere
Analystis mit dem gleichen experimentellen Setup zum gleichen Ergebnis
(wie in der ursprünglichen Analyse) kommen (Plesser, 2018). \(\square\)

\end{definition}

Definition~\ref{def-repro} ist, etwas überspitzt, in
Abbildung~\ref{fig-repro} wiedergegeben.

\begin{figure}[H]

\centering{

\includegraphics[width=0.5\linewidth,height=\textheight,keepaspectratio]{img/repro-star-struck.png}

}

\caption{\label{fig-repro}Daten + Syntax + genaue Beschreibung der
Messungen = reproduzierbar}

\end{figure}%

\begin{example}[Aus der Forschung: Reproduzierbarkeit in der
Psychologie]\protect\hypertarget{exm-repro}{}\label{exm-repro}

~

\begin{quote}
{\emoji{student}} Wie ist es um unsere Wissenschaft, Psychologie,
bestellt? Haben die Befunde Hand und Fuß?
\end{quote}

Obels et al. (2020) haben die Reproduzierbarkeit in psychologischen
Studien untersucht. Sie berichten folgendes Ergebnis (S. 229):

\begin{quote}
We examined data and code sharing for Registered Reports published in
the psychological literature from 2014 to 2018 and attempted to
independently computationally reproduce the main results in each
article. Of the 62 articles that met our inclusion criteria, 41 had data
available, and 37 had analysis scripts available. Both data and code for
36 of the articles were shared. We could run the scripts for 31
analyses, and we reproduced the main results for 21 articles.
\(\square\)
\end{quote}

\end{example}

\subsection{R \& RStudio}\label{r-rstudio}

Wenn wir sagen, \enquote{wir arbeiten mit R}, dann heißt das in unserem
Fall \enquote{wir arbeiten mit R und mit RStudio}.

\begin{figure}[H]

\begin{minipage}{0.40\linewidth}

\includegraphics[width=0.7\linewidth,height=\textheight,keepaspectratio]{img/rlogo.png}

\subcaption{\label{}R}
\end{minipage}%
%
\begin{minipage}{0.20\linewidth}

\includegraphics[width=0.4\linewidth,height=\textheight,keepaspectratio]{img/sparkling_heart.png}

\subcaption{\label{}und}
\end{minipage}%
%
\begin{minipage}{0.40\linewidth}

\includegraphics[width=0.4\linewidth,height=\textheight,keepaspectratio]{img/rstudio-logo-small.png}

\subcaption{\label{}RStudio}
\end{minipage}%

\caption{\label{fig-rlove}R und eine GUI wie RSTudio arbeiten gut
zusammen.}

\end{figure}%

Ismay \& Kim (2020) zeigen eine schöne Analogie, was der Unterschied von
\emph{R} und \emph{RStudio} ist, s. Abbildung~\ref{fig-r-rstudio}.
(Streng genommen ist RStudio für die Datenanalyse irrelevant, aber
RStudio ist praktisch, Sie werden es nicht missen wollen.)

\begin{figure}[H]

\centering{

\pandocbounded{\includegraphics[keepaspectratio]{img/r_vs_rstudio_1.png}}

}

\caption{\label{fig-r-rstudio}R vs.~RStudio: R macht die Arbeit, RStudio
ist für Komfort und Übersicht (Ismay \& Kim, 2020).}

\end{figure}%

Kurz gesagt: Das eigentlich Arbeiten besorgt R. Für den Komfort und die
Schönheit ist RStudio zuständig. Auch eine Art von Arbeitsteilung! Hier
sehen Sie einen Screenshot von der Oberfläche von RStudio, s.
\textbf{?@fig-rstudio}.

\section{Installation von R und RStudio}\label{sec-install-r}

\subsection{Installation von R}\label{installation-von-r}

R ist ein Softwarepaket für statistische Berechnungen. Laden Sie es für
Ihr Betriebssytem herunter unter \url{https://cloud.r-project.org}.

Wenn Sie beim Herunterladen gefragt werden, dass Sie einen
\enquote{Mirror} auswählen sollen, heißt das, Sie sollen einen Computer
(Server) wählen, von dem Sie R herunterladen. Der sollte möglichst nicht
zu weit weg stehen, dann spart es vielleicht etwas Zeit und Bandbreite.

Wenn Sie die Installationsdatei heruntergeladen haben, öffnen Sie diese
Datei (Doppelklick) und Sie werden durch die Installation geführt. (Sie
benötigen Admin-Rechte auf Ihrem Computer.)

\subsection{Installation von RStudio
Desktop}\label{installation-von-rstudio-desktop}

RStudio ist eine \emph{graphische Benutzeroberfläche} (graphical user
interface, GUI) für R, plus ein paar Goodies (in Form einer
\emph{intergrierten Entwicklungsumgebung} (integrated development
environment, IDE).

Laden Sie die \emph{Desktop-Version} von RStudio herunter für Ihr
Betriebssystem (Windows, MacOS, Linux) vom Anbieter (Posit) herunter.
\footnote{\url{https://posit.co/download/rstudio-desktop/}}

Wenn Sie die Installationsdatei heruntergeladen haben, öffnen Sie diese
Datei (Doppelklick) und Sie werden durch die Installation geführt. (Sie
benötigen u.U. Admin-Rechte auf Ihrem Computer.)

\subsection{Posit/RStudio Cloud}\label{positrstudio-cloud}

\subsubsection{RStudio Cloud als Alternative zu
RStudio}\label{rstudio-cloud-als-alternative-zu-rstudio}

Posit Cloud bzw. RStudio Cloud (\url{https://rstudio.cloud/}) ist ein
Webdienst von Posit (zum Teil kostenlos), also ein \emph{RStudio
online}: Man kann damit online mit R arbeiten. Die Oberfläche ist
praktisch identisch zur Desktop-Version, s.
Abbildung~\ref{fig-rstudio-cloud}. Sie können es als Alternative zur
Installation von RStudio auf Ihrem Computer verwenden. Ein Vorteil von
RStudio Cloud ist, dass man als Nutzer \emph{nichts installieren} muss
und dass es \emph{auch auf Tablets} läuft (im Gegensatz zur
Desktop-Version von RStudio). Ein Nachteil ist, dass es etwas langsamer
ist und nur für ein gewisses Zeitvolumen kostenlos. Sie müssen sich ein
Konto anlegen, um den Dienst nutzen zu können.

\begin{figure}[H]

\centering{

\includegraphics[width=0.75\linewidth,height=\textheight,keepaspectratio]{img/rstudio-cloud.png}

}

\caption{\label{fig-rstudio-cloud}So sieht RStudio Cloud aus. Fast genau
wie RStudio Desktop}

\end{figure}%

\subsubsection{Vertiefung}\label{vertiefung-1}

Wenn Ihre Lehrkraft Ihnen einen Projektordner bzw. einen Link dazu
bereitstellt, ist das komfortabel, da die Lehrkraft dann schon Pakete
installieren, Daten bereitstellen und andere Nettigkeit vorbereiten kann
für Sie. Allerdings müssen Sie den Projektordner in Ihrem Konto
abspeichern, wenn Sie etwas speichern möchten, da Sie vermutlich keine
Schreibrechte im Projektordner Ihrer Lehrkraft haben. Klicken Sie dazu
auf \enquote{Save a permanent copy}, s. Abbildung~\ref{fig-perm-copy}.

\begin{figure}[H]

\centering{

\pandocbounded{\includegraphics[keepaspectratio]{img/rstudio-save-a-permanent-copy.png}}

}

\caption{\label{fig-perm-copy}Einen Projektordner im eigenen Konto
abspeichern, um Schreibrechte zu haben}

\end{figure}%

Sie können auch von der Cloud exportieren, also Ihre Syntaxdatei
herunterladen. Klicken Sie dazu im Reiter \enquote{Files} auf
\texttt{More\ \textgreater{}\ Export}.

\section{RStudio starten, nicht R}\label{rstudio-starten-nicht-r}

Wir verwenden beide Programme (R und RStudio). Aber wir \emph{öffnen
nur} RStudio. RStudio findet selbständig R und öffnet dieses
\enquote{heimlich}. Öffnen Sie nicht noch extra R (sonst wäre R zweifach
geöffnet). Anstelle von \emph{RStudio Desktop} (auf Ihrem
Computer/Desktop) können Sie auch die \emph{RStudio Cloud} (die
Online-Version) starten.

\section{R-Pakete}\label{r-pakete}

\subsection{Was sind R-Pakete?}\label{was-sind-r-pakete}

Typisch für R ist sein modularer Aufbau: Man kann eine große Zahl an
Erweiterungen (\enquote{Pakete}, engl. \emph{packages}) installieren,
alle kostenlos. In R Paketen \enquote{wohnen} R-Befehle, also Dinge, die
R kann, \enquote{Skills} sozusagen. Außerdem können in R-Paketen auch
Daten bereitgestellt werden. Damit man die Inhalte eines R-Pakets nutzen
kann, muss man es zuerst installieren und dann starten. Man kann sich
daher ein R-Paket vorstellen wie ein Buch: Wenn R es gelesen hat, dann
kennt es die Inhalte. Diese Inhalte könnten irgendwelche Formeln, also
Berechnungen sein. Es könnte aber die \enquote{Bauanleitung} für ein
schönes Diagramm sein. Ist ein spezielles R-Paket auf Ihrem Computer
installiert, so können Sie diese Funktionalität nutzen.

\emph{Erweiterungen} kennt man von vielen Programmen, sie werden auch
\emph{Add-Ons}, \emph{Plug-Ins} oder sonstwie genannt. Man siehe zur
Verdeutlichung Erweiterungen beim Broswer Chrome,
Abbildung~\ref{fig-chrome}.

\begin{figure}[H]

\centering{

\includegraphics[width=0.5\linewidth,height=\textheight,keepaspectratio]{img/chrome-extensions.png}

}

\caption{\label{fig-chrome}Erweiterungen beim Browser Chrome}

\end{figure}%

Die Anzahl der R-Pakete ist groß; allein auf dem \enquote{offiziellen
Web-Store} (nennt sich \enquote{CRAN}) von R gibt es ca. 20,000 Pakete
(Hornik et al., 2023). Und es kommen immer mehr dazu.

\subsection{Pakete installieren}\label{sec-install-r-pckgs}

Wie jede Software muss man Pakete (Erweiterungen für R) erst einmal
installieren, bevor man sie verwenden kann. Ja, einmal installieren
reicht.

Das geht komfortabel, wenn man beim Reiter \emph{Packages} auf
\emph{Install} klickt und dann den Namen des zu installierenden Pakets
eingibt, s. Abbildung~\ref{fig-so-installieren}.

\begin{figure}[H]

\centering{

\pandocbounded{\includegraphics[keepaspectratio]{img/install-packages3.png}}

}

\caption{\label{fig-so-installieren}Geben Sie den Namen des zu
installierenden R-Pakets in dieser Maske ein}

\end{figure}%

\begin{quote}
{\emoji{student}} Welche R-Pakete sind denn schon installiert?
\end{quote}

Im Reiter \emph{Packages} können Sie nachschauen, welche Pakete auf
Ihrem Computer schon installiert sind. Diese Pakete brauchen Sie
logischerweise dann \emph{nicht} noch mal installieren, s.
Abbildung~\ref{fig-paket-installiert}; es sei denn, Sie wollen das Paket
updaten.

\begin{figure}[H]

\centering{

\pandocbounded{\includegraphics[keepaspectratio]{img/paket-installiert.png}}

}

\caption{\label{fig-paket-installiert}So sehen Sie, ob ein R-Paket auf
Ihrem System installiert ist}

\end{figure}%

Alternativ können Sie zum Installieren von Paketen auch den Befehl
\texttt{install.packages()} verwenden. Also zum Beispiel
\texttt{install.packages(tidyverse)} um das Paket \texttt{tidyverse} zu
installieren.

\begin{quote}
{\emoji{student}} Ja, aber welche R-Pakete \enquote{soll} ich denn
installieren, welche brauch ich denn?
\end{quote}

Im Moment sollten Sie die folgenden Pakete installiert haben:

\begin{itemize}
\tightlist
\item
  \texttt{tidyverse}
\item
  \texttt{easystats}
\end{itemize}

Wenn Sie die noch nicht installiert haben sollten, dann können Sie das
jetzt ja nachholen. (Übrigens sind \texttt{tidyverse} und
\texttt{easystats} Pakete, die nur dafür da sind, mehrere Pakete zu
installieren. So gehören z.B. zu \texttt{tidyverse} die Pakete
\texttt{ggplot} (Daten verbildlichen) und \texttt{dplyr} (Datenjudo).
Damit wir nicht alle Pakete einzeln installieren und starten müssen,
bietet uns das Paket \texttt{tidyverse} den Komfort, alle die Pakete
dieser \enquote{Sammlung} auf einmal zu starten. Praktisch.)

\begin{tcolorbox}[enhanced jigsaw, bottomtitle=1mm, leftrule=.75mm, breakable, title=\textcolor{quarto-callout-caution-color}{\faFire}\hspace{0.5em}{Vorsicht}, bottomrule=.15mm, titlerule=0mm, left=2mm, opacityback=0, colframe=quarto-callout-caution-color-frame, rightrule=.15mm, colback=white, coltitle=black, toprule=.15mm, toptitle=1mm, colbacktitle=quarto-callout-caution-color!10!white, arc=.35mm, opacitybacktitle=0.6]

Bevor Sie ein R-Paket (oder überhaupt irgendwelche Software)
installieren/updaten, sollten Sie das R-Paket schließen/beenden. Sonst
schrauben Sie sozusagen an einem elektrischen Gerät herum, das noch
unter Strom steht (nicht gut). Die einfachste Art, alle Pakete zu
beenden ist, \texttt{Session\ \textgreater{}\ Restart\ R} zu klicken (in
RStudio).\(\square\)

\end{tcolorbox}

\subsection{Pakete starten}\label{pakete-starten}

Wenn Sie ein Softwareprogramm installiert haben, müssen Sie es noch
\emph{starten} bzw. R bereitstellen. Sie erkennen leicht, ob ein Paket
bereit ist, wenn Sie ein Häkchen vor dem Namen des Pakets in der
Paketliste (Reiter \emph{Packages}) sehen.

\begin{tcolorbox}[enhanced jigsaw, bottomtitle=1mm, leftrule=.75mm, breakable, title=\textcolor{quarto-callout-note-color}{\faInfo}\hspace{0.5em}{Hinweis}, bottomrule=.15mm, titlerule=0mm, left=2mm, opacityback=0, colframe=quarto-callout-note-color-frame, rightrule=.15mm, colback=white, coltitle=black, toprule=.15mm, toptitle=1mm, colbacktitle=quarto-callout-note-color!10!white, arc=.35mm, opacitybacktitle=0.6]

Ein bestimmtes R-Paket muss man nur \emph{einmalig installieren}. Aber
man muss es \emph{jedes Mal neu starten}, wenn man R (bzw. RStudio)
startet. \(\square\)

\end{tcolorbox}

\section{Mit R arbeiten}\label{mit-r-arbeiten}

\subsection{Projekte in R}\label{projekte-in-r}

Ein \emph{Projekt} in RStudio (s. Abbildung~\ref{fig-rstudio-projekte})
ist letztlich ein Ordner, der als \enquote{Basis} für eine Reihe von
Dateien verwendet wird. Sagen wir, Sie nennen Ihr Projekt
\texttt{cool\_stuff}. RStudio legt uns diesen Ordner an einem von uns
gewählten Platz auf unserem Computer an. Das ist ganz praktisch, weil
man dann sagen kann \enquote{Hey R, nimmt die Datei
\enquote*{daten.csv}}, ohne einen Pfad anzugeben. Vorausgesetzt, die
Datei liegt auch im Projektordner (\texttt{cool\_stuff}).

Projekte kann anlegen mit Klick auf das Icon, das einen Quader mit dem
Buchstaben R darin anzeigt (s. Abbildung~\ref{fig-rstudio-projekte}).
Nutzen Sie RStudio-Projekte, das macht Ihr Leben leichter.
RStudio-Projekte zu nutzen ist viel sicherer als das Arbeitsverzeichnis
von Hand zu wählen oder mit Pfaden herumzubasteln.

\begin{figure}[H]

\centering{

\includegraphics[width=0.5\linewidth,height=\textheight,keepaspectratio]{img/rstudio-projekte.png}

}

\caption{\label{fig-rstudio-projekte}RStudio-Projekte, Beispiele}

\end{figure}%

\subsection{Skriptdateien}\label{skriptdateien}

Die R-Befehle (\enquote{Syntax}) schreiben Sie am besten in eine
speziell dafür vorgesehene Textdatei in RStudio. Eine Sammlung von
(R-)Computer-Befehlen nennt man auch ein \emph{Skript}, daher spricht
man auch von einer \emph{Skriptdatei}.

\subsubsection{So erstellen Sie eine neue
Skriptdatei}\label{so-erstellen-sie-eine-neue-skriptdatei}

Um eine neue R-Skriptdatei zu erstellen, klicken Sie auf das Icon, das
ein weißes Blatt mit einem grünen Pluszeichen zeigt, s.
Abbildung~\ref{fig-script-new}.

\begin{figure}[H]

\begin{minipage}{0.48\linewidth}

\centering{

\includegraphics[width=0.5\linewidth,height=\textheight,keepaspectratio]{img/script-new.png}

}

\subcaption{\label{fig-script-new1}So erstellen Sie eine neue
Skriptdatei}

\end{minipage}%
%
\begin{minipage}{0.05\linewidth}
~\end{minipage}%
%
\begin{minipage}{0.48\linewidth}

\centering{

\pandocbounded{\includegraphics[keepaspectratio]{img/neue-skriptdatei.png}}

}

\subcaption{\label{fig-script-new2}Klicken Sie auf das Icon mit dem
leeren Blatt und dem grünen Plus}

\end{minipage}%

\caption{\label{fig-script-new}Es gibt verschiedene Wege, um eine neue
R-Skript-Datei in RStudio zu öffnen.}

\end{figure}%

\subsubsection{So speichern Sie Ihre
Skripdatei}\label{so-speichern-sie-ihre-skripdatei}

Vergessen Sie nicht zu \emph{speichern}, wenn Sie ein tolles Skript
geschrieben haben. Dafür gibt es mehrere Möglichkeiten:

\begin{enumerate}
\def\labelenumi{\arabic{enumi}.}
\tightlist
\item
  Tastaturkürzel \emph{Strg+S}
\item
  Menü: \texttt{File\ \textgreater{}\ Save}
\item
  Klick auf das Icon mit der Diskette, s.
  Abbildung~\ref{fig-script-new}.
\end{enumerate}

\subsubsection{So öffnen Sie eine
Skriptdatei}\label{so-uxf6ffnen-sie-eine-skriptdatei}

Eine existierende Skriptdatei können Sie in typischer Manier
\emph{öffnen}:

\begin{enumerate}
\def\labelenumi{\arabic{enumi}.}
\tightlist
\item
  Strg+O
\item
  Klick auf das Icon mit der Akte und dem grünen Pfeil (vgl.
  Abbildung~\ref{fig-script-new})
\item
  Menü: \texttt{File\ \textgreater{}\ Open\ File...}
\end{enumerate}

\subsection{Quarto-Dokumente}\label{quarto-dokumente}

\href{https://quarto.org/}{Quarto}\footnote{\url{https://quarto.org/}}
ist ein Programm zum Erstellen von Texten, in die man R-Syntax einfügen
kann. Die Ausgaben der R-Befehle werden dann direkt im Dokument
eingebunden. Abbildung~\ref{fig-exm-quarto} zeit ein Beispiel für ein
Quarto-Dokument.

\begin{tcolorbox}[enhanced jigsaw, bottomtitle=1mm, leftrule=.75mm, breakable, title=\textcolor{quarto-callout-note-color}{\faInfo}\hspace{0.5em}{Hinweis}, bottomrule=.15mm, titlerule=0mm, left=2mm, opacityback=0, colframe=quarto-callout-note-color-frame, rightrule=.15mm, colback=white, coltitle=black, toprule=.15mm, toptitle=1mm, colbacktitle=quarto-callout-note-color!10!white, arc=.35mm, opacitybacktitle=0.6]

Quarto ist eine komfortable und leistungsfähige Methode, um Dokumente
mit R-Syntax zu schreiben. Sie sind aber nicht verpflichtet, Quarto zu
nutzen. Stattdessen können Sie Ihre Syntax auch in Skriptdateien
schreiben. \(\square\)

\end{tcolorbox}

\begin{figure}[H]

\centering{

\pandocbounded{\includegraphics[keepaspectratio]{index_files/mediabag/rstudio-hello.png}}

}

\caption{\label{fig-exm-quarto}Dokumente schreiben mit Quarto}

\end{figure}%

Wenn Sie Quarto nutzen möchten, müssen Sie es zunächst installieren,
d.h. herunterladen. Dann können Sie in RStudio Quarto-Dateien erstellen.
Ein neues Quarto-Dokument können Sie erstellen mit Klick auf \emph{File
\textgreater{} New File \textgreater{} Quarto Document}.

\section{Errisch für Einsteiger}\label{errisch-fuxfcr-einsteiger}

\begin{tcolorbox}[enhanced jigsaw, bottomtitle=1mm, leftrule=.75mm, breakable, title=\textcolor{quarto-callout-note-color}{\faInfo}\hspace{0.5em}{Hinweis}, bottomrule=.15mm, titlerule=0mm, left=2mm, opacityback=0, colframe=quarto-callout-note-color-frame, rightrule=.15mm, colback=white, coltitle=black, toprule=.15mm, toptitle=1mm, colbacktitle=quarto-callout-note-color!10!white, arc=.35mm, opacitybacktitle=0.6]

Sie finden den R-Code für jedes Kapitel
\href{https://github.com/sebastiansauer/statistik1/tree/main/R-code-for-all-chapters}{hier}:
\url{https://github.com/sebastiansauer/statistik1/tree/main/R-code-for-all-chapters}.
\(\square\)

\end{tcolorbox}

\subsection{Variablen}\label{sec-rvars}

In jeder Programmiersprache kann man Variablen definieren, so auch in R:

\begin{Shaded}
\begin{Highlighting}[]
\NormalTok{richtige\_antwort }\OtherTok{=} \DecValTok{42}
\NormalTok{falsche\_antwort }\OtherTok{=} \DecValTok{43}
\NormalTok{typ }\OtherTok{=} \StringTok{"Antwort"}
\NormalTok{ist\_korrekt }\OtherTok{=} \ConstantTok{TRUE}
\end{Highlighting}
\end{Shaded}

Alternativ zum Gleichheitszeichen \texttt{=} können Sie auch (synonym)
den Zuweisungspfeil \texttt{\textless{}-} verwenden. Beides führt zum
gleichen Ergebnis. Allerdings ist der Zuweisungspfeil präziser, und
sollte daher \emph{bevorzugt} werden.

Der \emph{Zuweisungspfeil} \texttt{\textless{}-} bzw. das
Gleichheitszeichen \texttt{=} definiert eine neue \emph{Variable} (oder
überschreibt den Inhalt, wenn die Variable schon existiert).

\begin{Shaded}
\begin{Highlighting}[]
\NormalTok{richtige\_antwort }\OtherTok{\textless{}{-}} \DecValTok{42}
\NormalTok{falsche\_antwort }\OtherTok{\textless{}{-}} \DecValTok{43}
\NormalTok{typ }\OtherTok{\textless{}{-}} \StringTok{"Antwort"}
\NormalTok{ist\_korrekt }\OtherTok{\textless{}{-}} \ConstantTok{TRUE}
\end{Highlighting}
\end{Shaded}

Sie können sich eine Variable wie einen Becher oder Behälter vorstellen,
der bestimmte Werte enthält. Auf dem Becher steht (mit Edding
geschrieben) der Name des Bechers. Natürlich können Sie die Werte aus
dem Becher entfernen und sie durch neue ersetzen (vgl.
Abbildung~\ref{fig-def-vars}).

\begin{figure}[H]

\centering{

\includegraphics[width=0.25\linewidth,height=\textheight,keepaspectratio]{img/Variablen_zuweisen.png}

}

\caption{\label{fig-def-vars}Variablen zuweisen}

\end{figure}%

R kann übrigens auch rechnen. Probieren Sie es doch gleich mal hier aus!

\begin{Shaded}
\begin{Highlighting}[]
\NormalTok{die\_summe }\OtherTok{\textless{}{-}}\NormalTok{ falsche\_antwort }\SpecialCharTok{+}\NormalTok{ richtige\_antwort}
\end{Highlighting}
\end{Shaded}

Aber was ist jetzt der Wert, der \enquote{Inhalt} der Variable
\texttt{die\_summe}?

Um den Wert, d.h. den Inhalt einer Variablen in R \emph{auszulesen},
geben wir einfach den Namen des Objekts ein:

\begin{Shaded}
\begin{Highlighting}[]
\NormalTok{die\_summe}
\DocumentationTok{\#\# [1] 85}
\end{Highlighting}
\end{Shaded}

Was passiert wohl, wenn wir \texttt{die\_summe} jetzt wie folgt
definieren?

\begin{Shaded}
\begin{Highlighting}[]
\NormalTok{die\_summe }\OtherTok{\textless{}{-}}\NormalTok{ falsche\_antwort }\SpecialCharTok{+}\NormalTok{ richtige\_antwort }\SpecialCharTok{+} \DecValTok{1}
\end{Highlighting}
\end{Shaded}

Wer hätt's geahnt:

\begin{Shaded}
\begin{Highlighting}[]
\NormalTok{die\_summe}
\DocumentationTok{\#\# [1] 86}
\end{Highlighting}
\end{Shaded}

Variablen können auch \enquote{leer} sein:

\begin{Shaded}
\begin{Highlighting}[]
\NormalTok{alter }\OtherTok{\textless{}{-}} \ConstantTok{NA}
\NormalTok{alter}
\DocumentationTok{\#\# [1] NA}
\end{Highlighting}
\end{Shaded}

\texttt{NA} steht für \emph{not available}, nicht verfügbar und macht
deutlich, dass hier ein Wert fehlt.

\begin{quote}
{\emoji{student}} Wozu brauche ich bitte fehlende Werte?!
\end{quote}

Fehlende Werte sind ein häufiges Problem in der Praxis. Vielleicht hat
sich die befragte Person geweigert, ihr Alter anzugeben (Datenschutz!).
Oder als Sie die Daten in Ihren Computer eingeben wollten, ist Ihre
Katze über die Tastatur gelaufen und alles war futsch\ldots{}

\subsection{\texorpdfstring{Funktionen
(\enquote{Befehle})}{Funktionen (``Befehle'')}}\label{funktionen-befehle}

Das, was R kann, ist in \enquote{Funktionen} hinterlegt. Genauer gesagt
ist ein \enquote{Befehl} eine Funktion.

\begin{definition}[Funktion]\protect\hypertarget{def-fun}{}\label{def-fun}

Eine Funktion ist eine Regel, die jedem Eingabewert (auch Argument
genannt) einen Ausgabewert zuordnet. Man kann sich Funktionen als
Maschinen vorstellen, die Eingabedaten in Ausgabedaten umwandeln, vgl.
Abbildung~\ref{fig-function-schema}. \(\square\)

\end{definition}

\subsubsection{Eine erste Funktion: Vektoren
erstellen}\label{eine-erste-funktion-vektoren-erstellen}

Ein Beispiel für eine solche Funktion könnte sein: \enquote{Berechne den
Mittelwert dieser Datenreihe} (schauen wir uns gleich an).

Das geht so:

\begin{Shaded}
\begin{Highlighting}[]
\NormalTok{Antworten }\OtherTok{\textless{}{-}} \FunctionTok{c}\NormalTok{(}\DecValTok{42}\NormalTok{, }\DecValTok{43}\NormalTok{)}
\end{Highlighting}
\end{Shaded}

Der Befehl \texttt{c} (c wie \emph{c}ombine) fügt mehrere Werte zusammen
zu einer \enquote{Liste} (einem Vektor). (Streng genommen sollte man
nicht von einer Liste sprechen, da es in R noch einen anderen Objekttyp
gibt, der \texttt{list} heißt, und eine verallgemeinerte Form eines
Vektors ist.)

\begin{definition}[Vektor]\protect\hypertarget{def-vektor}{}\label{def-vektor}

Als \emph{Vektor} (Datenreihe) bezeichnen wir eine geordnete Folge von
Werten. In R kann man sie mit der Funktion \texttt{c()} erstellen. Die
Werte eines Vektors bezeichnet man als \emph{Elemente}. \(\square\)

\end{definition}

Mit dem Zuweisungspfeil geben wir diesem Vektor einen Namen, hier
\texttt{Antworten}. Dieser Vektor besteht aus zwei Werten, zuerst
\texttt{42}, dann kommt \texttt{43}.

\begin{example}[Beispiele für
Vektoren]\protect\hypertarget{exm-vektoren}{}\label{exm-vektoren}

Vektoren können (praktisch) beliebig lang sein, z.B. drei Elemente.

\begin{Shaded}
\begin{Highlighting}[]
\NormalTok{x }\OtherTok{\textless{}{-}} \FunctionTok{c}\NormalTok{(}\DecValTok{1}\NormalTok{, }\DecValTok{2}\NormalTok{, }\DecValTok{3}\NormalTok{)}
\NormalTok{y }\OtherTok{\textless{}{-}} \FunctionTok{c}\NormalTok{(}\DecValTok{2}\NormalTok{, }\DecValTok{1}\NormalTok{, }\DecValTok{3}\NormalTok{)  }\CommentTok{\# x und y sind ungleich (Reihenfolge der Werte)}
\NormalTok{z }\OtherTok{\textless{}{-}} \FunctionTok{c}\NormalTok{(}\FloatTok{3.14}\NormalTok{, }\FloatTok{2.71}\NormalTok{)  }
\NormalTok{namen }\OtherTok{\textless{}{-}} \FunctionTok{c}\NormalTok{(}\StringTok{"Anni"}\NormalTok{, }\StringTok{"Bert"}\NormalTok{, }\StringTok{"Charli"}\NormalTok{) }\CommentTok{\# Text{-}Vektor}
\end{Highlighting}
\end{Shaded}

\end{example}

Zwei wichtige Typen von Vektoren sind numerische Vektoren (reelle
Zahlen; in R auch als \emph{numeric} oder \emph{double} bezeichnet) und
Textvektoren, in R auch als \emph{String} oder \emph{character}
bezeichnet.

\begin{example}[]\protect\hypertarget{exm-funs}{}\label{exm-funs}

Weitere Beispiel für Funktionen sind:

\begin{itemize}
\tightlist
\item
  \enquote{Erstelle eine Liste (Vektor) von Werten}.
\item
  \enquote{Lade dieses R-Paket.}
\item
  \enquote{Gib den größten Wert dieser Datenreihe aus.} \(\square\)
\end{itemize}

\end{example}

\subsection{Unsere erste statistische Funktion}\label{sec-first-fun}

Jetzt wird's ernst. Jetzt kommt die Statistik. Berechnen wir also unsere
erste statistische Funktion: Den Mittelwert. Puh.

\begin{Shaded}
\begin{Highlighting}[]
\FunctionTok{mean}\NormalTok{(Antworten)}
\DocumentationTok{\#\# [1] 42}
\end{Highlighting}
\end{Shaded}

Sie hätten \texttt{Antworten} auch durch \texttt{c(42,\ 43)} ersetzen
können, so haben Sie ja schließlich die Variable gerade definiert.

R arbeitet so einen \enquote{verschachtelten} Befehl \emph{von innen
nach außen} ab:

Start: \texttt{mean(Antworten)}

{\(\downarrow\)}

Schritt 1: \texttt{mean(c(42,\ 43))}

{\(\downarrow\)}

Schritt 2: \texttt{42.5}

Abbildung~\ref{fig-function-schema} stellt eine Funktion schematisch
dar.

\begin{figure}[H]

\centering{

\includegraphics[width=0.5\linewidth,height=\textheight,keepaspectratio]{img/function-schema.pdf}

}

\caption{\label{fig-function-schema}Schema einer Funktion}

\end{figure}%

Eine Funktion hat einen oder mehrere \emph{Inputs} (s.
Abbildung~\ref{fig-function-schema}), das sind Daten oder
Verarbeitungshinweise, die man in die Funktion \texttt{fun}
\emph{eingibt}, bevor sie loslegt. Eine Funktion hat immer (genau) eine
\emph{Ausgabe} (Output), in der das Ergebnis einer Funktion ausgegeben
wird.

\begin{definition}[Argumente einer
Funktion]\protect\hypertarget{def-args}{}\label{def-args}

Die \enquote{Trichter} einer (R-)Funktion, in denen man die Eingaben
\enquote{einfüllt}, nennt man auch \emph{Argumente}.\(\square\)

\end{definition}

So hat die Funktion \texttt{mean()} z.B. folgende Argumente, s.
Listing~\ref{lst-mean}.

\begin{codelisting}

\caption{\label{lst-mean}Die Argumente der R-Funktion \texttt{mean}}

\centering{

\begin{Shaded}
\begin{Highlighting}[]
\FunctionTok{mean}\NormalTok{(x, }\AttributeTok{trim =} \DecValTok{0}\NormalTok{, }\AttributeTok{na.rm =} \ConstantTok{FALSE}\NormalTok{, ...)}
\end{Highlighting}
\end{Shaded}

}

\end{codelisting}%

\begin{itemize}
\tightlist
\item
  \texttt{x}: das ist der Vektor, für den der Mittelwert berechnet
  werden soll
\item
  \texttt{trim\ =\ 0}: Sollen die extremsten Werte von \texttt{x} lieber
  \enquote{abgeschnitten} werden, also nicht in die Berechnung des
  Mittelwerts einfließen?
\item
  \texttt{na.rm\ =\ FALSE}: Wie soll mit fehlenden Werten \texttt{NA}
  umgegangen werden? Im Standard liefert \texttt{mean} (und viele andere
  arithmetische Funktionen in R) \texttt{NA} zurück. R schwenkt
  sozusagen die rote Fahne, um zu signalisieren: Achtung, Mensch, hier
  ist irgendwas nicht in Ordnung. Setzt man aber
  \texttt{na.rm\ =\ TRUE}, dann entfernt (remove, rm) R die fehlenden
  Werte und berechnet den Mittelwert, ohne weitere Hinweise zu den
  fehlenden Werten.
\item
  \texttt{...} heißt \enquote{sonstiges Zeugs, das manchmal eine Rolle
  spielen könnte}; darum kümmern wir uns jetzt nicht.
\end{itemize}

Einige Argumente haben einen \emph{Standardwert} bzw. eine
\emph{Voreinstellung} (engl. \emph{default}). So wird bei der Funktion
\texttt{mean()} im Standard nicht getrimmt (\texttt{trim\ =\ 0}) und
fehlende Werte werden nicht entfernt (\texttt{na.rm\ =\ FALSE)}.

\begin{tcolorbox}[enhanced jigsaw, bottomtitle=1mm, leftrule=.75mm, breakable, title=\textcolor{quarto-callout-note-color}{\faInfo}\hspace{0.5em}{Hinweis}, bottomrule=.15mm, titlerule=0mm, left=2mm, opacityback=0, colframe=quarto-callout-note-color-frame, rightrule=.15mm, colback=white, coltitle=black, toprule=.15mm, toptitle=1mm, colbacktitle=quarto-callout-note-color!10!white, arc=.35mm, opacitybacktitle=0.6]

Wenn ein R-Befehl ein Argument mit Voreinstellung hat, brauchen Sie
dieses Argument \emph{nicht} zu befüllen. In dem Fall wird auf den Wert
der Voreinstellung zurückgegriffen. Argumente ohne Voreinstellung -- wie
\texttt{x} bei \texttt{mean()} -- müssen Sie aber auf jeden Fall mit
einem Wert befüllen. Man würde also \texttt{mean} zumeist so aufrufen:
\texttt{mean(x)}. \(\square\)

\end{tcolorbox}

Bei jedem R-Befehl haben die Argumente eine bestimmte Reihenfolge, etwa
bei \texttt{mean()}:
\texttt{mean(x,\ trim\ =\ 0,\ na.rm\ =\ FALSE,\ ...)}.

(Nur) wenn man die Argumente in ihrer vorgegebenen Reihenfolge
anspricht, muss man \emph{nicht} den Namen des Arguments anführen:

\emoji{check-mark-button} \texttt{mean(Antworten,\ 0,\ FALSE)}

Hält man sich aber nicht an die vorgebene Reihenfolge, so weiß R nicht,
was zu tun ist und flüchtet sich in eine Fehlermeldung:

\begin{Shaded}
\begin{Highlighting}[]
\FunctionTok{mean}\NormalTok{(Antworten, }\ConstantTok{FALSE}\NormalTok{, }\DecValTok{0}\NormalTok{)  }\CommentTok{\# FALSCH, DON\textquotesingle{}T DO IT }
\DocumentationTok{\#\# Error in mean.default(Antworten, FALSE, 0): \textquotesingle{}trim\textquotesingle{} must be numeric of length one}
\end{Highlighting}
\end{Shaded}

Wenn man die Namen der Argumente anspricht, ist die Reihenfolge egal:

\begin{Shaded}
\begin{Highlighting}[]
\FunctionTok{mean}\NormalTok{(}\AttributeTok{na.rm =} \ConstantTok{FALSE}\NormalTok{, }\AttributeTok{x =}\NormalTok{ Antworten)  }\CommentTok{\# ok}
\FunctionTok{mean}\NormalTok{(}\AttributeTok{trim =} \DecValTok{0}\NormalTok{, }\AttributeTok{x =}\NormalTok{ Antworten, }\AttributeTok{na.rm =} \ConstantTok{TRUE}\NormalTok{)  }\CommentTok{\# ok}
\end{Highlighting}
\end{Shaded}

Übrigens: Leerzeichen sind R fast immer egal. Aus Gründen der
Übersichtlichkeit sollte man aber Leerzeichen verwenden. In diesen
Fällen sind Leerzeichen nicht erlaubt:

\begin{itemize}
\tightlist
\item
  \texttt{\textless{}-}
\item
  \texttt{\textless{}=} etc.
\item
  Variablennamen
\end{itemize}

\subsubsection{Achtung bei fehlenden
Werten}\label{achtung-bei-fehlenden-werten}

Sagen wir, wir haben einen fehlenden Wert in unseren Daten:

\begin{Shaded}
\begin{Highlighting}[]
\NormalTok{Antworten }\OtherTok{\textless{}{-}} \FunctionTok{c}\NormalTok{(}\DecValTok{42}\NormalTok{, }\DecValTok{43}\NormalTok{, }\ConstantTok{NA}\NormalTok{)}
\NormalTok{Antworten}
\DocumentationTok{\#\# [1] 42 43 NA}
\end{Highlighting}
\end{Shaded}

Wenn wir jetzt den Mittelwert berechnen wollen, quittiert R das mit
einem schnöden \texttt{NA}. \texttt{NA} steht für \emph{not available},
ist also ein Hinweis, dass Werte fehlen.

\begin{Shaded}
\begin{Highlighting}[]
\FunctionTok{mean}\NormalTok{(Antworten)}
\DocumentationTok{\#\# [1] NA}
\end{Highlighting}
\end{Shaded}

R meint es gut mit Ihnen\footnote{{\emoji{robot}} Naja, manchmal.}.
Stellen Sie sich vor, dass R Sie auf dieses Problem aufmerksam machen
möchte:

\begin{quote}
{\emoji{robot}} Achtung, NAs, fehlende Werte, lieber Herr und Gebieter,
du hast nicht mehr alle Latten am Zaun, will sagen, alle Daten im
Vektor!
\end{quote}

(Danke, R.)

Möchten Sie aber lieber R dieses Verhalten austreiben, so befüllen Sie
das Argument \texttt{na.rm} mit dem Wert \texttt{TRUE} (\texttt{na.rm}
steht für \emph{r}e\emph{m}ove die NA, also fehlenden Werte).

\begin{Shaded}
\begin{Highlighting}[]
\FunctionTok{mean}\NormalTok{(Antworten, }\AttributeTok{na.rm =} \ConstantTok{TRUE}\NormalTok{)}
\DocumentationTok{\#\# [1] 42}
\end{Highlighting}
\end{Shaded}

\subsection{Vektorielles Rechnen}\label{sec-veccalc}

\begin{definition}[Vektorielles
Rechnen]\protect\hypertarget{def-veccalc}{}\label{def-veccalc}

Das Rechnen mit Vektoren in R bezeichnen wir als \emph{vektorielles
Rechnen}. \(\square\)

\end{definition}

Vektorielles Rechnen ist ein praktische Angelegenheit, man kann z.B.
folgende Dinge einfach in R ausrechnen.

Gegeben sei \texttt{x} als Vektor \texttt{(1,\ 2,\ 3)}. Dann können wir
die Differenz (Abweichung) jedes Elements von \texttt{x} zum Mittelwert
von \texttt{x} komfortabel so ausrechnen:

\begin{Shaded}
\begin{Highlighting}[]
\NormalTok{x }\SpecialCharTok{{-}} \FunctionTok{mean}\NormalTok{(x)}
\DocumentationTok{\#\# [1] {-}1  0  1}
\end{Highlighting}
\end{Shaded}

Etwas fancier ausgedrückt: Wir haben die Funktion mit Namen
\enquote{Differenz} (\enquote{Minus-Rechnen}) auf jedes Element von
\texttt{x} angewandt. Im Einzelnen haben wir also folgenden drei
Differenzen ausgerechnet:

\begin{Shaded}
\begin{Highlighting}[]
\DecValTok{1} \SpecialCharTok{{-}} \DecValTok{2}
\DecValTok{2} \SpecialCharTok{{-}} \DecValTok{2}
\DecValTok{3} \SpecialCharTok{{-}} \DecValTok{2}
\end{Highlighting}
\end{Shaded}

Diese drei Rechenschritte sind symbolisch in
Abbildung~\ref{fig-vektoriell} dargestellt.

\begin{figure}[H]

\centering{

\includegraphics[width=0.5\linewidth,height=\textheight,keepaspectratio]{020-R_files/figure-pdf/fig-vektoriell-1.pdf}

}

\caption{\label{fig-vektoriell}Schema des vektoriellen Rechnens: Eine
Funktion wird auf jedes Element eines Vektors angewandt. Hier: 1-2=-1;
2-2=0; 3-2=1}

\end{figure}%

\subsection{R-Quiz}\label{r-quiz}

\begin{exercise}[]\protect\hypertarget{exr-rquiz}{}\label{exr-rquiz}

~

\begin{figure}

\begin{minipage}{0.80\linewidth}
Ihre R-Muskeln sind gestählt? Oder doch noch nicht so ganz ausdefiniert?
Macht nichts! Trainieren Sie sich mit dem R-Quiz auf der
\href{https://sebastiansauer.github.io/Datenwerk/posts/r-quiz/r-quiz}{Datenwerk-Webseite}!
\(\square\)\end{minipage}%
%
\begin{minipage}{0.20\linewidth}

\begin{center}
\includegraphics[width=0.75\linewidth,height=\textheight,keepaspectratio]{020-R_files/figure-pdf/unnamed-chunk-24-1.pdf}
\end{center}

\end{minipage}%

\end{figure}%

\end{exercise}

\subsection{Ich brauche R-Hilfe!}\label{r-faq}

\begin{itemize}
\tightlist
\item
  \emph{Wo finde ich Hilfe zu einer bestimmten Funktion, z.B.
  \texttt{fun()}?} Geben Sie dazu folgenden R-Befehl ein:
  \texttt{help(fun)}. Alternativ geben Sie den Namen der Funktion in
  RStudio im Suchfeld beim Reiter \texttt{Help} ein.
\item
  \emph{Wenn ich ein R-Paket installiere, fragt mich R manchmal, ob ich
  auch Pakete installieren, will, die \enquote{kompiliert} werden
  müssen. Soll ich das machen?} Nein, das ist zumeist nicht nötig; geben
  Sie \enquote{no} ein.
\item
  \emph{In welchem Paket wohnt meine R-Funktion}? Suchen Sie nach der
  Funktion auf der Webseite \emph{RDocumentation}\footnote{\url{https://www.rdocumentation.org/}}.
\item
  \emph{Ich weiß nicht, wie der R-Befehl funktioniert!} Vermutlich haben
  andere Ihr Problem auch, und meistens hat irgendwer das Problem schon
  gelöst. Am besten suchen Sie mal auf
  \textless www.stackoverflow.com\textgreater.
\item
  \emph{Ich muss mal grundlegend verstehen, wozu ein bestimmten R-Paket
  gut ist. Was tun?} Lesen Sie die Dokumenation (\enquote{Vignette})
  eines R-Pakets durch. Für das Paket \texttt{dplyr} bekommen Sie so
  einen Überblick über die verfügbaren Vignetten diese Pakets:
  \texttt{vignette(package\ =\ "dplyr")}. Dann suchen Sie sich aus der
  angezeigten Liste eine Vignette raus; mit \texttt{vignette("rowwise")}
  können Sie sich dann die gewünschte Vignette (z.B. \texttt{rowwise})
  anzeigen lassen.
\item
  \emph{Oh nein, ich seh rot, das heißt, R zeigt mir irgendwas in roter
  Schrift an. Ist jetzt was kaputt?} Keine Sorge, R ist in seiner
  Ausgabe nicht sparsam mit roter Frabe. Solange es nicht als
  Fehlermeldung (\texttt{ERROR}) erscheint, ist es meist kein Problem.
\item
  \emph{R hat sich aufgehängt oder bringt einen Fehler an einer Stelle,
  wo sonst alles funktioniert hat.} Probieren Sie auf jeden Fall mal das
  AEG-Prinzip (Aus-Ein-Gut): sprich R neu starten.
\item
  \emph{Ich suche schon seit einer Stunde einen Fehler und find ihn
  nicht. Ich habe schon verschiedene Gegenstände vor Wut an die Wand
  geworfen. Was soll ich tun?} Machen Sie eine Pause. Doch, das ist
  ernst gemeint. Meine Erfahrung: Mit etwas Abstand wird der Kopf klarer
  und man findet das Problem viel einfacher. (Und manchmal ist einem das
  Problem danach schlichtweg egal.)
\item
  \emph{Irgendwie reagiert R komisch, vielleicht hat es sich
  aufgehängt?} Starten Sie R neu. Klicken Sie auf \emph{Session
  \textgreater{} Restart R}.
\item
  \emph{Ich muss mal klar Schiff machen und alle (oder einige) Variablen
  löschen. Wie werd ich das Zeug wieder los?} Beim Neustart von R werden
  alle Objekte (Variablen) gelöscht. Einzelne Objekte können Sie
  selektiv löschen mit dem Befehl \texttt{rm}, so löscht
  \texttt{rm(mariokart)} das Objekt namens \texttt{mariokart}.
\end{itemize}

\begin{tcolorbox}[enhanced jigsaw, bottomtitle=1mm, leftrule=.75mm, breakable, title=\textcolor{quarto-callout-caution-color}{\faFire}\hspace{0.5em}{Vorsicht}, bottomrule=.15mm, titlerule=0mm, left=2mm, opacityback=0, colframe=quarto-callout-caution-color-frame, rightrule=.15mm, colback=white, coltitle=black, toprule=.15mm, toptitle=1mm, colbacktitle=quarto-callout-caution-color!10!white, arc=.35mm, opacitybacktitle=0.6]

R ist penibel: So sind \texttt{name} und \texttt{Name} zwei verschiedene
Variablen für R. Groß- und Kleinschreibung wird von R streng beachtet!
Hingegen ist es R egal, ob Sie zur besseren Übersichtlichkeit
Leerzeichen in Ihre Syntax tippen. Ausnahme sind spezielle Operatoren
wie \texttt{\textless{}-} oder \texttt{\textless{}=}.

Eine gute Nachricht: Wenn R etwas von \texttt{WARNING} (bzw. Warnung)
sagt, können Sie das zumeist ignorieren. Eine \emph{Warnung} ist kein
Fehler (\texttt{ERROR}) und meistens nicht gravierend oder nicht
dringend. Ihre Syntax läuft trotzdem durch. Im Zweifel ist Googeln eine
gute Idee. Nur wenn R von \texttt{Error} spricht, ist es auch ein Fehler
und Ihre Syntax läuft nicht durch.\(\square\)

\end{tcolorbox}

\section{Mit Daten arbeiten}\label{mit-daten-arbeiten}

\subsection{Wo sind meine Daten?}\label{wo-sind-meine-daten}

Damit Sie eine Datendatei importieren können, müssen Sie wissen, wo die
Datei ist. Schauen wir uns zwei Möglichkeiten an, wo Ihre Datei liegen
könnte.

\begin{enumerate}
\def\labelenumi{\arabic{enumi}.}
\tightlist
\item
  Irgendwo im Internet\footnote{z.B. hier:
    \url{https://vincentarelbundock.github.io/Rdatasets/csv/openintro/mariokart.csv}}
\item
  Irgendwo auf Ihrem Computer, z.B. in Ihrem R-Projektordner
\end{enumerate}

In beiden Fällen wird der \enquote{Aufenthaltsort} der Datei durch den
\emph{Pfad} (Der Pfad einer Datei sagt, in welchem Ordner und Unterorder
und Unter-Unterordner die gesuchte Datei liegt. Ein Pfad könnte z.B. so
aussehen: \enquote{/Users/sebastiansaueruser/github-repos/statistik1/}.)
und den Namen der Datei definiert.

\begin{tcolorbox}[enhanced jigsaw, bottomtitle=1mm, leftrule=.75mm, breakable, title=\textcolor{quarto-callout-note-color}{\faInfo}\hspace{0.5em}{Hinweis}, bottomrule=.15mm, titlerule=0mm, left=2mm, opacityback=0, colframe=quarto-callout-note-color-frame, rightrule=.15mm, colback=white, coltitle=black, toprule=.15mm, toptitle=1mm, colbacktitle=quarto-callout-note-color!10!white, arc=.35mm, opacitybacktitle=0.6]

Wir werden in diesem Kurs häufiger mit dem Daten \texttt{mariokart}
arbeiten; Sie finden ihn
\href{https://vincentarelbundock.github.io/Rdatasets/csv/openintro/mariokart.csv}{hier}.\footnotemark{}

\end{tcolorbox}

\footnotetext{Auf dieser Webseite
\url{https://vincentarelbundock.github.io/Rdatasets/articles/data.html}
finden Sie eine große Zahl an Datensätzen. Nur für den Fall, dass Ihnen
langweilig ist.}

\subsection{Gebräuchliche
Datenformate}\label{gebruxe4uchliche-datenformate}

Daten werden in verschiedenen Formaten im Computer abgespeichert;
Tabellen häufig als

\begin{itemize}
\tightlist
\item
  Excel-Datei
\item
  CSV-Datei
\end{itemize}

In der Datenanalyse ist das gebräuchlichste Format für Daten in
Tabellenform die \emph{CSV-Datei}. Das hat den Grund, weil dieses Format
technisch schön einfach ist. Für uns Endverbraucher tut das nichts groß
zur Sache, die CSV-Datei beherbergt einfach eine brave Tabelle in einer
\emph{Textdatei}, sonst nichts.

In diesem Buch werden wir mit einem Datensatz namens \texttt{mariokart}
arbeiten; hallo Mario (s. Abbildung~\ref{fig-mario})!

\begin{figure}[H]

\centering{

\includegraphics[width=0.25\linewidth,height=\textheight,keepaspectratio]{img/mario.jpg}

}

\caption{\label{fig-mario}Hallo, Mario}

\end{figure}%

\begin{exercise}[]\protect\hypertarget{exr-csv}{}\label{exr-csv}

~

\textbf{Aufgabe}

Öffnen Sie die CSV-Datei \texttt{mariokart.csv} mit einem
\emph{Texteditor} (nicht mit Word und auch nicht mit Excel). Schauen Sie
sich gut an, was Sie dort sehen und erklären Sie die Datenstruktur.

\textbf{Lösung}

Eine CSV-Datei repräsentiert eine Datentabelle. Eine Spaltengrenze wird
mittels eines Kommas dargestellt (man kann auch andere Zeichen wählen,
um Spalten voneinander abzugrenzen).

\end{exercise}

\subsection{Daten importieren}\label{daten-importieren}

\subsubsection{Importieren von einem
R-Paket}\label{importieren-von-einem-r-paket}

Ihr Datensatz schon in einem R-Paket gespeichert, können Sie ihn aus
diesem R-Paket starten. Das ist die bequemste Option. Zum Beispiel
\enquote{wohnt} der Datensatz \texttt{mariokart} im R-Paket
\texttt{openintro}.

\begin{tcolorbox}[enhanced jigsaw, bottomtitle=1mm, leftrule=.75mm, breakable, title=\textcolor{quarto-callout-tip-color}{\faLightbulb}\hspace{0.5em}{Tipp}, bottomrule=.15mm, titlerule=0mm, left=2mm, opacityback=0, colframe=quarto-callout-tip-color-frame, rightrule=.15mm, colback=white, coltitle=black, toprule=.15mm, toptitle=1mm, colbacktitle=quarto-callout-tip-color!10!white, arc=.35mm, opacitybacktitle=0.6]

Ein häufiger Fehler ist, dass man vergisst, dass man zuerst ein R-Paket
installieren muss, bevor man es nutzen kann. Auf der anderen Seite muss
man ein R-Paket (wie andere Software auch) nur ein Mal installieren --
das Paket muss man ein Paket nach jedem Neustart von RStudio mit
\texttt{library()} starten.

\end{tcolorbox}

\begin{Shaded}
\begin{Highlighting}[]
\FunctionTok{data}\NormalTok{(}\StringTok{"mariokart"}\NormalTok{, }\AttributeTok{package =} \StringTok{"openintro"}\NormalTok{)}
\end{Highlighting}
\end{Shaded}

\subsubsection{Importieren von einer
Webseite}\label{importieren-von-einer-webseite}

Hier ist eine Möglichkeit, Daten (in Form einer Tabelle) von einer
Webseite (URL) in R zu importieren:

\begin{Shaded}
\begin{Highlighting}[]
\NormalTok{mariokart }\OtherTok{\textless{}{-}} \FunctionTok{read.csv}\NormalTok{(}\FunctionTok{paste0}\NormalTok{(}
  \StringTok{"https://vincentarelbundock.github.io/Rdatasets/"}\NormalTok{,}
  \StringTok{"csv/openintro/mariokart.csv"}\NormalTok{))}
\end{Highlighting}
\end{Shaded}

Es ist egal, welchen Namen Sie der Tabelle geben. Ich nehme oft
\texttt{d}, \emph{d} die Daten. Außerdem ist \texttt{d} kurz, muss man
nicht so viel tippen. Auf der anderen Seite ist \texttt{d} nicht gerade
präzise und vielsagend.

Werfen Sie einen Blick in die Tabelle (engl. \emph{to glimpse}).

\begin{Shaded}
\begin{Highlighting}[]
\FunctionTok{glimpse}\NormalTok{(d)}
\end{Highlighting}
\end{Shaded}

\href{https://vincentarelbundock.github.io/Rdatasets/doc/openintro/mariokart.html}{Hier}
findet sich eine Erklärung (Data-Dictionary) des Datensatzes.\footnote{\url{https://vincentarelbundock.github.io/Rdatasets/doc/openintro/mariokart.html}}

\subsubsection{Importieren von Ihrem Computer in RStudio
Desktop}\label{importieren-von-ihrem-computer-in-rstudio-desktop}

Gehen wir davon aus, dass sich die Datendatei im gleichen Ordner wie die
R-Datei (\texttt{.R}- oder \texttt{.qmd}-Datei) befindet, in der Sie den
Befehl zum Importieren schreiben. Dann können Sie die Datei einfach so
importieren:

\begin{Shaded}
\begin{Highlighting}[]
\NormalTok{d }\OtherTok{\textless{}{-}} \FunctionTok{read.csv}\NormalTok{(}\StringTok{"mariokart.csv"}\NormalTok{)}
\end{Highlighting}
\end{Shaded}

\begin{figure}

\begin{minipage}{0.80\linewidth}
\href{https://youtu.be/B_nuN-M0pQM}{Dieses Video} erklärt die Schritte
des Importierens einer Datendatei von Ihrem Computer.\end{minipage}%
%
\begin{minipage}{0.20\linewidth}

\begin{center}
\includegraphics[width=0.75\linewidth,height=\textheight,keepaspectratio]{020-R_files/figure-pdf/unnamed-chunk-31-1.pdf}
\end{center}

\end{minipage}%

\end{figure}%

\subsubsection{Importieren von Ihrem Computer in RStudio
Cloud}\label{importieren-von-ihrem-computer-in-rstudio-cloud}

Das Importieren in von Ihrem Computer zu RStudio Cloud ist identisch zum
Importieren von Ihrem Computer in RStudio Desktop. Nur dass Sie die
Datendatei vorab hochladen müssen, schließlich ist RStudio Cloud in der
Cloud und nicht auf Ihrem Computer. Klicken Sie dazu auf das Icon
\texttt{Upload} im Reiter \texttt{Files}, s.
Abbildung~\ref{fig-upload-to-posit-cloud}. Wählen Sie am besten den
Ordner als Ziel, in dem sich auch die R-Datei, von der aus Sie den
Befehl zum Daten importieren schreiben, befindet.

\begin{figure}[H]

\centering{

\includegraphics[width=0.5\linewidth,height=\textheight,keepaspectratio]{img/upload-to-posit-cloud.png}

}

\caption{\label{fig-upload-to-posit-cloud}}

\end{figure}%

\begin{tcolorbox}[enhanced jigsaw, bottomtitle=1mm, leftrule=.75mm, breakable, title=\textcolor{quarto-callout-note-color}{\faInfo}\hspace{0.5em}{Hinweis}, bottomrule=.15mm, titlerule=0mm, left=2mm, opacityback=0, colframe=quarto-callout-note-color-frame, rightrule=.15mm, colback=white, coltitle=black, toprule=.15mm, toptitle=1mm, colbacktitle=quarto-callout-note-color!10!white, arc=.35mm, opacitybacktitle=0.6]

Es gibt verschiedene Formate, in denen (Tabellen-)Dateien in einem
Computer abgespeichert werden. Die gebräuchlichsten sind CSV und XLSX.
Es gibt auch mehrere R-Befehle, um Daten in R zu importieren, z.B.
\texttt{read.csv()} oder \texttt{data\_read()}. Praktischerweise kann
der R-Befehl \texttt{data\_read()} viele verschiedene Formate
automatisch einlesen, so dass wir uns nicht weiter um das Format kümmern
brauchen. Der Vorteil von \texttt{read.csv} ist, dass Sie kein
Extra-Paket installiert bzw. gestartet haben müssen.

\end{tcolorbox}

\subsubsection{Daten importieren per
Klick}\label{daten-importieren-per-klick}

RStudio Desktops GUI (Benutzeroberfläche) erlaubt es Ihnen auch, Daten
per Klick, also ohne R-Befehle, zu importieren, s.
Abbildung~\ref{fig-daten-rstudio}. Sie können über diese Maske sowohl
CSV-Dateien, Excel-Dateien (XLS, XLSX) oder Daten-Dateien aus anderen
Statistik-Programmen (z.B. SPSS) importieren auf diese Weise. Zur
Erinnerung: CSV-Dateien sind Textdateien, wählen Sie in dem Fall also
\texttt{From\ Text}. Ich empfehle die Variante
\texttt{From\ Text\ (readr)\ ...}.

\begin{figure}[H]

\centering{

\includegraphics[width=0.5\linewidth,height=\textheight,keepaspectratio]{img/import-rstudio.png}

}

\caption{\label{fig-daten-rstudio}Daten importieren per Klick}

\end{figure}%

In der folgenden Maske können Sie unter \texttt{Browse} die zu
importierende Datendatei auswählen. Mit Klick auf \texttt{Import} wird
die Datei schließlich in R importiert.

\subsection{Dataframes}\label{dataframes}

Eine in R importierte Tabelle (mit bestimmten Eigenschaften) heißt
\emph{Dataframe}. Dataframes sind in der Datenanalyse von großer
Bedeutung. Tabelle~\ref{tbl-mariokart} ist die Tabelle mit den
Mariokart-Daten; etwas präziser gesprochen ein Dataframe mit Namen
\texttt{mariokart}. Übrigens ist Tabelle~\ref{tbl-mariokart} in
Normalform (Tidy-Format), vgl. Definition~\ref{def-tidy}.

\begin{definition}[Dataframe]\protect\hypertarget{def-dataframe}{}\label{def-dataframe}

Ein Dataframe (data frame; auch \enquote{Tibble} genannt; von
\enquote{tbl} wie Table) ist ein Datenobjekt in R zur Darstellung von
Tabellen. Dataframes bestehen aus einer oder mehreren Spalten. Spalten
haben einen Namen, sozusagen einen \enquote{Spaltenkopf}. Alle Spalten
müssen die gleiche Länge haben; anschaulich gesprochen ist eine Tabelle
(in R) rechteckig. Jede Spalte einzeln betrachtet kann als Vektor
aufgefasst werden. \(\square\)

\end{definition}

\begin{tcolorbox}[enhanced jigsaw, bottomtitle=1mm, leftrule=.75mm, breakable, title=\textcolor{quarto-callout-note-color}{\faInfo}\hspace{0.5em}{Hinweis}, bottomrule=.15mm, titlerule=0mm, left=2mm, opacityback=0, colframe=quarto-callout-note-color-frame, rightrule=.15mm, colback=white, coltitle=black, toprule=.15mm, toptitle=1mm, colbacktitle=quarto-callout-note-color!10!white, arc=.35mm, opacitybacktitle=0.6]

Geben Sie den Namen eines Dataframes ein, um sich den Inhalt anzeigen zu
lassen. Beachten Sie, dass Sie die Daten auf diese Weise nur anschauen,
nicht ändern können. \(\square\)

\end{tcolorbox}

\subsection{Tabellen in R betrachten}\label{sec-viewtab}

Wenn Sie in R z.B. die Tabelle \texttt{mariokart} in einer
Excel-typischen Ansicht betrachten wollen, klicken Sie am besten auf das
Tabellen-Icon im Reiter \emph{Environment}, gleich neben dem Namen
\texttt{mariokart}, s. Abbildung~\ref{fig-view-mariokart}.

\begin{figure}[H]

\centering{

\includegraphics[width=0.5\linewidth,height=\textheight,keepaspectratio]{img/rstudio-environment-mariokart.png}

}

\caption{\label{fig-view-mariokart}Per Klick auf das Tabellen-Icon
können Sie eine Tabellenansicht der Tabelle \texttt{mariokart} öffnen}

\end{figure}%

Alternativ öffnet der Befehl \texttt{View(mariokart)} die gleiche
Ansicht.

\section{Logikprüfung}\label{sec-logic}

\begin{quote}
{\emoji{student}} Wer will schon wieder wen prüfen?!
\end{quote}

In diesem Abschnitt schauen wir uns \emph{Logikprüfungen} an: Wir lassen
R prüfen, ob eine Variable einen bestimmten Wert hat oder größer/kleiner
als ein Referenzwert ist.

Definieren wir zuerst eine Variable, \texttt{x}.

\begin{Shaded}
\begin{Highlighting}[]
\NormalTok{x }\OtherTok{\textless{}{-}} \DecValTok{42}
\end{Highlighting}
\end{Shaded}

Dann fragen wir R, ob diese Variable den Wert \texttt{42} hat.

\begin{Shaded}
\begin{Highlighting}[]
\NormalTok{x }\SpecialCharTok{==} \DecValTok{42}
\DocumentationTok{\#\# [1] TRUE}
\end{Highlighting}
\end{Shaded}

\begin{quote}
{\emoji{robot}} Hallo, Mensch. Ja, diese Variable hat den Wert 42.
\end{quote}

(Danke, R.)

Möchte man mit R prüfen, ob eine Variable \texttt{x} einen bestimmten
\texttt{Wert} (\enquote{Inhalt}) hat, so schreibt man:

\texttt{x\ ==\ Wert}.

Man beachte das \emph{doppelte} Gleichheitszeichen. Zur Prüfung auf
Gleichheit muss man das doppelte Gleichheitszeichen verwenden.

\begin{tcolorbox}[enhanced jigsaw, bottomtitle=1mm, leftrule=.75mm, breakable, title=\textcolor{quarto-callout-caution-color}{\faFire}\hspace{0.5em}{Vorsicht}, bottomrule=.15mm, titlerule=0mm, left=2mm, opacityback=0, colframe=quarto-callout-caution-color-frame, rightrule=.15mm, colback=white, coltitle=black, toprule=.15mm, toptitle=1mm, colbacktitle=quarto-callout-caution-color!10!white, arc=.35mm, opacitybacktitle=0.6]

Ein beliebter Fehler ist es, bei der Prüfung auf Gleichheit, nur ein
Gleichheitszeichen zu verwenden, z.B. so: \texttt{x\ =\ 73}. Mit einem
Gleichheitszeichen prüft man aber \emph{nicht} auf Gleichheit, sondern
man definiert die Variable oder bestimmt ein Funktionsargument, s.
Kapitel~\ref{sec-rvars}. \(\square\)

\end{tcolorbox}

Tabelle~\ref{tbl-lgl} gibt einen Überblick über wichtige Logikprüfungen
in R. (Um das Zeichen für das logische ODER, \texttt{\textbar{}} auf
einer Mac-Tastatur zu erhalten, drückt man \emph{Option+7}.)

\begin{table}

\caption{\label{tbl-lgl}Logische Prüfungen in R}

\centering{

\fontsize{12.0pt}{14.4pt}\selectfont
\begin{tabular*}{\linewidth}{@{\extracolsep{\fill}}ll}
\toprule
Prüfung.auf & R-Syntax \\ 
\midrule\addlinespace[2.5pt]
Gleichheit & x == Wert \\ 
Ungleichheit & x != Wert \\ 
Größer als Wert & x > Wert \\ 
Größer oder gleich Wert & x >= Wert \\ 
Kleiner als Wert & x < Wert \\ 
Kleiner oder gleich Wert & x <= Wert \\ 
Logisches UND & (x < Wert1) \& (x > Wert2) \\ 
Logisches ODER & (x < Wert1) | (x > Wert2) \\ 
\bottomrule
\end{tabular*}

}

\end{table}%

\section{Praxisbezug}\label{praxisbezug-1}

\begin{quote}
{\emoji{student}} R in der Praxis wirklich genutzt? Oder ist R nur der
Traum von (vielleicht verwirrten) Profs im Elfenbeinturm?
\end{quote}

Schauen wir uns dazu die Suchanfragen bei
\href{www.stackoverflow.com}{stackoverflow.com} an, dem größten
FAQ-Forum für Software-Entwicklung. Wir vergleichen Suchanfragen mit dem
Tag \texttt{{[}r{]}} zu Suchanfragen mit dem Tag
\texttt{{[}spss{]}}(SPSS ist eine an Hochschulen verbreitete
Statistik-Software). Die Ergebnisse sind in Abbildung
Abbildung~\ref{fig-stackoverflow1} dargestellt\footnote{Die Daten wurden
  am 2022-02-24, 17:21 CET, abgerufen.}

\begin{figure}[H]

\centering{

\includegraphics[width=0.75\linewidth,height=\textheight,keepaspectratio]{020-R_files/figure-pdf/fig-stackoverflow1-1.pdf}

}

\caption{\label{fig-stackoverflow1}Suchanfragen nach R bzw SPSS, Stand
2022-02-24}

\end{figure}%

Das ist grob gerechnet ein Faktor von 200 (der Unterschied von R zu
SPSS). Dieses Ergebnis lässt darauf schließen, dass R in der Praxis viel
mehr als SPSS gebraucht wird.

\begin{quote}
{\emoji{student}} Aber ist R wirklich ein Werkzeug, das mir im Job
hilft?
\end{quote}

\begin{quote}
{\emoji{teacher}} Viele Firmen weltweit nutzen R zur
Datenanalyse.\footnote{wie diese Liste zeigt:
  \url{https://www.quora.com/Which-organizations-use-R?share=1} zeigt}.
\end{quote}

\begin{quote}
{\emoji{woman-student}} R ist \emph{der} Place-to-be für die
Datenanalyse.
\end{quote}

\begin{quote}
{\emoji{student}} Aber ist Datenanalyse wirklich etwas, womit ich in
Zukunft einen guten Job bekomme?
\end{quote}

\begin{quote}
{\emoji{teacher}} Berufe mit Bezug zu Daten, Datenanalyse oder,
allgemeiner, Künstlicher Intelligenz (artificial intelligence) gehören
zu den stark wachsenden Berufen:
\end{quote}

\begin{quote}
Artificial intelligence (AI) continues to make a strong showing on our
Emerging Jobs lists, which is no surprise. Many jobs that have risen up
as a result of AI in fields like cybersecurity and data science and
because it's is so pervasive many roles may demand more knowledge of AI
than you may think. For example, real estate and business development
roles (Berger, 2019).
\end{quote}

\section{Aufgaben}\label{aufgaben-1}

\begin{exercise}[Statistik-Meme]\protect\hypertarget{exr-meme}{}\label{exr-meme}

Suchen Sie ein schönes Meme zum Thema Statistik, Datenanalyse und Data
Science. \(\square\)

\end{exercise}

Die Webseite \href{https://datenwerk.netlify.app}{datenwerk.netlify.app}
stellt eine Reihe von einschlägigen Übungsaufgaben bereit. Sie können
die Suchfunktion der Webseite nutzen, um die Aufgaben mit den folgenden
Namen zu suchen:

\begin{enumerate}
\def\labelenumi{\arabic{enumi}.}
\tightlist
\item
  \href{https://sebastiansauer.github.io/Datenwerk/posts/typ-fehler-r-01/typ-fehler-r-01.html}{Typ-Fehler-R-01}
\item
  \href{https://sebastiansauer.github.io/Datenwerk/posts/typ-fehler-r-02/typ-fehler-r-02.html}{Typ-Fehler-R-02}
\item
  \href{https://sebastiansauer.github.io/Datenwerk/posts/typ-fehler-r-03/typ-fehler-r-03.html}{Typ-Fehler-R-03}
\item
  \href{https://sebastiansauer.github.io/Datenwerk/posts/typ-fehler-r-04/typ-fehler-r-04.html}{Typ-Fehler-R-04}
\item
  \href{https://sebastiansauer.github.io/Datenwerk/posts/typ-fehler-r-06a/typ-fehler-r-06a.html}{Typ-Fehler-R-06a}
\item
  \href{https://sebastiansauer.github.io/Datenwerk/posts/typ-fehler-r-07/typ-fehler-r-07.html}{Typ-Fehler-R-07}
\item
  \href{https://sebastiansauer.github.io/Datenwerk/posts/typ-fehler-r-08-name-clash/typ-fehler-r-08-name-clash}{Typ-Fehler-R-08-name-clash}
\item
  \href{https://sebastiansauer.github.io/Datenwerk/posts/logikpruefung1/logikpruefung1}{Logikpruefung1}
\item
  \href{https://sebastiansauer.github.io/Datenwerk/posts/logikpruefung2/logikpruefung2}{Logikpruefung2}
\item
  \href{https://sebastiansauer.github.io/Datenwerk/posts/there-is-no-package/there-is-no-package.html}{there-is-no-package}
\item
  \href{https://sebastiansauer.github.io/Datenwerk/posts/wertberechnen2/wertberechnen2}{Wertberechnen2}
\item
  \href{https://sebastiansauer.github.io/Datenwerk/posts/wertzuweisen_mc/wertzuweisen_mc}{Wertzuweisen\_mc}
\item
  \href{https://sebastiansauer.github.io/Datenwerk/posts/argumente/argumente.html}{argumente}
\item
  \href{https://sebastiansauer.github.io/Datenwerk/posts/import-mtcars/import-mtcars.html}{import-mtcars}
\item
  \href{https://sebastiansauer.github.io/Datenwerk/posts/wertzuweisen/wertzuweisen}{Wertzuweisen}
\item
  \href{https://sebastiansauer.github.io/Datenwerk/posts/wertpruefen/wertpruefen}{Wertpruefen}
\item
  \href{https://sebastiansauer.github.io/Datenwerk/posts/wrangle1/wrangle1.html}{wrangle1}
\item
  \href{https://sebastiansauer.github.io/Datenwerk/posts/repro1-sessioninfo/repro1-sessioninfo.html}{repro1-sessioninfo}
\item
  \href{https://sebastiansauer.github.io/Datenwerk/posts/mw-berechnen/mw-berechnen}{mw-berechnen}
\end{enumerate}

Prüfen Sie Ihr Wissen zu R mit
\href{https://sebastiansauer.github.io/Datenwerk/posts/r-quiz/r-quiz}{einem
Quiz}!\footnote{\url{https://sebastiansauer.github.io/Datenwerk/posts/r-quiz/r-quiz}}

Noch nicht genug? Checken Sie alle Aufgaben mit dem Tag
\href{https://sebastiansauer.github.io/Datenwerk/\#category=R}{R} auf
dem Datenwerk aus.\footnote{\url{https://sebastiansauer.github.io/Datenwerk/\#category=R}}

\begin{tcolorbox}[enhanced jigsaw, bottomtitle=1mm, leftrule=.75mm, breakable, title=\textcolor{quarto-callout-note-color}{\faInfo}\hspace{0.5em}{Hinweis}, bottomrule=.15mm, titlerule=0mm, left=2mm, opacityback=0, colframe=quarto-callout-note-color-frame, rightrule=.15mm, colback=white, coltitle=black, toprule=.15mm, toptitle=1mm, colbacktitle=quarto-callout-note-color!10!white, arc=.35mm, opacitybacktitle=0.6]

Die Webseite
\href{https://sebastiansauer.github.io/Datenwerk/}{Datenwerk} stellt
eine Reihe von Aufgaben zum Thema Statistik bereit. Zu jeder Aufgabe
sind ein oder mehrere Schlagwörter (Tags) zugeordnet. Wenn Sie auf ein
Schlagwort klicken, sehen Sie die Liste der Aufgaben mit diesem
Schlagwort. Es kann aber sein, dass Sie einige Aufgabe nicht lösen
können, da Wissen vorausgesetzt wird, das Sie (noch) nicht haben. Lassen
Sie sich davon nicht ins Boxhorn jagen. Ignorieren Sie solche Aufgaben
fürs Erste. \(\square\)

\end{tcolorbox}

\section{Vertiefung}\label{vertiefung-2}

\subsection{\texorpdfstring{Varianten zu
\texttt{read.csv}}{Varianten zu read.csv}}\label{varianten-zu-read.csv}

Hier ist eine weitere Möglichkeit, um Daten von einem Ordner (egal ob
dieser sich im Internet oder auf Ihrem Computer befinde) einzulesen,
stellt die Funktion \texttt{data\_read} bereit:

\begin{Shaded}
\begin{Highlighting}[]
\FunctionTok{library}\NormalTok{(easystats)  }\CommentTok{\# Das Paket muss installiert sein}
\NormalTok{d }\OtherTok{\textless{}{-}} \FunctionTok{data\_read}\NormalTok{(}\FunctionTok{paste0}\NormalTok{(}
  \StringTok{"https://vincentarelbundock.github.io/Rdatasets/"}\NormalTok{,}
  \StringTok{"csv/openintro/mariokart.csv"}\NormalTok{))}
\end{Highlighting}
\end{Shaded}

Der Unterschied ist, dass \texttt{data\_read()} \emph{viele} Formate von
Daten (XLSX, CSV, SPSS, \ldots) verkraftet, wohingegen
\texttt{read.csv()} nur Standard-CSV einlesen kann.

Schauen wir uns die letzte R-Syntax en Detail an:

\begin{verbatim}
Hey R,
hol das "Buch" easystats aus der Bücherei und lies es
definiere als "d" die Tabelle,
die du unter der angegebenen URL findest.
\end{verbatim}

In R gibt es oft viele Möglichkeiten, ein Ziel zu erreichen. Zum
Beispiel haben wir hier den Befehl \texttt{data\_read()} verwendet, um
Daten zu importieren. Andere, gebräuchliche Befehle, die CSV-Dateien
importieren, heißen \texttt{read.csv()} (aus dem Standard-R, kein
Extra-Paket nötig) und \texttt{read\_csv()} (aus dem Meta-Paket
\texttt{tidyverse}).

\subsection{Importieren von
Excel-Tabellen}\label{importieren-von-excel-tabellen}

Mit der Funktion \texttt{data\_read} aus \texttt{\{easystats\}} kann man
viele verschiedene Datenformate importieren, auch Excel-Tabellen (.xls,
.xlsx).

Als Beispiel betrachten wir den Datensatz \texttt{extra} aus dem R-Paket
\texttt{\{pradadata\}}\footnote{\url{https://github.com/sebastiansauer/pradadata}}.
In diesem Datensatz werden die Ergebnisse einer Umfrage zu den
Korrelaten von Extraversion beschrieben. Details zu der
zugrundeliegenden Studie finden Sie hier:
\url{https://osf.io/4kgzh}.\footnote{Ein Daten-Dictionary findet sich
  hier:
  \url{https://github.com/sebastiansauer/statistik1/raw/main/data/extra-dictionary.md}.}

Laden Sie die Excel-Datei herunter. Angenommen, Sie speichern die
Excel-Datei in einem Unterordner namens \texttt{daten} Ihres aktuellen
Projektordners. Dann können Sie die Daten so importieren:

\begin{Shaded}
\begin{Highlighting}[]
\FunctionTok{library}\NormalTok{(easystats)}
\NormalTok{extra }\OtherTok{\textless{}{-}} \FunctionTok{data\_read}\NormalTok{(}\StringTok{"data/extra.xls"}\NormalTok{)}
\end{Highlighting}
\end{Shaded}

Allerdings kann \texttt{data\_read()} keine Dateien aus dem Internet
importieren, was praktisch wäre. Stattdessen muss die Datei lokal auf
Ihrer Festplatte liegen.

Wenn Sie allerdings \enquote{remote}, also aus dem Internet, eine
Excel-Datei importieren möchten, so können Sie das mit \texttt{import()}
aus dem R-Paket \texttt{\{rio\}} tun:

\begin{Shaded}
\begin{Highlighting}[]
\FunctionTok{library}\NormalTok{(rio)}
\NormalTok{extra\_path }\OtherTok{\textless{}{-}} \FunctionTok{paste0}\NormalTok{(}
  \StringTok{"https://github.com/sebastiansauer/statistik1/"}\NormalTok{,}
  \StringTok{"raw/main/data/extra.xls"}\NormalTok{)}
\NormalTok{extra }\OtherTok{\textless{}{-}} \FunctionTok{import}\NormalTok{(extra\_path)}
\end{Highlighting}
\end{Shaded}

\begin{tcolorbox}[enhanced jigsaw, bottomtitle=1mm, leftrule=.75mm, breakable, title=\textcolor{quarto-callout-note-color}{\faInfo}\hspace{0.5em}{Hinweis}, bottomrule=.15mm, titlerule=0mm, left=2mm, opacityback=0, colframe=quarto-callout-note-color-frame, rightrule=.15mm, colback=white, coltitle=black, toprule=.15mm, toptitle=1mm, colbacktitle=quarto-callout-note-color!10!white, arc=.35mm, opacitybacktitle=0.6]

CSV-Dateien werden auf vielen Computern als eine Datei erkannt, die
Excel öffnen kann und das auch tut, wenn man eine CSV-Datei
doppelklickt. Dennoch ist das CSV-Format keine Datei im Excel-Format,
sondern eine einfache Text-Datei, die auch mit jedem Text-Editor
geöffnet und bearbeitet werden kann. \(\square\)

\end{tcolorbox}

Alternativ können Sie in RStudio auch Excel-Dateien \emph{ohne} R-Code
importieren, s. Abbildung~\ref{fig-daten-rstudio}.

\subsection{Der Dollar-Operator}\label{sec-dollar-op}

In Definition~\ref{def-veccalc} hatten wir Vektoren definiert. Solche
Vektoren fliegen sozusagen frei in Ihrem \texttt{Environment} herum
(Schauen Sie mal dort nach!) Die Spalten einer Tabelle sind aber auch
Vektoren, nur eben nicht frei im \texttt{Environment}, sondern in eine
Tabelle eingebunden.

Möchte man diese Vektoren direkt ansprechen, so kann man das mit dem
sog. \emph{Dollar-Operator} \texttt{\$} tun.

Angenommen, Sie möchten sich die Verkaufspreise (\texttt{total\_pr}) aus
der Tabelle \texttt{mariokart} herausziehen, dann können Sie das mit dem
Dollar-Operator tun:

\begin{Shaded}
\begin{Highlighting}[]
\NormalTok{mariokart}\SpecialCharTok{$}\NormalTok{total\_pr }\SpecialCharTok{|\textgreater{}} \FunctionTok{head}\NormalTok{()}
\DocumentationTok{\#\# [1] 52 37 46 44 71 45}
\end{Highlighting}
\end{Shaded}

Der Dollar-Operator trennt den Namen der Tabelle vom Namen der Spalte.

Natürlich können Sie mit dem resultierenden Vektor beliebig
weiterarbeiten, etwa ihn in einem anderen Vektor speichern oder eine
Funktion anwenden:

\begin{Shaded}
\begin{Highlighting}[]
\NormalTok{verkaufspreise }\OtherTok{\textless{}{-}}\NormalTok{ mariokart}\SpecialCharTok{$}\NormalTok{total\_pr}
\FunctionTok{mean}\NormalTok{(verkaufspreise)}
\DocumentationTok{\#\# [1] 50}
\FunctionTok{mean}\NormalTok{(mariokart}\SpecialCharTok{$}\NormalTok{total\_pr)  }\CommentTok{\# synonym zur obigen Zeile}
\DocumentationTok{\#\# [1] 50}
\end{Highlighting}
\end{Shaded}

\subsection{R-Funktionen
verschachteln}\label{r-funktionen-verschachteln}

Das Kombinieren von Funktionen kann kompliziert werden:

\begin{codelisting}

\caption{\label{lst-schachtel}Verschachtelte Funktionen}

\centering{

\begin{Shaded}
\begin{Highlighting}[]
\NormalTok{x }\OtherTok{\textless{}{-}} \FunctionTok{c}\NormalTok{(}\DecValTok{1}\NormalTok{, }\DecValTok{2}\NormalTok{, }\DecValTok{3}\NormalTok{)}
\FunctionTok{sum}\NormalTok{(}\FunctionTok{abs}\NormalTok{(}\FunctionTok{mean}\NormalTok{(x)}\SpecialCharTok{{-}}\NormalTok{x)) }
\DocumentationTok{\#\# [1] 2}
\end{Highlighting}
\end{Shaded}

}

\end{codelisting}%

Die Funktion \texttt{abs(x)} gibt den (Absolut-)Betrag von \texttt{x}
zurück (entfernt das Vorzeichen, mit anderen Worten).

Verschachtelte Ausdrücke lesen sich von innen nach außen (und werden in
dieser Reihenfolge abgearbeitet). Für unser Beispiel
(Listing~\ref{lst-schachtel}):

\begin{enumerate}
\def\labelenumi{\arabic{enumi}.}
\tightlist
\item
  Berechne den Mittelwert von \texttt{x}
\item
  Ziehe vom Mittelwert jeweils die Elemente von \texttt{x} ab
\item
  Nimm vom Ergebnis jeweils den Absolutwert
\item
  Summiere diese Werte
\end{enumerate}

Kurz gesagt: Hier haben wir die mittlere Absolutabweichung der Elemente
von \texttt{x} zum Mittelwert ausgerechnet.

\subsection{R und Friends updaten}\label{r-und-friends-updaten}

Irgendwann werden Ihr R, Ihr RStudio und Ihre R-Pakete veraltet sein, s.
Abbildung~\ref{fig-arnie}. Installieren Sie dann einfach die neue
Version von R und RStudio wie oben beschreiben, s.
Kapitel~\ref{sec-install-r}.

So updaten Sie Ihre R-Pakete: Klicken Sie im Reiter \texttt{Packages}
(in RStudio) und dort auf den Button \texttt{Update}. Wenn die Anzahl
der zu aktualisierenden Pakete groß ist, dann besser nicht alle
auswählen, sondern nur ein paar. Dann die nächsten paar Pakete usw.
Denken Sie daran, dass Sie die Software (R, RStudio, R-Paket), die Sie
updata/installieren, nicht laufen darf.

\begin{tcolorbox}[enhanced jigsaw, bottomtitle=1mm, leftrule=.75mm, breakable, title=\textcolor{quarto-callout-note-color}{\faInfo}\hspace{0.5em}{Hinweis}, bottomrule=.15mm, titlerule=0mm, left=2mm, opacityback=0, colframe=quarto-callout-note-color-frame, rightrule=.15mm, colback=white, coltitle=black, toprule=.15mm, toptitle=1mm, colbacktitle=quarto-callout-note-color!10!white, arc=.35mm, opacitybacktitle=0.6]

Ihre R-Pakete sollten aktuell sein. Klicken Sie beim Reiter
\emph{Packages} auf \enquote{Update}, um Ihre R-Pakete zu aktualisieren.
Arnold Schwarzenegger rät, Ihre R-Pakete aktuell zu halten, s.
Abbildung~\ref{fig-arnie}.

\end{tcolorbox}

\begin{figure}[H]

\centering{

\includegraphics[width=0.5\linewidth,height=\textheight,keepaspectratio]{img/terminator.jpg}

}

\caption{\label{fig-arnie}R-Pakete sollten stets aktuell sein, so Arnold
Schwarzenegger (imgflip, 2024a)}

\end{figure}%

\subsection{Benötigte Daten}\label{benuxf6tigte-daten}

Sie benötigen in diesem Kapitel den Datensatz \texttt{mariokart}, der
entweder online\footnote{ über diese Internetadresse:
  \url{https://vincentarelbundock.github.io/Rdatasets/csv/openintro/mariokart.csv}}
oder über R-Paket \texttt{openintro} importiert werden kann:

\subsubsection{Import via Download}\label{import-via-download}

\begin{Shaded}
\begin{Highlighting}[]
\NormalTok{mariokart }\OtherTok{\textless{}{-}} \FunctionTok{read.csv}\NormalTok{(}\FunctionTok{paste0}\NormalTok{(}
  \StringTok{"https://vincentarelbundock.github.io/Rdatasets/"}\NormalTok{,}
  \StringTok{"csv/openintro/mariokart.csv"}\NormalTok{))}
\end{Highlighting}
\end{Shaded}

\subsubsection{Import via R-Paket}\label{import-via-r-paket}

\begin{Shaded}
\begin{Highlighting}[]
\CommentTok{\# Das Paket \textquotesingle{}openintro\textquotesingle{} muss installiert sein:}
\FunctionTok{data}\NormalTok{(mariokart, }\AttributeTok{package =} \StringTok{"openintro"}\NormalTok{) }
\end{Highlighting}
\end{Shaded}

\section{Literaturhinweise}\label{literaturhinweise-1}

\enquote{Warum R? Warum, R?} heißt ein Kapitel in Sauer (2019), das
einiges zum Pro und Contra von R ausführt. In Kapitel 3 in der gleichen
Quelle finden sich viele Hinweise, wie man R startet; In Kapitel 4
werden Grundlagen von \enquote{Errisch} erläutert; Kapitel 5 führt in
Datenstrukturen von R ein (schon etwas anspruchsvoller). Alternativ
bietet \href{https://moderndive.com/1-getting-started.html}{Kapitel 1}
von Ismay \& Kim (2020) einen guten und sehr anwenderfreundlichen
Überblick. Das Buch hat auch den Vorteil, dass es komplett frei online
verfügbar ist. Vergleichbar dazu ist Cetinkaya-Rundel \& Hardin (2021),
vielleicht einen Tick formaler; auf jeden Fall genau das richtige Niveau
für Bachelor-Statistik in angewandten nicht-technischen Studiengängen.

\chapter{Daten umformen}\label{daten-umformen}

\section{Lernsteuerung}\label{lernsteuerung-2}

Abb. Abbildung~\ref{fig-ueberblick} zeigt den Standort dieses Kapitels
im Lernpfad und gibt damit einen Überblick über das Thema dieses
Kapitels im Kontext aller Kapitel.

\subsection{Lernziele}\label{lernziele-3}

\begin{itemize}
\tightlist
\item
  Sie können folgende Verben des Datenjudo anwenden: \texttt{arrange()},
  \texttt{filter()}, \texttt{select()}, \texttt{summarise()},
  \texttt{group\_by()}, \texttt{mutate()}.
\item
  Sie können R-Befehle mit der \enquote{Pfeife} verketten.
\end{itemize}

\subsection{Benötigte R-Pakete}\label{benuxf6tigte-r-pakete}

\begin{Shaded}
\begin{Highlighting}[]
\FunctionTok{library}\NormalTok{(tidyverse)}
\FunctionTok{library}\NormalTok{(easystats)}
\end{Highlighting}
\end{Shaded}

\subsection{Benötigte Daten}\label{benuxf6tigte-daten-1}

\begin{Shaded}
\begin{Highlighting}[]
\NormalTok{mariokart }\OtherTok{\textless{}{-}} \FunctionTok{paste0}\NormalTok{(}
  \StringTok{"https://vincentarelbundock.github.io/Rdatasets/"}\NormalTok{,}
  \StringTok{"csv/openintro/mariokart.csv"}\NormalTok{)}

\NormalTok{mariokart }\OtherTok{\textless{}{-}} \FunctionTok{read.csv}\NormalTok{(mariokart\_path)}
\end{Highlighting}
\end{Shaded}

\begin{exercise}[]\protect\hypertarget{exr-los}{}\label{exr-los}

Importieren Sie den Datensatz \texttt{mariokart}. Probieren Sie alle
vorgestellten Verben des Datenjudos aus, zunächst in dem Sie einfach den
R-Code kopieren und selber ausführen.\(\square\)

\end{exercise}

\subsection{Frag den Bot}\label{frag-den-bot}

\begin{quote}
{\emoji{student}} Erkläre einem Fünfjährigen, warum Statistik nützlich
ist.
\end{quote}

\begin{quote}
{\emoji{robot}} Okay, hör zu. Stell dir vor, du hast viele
Spielzeugautos und du willst herausfinden, welches Auto am meisten
bespielt wird. Du kannst jeden Tag zählen, wie oft du jedes Auto benutzt
hast. Aber das wäre sehr mühsam und könnte schnell durcheinander
gebracht werden. Stattdessen kannst du mit Statistik eine einfachere
Methode finden. Du kannst zum Beispiel eine Liste machen und jedes Auto
mit einer bestimmten Farbe markieren. Dann kannst du jeden Tag einfach
zählen, wie oft du jede Farbe verwendet hast. Am Ende kannst du
herausfinden, welches Auto am meisten bespielt wurde, indem du siehst,
welche Farbe am meisten verwendet wurde. Das ist ein Beispiel dafür, wie
Statistik uns dabei helfen kann, Muster in großen Datensätzen zu finden
und Entscheidungen zu treffen, ohne alles einzeln zu zählen.\footnote{Quelle:
  ChatGTP 3.5, 2023-02-09}
\end{quote}

\section{Datenjudo}\label{datenjudo}

\begin{definition}[Datenjudo]\protect\hypertarget{def-datenjudo}{}\label{def-datenjudo}

Mit \emph{Datenjudo} meint man den Prozess der Aufbereitens, Umformens
oder Zusammenfassen von Daten, sowohl für einzelne Beobachtungen (Zeilen
einer Datentabelle) oder Variablen (Spalten einer Datentabelle) oder
einer ganzen Datentabelle. \(\square\)

\end{definition}

\subsection{Praxisbezug: Aus dem Alltag des Data
Scientisten}\label{praxisbezug-aus-dem-alltag-des-data-scientisten}

Denkt man an Data Science, stellt man sich coole Leute vor (in San
Francisco oder Berlin), die an abgefahrenen Berechnungen mit hoch
komplexen statistischen Modellen für gigantische Datenmengen basteln.
Laut dem \emph{Harvard Business Review} allerdings, verbringen Data
Scientisten \enquote{80\%} ihrer Zeit mit dem \emph{Aufbereiten} von
Daten (Bowne-Anderson, 2018). Ja: mit uncoolen Tätigkeiten wie
Tippfehler aus Datensätzen entfernen oder die Daten überhaupt nutzbar
und verständlich zu machen.

Das zeigt zumindest, dass das Aufbereiten von Daten a) wichtig ist und
b) dass man allein damit schon weit kommen kann. Eine gute Nachricht ist
(vielleicht), dass das Aufbereiten von Daten keine aufwändige Mathematik
verlangt, stattdessen muss man ein paar Handgriffe und Kniffe kennen.
Daher passt der Begriff \emph{Datenjudo} vielleicht ganz gut. Kümmern
wir uns also um das Aufbereiten bzw. Umformen von Daten, um das
Datenjudo. \(\square\)

\begin{example}[]\protect\hypertarget{exm-datenjudo}{}\label{exm-datenjudo}

Beispiele für typische Tätigkeiten des Datenjudos sind:

\begin{itemize}
\tightlist
\item
  Zeilen \emph{filtern} (z.B. nur Studentis des Studiengangs X)
\item
  Zeilen \emph{sortieren} (z.B. Studenten mit guten Noten in den oberen
  Zeilen)
\item
  Spalten \emph{wählen} (z.B. 100 weitere Produkte ausblenden)
\item
  Spalten in eine Zahl \emph{zusammenfassen} (z. B. Notenschnitt der 1.
  Klausur)
\item
  Tabelle \emph{gruppieren} (z.B. Analyse getrennt nach Standorten)
\item
  Werte aus einer Spalte \emph{verändern} oder \emph{neue Spalte} bilden
  (z. B. Punkte in Prozent-Richtige umrechnen).
\item
  \ldots{} \(\square\)
\end{itemize}

\end{example}

\subsection{Mach's einfach}\label{machs-einfach}

Es gibt einen (einfachen) Trick, wie man umfangreiche Datenaufbereitung
elegant geregelt kriegt. Flingt fast zu schön, um wahr zu sein (s.
Abbildung~\ref{fig-that-would-be-great}).

\begin{figure}[H]

\centering{

\includegraphics[width=0.5\linewidth,height=\textheight,keepaspectratio]{img/thatwouldbegreat.jpg}

}

\caption{\label{fig-that-would-be-great}Mach's einfach (imgflip, 2024a)}

\end{figure}%

Der Trick besteht darin, komplexe Operationen in mehrere einfache
Teilschritte zu zergliedern. (In gewisser Weise besteht das Wesen einer
Analyse eben darin: die Zerlegung eines Gegenstands in seine
Bestandteile.) Man könnte vom \enquote{Lego-Prinzip} sprechen, s.
Abbildung~\ref{fig-lego}. Im linken Teil von Abbildung~\ref{fig-lego}
sieht man ein (recht) komplexes Gebilde. Zerlegt man es aber in seine
Einzelteile, so sind es deutlich einfachere geometrische Objekte wie
Dreiecke oder Quadrate (rechter Teil des Diagramms).

\begin{figure}[H]

\centering{

\includegraphics[width=0.75\linewidth,height=\textheight,keepaspectratio]{img/Bausteine_dplyr_crop.pdf}

}

\caption{\label{fig-lego}Das Lego-Prinzip (Sauer, 2019)}

\end{figure}%

Damit Sie es selber einfach machen können, müssen Sie selber Hand
anlegen. Importieren Sie daher den Datensatz \texttt{mariokart}, z.B.
so:

\begin{Shaded}
\begin{Highlighting}[]
\NormalTok{mariokart }\OtherTok{\textless{}{-}} \FunctionTok{read.csv}\NormalTok{(mariokart\_path)}
\end{Highlighting}
\end{Shaded}

\begin{example}[]\protect\hypertarget{exm-datenjudo}{}\label{exm-datenjudo}

Sie arbeiten immer noch bei dem großen Online-Auktionshaus. Mittlerweile
haben Sie sich den Ruf des \enquote{Datenguru} erworben. Vielleicht weil
Sie behauptet haben, Data Science sei zu 80\% Datenjudo, das hat
irgendwie Eindruck geschindet\ldots{} Naja, jedenfalls müssen Sie jetzt
mal zeigen, dass Sie nicht nur schlaue Sprüche draufhaben, sondern auch
die Daten ordentlich abbürsten können. Sie analysieren dafür im
Folgenden den Datensatz \texttt{mariokart}. Na, dann los.\(\square\)

\end{example}

\section{Die Verben des Datenjudos}\label{die-verben-des-datenjudos}

Im R-Paket \texttt{dplyr}, das wiederum Teil des R-Pakets
\texttt{tidyverse} ist, gibt es eine Reihe von R-Befehlen, die das
Datenjudo in eine Handvoll einfacher Verben runterbrechen. (Falls Sie
das R-Paket \texttt{tidyverse} noch nicht installiert haben sollten,
wäre jetzt ein guter Zeitpunkt dafür.) Die wichtigsten Verben des
Datenjudos schauen wir uns im Folgenden an.

Wir betrachten dazu im Folgenden einen einfachen (Spielzeug-)Datensatz,
an dem wir zunächst die Verben des Datenjudos vorstellen, s.
Tabelle~\ref{tbl-datenjudo}.

\begin{table}

\caption{\label{tbl-datenjudo}Ein einfacher Datensatz von schlichtem
Gemüt}

\centering{

\fontsize{12.0pt}{14.4pt}\selectfont
\begin{tabular*}{\linewidth}{@{\extracolsep{\fill}}rllr}
\toprule
id & name & gruppe & note \\ 
\midrule\addlinespace[2.5pt]
1 & Anni & A & 2.7 \\ 
2 & Berti & A & 2.7 \\ 
3 & Charli & B & 1.7 \\ 
\bottomrule
\end{tabular*}

}

\end{table}%

\begin{tcolorbox}[enhanced jigsaw, bottomtitle=1mm, leftrule=.75mm, breakable, title=\textcolor{quarto-callout-important-color}{\faExclamation}\hspace{0.5em}{Wichtig}, bottomrule=.15mm, titlerule=0mm, left=2mm, opacityback=0, colframe=quarto-callout-important-color-frame, rightrule=.15mm, colback=white, coltitle=black, toprule=.15mm, toptitle=1mm, colbacktitle=quarto-callout-important-color!10!white, arc=.35mm, opacitybacktitle=0.6]

Die Verben des Datenjudos wohnen im Paket \texttt{dplyr}, welches
gestartet wird, wenn Sie \texttt{library(tidyverse)} eingeben. Falls Sie
vergessen , das Paket \texttt{tidyverse} zu starten, dann funktionieren
diese Befehle nicht.\(\square\)

\end{tcolorbox}

\subsection{\texorpdfstring{Tabelle sortieren:
\texttt{arrange()}}{Tabelle sortieren: arrange()}}\label{tabelle-sortieren-arrange}

\emph{Sortieren} der Zeilen ist eine einfache, aber häufige Tätigkeit
des Datenjudos, s. Abbildung~\ref{fig-arrange}.

\begin{figure}[H]

\centering{

\includegraphics[width=0.7\linewidth,height=\textheight,keepaspectratio]{030-aufbereiten_files/figure-pdf/fig-arrange-1.pdf}

}

\caption{\label{fig-arrange}Sinnbild für das Sortieren einer Tabelle mit
\texttt{arrange()}}

\end{figure}%

\begin{example}[Was sind die höchsten
Preise?]\protect\hypertarget{exm-arrange1}{}\label{exm-arrange1}

Sie wollen mal locker anfangen. Daher stellen Sie sich folgende Frage:
Was sind denn eigentlich die höchsten Preise, für die das Spiel
\emph{Mariokart} über den Online-Ladentisch geht? Die Spalte des
Verkaufspreis heißt offenbar \texttt{total\_pr} (s. Datensatz
\texttt{mariokart}). In Excel kann die Spalte, nach der man die Tabelle
sortieren möchte, einfach anklicken. Ob das in R auch so einfach geht?
Die Funktion \texttt{arrange()} macht es uns ziemlich einfach, s.
tbl-arrange.

Die Funktion \texttt{arrange()} macht es uns ziemlich einfach, s.
tbl-arrange-pdf.

\begin{Shaded}
\begin{Highlighting}[]
\FunctionTok{arrange}\NormalTok{(mariokart, total\_pr) }
\end{Highlighting}
\end{Shaded}

\begin{longtable}[]{@{}rr@{}}

\caption{\label{tbl-arrange-pdf}Die Datentabelle, soritert nach
\texttt{total\_pr}}

\tabularnewline

\toprule\noalign{}
total\_pr & start\_pr \\
\midrule\noalign{}
\endhead
\bottomrule\noalign{}
\endlastfoot
29 & 0.99 \\
30 & 0.01 \\
31 & 0.99 \\
31 & 1.99 \\
31 & 30.00 \\
31 & 0.01 \\

\end{longtable}

Übersetzen wir die R-Syntax ins Deutsche:

\begin{verbatim}
Hey R,
arrangiere (sortiere) `mariokart` nach der Spalte `total_pr` (aufsteigend).
\end{verbatim}

Gar nicht so schwer.\(\square\)

\end{example}

Übrigens wird in \texttt{arrange()} per Voreinstellung aufsteigend
sortiert. Setzt man ein Minus vor der zu sortierenden Spalte, wird
umgekehrt, also \emph{absteigend} sortiert:

\begin{Shaded}
\begin{Highlighting}[]
\NormalTok{mario\_sortiert }\OtherTok{\textless{}{-}} \FunctionTok{arrange}\NormalTok{(mariokart, }\SpecialCharTok{{-}}\NormalTok{total\_pr)}
\end{Highlighting}
\end{Shaded}

\begin{exercise}[]\protect\hypertarget{exr-arrange2}{}\label{exr-arrange2}

Sortieren Sie die Mariokart-Daten absteigend nach der Anzahl der
beigelegten Lenkräder.\(\square\)

\end{exercise}

\subsection{\texorpdfstring{Zeilen filtern:
\texttt{filter()}}{Zeilen filtern: filter()}}\label{zeilen-filtern-filter}

\subsubsection{Nur bestimmte Zeilen
behalten}\label{nur-bestimmte-zeilen-behalten}

Zeilen \emph{filtern} bedeutet, dass man nur \emph{bestimmte}
\emph{Zeilen} (Beobachtungen) \emph{behalten} möchte, die restlichen
Zeilen brauchen wir nicht, weg mit ihnen. Wir haben also ein
Filterkriterium im Kopf, anhand dessen wir die Tabelle filern, s.
Abbildung~\ref{fig-filter}.

\begin{figure}[H]

\centering{

\includegraphics[width=0.7\linewidth,height=\textheight,keepaspectratio]{030-aufbereiten_files/figure-pdf/fig-filter-1.pdf}

}

\caption{\label{fig-filter}Sinnbild für das Filtern einer Tabelle mit
\texttt{filter()}}

\end{figure}%

\begin{example}[Ob ein Foto für den Verkaufspreis nützlich
ist?]\protect\hypertarget{exm-filter}{}\label{exm-filter}

Als nächstes kommt Ihnen die Idee, mal zu schauen, ob Auktionen mit
Photo der Ware einen höheren Verkaufspreis erzielen als Auktionen ohne
Photo.

\begin{Shaded}
\begin{Highlighting}[]
\NormalTok{mariokart\_neu }\OtherTok{\textless{}{-}} \FunctionTok{filter}\NormalTok{(mariokart, stock\_photo }\SpecialCharTok{==} \StringTok{"yes"}\NormalTok{)}
\end{Highlighting}
\end{Shaded}

Sie filtern also die Tabelle so, dass \emph{nur} diese Auktionen im
Datensatz verbleiben, welche mind. ein Photo haben, mit anderen Worten,
Auktionen (Beobachtungen) bei denen gilt:
\texttt{stock\_photo\ ==\ TRUE}.\(\square\)

\end{example}

\subsubsection{Komplexeres Filtern}\label{komplexeres-filtern}

Angestachelt von Ihren Erfolgen möchten Sie jetzt komplexere Hypothesen
prüfen: Ob wohl Auktionen von \emph{neuen} Spielen und zwar \emph{mit}
Photo einen höheren Preis erzielen als die übrigen Auktionen?

Anders gesagt haben Sie zwei Filterkriterien im Blick: Neuheit
\texttt{cond} und Photo \texttt{stock\_photo}. Nur diejenigen Auktionen,
die \emph{sowohl} Neuheit \emph{als auch} Photo erfüllen, möchten Sie
näher untersuchen (Filtern mit dem logischen UND):

\begin{Shaded}
\begin{Highlighting}[]
\NormalTok{mario\_filter1 }\OtherTok{\textless{}{-}} 
  \FunctionTok{filter}\NormalTok{(mariokart,  }\CommentTok{\# "\&" heißt UND:}
\NormalTok{         stock\_photo }\SpecialCharTok{==} \StringTok{"yes"} \SpecialCharTok{\&}\NormalTok{ cond }\SpecialCharTok{==} \StringTok{"new"}\NormalTok{)}
\end{Highlighting}
\end{Shaded}

Hm. Was ist mit den Auktionen, die \emph{entweder} über (mind.) ein
Photo verfügen \emph{oder auch} neu sind, oder beides (Filtern mit dem
logischen ODER)?

\begin{Shaded}
\begin{Highlighting}[]
\NormalTok{mario\_filter2 }\OtherTok{\textless{}{-}} 
  \FunctionTok{filter}\NormalTok{(mariokart,  }\CommentTok{\# "|" heißt ODER:}
\NormalTok{         stock\_photo }\SpecialCharTok{==} \StringTok{"yes"} \SpecialCharTok{|}\NormalTok{ cond }\SpecialCharTok{==} \StringTok{"new"}\NormalTok{)}
\end{Highlighting}
\end{Shaded}

Zur Erinnerung: Logische Operatoren sind in Kapitel~\ref{sec-logic}
erläutert.

\begin{exercise}[]\protect\hypertarget{exr-hyps-filter}{}\label{exr-hyps-filter}

Hier könnte man noch viele interessante Hypothesen prüfen, denken Sie
sich und tun das auch. \(\square\)

\end{exercise}

\begin{exercise}[]\protect\hypertarget{exr-filter2}{}\label{exr-filter2}

Filtern Sie die Spiele mit nur einem Lenkrad und ohne
Versandkosten.\(\square\)

\end{exercise}

\begin{exercise}[]\protect\hypertarget{exr-filter3}{}\label{exr-filter3}

Filtern Sie die Spiele mit nur einem Lenkrad, die einen
überdurchschnittlichen Verkaufspreis erzielen. Tipp: Nutzen Sie die
Funktion \texttt{describe\_distribution}, um den Mittelwert einer
Variable des Datensatzes zu erfahren (diese Funktion wohnt im R-Paket
\texttt{easystats}). \(\square\)

\end{exercise}

\subsection{\texorpdfstring{Spalten auswählen mit
\texttt{select()}}{Spalten auswählen mit select()}}\label{spalten-auswuxe4hlen-mit-select}

Eine Tabelle mit vielen Spalten kann schnell unübersichtlich werden. Da
lohnt es sich, eine alte goldene Regel zu beachten: Mache die Dinge so
einfach wie möglich, aber nicht einfacher. Wählen wir also \emph{nur}
die Spalten aus, die uns interessieren und entfernen wir die restlichen,
s. Abbildung~\ref{fig-select}.

\begin{figure}[H]

\centering{

\includegraphics[width=0.7\linewidth,height=\textheight,keepaspectratio]{030-aufbereiten_files/figure-pdf/fig-select-1.pdf}

}

\caption{\label{fig-select}Sinnbild für das Auswählen von Spalten mit
\texttt{select()}}

\end{figure}%

\begin{example}[Fokus auf nur zwei
Spalten]\protect\hypertarget{exm-select}{}\label{exm-select}

Ob wohl gebrauchte Spiele deutlich geringere Preise erzielen im
Vergleich zu neuwertigen Spielen? Sie entschließen sich, mal ein
Stündchen auf die relevanten Daten zu starren.

\begin{Shaded}
\begin{Highlighting}[]
\NormalTok{mario\_select1 }\OtherTok{\textless{}{-}} \FunctionTok{select}\NormalTok{(mariokart, cond, total\_pr)}
\end{Highlighting}
\end{Shaded}

Aha (?)\(\square\)

\end{example}

Der Befehl \texttt{select()} erwartet als Input eine Tabelle und gibt
(als Output) eine Tabelle zurück -- genau wie die meisten anderen
Befehle des Datenjudos. Auch wenn Sie nur eine Spalte auswählen, bleibt
es eine Tabelle, eben eine Tabelle mit nur einer Spalte.

\texttt{select()} erlaubt Komfort; Sie können Spalten auf mehrere Arten
auswählen, z.B.

\begin{Shaded}
\begin{Highlighting}[]
\FunctionTok{select}\NormalTok{(mariokart, }\DecValTok{1}\NormalTok{, }\DecValTok{2}\NormalTok{)  }\CommentTok{\# Spalte 1 und 2}
\FunctionTok{select}\NormalTok{(mariokart, }\DecValTok{2}\SpecialCharTok{:}\DecValTok{5}\NormalTok{)  }\CommentTok{\#  Spalten 2 *bis* 5 }
\FunctionTok{select}\NormalTok{(mariokart, }\SpecialCharTok{{-}}\DecValTok{1}\NormalTok{)  }\CommentTok{\# Alle Spalte *außer* Spalte 1}
\end{Highlighting}
\end{Shaded}

\begin{exercise}[]\protect\hypertarget{exr-select}{}\label{exr-select}

Wählen Sie die Spalten \texttt{total\_pr}, \texttt{cond} sowie die
zweite Spalte der Tabelle \texttt{mariokart} aus!\footnote{\texttt{select(mariokart,\ total\_pr,\ cond,\ 2)}}
\(\square\)

\end{exercise}

Vertiefte Informationen zum Auswählen von Spalten mit \texttt{select()}
findet sich
\href{https://tidyr.tidyverse.org/reference/tidyr_tidy_select.html}{hier}.\footnote{\url{https://tidyr.tidyverse.org/reference/tidyr_tidy_select.html}}

\subsection{\texorpdfstring{Spalten zu einer Zahl zusammenfassen mit
\texttt{summarise()}}{Spalten zu einer Zahl zusammenfassen mit summarise()}}\label{spalten-zu-einer-zahl-zusammenfassen-mit-summarise}

So eine lange Spalte mit Zahlen -- mal ehrlich: wer blickt da schon
durch? Viel besser wäre es doch, die Spalte \texttt{total\_pr} zu einer
Zahl zusammenzufassen, das ist doch viel handlicher. Kurz entschlossen
fassen Sie die Spalte \texttt{total\_pr}, den Verkaufspreis, zum
Mittelwert zusammen, s. Abbildung~\ref{fig-summarise}.

\begin{figure}[H]

\centering{

\includegraphics[width=0.7\linewidth,height=\textheight,keepaspectratio]{030-aufbereiten_files/figure-pdf/fig-summarise-1.pdf}

}

\caption{\label{fig-summarise}Spalten zu einer einzelnen Zahl
zusammenfassen mit \texttt{summarise()}}

\end{figure}%

\begin{example}[Was ist der mittlere
Verkaufspreis?]\protect\hypertarget{exm-summarise}{}\label{exm-summarise}

Mit \texttt{summarise()}, s. Listing~\ref{lst-summarise}, können wir den
mittleren Verkaufspreis der Mariokart-Spiele berechnen.

\begin{codelisting}

\caption{\label{lst-summarise}Die R-Funktion summarise fasst einen
Vektor z u einer Zahl zusammen}

\centering{

\begin{Shaded}
\begin{Highlighting}[]
\NormalTok{mariokart\_mittelwert }\OtherTok{\textless{}{-}} \FunctionTok{summarise}\NormalTok{(mariokart,}
                                  \AttributeTok{preis\_mw =} \FunctionTok{mean}\NormalTok{(total\_pr))}
\NormalTok{mariokart\_mittelwert}
\end{Highlighting}
\end{Shaded}

}

\end{codelisting}%

\begin{longtable}[]{@{}r@{}}
\toprule\noalign{}
preis\_mw \\
\midrule\noalign{}
\endhead
\bottomrule\noalign{}
\endlastfoot
50 \\
\end{longtable}

Aha! Etwa 50€ erzielt so eine Auktion im Schnitt.\(\square\)

\end{example}

Übersetzen wir Listing~\ref{lst-summarise} vom Errischen ins Deutsche:

\begin{quote}
{\emoji{student}} Hey R, fasse die Zeilen von \texttt{total\_pr} aus
\texttt{mariokart} zu einer Zahl zusammen, und zwar mit Hilfe des
Mittelwerts. Die resultierende Tabelle nennen wir
\texttt{mariokart\_mittelwert}, sehr kreativ. Und die resultierende
Spalte, die einzige in \texttt{mariokart\_mittelwert}, nennen wir
\texttt{preis\_mw}.
\end{quote}

Ein bisschen abstrakter gesprochen, fasst \texttt{summarise()} also eine
\emph{Spalte} zu einer (einzelnen) \emph{Zahl} zusammen, s.
Gleichung~\ref{eq-desk-summ}.

\begin{tcolorbox}[enhanced jigsaw, bottomtitle=1mm, leftrule=.75mm, breakable, title=\textcolor{quarto-callout-note-color}{\faInfo}\hspace{0.5em}{Hinweis}, bottomrule=.15mm, titlerule=0mm, left=2mm, opacityback=0, colframe=quarto-callout-note-color-frame, rightrule=.15mm, colback=white, coltitle=black, toprule=.15mm, toptitle=1mm, colbacktitle=quarto-callout-note-color!10!white, arc=.35mm, opacitybacktitle=0.6]

Eine Alternative, um eine Spalte zu einer Zahl zusammenzufassen, bietet
der \enquote{Dollar-Operator} (\texttt{\$}):
\texttt{mean(mariokart\$total\_pr)}. Der Dollar-Operator trennt hier die
Tabelle von der Spalte: \texttt{tibble\$spalte}. Im Gegensatz zu den
Verben des Tidyverse (die immer einer Tabelle zurückliefern), liefert
der Dollar-Operator einen Vektor (Spalte) zurück. (Diese wird von
\texttt{mean} dann zu einer einzelnen Zahl zusammengefasst.) \(\square\)

\end{tcolorbox}

\emph{Auf welche Art} zusammengefasst werden soll, z.B. anhand des
Mittelwerts oder Maximalwerts, muss noch zusätzlich innerhalb von
\texttt{summarise()} angegeben werden.

\begin{equation}\phantomsection\label{eq-desk-summ}{\begin{array}{|c|} \hline \\ \hline \\  \\  \\ \\ \hline \end{array} \qquad \rightarrow  \qquad \begin{array}{|c|} \hline \\  \hline \end{array}}\end{equation}

\begin{exercise}[]\protect\hypertarget{exr-summarise}{}\label{exr-summarise}

Identifizieren Sie den höchsten Kaufpreis eines
Mariokart-Spiels!\footnote{\texttt{summarise(mariokart,\ hoechster\_preis\ =\ max(total\_pr))}}
\(\square\)

\end{exercise}

\subsection{Tabelle gruppieren}\label{tabelle-gruppieren}

Es ist ja gut und schön, zu wissen, was so ein Spiel im Schnitt kostet.
Aber viel interessanter wäre es doch, denken Sie sich, zu wissen, ob die
neuen Spiele im Schnitt mehr kosten als die alten? Ob R Ihnen so etwas
ausrechnen kann?

\begin{quote}
{\emoji{robot}} Ich tue fast alles für dich. {\emoji{heart}}
\end{quote}

Also gut, R, dann gruppiere die Tabelle, s. Abbildung~\ref{fig-group}.

\begin{figure}[H]

\centering{

\includegraphics[width=0.7\linewidth,height=\textheight,keepaspectratio]{030-aufbereiten_files/figure-pdf/fig-group-1.pdf}

}

\caption{\label{fig-group}Gruppieren von Datensätzen mit
\texttt{group\_by()}}

\end{figure}%

Durch das Gruppieren wird die Tabelle in \enquote{Teiltabellen} --
entsprechend der Gruppen -- aufgeteilt. Das sieht man der R-Tabelle aber
nicht wirklich an. Aber alle nachfolgenden Berechnungen werden \emph{für
jede Teiltabelle} einzeln ausgeführt.

\begin{example}[Mittlerer Preis pro
Gruppe]\protect\hypertarget{exm-groupby}{}\label{exm-groupby}

Gruppieren alleine liefert Ihnen zwei (oder mehrere) Teiltabellen, etwa
neue Spiele (Gruppe 1, \texttt{new}) vs.~gebrauchte Spiele (Gruppe 2,
\texttt{used}). Mit anderen Worten: Wir gruppieren anhand der Variable
\texttt{cond}.

\begin{Shaded}
\begin{Highlighting}[]
\NormalTok{mariokart\_gruppiert }\OtherTok{\textless{}{-}} \FunctionTok{group\_by}\NormalTok{(mariokart, cond)}
\end{Highlighting}
\end{Shaded}

Wenn Sie die neue Tabelle betrachte, sehen Sie wenig Aufregendes, nur
einen Hinweis, dass die Tabelle gruppiert ist. Jetzt können Sie an jeder
Teiltabelle Ihre weiteren Berechnungen vornehmen, etwa die Berechnung
des mittleren Verkaufspreises.

\begin{Shaded}
\begin{Highlighting}[]
\FunctionTok{summarise}\NormalTok{(mariokart\_gruppiert, }\AttributeTok{preis\_mw =} \FunctionTok{mean}\NormalTok{(total\_pr))}
\end{Highlighting}
\end{Shaded}

\begin{longtable}[]{@{}lr@{}}
\toprule\noalign{}
cond & preis\_mw \\
\midrule\noalign{}
\endhead
\bottomrule\noalign{}
\endlastfoot
new & 54 \\
used & 47 \\
\end{longtable}

Langsam fühlen Sie sich als Datenchecker \ldots{} \(\square\)

\end{example}

\begin{exercise}[]\protect\hypertarget{exr-groupby}{}\label{exr-groupby}

~

\textbf{Aufgabe}

Berechnen Sie den mittleren und maximalen Verkaufspreis getrennt für
Spiele mit und ohne Foto!

\textbf{Lösung}

\begin{Shaded}
\begin{Highlighting}[]
\NormalTok{mariokart\_gruppiert\_foto }\OtherTok{\textless{}{-}} \FunctionTok{group\_by}\NormalTok{(mariokart, stock\_photo)}

\NormalTok{mariokart\_verkaufspreis\_foto }\OtherTok{\textless{}{-}} 
  \FunctionTok{summarise}\NormalTok{(mariokart\_gruppiert\_foto,}
            \AttributeTok{total\_pr\_avg =} \FunctionTok{mean}\NormalTok{(total\_pr),}
            \AttributeTok{total\_pr\_max =} \FunctionTok{max}\NormalTok{(total\_pr))}

\NormalTok{mariokart\_verkaufspreis\_foto}
\end{Highlighting}
\end{Shaded}

\begin{longtable}[]{@{}lrr@{}}
\toprule\noalign{}
stock\_photo & total\_pr\_avg & total\_pr\_max \\
\midrule\noalign{}
\endhead
\bottomrule\noalign{}
\endlastfoot
no & 54 & 327 \\
yes & 48 & 75 \\
\end{longtable}

\end{exercise}

\subsection{\texorpdfstring{Spalten verändern mit
\texttt{mutate()}}{Spalten verändern mit mutate()}}\label{spalten-veruxe4ndern-mit-mutate}

Immer mal wieder möchte man \emph{Spalten verändern}, bzw. deren Werte
umrechnen, s. Abbildung~\ref{fig-mutate}.

\begin{figure}[H]

\centering{

\includegraphics[width=0.7\linewidth,height=\textheight,keepaspectratio]{030-aufbereiten_files/figure-pdf/fig-mutate-1.pdf}

}

\caption{\label{fig-mutate}Spalten verändern/neu berechnen mit
\texttt{mutate()}}

\end{figure}%

\begin{example}[]\protect\hypertarget{exm-mutate}{}\label{exm-mutate}

Der Hersteller des Computerspiels \emph{Mariokart} kommt aus Japan;
daher erscheint es Ihnen opportun für ein anstehendes Meeting mit dem
Hersteller die Verkaufspreise von Dollar in japanische Yen umzurechnen.
Nach etwas Googeln finden Sie einen Umrechnungskurs von 1:133.

\begin{Shaded}
\begin{Highlighting}[]
\NormalTok{mariokart\_yen }\OtherTok{\textless{}{-}} 
  \FunctionTok{mutate}\NormalTok{(mariokart, }\AttributeTok{total\_pr\_yen =}\NormalTok{ total\_pr }\SpecialCharTok{*} \DecValTok{133}\NormalTok{)}
\NormalTok{mariokart\_yen }\OtherTok{\textless{}{-}} \FunctionTok{select}\NormalTok{(mariokart\_yen, total\_pr\_yen, total\_pr)}
\NormalTok{mariokart\_yen }\SpecialCharTok{|\textgreater{}} \FunctionTok{head}\NormalTok{()  }\CommentTok{\# nur die ersten paar Zeilen}
\end{Highlighting}
\end{Shaded}

\begin{longtable}[]{@{}rr@{}}
\toprule\noalign{}
total\_pr\_yen & total\_pr \\
\midrule\noalign{}
\endhead
\bottomrule\noalign{}
\endlastfoot
6856 & 52 \\
4926 & 37 \\
6052 & 46 \\
5852 & 44 \\
9443 & 71 \\
5985 & 45 \\
\end{longtable}

Sicherlich werden Sie Ihre Gesprächspartner schwer
beeindrucken.\(\square\)

\end{example}

Mit \texttt{mutate()} berechnen Sie eine Spalte \texttt{x} (in einer
Tabelle) neu. Die Funktion, die Sie in \texttt{mutate()} benennen wird
für jede Zeile der Spalte \texttt{x} angewendet.

\begin{example}[Beispiele für Funktionen für
\texttt{mutate()}]\protect\hypertarget{exm-mutate2}{}\label{exm-mutate2}

\texttt{mutate()} eignet sich, z.B. um Spalten zu addieren, zu
multiplizieren oder sonst wie zu transformieren (z.B. den Logarithmus
anwenden oder den Mittelwert der Spalte von jeder Zeile abziehen).
\(\square\)

\end{example}

\begin{quote}
{\emoji{woman-zombie}}️ Statistik, wann braucht man schon sowas!?
\end{quote}

\begin{quote}
{\emoji{robot}} Eigentlich nur dann, wenn man die Fakten gut verstehen
will, sonst nicht.
\end{quote}

\subsection{\texorpdfstring{Zeilen zählen mit
\texttt{count()}}{Zeilen zählen mit count()}}\label{zeilen-zuxe4hlen-mit-count}

Arbeitet man mit nominalskalierten Daten, ist (fast) alles, was man tun
kann, die entsprechenden Zeilen der Tabelle zu zählen: Man könnte z.B.
fragen, wie viele neue und wie viele alte Spiele in der Tabelle
(Dataframe) \texttt{mariokart} vorhanden sind.

\begin{example}[]\protect\hypertarget{exm-count}{}\label{exm-count}

Nach der letzten Präsentation Ihrer Analyse hat Ihre Chefin gestöhnt:
\enquote{Oh nein, alles so kompliziert. Statistik! Himmel hilf! Kann man
das nicht einfacher machen?} Anstelle von irgendwelchen komplizierten
Berechnungen (Mittelwert?) möchten Sie ihr beim nächsten Treffen nur
zeigen, wie viele Computerspiele neu und wie viele gebraucht sind (in
Ihrem Datensatz). Schlichte Häufigkeiten also. Hoffentlich ist Ihre
Chefin nicht wieder überfordert\ldots{}

\begin{Shaded}
\begin{Highlighting}[]
\NormalTok{mariocart\_counted }\OtherTok{\textless{}{-}} \FunctionTok{count}\NormalTok{(mariokart, cond)}
\NormalTok{mariocart\_counted}
\end{Highlighting}
\end{Shaded}

\begin{longtable}[]{@{}lr@{}}
\toprule\noalign{}
cond & n \\
\midrule\noalign{}
\endhead
\bottomrule\noalign{}
\endlastfoot
new & 59 \\
used & 84 \\
\end{longtable}

Aha! Es gibt mehr gebrauchte als neue Spiele.\(\square\)

\end{example}

Jetzt könnte man noch den \emph{Anteil} (engl. \emph{proportion})
ergänzen: Welcher \emph{Anteil} (der 143 Spiele in \texttt{mariokart})
ist neu, welcher gebraucht?

\begin{Shaded}
\begin{Highlighting}[]
\FunctionTok{mutate}\NormalTok{(mariocart\_counted, }\AttributeTok{Anteil =}\NormalTok{ n }\SpecialCharTok{/} \FunctionTok{sum}\NormalTok{(n))}
\end{Highlighting}
\end{Shaded}

\begin{longtable}[]{@{}lrr@{}}
\toprule\noalign{}
cond & n & Anteil \\
\midrule\noalign{}
\endhead
\bottomrule\noalign{}
\endlastfoot
new & 59 & 0.41 \\
used & 84 & 0.59 \\
\end{longtable}

\begin{exercise}[]\protect\hypertarget{exr-count}{}\label{exr-count}

Zählen Sie Sie, wie viele Auktionen ein Foto enthalten.\footnote{\texttt{count(mariokart,\ stock\_photo)}}
\(\square\)

\end{exercise}

\begin{exercise}[]\protect\hypertarget{exr-count2}{}\label{exr-count2}

Zählen Sie Sie, wie viele Auktionen ein Foto enthalten -- innerhalb der
gebrauchten Spiele und innerhalb der neuen Spiele. Anders gesagt: Teilen
Sie den Datensatz sowohl nach Zustand als auch nach Foto auf und zählen
Sie jeweils, wie viele Spiele/Auktionen in die jeweilige Gruppe
gehören.\footnote{\texttt{count(mariokart,\ stock\_photo,\ cond)}}
\(\square\)

\end{exercise}

\subsection{Fazit: Verben am
Fließband}\label{fazit-verben-am-flieuxdfband}

Die Befehle (\enquote{Verben}) des Tidyverse sind jeweils für einzelne,
typische Aufgaben des Datenaufbereitens (\enquote{Datenjudo}) zuständig.
Typischerweise erwarten diese Befehle eine Tabelle (\faIcon{table}) als
Input und liefern eine Tabelle aus Output zurück, s.
Abbildung~\ref{fig-tbl-in-out}.

\begin{figure}[H]

\centering{

\includegraphics[width=4in,height=0.63in]{030-aufbereiten_files/figure-latex/mermaid-figure-1.png}

}

\caption{\label{fig-tbl-in-out}Tidyverse-Befehle erwarten normalerweise
eine Tabelle (tibble) als Input und geben auch eine Tabelle zurück als
Output}

\end{figure}%

\section{Die Pfeife}\label{sec-pipe}

\href{https://en.wikipedia.org/wiki/The_Treachery_of_Images}{Das ist
keine Pfeife}, wie René Magritte 1929 in seinem
\href{https://en.wikipedia.org/wiki/File:MagrittePipe.jpg}{berühmten
Bild} schrieb, s. Abbildung~\ref{fig-pfeifen}.

\begin{figure}[H]

\centering{

\begin{figure}[H]

\begin{minipage}{0.42\linewidth}
\pandocbounded{\includegraphics[keepaspectratio]{img/800px-Pipa_savinelli.jpg}}\end{minipage}%
%
\begin{minipage}{0.08\linewidth}

\end{minipage}%
%
\begin{minipage}{0.25\linewidth}

\%\textgreater\%

\end{minipage}%
%
\begin{minipage}{0.25\linewidth}

\textbar\textgreater{}

\end{minipage}%

\end{figure}%

}

\caption{\label{fig-pfeifen}So sieht die Pfeife in R aus (Jaja, das ist
keine Pfeife, sondern ein Symbol einer Pfeife\ldots). Links: Ein
\emph{Bild} einer Pfeife (M7, 2004). Mitte und Rechts: Die zwei
R-Symbole für eine \enquote{Pfeife} (pipe).}

\end{figure}%

\subsection{Russische Puppen}\label{russische-puppen}

Computerbefehle, und im Speziellen R-Befehle kann man
\enquote{aufeinander} -- oder vielmehr: ineinander -- stapeln, so
ähnlich wie eine russische Puppe (vgl. Kapitel~\ref{sec-first-fun}).
Schauen wir uns das in einem Beispiel an. Dazu definieren wir zuerst
einen Vektor \texttt{x} aus drei Zahlen:

\begin{Shaded}
\begin{Highlighting}[]
\NormalTok{x }\OtherTok{\textless{}{-}} \FunctionTok{c}\NormalTok{(}\DecValTok{1}\NormalTok{, }\DecValTok{2}\NormalTok{, }\DecValTok{3}\NormalTok{)}
\end{Highlighting}
\end{Shaded}

Und dann kommt unser verschachtelter Befehl:

\begin{Shaded}
\begin{Highlighting}[]
\FunctionTok{sum}\NormalTok{(x }\SpecialCharTok{{-}} \FunctionTok{mean}\NormalTok{(x))}
\DocumentationTok{\#\# [1] 0}
\end{Highlighting}
\end{Shaded}

Wie schon erwähnt, arbeitet R so einen \enquote{verschachtelten} Befehl
\emph{von innen nach außen} ab:

Start: \texttt{sum(x\ -\ mean(x))}

{\(\downarrow\)}

Schritt 1: \texttt{sum(x\ -\ 2)}

{\(\downarrow\)}

Schritt 2: \texttt{sum(-1,\ 0,\ 1)}

{\(\downarrow\)}

Schritt 3: \texttt{0}. Fertig. Puh. Kompliziert.

Soweit kann man noch einigermaßen folgen. Aber das Verschachteln kann
man noch extremer machen, dann wird's wild. Schauen Sie sich mal
folgende (Pseudo-)Syntax an:

\begin{codelisting}

\caption{\label{lst-schachtel}Eine wild verschachtelte Sequenz von
R-Befehlen}

\centering{

\begin{Shaded}
\begin{Highlighting}[]
\FunctionTok{fasse\_zusammen}\NormalTok{(}
  \FunctionTok{gruppiere}\NormalTok{(}
\NormalTok{    wähle}\FunctionTok{\_spalten}\NormalTok{(}
      \FunctionTok{filter\_zeilen}\NormalTok{(meine\_daten))))}
\end{Highlighting}
\end{Shaded}

}

\end{codelisting}%

\begin{tcolorbox}[enhanced jigsaw, bottomtitle=1mm, leftrule=.75mm, breakable, title=\textcolor{quarto-callout-caution-color}{\faFire}\hspace{0.5em}{Vorsicht}, bottomrule=.15mm, titlerule=0mm, left=2mm, opacityback=0, colframe=quarto-callout-caution-color-frame, rightrule=.15mm, colback=white, coltitle=black, toprule=.15mm, toptitle=1mm, colbacktitle=quarto-callout-caution-color!10!white, arc=.35mm, opacitybacktitle=0.6]

Ein beliebter Fehler ist es übrigens, nicht die richtige Zahl an
schließenden Klammern hinzuschreiben, z.B.
\texttt{d(c(b(a(meine\_daten))} FALSCHE ZAHL AN KLAMMERN. \(\square\)

\end{tcolorbox}

\subsection{Die Pfeife zur Rettung}\label{die-pfeife-zur-rettung}

Listing~\ref{lst-schachtel} ist schon harter Tobak, was für echte Fans.
Wäre es nicht einfacher, man könnte Listing~\ref{lst-schachtel} wie
folgt schreiben:

\begin{verbatim}
Nimm "meine_daten" *und dann*
  filter gewünschte Zeilen *und dann*
  wähle gewünschte Spalten *und dann*
  teile in Subgruppen *und dann*
  fasse sie zusammen.
\end{verbatim}

\begin{definition}[Pfeife]\protect\hypertarget{def-pipe}{}\label{def-pipe}

\enquote{Und dann} heißt auf Errisch \texttt{\%\textgreater{}\%} oder
(synonym) \texttt{\textbar{}\textgreater{}}. Man nennt diesen Befehl
\enquote{Pfeife} (engl. \emph{pipe}). \(\square\)

\end{definition}

\begin{tcolorbox}[enhanced jigsaw, bottomtitle=1mm, leftrule=.75mm, breakable, title=\textcolor{quarto-callout-note-color}{\faInfo}\hspace{0.5em}{Hinweis}, bottomrule=.15mm, titlerule=0mm, left=2mm, opacityback=0, colframe=quarto-callout-note-color-frame, rightrule=.15mm, colback=white, coltitle=black, toprule=.15mm, toptitle=1mm, colbacktitle=quarto-callout-note-color!10!white, arc=.35mm, opacitybacktitle=0.6]

Der Befehl \texttt{\%\textgreater{}\%} \emph{verknüpft} Befehle. Der
Shortcut für diesen Befehl ist Strg-Shift-M. Die Pfeife
\texttt{\%\textgreater{}\%} \enquote{wohnt} im Paket
\texttt{tidyverse}.\footnotemark{}

\end{tcolorbox}

\footnotetext{Genauer gesagt im Paket \texttt{magrittr}, welches aber
von \texttt{tidyverse} geladen wird. Also nichts, um das Sie sich
kümmern müssten.}

Mittlerweile (Seit R 4.1) ist auch im Standard-R eine Pfeife eingebaut.
die sieht so aus: \texttt{\textbar{}\textgreater{}}. Die eingebaute
Pfeife funktioniert praktisch gleich zur anderen Pfeife
\texttt{\%\textgreater{}\%}, hat aber den Vorteil, dass Sie nicht
\texttt{tidyverse} starten müssen. Da wir \texttt{tidyverse} aber
sowieso praktisch immer starten werden, bringt es uns keinen Vorteil,
die neuere Pfeife des Standard-R \texttt{\textbar{}\textgreater{}} zu
verwenden. Aber auch keinen Nachteil. Unter \emph{Tools \textgreater{}
Global Options\ldots{}} können Sie einstellen, dass der Shortcut
Strg-Shift-M die eingebaute Pfeife verwendet.

\begin{figure}[H]

\centering{

\includegraphics[width=1.15in,height=3.5in]{030-aufbereiten_files/figure-latex/mermaid-figure-2.png}

}

\caption{\label{fig-pfeife}Illustration für eine Pfeifensequenz, es geht
vorwärts wie am Fließband.}

\end{figure}%

Und jetzt kommt's: So eine Art von Befehls-Verkettung gibt es in R.
Schauen Sie sich mal Listing~\ref{lst-pfeife} an.

\begin{codelisting}

\caption{\label{lst-pfeife}Eine Pfeifen-Befehlssequenz (Pseudo-Syntax)}

\centering{

\begin{Shaded}
\begin{Highlighting}[]
\NormalTok{meine\_daten }\SpecialCharTok{\%\textgreater{}\%}
\NormalTok{  filter\_gewünschte}\FunctionTok{\_zeilen}\NormalTok{() }\SpecialCharTok{\%\textgreater{}\%}
\NormalTok{  wähle\_gewünschte}\FunctionTok{\_spalten}\NormalTok{() }\SpecialCharTok{\%\textgreater{}\%}
  \FunctionTok{gruppiere}\NormalTok{() }\SpecialCharTok{\%\textgreater{}\%}
  \FunctionTok{fasse\_zusammen}\NormalTok{() }
\end{Highlighting}
\end{Shaded}

}

\end{codelisting}%

So eine Pfeifen-Befehlsequenz ist ein wie ein Fließband, an dem es
mehrere Arbeitsstationen gibt, s. Abbildung~\ref{fig-pfeife}. Unser
Datensatz wird am Fließband von Station zu Station weitergereicht und an
jeder Stelle weiterverarbeitet.

So könnte Ihre \enquote{Pfeifen-Sequenz} aussehen:

\begin{Shaded}
\begin{Highlighting}[]
\CommentTok{\# Hey R, nimm die Tabelle "mariokart":}
\NormalTok{mariokart }\SpecialCharTok{\%\textgreater{}\%}  
   \CommentTok{\# filter nur die günstigen Spiele:}
  \FunctionTok{filter}\NormalTok{(total\_pr }\SpecialCharTok{\textless{}} \DecValTok{100}\NormalTok{) }\SpecialCharTok{\%\textgreater{}\%} 
  \CommentTok{\# wähle die zwei Spalten:}
  \FunctionTok{select}\NormalTok{(cond, total\_pr) }\SpecialCharTok{\%\textgreater{}\%}  
  \CommentTok{\# gruppiere die Tabelle nach Zustand des Spiels:}
  \FunctionTok{group\_by}\NormalTok{(cond) }\SpecialCharTok{\%\textgreater{}\%}  
  \CommentTok{\# fasse beide Gruppen nach dem mittleren Preis zusammen:}
  \FunctionTok{summarise}\NormalTok{(}\AttributeTok{total\_pr\_mean =} \FunctionTok{mean}\NormalTok{(total\_pr))  }
\end{Highlighting}
\end{Shaded}

\begin{longtable}[]{@{}lr@{}}
\toprule\noalign{}
cond & total\_pr\_mean \\
\midrule\noalign{}
\endhead
\bottomrule\noalign{}
\endlastfoot
new & 54 \\
used & 43 \\
\end{longtable}

\begin{tcolorbox}[enhanced jigsaw, bottomtitle=1mm, leftrule=.75mm, breakable, title=\textcolor{quarto-callout-important-color}{\faExclamation}\hspace{0.5em}{Wichtig}, bottomrule=.15mm, titlerule=0mm, left=2mm, opacityback=0, colframe=quarto-callout-important-color-frame, rightrule=.15mm, colback=white, coltitle=black, toprule=.15mm, toptitle=1mm, colbacktitle=quarto-callout-important-color!10!white, arc=.35mm, opacitybacktitle=0.6]

Die Syntax \texttt{filter(mariokart,\ total\_pr\ \textless{}\ 100)} und
die Syntax
\texttt{mariokart\ \textbar{}\textgreater{}\ filter(total\_pr\ \textless{}\ 100)}
sind identisch.

Allgemeiner: \texttt{d\ \textbar{}\textgreater{}\ f(x)} =
\texttt{f(d,\ x)}.

\end{tcolorbox}

\section{Beispiele für
Forschungsfragen}\label{beispiele-fuxfcr-forschungsfragen}

\begin{exercise}[]\protect\hypertarget{exr-fallbsps}{}\label{exr-fallbsps}

Bevor Sie die Lösungen der folgenden Fallbeispiele lesen, versuchen Sie
die Aufgaben selber zu lösen. Ja, ich weiß, es ist hart, nicht gleich
auf die Lösungen zu schauen! \(\square\)

\end{exercise}

Sie arbeiten als \st{Diener} strategischer Assistent der
Geschäftsführerin und sind für Faktenchecks und andere Daten-Aufgaben
zuständig. Heute sollen Sie zeigen, was Sie können (Schluck).

\begin{exercise}[Das teuerste
Spiel?]\protect\hypertarget{exr-fofrage1}{}\label{exr-fofrage1}

~

\begin{quote}
{\emoji{woman}} Ich würde von Ihnen gerne wissen, was das teuerste Spiel
ist, aber jeweils für neue und gebrauchte Spiele. Aber nur für Spiele,
die mit Foto verkauft wurden!
\end{quote}

\textbf{Antwort}

\begin{Shaded}
\begin{Highlighting}[]
\NormalTok{mariokart }\SpecialCharTok{\%\textgreater{}\%} 
  \FunctionTok{filter}\NormalTok{(stock\_photo }\SpecialCharTok{==} \StringTok{"yes"}\NormalTok{) }\SpecialCharTok{\%\textgreater{}\%} 
  \FunctionTok{group\_by}\NormalTok{(cond) }\SpecialCharTok{\%\textgreater{}\%} 
  \FunctionTok{summarise}\NormalTok{(}\AttributeTok{total\_pr\_max =} \FunctionTok{max}\NormalTok{(total\_pr))}
\end{Highlighting}
\end{Shaded}

\begin{longtable}[]{@{}lr@{}}
\toprule\noalign{}
cond & total\_pr\_max \\
\midrule\noalign{}
\endhead
\bottomrule\noalign{}
\endlastfoot
new & 75 \\
used & 62 \\
\end{longtable}

Die Funktion \texttt{max} liefert den größten Wert eines Vektors zurück:

\begin{Shaded}
\begin{Highlighting}[]
\NormalTok{x }\OtherTok{\textless{}{-}} \FunctionTok{c}\NormalTok{(}\DecValTok{1}\NormalTok{, }\DecValTok{2}\NormalTok{, }\DecValTok{10}\NormalTok{)}
\FunctionTok{max}\NormalTok{(x)}
\DocumentationTok{\#\# [1] 10}
\end{Highlighting}
\end{Shaded}

\end{exercise}

\begin{exercise}[Die mittlere
Versandpauschale?]\protect\hypertarget{exr-Forschungsfrage2}{}\label{exr-Forschungsfrage2}

~

\begin{quote}
{\emoji{woman}} Ich würde gerne die mittlere Versandpauschale wissen,
aber getrennt nach Anzahl der Lenkräder, die dem Spiel beigelegt sind.
Und ich will nur Gruppen berücksichtigen, die aus mindestens 10 Spielen
bestehen!
\end{quote}

\textbf{Antwort}

Wenn wir die Anzahl der Spiele zählen in Abhängigkeit der beigelegten
Lenkräder (\texttt{wheels}), bekommen wir eine Tabelle mit zwei Spalten:
\texttt{wheels} und \texttt{n}. \texttt{n} zählt, wie viele Spiele
(Zeilen) in der jeweiligen Gruppe (\enquote{Teiltabelle}) von
\texttt{wheels} sind.

\begin{Shaded}
\begin{Highlighting}[]
\NormalTok{mariokart }\SpecialCharTok{\%\textgreater{}\%}
  \FunctionTok{count}\NormalTok{(wheels)}
\end{Highlighting}
\end{Shaded}

\begin{longtable}[]{@{}rr@{}}
\toprule\noalign{}
wheels & n \\
\midrule\noalign{}
\endhead
\bottomrule\noalign{}
\endlastfoot
0 & 37 \\
1 & 52 \\
2 & 51 \\
3 & 2 \\
4 & 1 \\
\end{longtable}

Aus dieser Tabellet sehen wir, dass 3 oder 4 Lenkräder nur selten (2
bzw. 1 Mal) beigelegt wurden und wir solche Spiele herausfiltern
sollten, bevor wir den Mittelwert der Versandkosten ausrechnen:

\begin{Shaded}
\begin{Highlighting}[]
\NormalTok{mariokart }\SpecialCharTok{\%\textgreater{}\%}
  \FunctionTok{filter}\NormalTok{(wheels }\SpecialCharTok{\textless{}} \DecValTok{3}\NormalTok{) }\SpecialCharTok{\%\textgreater{}\%} 
  \FunctionTok{group\_by}\NormalTok{(wheels) }\SpecialCharTok{\%\textgreater{}\%} 
  \FunctionTok{summarise}\NormalTok{(}\AttributeTok{mittlere\_versandkosten =} \FunctionTok{mean}\NormalTok{(ship\_pr),}
            \AttributeTok{anzahl\_spiele =} \FunctionTok{n}\NormalTok{())}
\end{Highlighting}
\end{Shaded}

\begin{longtable}[]{@{}rrr@{}}
\toprule\noalign{}
wheels & mittlere\_versandkosten & anzahl\_spiele \\
\midrule\noalign{}
\endhead
\bottomrule\noalign{}
\endlastfoot
0 & 2.7 & 37 \\
1 & 3.6 & 52 \\
2 & 2.9 & 51 \\
\end{longtable}

Die Funktion \texttt{n()} gibt die Anzahl der Zeilen pro Teiltabelle
zurück.

\end{exercise}

\begin{exercise}[Verkaufspreis in
Yen?]\protect\hypertarget{exr-Forschungsfrage3}{}\label{exr-Forschungsfrage3}

~

\begin{quote}
{\emoji{woman}} Ich würde gerne den Verkaufspreis in Yen wissen, nicht
in Euro. Dann rechne mal den mittleren Verkaufspreis aus und ziehe 10\%
ab, die wir als Provision unseren Verkäufern zahlen müssen.
\end{quote}

\textbf{Antwort}

\begin{Shaded}
\begin{Highlighting}[]
\NormalTok{mariokart }\SpecialCharTok{\%\textgreater{}\%} 
  \FunctionTok{select}\NormalTok{(total\_pr) }\SpecialCharTok{\%\textgreater{}\%} 
  \FunctionTok{mutate}\NormalTok{(}\AttributeTok{total\_pr\_yen =}\NormalTok{ total\_pr }\SpecialCharTok{*} \DecValTok{133}\NormalTok{) }\SpecialCharTok{\%\textgreater{}\%} 
  \FunctionTok{summarise}\NormalTok{(}
    \AttributeTok{preis\_yen\_mw =} \FunctionTok{mean}\NormalTok{(total\_pr\_yen),}
    \AttributeTok{preis\_yen\_mw\_minus\_10proz =}\NormalTok{ preis\_yen\_mw }\SpecialCharTok{{-}} \FloatTok{0.1}\SpecialCharTok{*}\NormalTok{preis\_yen\_mw)}
\end{Highlighting}
\end{Shaded}

\begin{longtable}[]{@{}rr@{}}
\toprule\noalign{}
preis\_yen\_mw & preis\_yen\_mw\_minus\_10proz \\
\midrule\noalign{}
\endhead
\bottomrule\noalign{}
\endlastfoot
6634 & 5971 \\
\end{longtable}

Wie man sieht kann man in \texttt{summarise()} auch mehr als eine
Berechnung einstellen. In diesem Fall haben wir zwei Berechnungen
angestellt: Einmal den Mittelwert und einmal den Mittelwert minus 10\%
(des Mittelwerts).

\end{exercise}

\begin{exercise}[Do It
Yourself]\protect\hypertarget{exr-diy}{}\label{exr-diy}

Denken Sie sich selber ähnliche Forschungsfragen aus. Stellen Sie diese
einer vertrauenswürdigen Kommilitonen bzw. einem vertrauenswürdigen
Kommilitonen. DIY! Schauen Sie, ob Ihre Aufgabe richtig gelöst wird.
Prüfen Sie streng.\(\square\)

\end{exercise}

\section{Praxisbezug}\label{praxisbezug-2}

Die Covid19-Epidemie hatte weltweit massive Auswirkungen; auch
psychologischer Art wie Vereinsamung, Angst oder Depression. Eine
Studie, die die psychologischen Auswirkungen von Mulukom et al. (2020),
die \href{https://osf.io/tsjnb/}{unter der Projekt-ID tsjnb bei der Open
Science Foundation (OSF), \textless https://osf.io/tsjnb/\textgreater,
angemeldet ist}. Die Daten wurden mit R ausgewertet. Beispielhaft ist
unter \url{https://osf.io/4b9p2} die R-Syntax zu sehen, die die Autoren
zur Datenaufbereitung verwendet haben. Einen guten Teil dieser Syntax
kennen Sie aus diesem Kapitel. Diese Studie ist, neben einigen
vergleichbaren, ein schönes Beispiel, wie Forschung und Praxis
ineinander greifen können: Angewandte Forschung als Beitrag zur Lösung
eines akuten Problems, der Corona-Pandemie.

\section{Wie man mit Statistik
lügt}\label{wie-man-mit-statistik-luxfcgt-1}

Ein (leider) immer mal wieder zu beobachtender \enquote{Trick}, um Daten
zu frisieren ist, nur die Daten zu berichten, die einem in den Kram
passen.

\begin{example}[]\protect\hypertarget{exm-luege-filter}{}\label{exm-luege-filter}

Eine Analystin {\emoji{woman}} möchte zeigen, dass der Verkaufspreis von
Mariokart-Spielen \enquote{viel zu niedrig} ist. Es muss ein höherer
Wert rauskommen, findet die Analystin. Der mittlere Verkaufspreis (im
Datensatz \texttt{mariokart}) liegt bei 50 Euro.

\begin{quote}
{\emoji{woman}} Kann man den Wert nicht \ldots{} \enquote{kreativ
verbessern}? Ein paar Statistik-Tricks anwenden?
\end{quote}

Um dieses Ziel zu erreichen, teilt die Analystin den Datensatz in
Gruppen nach Anzahl der dem Spiel beigelegten Lenkräder
(\texttt{wheels}). Dann wird der Mittelwert pro Gruppe berechnet.

\begin{Shaded}
\begin{Highlighting}[]
\NormalTok{mariokart\_wheels }\OtherTok{\textless{}{-}} 
\NormalTok{mariokart }\SpecialCharTok{\%\textgreater{}\%} 
  \FunctionTok{group\_by}\NormalTok{(wheels) }\SpecialCharTok{\%\textgreater{}\%} 
  \FunctionTok{summarise}\NormalTok{(}\AttributeTok{pr\_mean =} \FunctionTok{mean}\NormalTok{(total\_pr),}
            \AttributeTok{count\_n =} \FunctionTok{n}\NormalTok{())  }\CommentTok{\# n() gibt die Anzahl der Zeilen pro Gruppe an}

\NormalTok{mariokart\_wheels}
\end{Highlighting}
\end{Shaded}

\begin{longtable}[]{@{}rrr@{}}
\toprule\noalign{}
wheels & pr\_mean & count\_n \\
\midrule\noalign{}
\endhead
\bottomrule\noalign{}
\endlastfoot
0 & 41 & 37 \\
1 & 44 & 52 \\
2 & 61 & 51 \\
3 & 70 & 2 \\
4 & 65 & 1 \\
\end{longtable}

Schließlich berechnet unsere Analystin den \emph{ungewichteten}
Mittelwert über diese 5 Gruppen:

\begin{Shaded}
\begin{Highlighting}[]
\NormalTok{mariokart\_wheels }\SpecialCharTok{\%\textgreater{}\%} 
  \FunctionTok{summarise}\NormalTok{(}\FunctionTok{mean}\NormalTok{(pr\_mean))}
\end{Highlighting}
\end{Shaded}

\begin{longtable}[]{@{}r@{}}
\toprule\noalign{}
mean(pr\_mean) \\
\midrule\noalign{}
\endhead
\bottomrule\noalign{}
\endlastfoot
56 \\
\end{longtable}

Und das Ergebnis lautet: 56 Euro! Das ist doch schon etwas
\enquote{besser} als 50 Euro.

Natürlich ist es \emph{falsch} und irreführend, hier einen ungewichteten
Mittelwert zu berechnen. Der gewichtete Mittelwert würde wiederum zum
korrekten Ergebnis, 50 Euro, führen.\(\square\)

\end{example}

\section{Fallstudien}\label{fallstudien}

\subsection{Die Pinguine}\label{die-pinguine}

\begin{figure}[H]

\centering{

\includegraphics[width=0.5\linewidth,height=\textheight,keepaspectratio]{img/penguins.png}

}

\caption{\label{fig-penguins}Possierlich: Die Pinguine (Horst, 2024)}

\end{figure}%

\begin{exercise}[]\protect\hypertarget{exr-peng-start}{}\label{exr-peng-start}

Machen Sie sich zunächst mit dem Pinguin-Datensatz vertraut. Sie finden
den Datensatz \texttt{penguins} im R-Paket \texttt{palmerpenguins}, das
Sie auf gewohnte Art installieren können (vgl.
Kapitel~\ref{sec-install-r-pckgs}); im Internet findet man den Datensatz
auch als CSV-Datei. Fokussieren Sie Ihre Analyse auf die Zielvariable
\emph{Gewicht}. \(\square\)

\end{exercise}

Forschungsfragen:

\begin{enumerate}
\def\labelenumi{\arabic{enumi}.}
\tightlist
\item
  Was ist das mediane Gewicht von Pinguinen, gruppiert nach Spezies und
  nach Gewicht?
\item
  Wie viele Pinguine gibt es pro Spezies?
\item
  Wie viel wiegt der schwerste und der leichteste Pinguin pro Spezies?
\end{enumerate}

\subsection{Fallstudie COVIDiSTRESS}\label{fallstudie-covidistress}

\begin{figure}[H]

{\centering \includegraphics[width=0.5\linewidth,height=\textheight,keepaspectratio]{img/Covidistress1.jpg}

}

\caption{Studie COVIDiSTTRESS (Lieberoth et al., 2020)}

\end{figure}%

Lesen Sie die Beschreibung der Studie COVIDiSTRESS (Lieberoth et al.,
2022). Hier ist ein Abstract:

\begin{quote}
The COVIDiSTRESS global survey is an international collaborative
undertaking for data gathering on human experiences, behavior and
attitudes during the COVID-19 pandemic. In particular, the survey
focuses on psychological stress, compliance with behavioral guidelines
to slow the spread of Coronavirus, and trust in governmental
institutions and their preventive measures, but multiple further items
and scales are included for descriptive statistics, further analysis and
comparative mapping between participating countries. Round one data
collection was concluded May 30. 2020. To gather comparable data swiftly
from across the globe, when the Coronavirus started making a critical
impact on societies and individuals, the collaboration and survey was
constructed as an urgent collaborative process. Individual contributors
and groups in the COVIDiSTRESS network (see below) conducted
translations to each language and shared online links by their own best
means in each country.
\end{quote}

\href{https://osf.io/z39us/files/osfstorage}{Die Daten} stehen unter
\url{https://osf.io/z39us} zur freien Verfügung. Sie können diese echten
Daten eigenständig analysieren.

\section{Aufgaben}\label{aufgaben-2}

\begin{tcolorbox}[enhanced jigsaw, bottomtitle=1mm, leftrule=.75mm, breakable, title=\textcolor{quarto-callout-tip-color}{\faLightbulb}\hspace{0.5em}{ChatGPT}, bottomrule=.15mm, titlerule=0mm, left=2mm, opacityback=0, colframe=quarto-callout-tip-color-frame, rightrule=.15mm, colback=white, coltitle=black, toprule=.15mm, toptitle=1mm, colbacktitle=quarto-callout-tip-color!10!white, arc=.35mm, opacitybacktitle=0.6]

Nutzen Sie einen Chat-Bot wie ChatGPT, um sich Hilfe für die R-Syntax
geben zu lassen. \(\square\)

\end{tcolorbox}

Die Webseite \href{https://datenwerk.netlify.app}{datenwerk.netlify.app}
stellt eine Reihe von einschlägigen Übungsaufgaben bereit. Sie können
die Suchfunktion der Webseite nutzen, um die Aufgaben mit den folgenden
Namen zu suchen:

\begin{enumerate}
\def\labelenumi{\arabic{enumi}.}
\tightlist
\item
  \href{https://sebastiansauer.github.io/Datenwerk/posts/wrangle3/wrangle3.html}{wrangle3}
\item
  \href{https://sebastiansauer.github.io/Datenwerk/posts/wrangle4/wrangle4.html}{wrangle4}
\item
  \href{https://sebastiansauer.github.io/Datenwerk/posts/wrangle5/wrangle5.html}{wrangle5}
\item
  \href{https://sebastiansauer.github.io/Datenwerk/posts/wrangle7/wrangle7.html}{wrangle7}
\item
  \href{https://sebastiansauer.github.io/Datenwerk/posts/wrangle9/wrangle9.html}{wrangle9}
\item
  \href{https://sebastiansauer.github.io/Datenwerk/posts/wrangle10/wrangle10.html}{wrangle10}
\item
  \href{https://sebastiansauer.github.io/Datenwerk/posts/tidydata1/tidydata1.html}{tidydata1}
\item
  \href{https://sebastiansauer.github.io/Datenwerk/posts/affairs-dplyr/affairs-dplyr.html}{affairs-dplyr}
\item
  \href{https://sebastiansauer.github.io/Datenwerk/posts/dplyr-uebersetzen/dplyr-uebersetzen.html}{dplyr-uebersetzen}
\item
  \href{https://sebastiansauer.github.io/Datenwerk/posts/haeufigkeit01/haeufigkeit01.html}{haeufigkeit01}
\item
  \href{https://sebastiansauer.github.io/Datenwerk/posts/mariokart-mean1/mariokart-mean1.html}{mariokart-mean1}
\item
  \href{https://sebastiansauer.github.io/Datenwerk/posts/mariokart-mean2/mariokart-mean2.html}{mariokart-mean2}
\item
  \href{https://sebastiansauer.github.io/Datenwerk/posts/mariokart-mean3/mariokart-mean3.html}{mariokart-mean3}
\item
  \href{https://sebastiansauer.github.io/Datenwerk/posts/mariokart-mean4/mariokart-mean4.html}{mariokart-mean4}
\item
  \href{https://sebastiansauer.github.io/Datenwerk/posts/mariokart-max1/mariokart-max1.html}{mariokart-max1}
\item
  \href{https://sebastiansauer.github.io/Datenwerk/posts/mariokart-max2/mariokart-max2.html}{mariokart-max2}
\item
  \href{https://sebastiansauer.github.io/Datenwerk/posts/filter01/filter01.html}{filter01}
\item
  \href{https://sebastiansauer.github.io/Datenwerk/posts/affairs-dplyr/affairs-dplyr.html}{affairs-dplyr}
\item
  \href{https://sebastiansauer.github.io/Datenwerk/posts/summarise01/summarise01.html}{summarise01}
\item
  \href{https://sebastiansauer.github.io/Datenwerk/posts/summarise02/summarise02.html}{summarise02}
\item
  \href{https://sebastiansauer.github.io/Datenwerk/posts/mutate01/mutate01.html}{mutate01}
\item
  \href{https://sebastiansauer.github.io/Datenwerk/posts/wrangle3/wrangle3}{wrangle3}
\end{enumerate}

\section{Vertiefung}\label{vertiefung-3}

\subsection{Tidydatatutor}\label{tidydatatutor}

Die Verben des Datenjudos werden beim
\href{https://tidydatatutor.com/}{\enquote{Tidydatatutor}} anschaulich
illustriert.\footnote{\url{https://tidydatatutor.com}}

\subsection{Fortgeschrittenes R}\label{fortgeschrittenes-r}

\begin{tcolorbox}[enhanced jigsaw, bottomtitle=1mm, leftrule=.75mm, breakable, title=\textcolor{quarto-callout-note-color}{\faInfo}\hspace{0.5em}{Hinweis}, bottomrule=.15mm, titlerule=0mm, left=2mm, opacityback=0, colframe=quarto-callout-note-color-frame, rightrule=.15mm, colback=white, coltitle=black, toprule=.15mm, toptitle=1mm, colbacktitle=quarto-callout-note-color!10!white, arc=.35mm, opacitybacktitle=0.6]

In weiterführendem Material werden Sie immer wieder auf Inhalte treffen,
die Sie noch nicht kennen, die etwa noch nicht im Unterricht behandelt
wurden. Seien Sie unbesorgt: In der Regel können Sie diese Inhalte
einfach auslassen, ohne den Anschluss zu verlieren. Einfach ignorieren.
😄

\end{tcolorbox}

Häufig ist es nützlich, die Werte einer Variablen umzukodieren, z.B.
\enquote{weiblich} in \enquote{w} oder in \texttt{0}. Eine gute
Möglichkeit, dies in R umzusetzen, bietet der Befehl
\texttt{case\_when()}; der Befehl wohnt im Tidyverse.\footnote{\url{https://www.statology.org/dplyr-case_when/}}
Im Datenwerk finden Sie dazu Übungen, etwa
\href{https://sebastiansauer.github.io/Datenwerk/posts/mutate03/mutate03.html}{mutate03}.

\subsection{Hilfe?! Erbie!}\label{sec-erbie}

R will nicht, so wie Sie wollen? Sie haben das Gefühl, R verweigert
störrisch den Dienst, vermutlich rein aus Boshaftigkeit, rein um Sie zu
ärgern? Ausführliches Googeln und ChatGPT befragen hat keine Lösung
gebracht? Kurz, Sie brauchen die Hilfe eines kundigen Menschens?
\href{https://data-se.netlify.app/2022/01/31/erbie-einfache-reproduzierbare-beispiele-ihres-problems-mit-r-syntax/}{Sie
sollten Ihren Hilfeschrei so artikulieren}, dass er nicht nur gehört,
sondern auch verstanden wird und einen anderen Menschen veranlasst und
ermöglicht Ihnen zu helfen.

Also: Sie müssen Ihr Problem nachvollziehbar aber prägnant formulieren.
Das nennt man auch ein \emph{ERBie}, ein \emph{einfaches,
reproduzierbare Beispiel} Ihres Problems mit (R-)Syntax:

\begin{itemize}
\tightlist
\item
  \emph{einfach}: die einfachste Syntax, die Ihr Problem bzw. die
  Fehlermeldung produziert. Es bietet sich an, einen einfachen,
  allgemein bekannten Datensatz zu verwenden, etwa \texttt{mtcars}
\item
  \emph{reproduzierbar}: Code (z.B. als Textdatei oder in einem Post),
  der die Fehlermeldung entstehen lässt
\end{itemize}

\begin{example}[Beispiel für ein
Erbie]\protect\hypertarget{exm-erbie}{}\label{exm-erbie}

\emph{Problem:} Ich verstehe nicht, warum eine Fehlermeldung kommt

\emph{Ziel:} Ich möchte die Automatikautos filtern (am = 0)

\emph{Was ich schon versucht habe:} Ich habe folgende Posts gelesen
\ldots, aber ohne Erfolg

\emph{Erbie}:

\begin{Shaded}
\begin{Highlighting}[]
\FunctionTok{data}\NormalTok{(mtcars)}
\FunctionTok{library}\NormalTok{(dplyr)  }\CommentTok{\# nicht "tidyverse", denn "dplyr" reicht}

\NormalTok{mtcars }\SpecialCharTok{\%\textgreater{}\%} 
  \FunctionTok{filter}\NormalTok{(}\AttributeTok{am =} \DecValTok{0}\NormalTok{)  }\CommentTok{\# den kürzesten Code, der Ihren Fehler entstehen lässt!}

\FunctionTok{sessionInfo}\NormalTok{()  }\CommentTok{\# gibt Infos zur R{-}Version etc. aus}
\end{Highlighting}
\end{Shaded}

Mit dem Paket \texttt{reprex} kann man sich R-Syntax schön formuliert
ausgeben lassen. Das ist perfekt, um den Code dann in einem Forum (oder
Mail) einzustellen. Dafür müssen Sie nur den Code auswählen,
\texttt{Strg-c} drücken und dann \texttt{reprex::reprex} ausführen. Mit
\texttt{Strg-v} können Sie die schön formatierte Syntax (sowie die
Ausgabe, auch schön formatiert) dann irgendwohin pasten.

\end{example}

\begin{tcolorbox}[enhanced jigsaw, bottomtitle=1mm, leftrule=.75mm, breakable, title=\textcolor{quarto-callout-tip-color}{\faLightbulb}\hspace{0.5em}{Tipp}, bottomrule=.15mm, titlerule=0mm, left=2mm, opacityback=0, colframe=quarto-callout-tip-color-frame, rightrule=.15mm, colback=white, coltitle=black, toprule=.15mm, toptitle=1mm, colbacktitle=quarto-callout-tip-color!10!white, arc=.35mm, opacitybacktitle=0.6]

\begin{figure}[H]

\begin{minipage}{0.80\linewidth}
Posten Sie Ihr Erbie bei \url{https://gist.github.com/} als
\enquote{public gist}.
\href{https://gist.github.com/sebastiansauer/0649a0453b5cc7c6a1d16ac760667215}{Hier}
ist ein Beispiel.\(\square\)\end{minipage}%
%
\begin{minipage}{0.20\linewidth}

\begin{center}
\includegraphics[width=0.75\linewidth,height=\textheight,keepaspectratio]{030-aufbereiten_files/figure-pdf/unnamed-chunk-50-1.pdf}
\end{center}

\end{minipage}%

\end{figure}%

\end{tcolorbox}

\section{Exkurs}\label{exkurs}

\href{https://openai.com/dall-e-2/}{Dall-E 2} ist eine KI, die
\enquote{realistische Bilder und Kunst aus einer Beschreibung in
natürlicher Sprache} erstellt.

\begin{quote}
{\emoji{teacher}} I'd like a mixture between robot und professor, in oil
painting
\end{quote}

\begin{quote}
{\emoji{robot}} \ldots{} s. Abbildung~\ref{fig-mix-rob-prof}
\end{quote}

\begin{figure}[H]

\centering{

\includegraphics[width=0.5\linewidth,height=\textheight,keepaspectratio]{img/mix-prof-robot.png}

}

\caption{\label{fig-mix-rob-prof}Bild erzeugt von künstlicher
Intelligenz, Quelle: DALL-E 2, 2023-02-09}

\end{figure}%

\begin{tcolorbox}[enhanced jigsaw, bottomtitle=1mm, leftrule=.75mm, breakable, title=\textcolor{quarto-callout-note-color}{\faInfo}\hspace{0.5em}{Hinweis}, bottomrule=.15mm, titlerule=0mm, left=2mm, opacityback=0, colframe=quarto-callout-note-color-frame, rightrule=.15mm, colback=white, coltitle=black, toprule=.15mm, toptitle=1mm, colbacktitle=quarto-callout-note-color!10!white, arc=.35mm, opacitybacktitle=0.6]

Der Nutzen künstlicher Intelligenz für die Datenanalyse ist natürlich
breiter: Wenn Sie sich z.B. über die Syntax eines bestimmten Befehls
(oder allgemeiner: Vorhabens) nicht sicher sind, fragen Sie sich doch
mal einen Bot wie ChatGPT.

\end{tcolorbox}

\section{Literaturhinweise}\label{literaturhinweise-2}

Sauer (2019), Kap. 7, gibt eine Einführung in die Datenaufbereitung (mit
Hilfe von R), ähnlich zu den Inhalten dieses Kapitels. Mehr in die Tiefe
des \enquote{Datenjudo} führen Wickham \& Grolemund (2018); der Autor
Hadley Wickham ist die Leitfigur in der R-Community schlechthin. Kap. 5
behandelt (etwas ausführlicher) die Themen dieses Kapitels. Er ist einer
der Hauptautoren von beliebten R-Paketen wie \texttt{dplyr} und
\texttt{ggplot2}.

\part{Modellieren}

\chapter{Daten verbildlichen}\label{daten-verbildlichen}

\section{Lernsteuerung}\label{lernsteuerung-3}

Abb. Abbildung~\ref{fig-ueberblick} zeigt den Standort dieses Kapitels
im Lernpfad und gibt damit einen Überblick über das Thema dieses
Kapitels im Kontext aller Kapitel.

\subsection{Lernziele}\label{lernziele-4}

\begin{itemize}
\tightlist
\item
  Sie können erläutern, wann und wozu das Visualisieren statistischer
  Inhalte sinnvoll ist.
\item
  Sie kennen typische Arte von Datendiagrammen.
\item
  Sie können typische Datendiagramme mit R visualisieren.
\item
  Sie können zentrale Ergebnisse aus Datendiagrammen herauslesen.
\end{itemize}

\subsection{Benötigte R-Pakete}\label{benuxf6tigte-r-pakete-1}

\begin{Shaded}
\begin{Highlighting}[]
\FunctionTok{library}\NormalTok{(tidyverse)}
\FunctionTok{library}\NormalTok{(easystats)}
\FunctionTok{library}\NormalTok{(DataExplorer)  }\CommentTok{\# nicht vergessen zu installieren}
\FunctionTok{library}\NormalTok{(ggpubr)  }\CommentTok{\# optional, Datenvisualisierung}
\FunctionTok{library}\NormalTok{(ggstatsplot)  }\CommentTok{\# optional, Datenvisualisierung}
\end{Highlighting}
\end{Shaded}

\subsection{Benötigte Daten}\label{benuxf6tigte-daten-2}

Zuerst definieren wir den Pfad, wo wir die Daten finden, s.
Listing~\ref{lst-mariokart}. Dann importieren wir die Mariokart-Daten.

\begin{codelisting}

\caption{\label{lst-mariokart}Pfad zu den Mariokart-Daten}

\centering{

\begin{Shaded}
\begin{Highlighting}[]
\NormalTok{mariokart\_path }\OtherTok{\textless{}{-}} \FunctionTok{paste0}\NormalTok{(}
  \StringTok{"https://vincentarelbundock.github.io/Rdatasets"}\NormalTok{,}
  \StringTok{"/csv/openintro/mariokart.csv"}\NormalTok{)}

\NormalTok{mariokart }\OtherTok{\textless{}{-}} \FunctionTok{read.csv}\NormalTok{(mariokart\_path)}
\end{Highlighting}
\end{Shaded}

}

\end{codelisting}%

\subsection{Wozu das alles?}\label{wozu-das-alles}

\begin{quote}
{\emoji{ninja}} Wir müssen die Galaxis retten, Kermit.
\end{quote}

\begin{quote}
{\emoji{frog}} \emph{Schlock}
\end{quote}

\section{Ein Dino sagt mehr als 1000
Worte}\label{ein-dino-sagt-mehr-als-1000-worte}

Es heißt, ein Bild sage mehr als 1000 Worte. Schon richtig, aber ein
Dinosaurier sagt auch mehr als 1000 Worte, s. Abbildung~\ref{fig-dino1}.
In Abbildung~\ref{fig-dino1} sieht man verschiedene \enquote{Bilder},
also Datensätze: etwa einen Dino und einmal einen Kreis. Obwohl die
Bilder grundverschiedene sind, sind die zentralen statistischen
Kennwerte (praktisch) identisch. In die gleiche Bresche schlägt
\enquote{Anscombes Quartett} (Anscombe, 1973), s.
Abbildung~\ref{fig-dino2}: Es zeigt vier Datensätze, in denen die
zentralen Statistiken fast identisch sind,\\
also Mittelwerte, Streuungen, Korrelationen. Aber die Streudiagramme
sind grundverschieden. Anscombes Beispiel zeigt (zugespitzt): Eine
Visualisierung enthüllt, was der Statistik (als Kennzahl) verhüllt
bleibt.

\begin{figure}[H]

\centering{

\includegraphics[width=1\linewidth,height=\textheight,keepaspectratio]{040-verbildlichen_files/figure-pdf/fig-dino1-1.pdf}

}

\caption{\label{fig-dino1}Alle Diagramme haben gleiche statistische
Koeffizienten, wie Mittelwert und Streuung und Korrelation, aber die
Datengrundlage sind komplett verschieden.}

\end{figure}%

\begin{tcolorbox}[enhanced jigsaw, bottomtitle=1mm, leftrule=.75mm, breakable, title=\textcolor{quarto-callout-important-color}{\faExclamation}\hspace{0.5em}{Wichtig}, bottomrule=.15mm, titlerule=0mm, left=2mm, opacityback=0, colframe=quarto-callout-important-color-frame, rightrule=.15mm, colback=white, coltitle=black, toprule=.15mm, toptitle=1mm, colbacktitle=quarto-callout-important-color!10!white, arc=.35mm, opacitybacktitle=0.6]

Statistische Diagramme können Einblicke geben, die sich nicht (leicht)
in grundlegenden Statistiken (Kennwerten) abbilden. \(\square\)

\end{tcolorbox}

\begin{figure}[H]

\centering{

\includegraphics[width=0.75\linewidth,height=\textheight,keepaspectratio]{img/anscombe.png}

}

\caption{\label{fig-dino2}Anscombes Quartet: Gleiche statistischen
Kennwerte in vier Datensätzen}

\end{figure}%

Unter visueller Cortex ist sehr leistungsfähig. Wir können ohne Mühe
eine große Anzahl an Informationen aufnehmen und parallel verarbeiten.
Aus diesem Grund sind Datendiagramme eine effektive und einfache Art,
aus Daten Erkenntnisse zu ziehen.

\begin{tcolorbox}[enhanced jigsaw, bottomtitle=1mm, leftrule=.75mm, breakable, title=\textcolor{quarto-callout-tip-color}{\faLightbulb}\hspace{0.5em}{Tipp}, bottomrule=.15mm, titlerule=0mm, left=2mm, opacityback=0, colframe=quarto-callout-tip-color-frame, rightrule=.15mm, colback=white, coltitle=black, toprule=.15mm, toptitle=1mm, colbacktitle=quarto-callout-tip-color!10!white, arc=.35mm, opacitybacktitle=0.6]

Nutzen Sie Datendiagramme umfassend; sie sind einfach zu verstehen und
doch sehr mächtig.

\end{tcolorbox}

\subsection{Datendiagramm}\label{datendiagramm}

Ein \emph{Datendiagramm} (kurz: Diagramm) ist ein Diagramm, das Daten
und Statistiken zeigt, mit dem Zweck, Erkenntnisse daraus zu ziehen.

\begin{example}[Aus der Forschung: Ein aufwändiges (und ansprechendes)
Datendiagramm]\protect\hypertarget{exm-datendiagramm}{}\label{exm-datendiagramm}

~

Auf Basis des Korruptionsindex von Transparency International (2017)
erstellt Wilke (2019/2024) ein Diagramm zum Zusammenhang vom
Entwicklungsindex (Lebenserwartung, Bildung, Einkommen; vgl. Hou et al.
(2015)) und Korruption, jeweils auf Landesebene, s.
Abbildung~\ref{fig-develop-corrupt}.

Es finden sich in der Literatur (im Internet) viele weitere Beispiele
für handwerklich meisterhaft erstelle Datendiagramme, die in vielen
Fällen mit R erstellt werden (vgl. Scherer et al., 2019).

\begin{figure}[H]

\centering{

\includegraphics[width=0.75\linewidth,height=\textheight,keepaspectratio]{img/develop-corrupt.png}

}

\caption{\label{fig-develop-corrupt}Der Zusammenhang von
Entwicklungindex und und Korruption}

\end{figure}%

\end{example}

\subsection{Ein Bild hat nicht so viele
Dimensionen}\label{ein-bild-hat-nicht-so-viele-dimensionen}

Abbildung~\ref{fig-many-dims} zeigt ein Bild mit mehreren (5) Variablen,
die jeweils einer \enquote{Dimension} entsprechen. Wie man (nicht)
sieht, wird es langsam unübersichtlich. Offenbar kann man in einem Bild
nicht beliebig viele Variablen sinnvoll reinquetschen. Die
\enquote{Dimensionalität} eines Diagramms hat ihre Grenzen, vielleicht
bei 4-6 Variablen.

\begin{figure}[H]

\centering{

\includegraphics[width=0.75\linewidth,height=\textheight,keepaspectratio]{040-verbildlichen_files/figure-pdf/fig-many-dims-1.pdf}

}

\caption{\label{fig-many-dims}Ein Diagramm kann nur eine begrenzte
Anzahl von Variablen zeigen. Wenn Sie dieses Bild nicht checken: Prima.
Genau das soll das Bild zeigen.}

\end{figure}%

Möchten wir den Zusammenhang von vielen Variablen, z.B. mehr als 5,
verstehen, kommen wir mit Bildern nicht weiter. Dann brauchen wir andere
Werkzeuge: statistics to the rescue.

\begin{tcolorbox}[enhanced jigsaw, bottomtitle=1mm, leftrule=.75mm, breakable, title=\textcolor{quarto-callout-note-color}{\faInfo}\hspace{0.5em}{Hinweis}, bottomrule=.15mm, titlerule=0mm, left=2mm, opacityback=0, colframe=quarto-callout-note-color-frame, rightrule=.15mm, colback=white, coltitle=black, toprule=.15mm, toptitle=1mm, colbacktitle=quarto-callout-note-color!10!white, arc=.35mm, opacitybacktitle=0.6]

Bei klaren Zusammenhängen und wenig Variablen braucht man keine
(aufwändige) Statistik. Ein Bild (Datendiagramm) ist dann (oft)
ausreichend. Man könnte sagen, dass es Statistik nur deshalb gibt, weil
unser Auge mit mehr als ca. 4-6 Variablen nicht gleichzeitig umgehen
kann.

\end{tcolorbox}

\begin{exercise}[]\protect\hypertarget{exr-anz-dims}{}\label{exr-anz-dims}

Wie viele Variablen sind in Abbildung~\ref{fig-many-dims}
dargestellt?\footnote{5}

\end{exercise}

\section{Nomenklatur von
Datendiagrammen}\label{nomenklatur-von-datendiagrammen}

Tabelle~\ref{tbl-nom-plots} zeigt eine -- sehr kurze Nomenklatur -- an
Datendiagrammen. Weitere Nomenklaturen sind möglich, aber wir halten
hier die Sache einfach. Wer an Vertiefung interessiert ist, findet bei
data-to-vis einen Überblick über verschiedene Typen an Diagrammen, sogar
in Form einer systematischen Nomenklatur:
\url{https://www.data-to-viz.com/}.

\begin{longtable}[]{@{}
  >{\raggedright\arraybackslash}p{(\linewidth - 4\tabcolsep) * \real{0.2143}}
  >{\raggedright\arraybackslash}p{(\linewidth - 4\tabcolsep) * \real{0.3571}}
  >{\raggedright\arraybackslash}p{(\linewidth - 4\tabcolsep) * \real{0.4286}}@{}}

\caption{\label{tbl-nom-plots}Ein (sehr kurze) Nomenklatur von
Datendiagrammen}

\tabularnewline

\toprule\noalign{}
\begin{minipage}[b]{\linewidth}\raggedright
Erkenntnisziel
\end{minipage} & \begin{minipage}[b]{\linewidth}\raggedright
qualitativ
\end{minipage} & \begin{minipage}[b]{\linewidth}\raggedright
quantitativ
\end{minipage} \\
\midrule\noalign{}
\endhead
\bottomrule\noalign{}
\endlastfoot
Verteilung & Balkendiagramm & Histogramm und Dichtediagramm \\
Zusammenhang & gefülltes Balkendiagramm & Streudiagramm \\
Unterschied & gefülltes Balkendiagramm & Boxplot \\

\end{longtable}

\begin{tcolorbox}[enhanced jigsaw, bottomtitle=1mm, leftrule=.75mm, breakable, title=\textcolor{quarto-callout-note-color}{\faInfo}\hspace{0.5em}{Hinweis}, bottomrule=.15mm, titlerule=0mm, left=2mm, opacityback=0, colframe=quarto-callout-note-color-frame, rightrule=.15mm, colback=white, coltitle=black, toprule=.15mm, toptitle=1mm, colbacktitle=quarto-callout-note-color!10!white, arc=.35mm, opacitybacktitle=0.6]

Wir arbeiten hier mit dem Datensatz \texttt{mariokart}. Hilfe bzw. ein
Data-Dictionary (Codebook) finden Sie
\href{https://www.rdocumentation.org/packages/openintro/versions/2.4.0/topics/mariokart}{hier}.

\end{tcolorbox}

\section{Verteilungen verbildlichen}\label{verteilungen-verbildlichen}

\subsection{Verteilung: nominale
Variable}\label{verteilung-nominale-variable}

\begin{definition}[Verteilung]\protect\hypertarget{def-verteilung}{}\label{def-verteilung}

Eine (Häufigkeits-)Verteilung einer Variablen \(X\) schlüsselt auf, wie
häufig jede Ausprägung von \(X\) ist.\(\square\)

\end{definition}

\begin{example}[]\protect\hypertarget{exm-verteilung1}{}\label{exm-verteilung1}

Tabelle~\ref{tbl-wheels-n} zeigt die Häufigkeitsverteilung von
\texttt{cond} (condition, also der Zustand des Artikels, neu oder
gebraucht) aus dem Datensatz \texttt{mariokart}. Die Variable hat 5
Ausprägungen; z.B. kommt die Ausprägung \texttt{new} 59 mal
vor.\(\square\)

\end{example}

\begin{longtable}[]{@{}ll@{}}

\caption{\label{tbl-wheels-n}Häufigkeitsverteilung von \texttt{cond} aus
dem Datensatz \texttt{mariokart}}

\tabularnewline

\toprule\noalign{}
cond & n \\
\midrule\noalign{}
\endhead
\bottomrule\noalign{}
\endlastfoot
new & 59 \\
used & 84 \\

\end{longtable}

Zugegeben, das Datendiagramm von \texttt{cond} ist nicht so aufregend,
s. Abbildung~\ref{fig-mario-n-plot-cond}. Wie man sieht, besteht so ein
Diagramm als \emph{Balken}, daher heißt es \emph{Balkendiagramm}
(synonym: Säulendiagramm). Man kann so ein Diagramm um 90° drehen; keine
Ausrichtung ist unbedingt besser als die andere.

\begin{definition}[Balkendiagramm]\protect\hypertarget{def-balken}{}\label{def-balken}

Ein Balkendiagramm ist eine grafische Darstellung von Werten, zumeist
für die Häufigkeiten bestimmter Kategorien (Ausprägungen nominaler
Variablen). Dabei werden rechteckige Balken verwendet werden, und die
Länge eines Balkens ist proportional zur dargestellten Häufigkeit.
\(\square\)

\end{definition}

\begin{figure}[H]

\begin{minipage}{0.50\linewidth}

\centering{

\includegraphics[width=0.75\linewidth,height=\textheight,keepaspectratio]{040-verbildlichen_files/figure-pdf/fig-mario-n-plot-cond-1.pdf}

}

\subcaption{\label{fig-mario-n-plot-cond-1}horizontale Balken}

\end{minipage}%
%
\begin{minipage}{0.50\linewidth}

\centering{

\includegraphics[width=0.75\linewidth,height=\textheight,keepaspectratio]{040-verbildlichen_files/figure-pdf/fig-mario-n-plot-cond-2.pdf}

}

\subcaption{\label{fig-mario-n-plot-cond-2}vertikale Balken}

\end{minipage}%

\caption{\label{fig-mario-n-plot-cond}Häufigkeitsverteilung der Variable
\texttt{cond}}

\end{figure}%

Es gibt viele Methoden, sich mit R ein Balkendiagramm ausgeben zu
lassen. Eine einfache, komfortable ist die mit dem Paket
\texttt{DataExplorer}, s. Abbildung~\ref{fig-mario-n-plot-cond}; wir
betrachten gleich die Syntax. Zuerst importieren wir die Daten, s.
Listing~\ref{lst-mariokart}. Außerdem nicht vergessen, das Paket
\texttt{DataExplorer} mit dem Befehle \texttt{library} zu starten.
(Natürlich müssen Sie das Paket einmalig installiert haben, bevor Sie es
starten können.) In diesem Paket \enquote{wohnen} die Befehle, die wir
zum Erstellen der Datendiagramme nutzen werden. Listing~\ref{lst-de1}
zeigt die Syntax, um ein Balkendiagramm zu erstellen. Auf der Hilfeseite
der Funktion finden Sie weitere Details zur Funktion.

\begin{codelisting}

\caption{\label{lst-de1}Syntax zur Erstellung eines Balkendiagramms}

\centering{

\begin{Shaded}
\begin{Highlighting}[]
\FunctionTok{library}\NormalTok{(DataExplorer)}
\NormalTok{mariokart }\OtherTok{\textless{}{-}} \FunctionTok{read.csv}\NormalTok{(mariokart\_path)}

\NormalTok{mariokart }\SpecialCharTok{\%\textgreater{}\%} 
  \FunctionTok{select}\NormalTok{(cond) }\SpecialCharTok{\%\textgreater{}\%} 
  \FunctionTok{plot\_bar}\NormalTok{()}
\end{Highlighting}
\end{Shaded}

}

\end{codelisting}%

\begin{figure}[H]

\centering{

\includegraphics[width=0.5\linewidth,height=\textheight,keepaspectratio]{040-verbildlichen_files/figure-pdf/fig-de1-1.pdf}

}

\caption{\label{fig-de1}Ein Balkendiagramm. Unglaublich.}

\end{figure}%

Die Syntax ist in Listing~\ref{lst-de1} abgedruckt (Zur Erinnerung:
\texttt{\%\textgreater{}\%} nennt man die \enquote{Pfeife und lässt sich
als}und dann'' übersetzen, vgl. Kapitel~\ref{sec-pipe}). Übersetzen wir
die Syntax ins Deutsche:

\begin{verbatim}
Nimm den Datensatz mariokart *und dann*
  wähle die Spalte cond *und dann*
  zeichne ein Balkendiagramm.
\end{verbatim}

\begin{exercise}[Spalten wählen für das
Balkendiagramm]\protect\hypertarget{exr-de1}{}\label{exr-de1}

Hätten wir andere Spalten ausgewählt, so würde das Balkendiagramm die
Verteilung jener Variablen zeigen. Ja, Sie können auch mehrere Variablen
auf einmal auswählen. Probieren Sie das doch mal aus!

\end{exercise}

\begin{exercise}[Visualisieren Sie die Verteilung von
\texttt{stock\_photo}!]\protect\hypertarget{exr-balken}{}\label{exr-balken}

~

\begin{Shaded}
\begin{Highlighting}[]
\NormalTok{mariokart }\SpecialCharTok{|\textgreater{}} 
  \FunctionTok{select}\NormalTok{(stock\_photo) }\SpecialCharTok{|\textgreater{}} 
  \FunctionTok{plot\_bar}\NormalTok{()}
\end{Highlighting}
\end{Shaded}

\end{exercise}

\subsection{Verteilung: quantitative
Variable}\label{verteilung-quantitative-variable}

\subsubsection{Histogramm}\label{histogramm}

Bei einer quantitativen Variablen mit vielen Ausprägungen wäre ein
Balkendiagramm nicht so aussagekräftig, s.
Abbildung~\ref{fig-balken-hist} (links). Es gibt einfach zu viele
Ausprägungen.

Die Lösung: Wir reduzieren die Anzahl der Ausprägungen, in dem wir auf
ganze Dollar runden. Oder, um noch weniger Ausprägungen zu bekommen,
können wir einfach Gruppen definieren, z.B.

\begin{itemize}
\tightlist
\item
  Gruppe 1: 0-5 Dollar
\item
  Gruppe 2: 6-10 Dollar
\item
  Gruppe 2: 11-15 Dollar \ldots{}
\end{itemize}

In Abbildung~\ref{fig-balken-hist} (rechts) sind z.B. die Ausprägungen
des Verkaufspreis (\texttt{total\_pr}) in in Gruppen der Breite von 5
Dollar aufgeteilt worden. Zusätzlich sind noch die einzelnen Werte als
schwarze Punkte gezeigt.

\begin{figure}[H]

\begin{minipage}{0.50\linewidth}

\centering{

\includegraphics[width=1\linewidth,height=\textheight,keepaspectratio]{040-verbildlichen_files/figure-pdf/fig-balken-hist-1.pdf}

}

\subcaption{\label{fig-balken-hist-1}Balkendiagramm}

\end{minipage}%
%
\begin{minipage}{0.50\linewidth}

\centering{

\includegraphics[width=1\linewidth,height=\textheight,keepaspectratio]{040-verbildlichen_files/figure-pdf/fig-balken-hist-2.pdf}

}

\subcaption{\label{fig-balken-hist-2}Histogramm}

\end{minipage}%

\caption{\label{fig-balken-hist}Balkendiagramm vs.~Histogramm für den
Gesamtpreis (\texttt{total\_pr})}

\end{figure}%

\begin{definition}[Histogramm]\protect\hypertarget{def-histogramm}{}\label{def-histogramm}

Ein Histogramm ist ein Diagramm zur Darstellung der
Häufigkeitsverteilung einer quantitativen Variablen. Die Daten werden in
Gruppen (Klassen) eingeteilt, die dann durch einen Balken (pro Klasse)
dargestellt sind. Die Höhe der Balken zeigt die Häufigkeit der Daten in
dieser Gruppe/in diesem Balken (bei konstanter Balkenbreite).

\end{definition}

Es gibt keine klare Regel, in wie viele Balken ein Histogramm gegliedert
sein sollte. Nur: Es sollten nicht sehr viele und nicht sehr wenig sein,
s. Abbildung~\ref{fig-zu-wenig-viele} links bzw.
Abbildung~\ref{fig-zu-wenig-viele}, rechts.

\begin{figure}[H]

\begin{minipage}{0.50\linewidth}

\centering{

\includegraphics[width=1\linewidth,height=\textheight,keepaspectratio]{040-verbildlichen_files/figure-pdf/fig-zu-wenig-viele-1.pdf}

}

\subcaption{\label{fig-zu-wenig-viele-1}Zu viele Gruppen (Balken)}

\end{minipage}%
%
\begin{minipage}{0.50\linewidth}

\centering{

\includegraphics[width=1\linewidth,height=\textheight,keepaspectratio]{040-verbildlichen_files/figure-pdf/fig-zu-wenig-viele-2.pdf}

}

\subcaption{\label{fig-zu-wenig-viele-2}Zu wenige Gruppen (Balken)}

\end{minipage}%

\caption{\label{fig-zu-wenig-viele}Nicht zu wenig und nicht zu viele
Balken im Histogramm}

\end{figure}%

Zur Erstellung eines Histogramms können Sie die Syntax
Listing~\ref{lst-de2} nützen, vgl. Abbildung~\ref{fig-de-hist-density},
links.

\begin{codelisting}

\caption{\label{lst-de2}Syntax zur Erstellung eines Histogramms}

\centering{

\begin{Shaded}
\begin{Highlighting}[]
\NormalTok{mariokart }\SpecialCharTok{\%\textgreater{}\%} 
  \FunctionTok{select}\NormalTok{(total\_pr) }\SpecialCharTok{\%\textgreater{}\%} 
  \FunctionTok{filter}\NormalTok{(total\_pr }\SpecialCharTok{\textless{}} \DecValTok{100}\NormalTok{) }\SpecialCharTok{\%\textgreater{}\%}  \CommentTok{\# ohne Extremwerte}
  \FunctionTok{plot\_histogram}\NormalTok{()}
\end{Highlighting}
\end{Shaded}

}

\end{codelisting}%

\begin{figure}[H]

\begin{minipage}{0.50\linewidth}

\centering{

\includegraphics[width=1\linewidth,height=\textheight,keepaspectratio]{040-verbildlichen_files/figure-pdf/fig-de-hist-density-1.pdf}

}

\subcaption{\label{fig-de-hist-density-1}Histogramm}

\end{minipage}%
%
\begin{minipage}{0.50\linewidth}

\centering{

\includegraphics[width=1\linewidth,height=\textheight,keepaspectratio]{040-verbildlichen_files/figure-pdf/fig-de-hist-density-2.pdf}

}

\subcaption{\label{fig-de-hist-density-2}Dichtediagramm}

\end{minipage}%

\caption{\label{fig-de-hist-density}Eine stetige Verteilung
verbildlichen}

\end{figure}%

\begin{exercise}[Visualisieren Sie die Verteilung von \texttt{ship\_pr}
anhand eines
Histogramms!]\protect\hypertarget{exr-histo-ship-pr}{}\label{exr-histo-ship-pr}

~

\begin{Shaded}
\begin{Highlighting}[]
\NormalTok{mariokart }\SpecialCharTok{|\textgreater{}} 
  \FunctionTok{select}\NormalTok{(ship\_pr) }\SpecialCharTok{|\textgreater{}} 
  \FunctionTok{plot\_histogram}\NormalTok{()}
\end{Highlighting}
\end{Shaded}

\end{exercise}

\subsubsection{Dichtediagramm}\label{dichtediagramm}

Abbildung~\ref{fig-balken-total-pr-hist-dens} fügt zum Histogramm ein
\emph{Dichtediagramm} hinzu (durchgezogene Linie). Ein Dichtediagramm
ähnelt einem \enquote{glattgeschmirgeltem} Histogramm.

\begin{definition}[Dichtediagramm]\protect\hypertarget{def-dichtediagramm}{}\label{def-dichtediagramm}

Ein Dichtediagramm visualisiert die Verteilung einer stetigen Variablen.
Im Gegensatz zum Histogramm wird der Verlauf der Kurve geglättet, so
kann Rauschen (Zufallsschwankung) besser ausgeblendet werden. (Mit
\emph{Dichte} ist die Anzahl der Beobachtungen pro Einheit der Variablen
auf der X-Achse gemeint.)

\end{definition}

\begin{figure}[H]

\centering{

\includegraphics[width=0.75\linewidth,height=\textheight,keepaspectratio]{040-verbildlichen_files/figure-pdf/fig-balken-total-pr-hist-dens-1.pdf}

}

\caption{\label{fig-balken-total-pr-hist-dens}Histogramm (graue Balken)
und Dichtediagramm (orange Linie) für \texttt{total\_pr}}

\end{figure}%

\begin{exercise}[]\protect\hypertarget{exr-plot-density}{}\label{exr-plot-density}

Erstellen Sie das Diagramm Abbildung~\ref{fig-de-hist-density}, rechtes
Teildiagramm!\footnote{Grob gesagt:
  \texttt{mariokart\ \%\textgreater{}\%\ plot\_density()}.}\(\square\)

\end{exercise}

\subsubsection{Eigenschaften von
Verteilungen}\label{eigenschaften-von-verteilungen}

Verteilungen unterscheiden sich z.B. einerseits in ihrem
\enquote{typischen} oder \enquote{mittleren} Wert (vgl.
Kapitel~\ref{sec-lage}) und anderseits in ihrer Streuung (vgl.
Kapitel~\ref{sec-streuung}.) (Diagramme von) Verteilungen können
symmetrisch oder schief (nicht symmetrisch) sein, s.
Abbildung~\ref{fig-symm-schief}.

\begin{figure}[H]

\begin{minipage}{0.50\linewidth}

\centering{

\includegraphics[width=1\linewidth,height=\textheight,keepaspectratio]{040-verbildlichen_files/figure-pdf/fig-symm-schief-1.pdf}

}

\subcaption{\label{fig-symm-schief-1}Symmetrisch (Normal)}

\end{minipage}%
%
\begin{minipage}{0.50\linewidth}

\centering{

\includegraphics[width=1\linewidth,height=\textheight,keepaspectratio]{040-verbildlichen_files/figure-pdf/fig-symm-schief-2.pdf}

}

\subcaption{\label{fig-symm-schief-2}Schief}

\end{minipage}%

\caption{\label{fig-symm-schief}Symmetrische vs.~schiefe Verteilung,
verbildlicht}

\end{figure}%

Abbildung~\ref{fig-plot-distribs} zeigt verschiedene Formen von
Verteilungen. \enquote{Bimodal} meint \enquote{zweigipflig} und
\enquote{multimodal} entsprechend \enquote{mehrgipflig}.\footnote{Quelle:
  ifes/FOM Hochschule,
  \url{https://github.com/FOM-ifes/VL-Vorlesungsfolien}}

\begin{figure}[H]

\centering{

\includegraphics[width=1\linewidth,height=\textheight,keepaspectratio]{040-verbildlichen_files/figure-pdf/fig-plot-distribs-1.pdf}

}

\caption{\label{fig-plot-distribs}Verschiedene Verteilungsformen}

\end{figure}%

\begin{exercise}[Verteilungform von
\texttt{total\_pr}?]\protect\hypertarget{exr-verteilungsform-total-pr}{}\label{exr-verteilungsform-total-pr}

Benennen Sie die am besten passende Verteilungsform für die Variable
\texttt{total\_pr}.

\textbf{Lösung}

Die Verteilung ist rechtsschief.

\end{exercise}

\subsection{Normalverteilung}\label{normalverteilung}

Eine Normalverteilung ist eine bestimmte Art von Verteilung einer
quantitativen Variablen. Aber sie ist besonders wichtig, und ist daher
hier herausgestellt. Eine Normalverteilung sehen Sie in
Abbildung~\ref{fig-symm-schief}, links. Die Normalverteilung ist in der
Statistik von hoher Bedeutung, da sich diese Verteilung unter (recht
häufigen) Bedingungen zwangsläufig ergeben muss. Sie hat u.a. folgende
Eigenschaften:

\begin{itemize}
\tightlist
\item
  symmetrisch
\item
  glockenförmig
\item
  stetig
\item
  eingipflig (unimodal)
\item
  Mittelwert, Median und Modus sind identisch
\end{itemize}

\begin{example}[]\protect\hypertarget{exm-norm}{}\label{exm-norm}

Beispiele für normalverteilte Variablen sind Körpergröße von Männern
oder Frauen, IQ-Werte, Prüfungsergebnisse, Messfehler, Lebensdauer von
Glühbirnen, Gewichte von Brotlaiben, Milchproduktion von Kühen,
Brustumfang schottischer Soldaten (Lyon, 2014).\(\square\)

\end{example}

\begin{definition}[Normalverteilung]\protect\hypertarget{def-norm}{}\label{def-norm}

Eine Normalverteilung ist eine spezielle Art von Verteilung einer
quantitativen Variablen. Sie ist symmetrisch, glockenförmig, stetig,
unimodal und hat Mittelwert, Median und Modus identisch. Sie lässt sich
durch zwei Parameter vollständig beschreiben: Mittelwert (\(\mu\)) und
Streuung (\(\sigma\)). \(\square\)

\end{definition}

\begin{definition}[Entstehung einer
Normalverteilung]\protect\hypertarget{def-normal-galton}{}\label{def-normal-galton}

Wenn sich eine Variable \(X\) als Summe mehrerer, unabhängiger, etwa
gleich starker Summanden, dann kann man erwarten, dass sich diese
Variable \(X\) tendenziell normalverteilt. \(\square\)

\end{definition}

\begin{figure}

\begin{minipage}{0.80\linewidth}
Dieses Phänomen kann man gut anhand des
\href{https://www.youtube.com/watch?v=3m4bxse2JEQ}{Galton-Bretts}
veranschaulichen.\end{minipage}%
%
\begin{minipage}{0.20\linewidth}

\begin{center}
\includegraphics[width=0.75\linewidth,height=\textheight,keepaspectratio]{040-verbildlichen_files/figure-pdf/unnamed-chunk-17-1.pdf}
\end{center}

\end{minipage}%

\end{figure}%

\begin{tcolorbox}[enhanced jigsaw, bottomtitle=1mm, leftrule=.75mm, breakable, title=\textcolor{quarto-callout-important-color}{\faExclamation}\hspace{0.5em}{Parameter der Normalverteilung}, bottomrule=.15mm, titlerule=0mm, left=2mm, opacityback=0, colframe=quarto-callout-important-color-frame, rightrule=.15mm, colback=white, coltitle=black, toprule=.15mm, toptitle=1mm, colbacktitle=quarto-callout-important-color!10!white, arc=.35mm, opacitybacktitle=0.6]

Eine Normalverteilung lässt sich exakt beschreiben anhand zweier
Parameter: ihres zentralen Werts (Mittelwerts), \(\mu\), und ihrer
Streuung (Standardabweichung), \(\sigma\). \(\square\)

\end{tcolorbox}

Kennt man diese beiden Parameter, so kann man einfach angeben, welcher
Anteil der Fläche sich in einem bestimmten Bereich befindet, s.
Abbildung~\ref{fig-norm-perc}.

Davon leitet sich die \enquote{68-95-99.7-Prozentregel} ab:

\begin{itemize}
\tightlist
\item
  \(68\,\%\) der Werte im Bereich \(\mu\pm 1 \cdot \sigma\)
\item
  \(95\,\%\) der Werte im Bereich \(\mu\pm 2 \cdot \sigma\)
\item
  \(99{,}7\,\%\) der Werte im Bereich \(\mu\pm 3 \cdot \sigma\)
\end{itemize}

\begin{figure}[H]

\centering{

\includegraphics[width=0.5\linewidth,height=\textheight,keepaspectratio]{img/Standard_deviation_diagram_micro.svg.png}

}

\caption{\label{fig-norm-perc}Die Flächeninhalte
(Wahrscheinlichkeitsmasse) einer Normalverteilung in Abhängigkeit der
SD-Einheiten (Ainali, 2007)}

\end{figure}%

\section{Zusammenhänge
verbildlichen}\label{zusammenhuxe4nge-verbildlichen}

\subsection{Zusammenhang: nominale
Variablen}\label{zusammenhang-nominale-variablen}

\begin{example}[Beispiele für Zusammenhänge bei nominalen
Variablen]\protect\hypertarget{exm-nom-zshg}{}\label{exm-nom-zshg}

~

\begin{itemize}
\tightlist
\item
  Hängt Berufserfolg (Führungskraft ja/nein) mit dem Geschlecht
  zusammen?
\item
  Hängt der Beruf des Vaters mit dem Schulabschluss des Kindes (Abitur,
  Realschule, Mittelschule) zusammen?
\item
  Gibt es einen Zusammenhang zwischen Automarke und politische Präferenz
  einer Partei? \(\square\)
\end{itemize}

\end{example}

Sagen wir, Sie arbeiten immer noch beim Online-Auktionshaus und Sie
fragen sich, ob ein Produktfoto wohl primär bei neuwertigen Produkten
beiliegt, aber nicht bei gebrauchten? Dazu betrachten Sie wieder die
\texttt{mariokart}-Daten, s. Abbildung~\ref{fig-zshg-nom1}.

\begin{figure}[H]

\begin{minipage}{0.45\linewidth}

\centering{

\includegraphics[width=1\linewidth,height=\textheight,keepaspectratio]{040-verbildlichen_files/figure-pdf/fig-zshg-nom1-1.pdf}

}

\subcaption{\label{fig-zshg-nom1-1}Es findet sich ein Zusammenhang von
Foto und Zustand in den Daten}

\end{minipage}%
%
\begin{minipage}{0.10\linewidth}
~\end{minipage}%
%
\begin{minipage}{0.45\linewidth}

\centering{

\includegraphics[width=1\linewidth,height=\textheight,keepaspectratio]{040-verbildlichen_files/figure-pdf/fig-zshg-nom1-2.pdf}

}

\subcaption{\label{fig-zshg-nom1-2}Es findet sich (fast) kein
Zusammenhang von \texttt{wheel} und Foto in den Daten}

\end{minipage}%

\caption{\label{fig-zshg-nom1}Zusammenhang zwischen nominalskalierten
Variablen verbildlichen}

\end{figure}%

Tatsächlich: Es findet sich ein Zusammenhang zwischen der Tatsache, ob
dem versteigerten Produkt ein Foto bei lag und ob es neuwertig oder
gebraucht war (Abbildung~\ref{fig-zshg-nom1}, links). Bei neuen Spielen
war fast immer (ca. 90\%) ein Foto dabei; bei gebrauchten Spielen
immerhin bei gut der Hälfte der Fälle.

Anders sieht es aus für die Frage, ob ein (oder mehrere) Lenkräder dem
Spiel beilagen (oder nicht) in Zusammenhang mit der Fotofrage Hier gab
es fast keinen Unterschied zwischen neuen und alten Spielen, was die
Frage nach \enquote{Foto des Produkts dabei} betraf
(Abbildung~\ref{fig-zshg-nom1}, rechts), der Anteil betrug jeweils ca.
70\%. Das zeigt, dass es keinen Zusammenhang zwischen Foto und
Neuwertigkeit des Spiels gibt (laut unseren Daten).

Anders gesagt: Unterscheiden sich die \enquote{Füllhöhe} in den
Diagrammen, so gibt es einen Unterschied hinsichtlich \enquote{Foto ist
dabei} zwischen den beiden Gruppen (linker vs.~rechter Balken).
Unterscheiden sich die Anteile in den Gruppen (neuwertige vs.~gebrauchte
Spiele), so spielt z.B. die Variable \enquote{Foto dabei} offenbar eine
Rolle. Dann hängen Neuwertigkeit und \enquote{Foto dabei} also zusammen!

So können Sie sich in R ein gefülltes Balkendiagramm ausgeben lassen, s.
Abbildung~\ref{fig-de-bar-filled}. Diese Darstellung eignet sich, um
Zusammenhänge zwischen zwei zweistufigen nominal skalierten Variablen zu
verbildlichen. Die verschiedenen Werte der Füllfarbe werden den Stufen
der Variablen \texttt{cond} zugewiesen, s. Listing~\ref{lst-plot-bar}.

\begin{codelisting}

\caption{\label{lst-plot-bar}R-Syntax für ein gefülltes Balkendiagramm}

\centering{

\begin{Shaded}
\begin{Highlighting}[]
\NormalTok{mariokart }\SpecialCharTok{\%\textgreater{}\%} 
  \FunctionTok{select}\NormalTok{(cond, stock\_photo) }\SpecialCharTok{\%\textgreater{}\%} 
  \FunctionTok{plot\_bar}\NormalTok{(}\AttributeTok{by =} \StringTok{"cond"}\NormalTok{)  }\CommentTok{\# aus dem Paket DataExplorer}
\end{Highlighting}
\end{Shaded}

}

\end{codelisting}%

\begin{figure}[H]

\centering{

\includegraphics[width=1\linewidth,height=\textheight,keepaspectratio]{040-verbildlichen_files/figure-pdf/fig-de-bar-filled-1.pdf}

}

\caption{\label{fig-de-bar-filled}Ein gefülltes Balkendiagramm zur
Untersuchung eines Zusammenhangs zwischen nominalskalierter Variablen}

\end{figure}%

\begin{tcolorbox}[enhanced jigsaw, bottomtitle=1mm, leftrule=.75mm, breakable, title=\textcolor{quarto-callout-note-color}{\faInfo}\hspace{0.5em}{Hinweis}, bottomrule=.15mm, titlerule=0mm, left=2mm, opacityback=0, colframe=quarto-callout-note-color-frame, rightrule=.15mm, colback=white, coltitle=black, toprule=.15mm, toptitle=1mm, colbacktitle=quarto-callout-note-color!10!white, arc=.35mm, opacitybacktitle=0.6]

\emph{Gefüllte Balkendiagramme} eignen sich zur Analyse eines
Zusammenhangs zwischen nominalskalierten Variablen. Allerdings sollte
eine der beiden Variablen nur zwei Ausprägungen aufweisen, sonst sind
die Zusammenhänge nicht mehr so gut zu erkennen. Außerdem sollten die
Balken auf gleiche Länge (100\%) ausgerichtet sein.\(\square\)

\end{tcolorbox}

\begin{exercise}[Zusammenhang
visualisieren]\protect\hypertarget{exr-zsmnhang-cond-wheels}{}\label{exr-zsmnhang-cond-wheels}

Aufgabe Visualisieren Sie den Zusammenhang der beiden nominalen
Variablen \texttt{cond} und \texttt{wheels}!

\textbf{Lösung}

\texttt{wheels} ist als metrische Variable (\texttt{int}: Integer, d.h.
Ganzzahl) formatiert im Datensatz \texttt{mariokart}. Wir müssen Sie
zunächst als Faktorvariable umformatieren, damit R sie als nominal
skalierte Variable erkennt.

\begin{Shaded}
\begin{Highlighting}[]
\NormalTok{mariokart }\SpecialCharTok{|\textgreater{}} 
  \CommentTok{\# Mache aus einer metrischen eine nominale Variable: }
  \FunctionTok{mutate}\NormalTok{(}\AttributeTok{wheels =} \FunctionTok{factor}\NormalTok{(wheels)) }\SpecialCharTok{|\textgreater{}} 
  \FunctionTok{select}\NormalTok{(cond, wheels) }\SpecialCharTok{|\textgreater{}} 
  \FunctionTok{plot\_bar}\NormalTok{(}\AttributeTok{by =} \StringTok{"cond"}\NormalTok{)}
\end{Highlighting}
\end{Shaded}

\begin{center}
\includegraphics[width=1\linewidth,height=\textheight,keepaspectratio]{040-verbildlichen_files/figure-pdf/unnamed-chunk-20-1.pdf}
\end{center}

\end{exercise}

\subsection{Zusammenhang: metrisch}\label{sec-zshg-metr}

Den (etwaigen) Zusammenhang zweier metrischer Variablen kann man mit
einem \emph{Streudiagramm} visualisieren (engl. scatterplot).
Abbildung~\ref{fig-streu1} links untersucht den Zusammenhang des
Einstiegpreises (X-Achse) und Abschlusspreises (Y-Achse) von Geboten bei
Versteigerungen des Computerspiels Mariokart. In dem Diagramm ist eine
\enquote{Trendgerade} (Regressionsgerade), um die Art des Zusammenhangs
besser zu verdeutlichen. Die Trendgerade steigt an (von links nach
recht). Daraus kann man schließen: Es handelt sich um einen
\emph{gleichsinnigen} (positiven) Zusammenhang: Je höher der Startpreis,
desto \emph{höher} der Abschlusspreis, zumindest tendenziell. Diese
Gerade liegt \enquote{mittig} in den Daten (wir definieren dies später
genauer). Diese Trendgerade gibt Aufschluss über \enquote{typische}
Werte: Welcher Y-Wert ist \enquote{typisch} für einen bestimmten X-Wert?

Abbildung~\ref{fig-streu1} rechts untersucht den Zusammenhang zwischen
Anzahl der Gebote (X-Achse) und Abschlusspreises (Y-Achse). Es handelt
sich um einen negativen Zusammenhang: Je mehr Gebote, desto
\emph{geringer} der Abschlusspreis. Das erkennt man an der sinkenden
Trendgeraden.

Die Ellipse zeigt an, wie eng die Daten um die Trendgerade streuen.
Daraus kann man ableiten, wie stark der Absolutwert des Zusammenhangs
ist, vgl. Abbildung~\ref{fig-cors}.

\begin{figure}[H]

\begin{minipage}{0.50\linewidth}

\centering{

\includegraphics[width=1\linewidth,height=\textheight,keepaspectratio]{040-verbildlichen_files/figure-pdf/fig-streu1-1.pdf}

}

\subcaption{\label{fig-streu1-1}positiver, mittelstarker Zusammenhang}

\end{minipage}%
%
\begin{minipage}{0.50\linewidth}

\centering{

\includegraphics[width=1\linewidth,height=\textheight,keepaspectratio]{040-verbildlichen_files/figure-pdf/fig-streu1-2.pdf}

}

\subcaption{\label{fig-streu1-2}negativer, eher schwacher Zusammenhang}

\end{minipage}%

\caption{\label{fig-streu1}Streudiagramm zur Darstellung eines
Zusammenhangs zweier metrischer Variablen}

\end{figure}%

\begin{definition}[Linearer
Zusammenhang]\protect\hypertarget{def-lin-zshg}{}\label{def-lin-zshg}

Lässt sich die Beziehung zwischen zwei Variablen mit einer Gerade
visualisieren, so spricht man von einem linearen Zusammenhang. Ändert
man eine der beiden Variablen um einen bestimmten Wert (z.B. 1), so
ändert sich die andere um einen proportionalen Wert (z.B. 0.5).
\(\square\)

\end{definition}

Natürlich könnte man auch nicht-lineare Zusammenhänge untersuchen, aber
der Einfachheit halber konzentrieren wir uns hier mit linearen;
Beispiele für nicht-lineare Zusammenhänge sind in
Abbildung~\ref{fig-nonlinear} zu sehen.

\begin{figure}[H]

\centering{

\includegraphics[width=1\linewidth,height=\textheight,keepaspectratio]{040-verbildlichen_files/figure-pdf/fig-nonlinear-1.pdf}

}

\caption{\label{fig-nonlinear}Beispiele nichtlinearer Zusammenhänge}

\end{figure}%

\begin{definition}[Richtung und Stärke eines
Zusammenhang]\protect\hypertarget{def-zshg}{}\label{def-zshg}

\emph{Gleichsinnige} (positive) Zusammenhänge erkennt man an
\emph{aufsteigenden} Trendgeraden \(\nearrow\); \emph{gegensinnige}
(negative) Zusammenhänge an \emph{absteigenden} Trendgeraden
\(\searrow\). \(\square\)

\end{definition}

Starke Zusammenhänge erkennt man an schmalen Ellipsen
(\enquote{Baguette} ); schwache Zusammenhänge an breiten Ellipsen
(\enquote{Torte} ). Abbildung~\ref{fig-cors} bietet einen Überblick über
verschiedene Beispiele von Richtung und Stärke von
Zusammenhängen.\footnote{Quelle: Aufbauend auf FOM/ifes, Autor: Norman
  Markgraf}

\begin{figure}[H]

\centering{

\includegraphics[width=0.75\linewidth,height=\textheight,keepaspectratio]{040-verbildlichen_files/figure-pdf/fig-cors-1.pdf}

}

\caption{\label{fig-cors}Lineare Zusammenhänge verschiedener Stärke und
Richtung}

\end{figure}%

In Abbildung~\ref{fig-cors} ist für jedes Teildiagramm eine Zahl
angegeben: der \emph{Korrelationskoeffizient}. Diese Statistik
quantifiziert Richtung und Stärke des Zusammenhangs (mehr dazu in Kap.
\textbf{?@sec-zusammenhaenge}). Ein positives Vorzeichen steht für einen
positiven Zusammenhang, ein negatives Vorzeichen für einen negativen
Zusammenhang. Der (Absolut-)Wert gibt die Stärke des linearen
Zusammenhangs an (Cohen, 1992):

\begin{itemize}
\tightlist
\item
  ±0: Kein Zusammenhang
\item
  ±0.1: schwacher Zusammenhang
\item
  ±0.3: mittlerer Zusammenhang
\item
  ±0.5: starker Zusammenhang
\item
  ±1: perfekter Zusammenhang
\end{itemize}

Abbildung~\ref{fig-cors2} hat die gleiche Aussage wie
Abbildung~\ref{fig-cors}, ist aber plakativer, indem \emph{Stärke}
(schwach, stark) und \emph{Richtung} (positiv, negativ)
gegenübergestellt sind.

\begin{figure}[H]

\centering{

\includegraphics[width=0.75\linewidth,height=\textheight,keepaspectratio]{040-verbildlichen_files/figure-pdf/fig-cors2-1.pdf}

}

\caption{\label{fig-cors2}Überblick über starke vs.~schwache bzw.
positive vs.~negative Zusammenhänge}

\end{figure}%

Man sieht in Abbildung~\ref{fig-cors} und Abbildung~\ref{fig-cors2},
dass ein \emph{negativer} Korrelationskoeffizient mit einer
\emph{absinkenden} Trendgerade (synonym: Regressionsgerade; blaue Linie)
einhergeht. Umgekehrt geht ein \emph{positiver} Trend mit einer
\emph{ansteigenden} Trendgerade einher. Zweitens erkennt man, dass
\emph{starke} Zusammenhänge mit einer \emph{schmaler} Ellipse
einhergehen und \emph{schwache} Zusammenhänge mit einer \emph{breiten}
Ellipse einhergehen.

\begin{example}[]\protect\hypertarget{exm-scatter}{}\label{exm-scatter}

Sie arbeiten nach wie vor bei einem Online-Auktionshaus, und manchmal
gehört Datenanalyse zu Ihren Aufgaben. Daher interessiert Sie, ob welche
Variablen mit dem Abschlusspreis (\texttt{total\_pr}) im Datensatz
\texttt{mariokart} zusammenhängen. Sie verbildlichen die Daten mit R,
und zwar nutzen Sie das Paket \texttt{DataExplorer}. Außerdem müssen wir
noch die Daten importieren, falls noch nicht getan, s.
Listing~\ref{lst-mariokart}.

So, jetzt kann die eigentliche Arbeit losgehen. Da Sie sich nur auf
metrische Variablen konzentrieren wollen, wählen Sie (mit
\texttt{select}) nur diese Variablen aus. Dann weisen Sie R an, einen
Scatterplot zu malen (\texttt{plot\_scatterplot}) und zwar jeweils den
Zusammenhang einer der gewählten Variablen mit dem Abschlusspreis
(\texttt{total\_pr}), da das die Variable ist, die Sie primär
interessiert. Das Ergebnis sieht man in
Abbildung~\ref{fig-mario-scatter} bzw. Listing~\ref{lst-scatterplot}.
\(\square\)

\end{example}

\begin{codelisting}

\caption{\label{lst-scatterplot}Streudiagramm erstellen mit dem R-Paket
\enquote*{DataExplorer}}

\centering{

\begin{Shaded}
\begin{Highlighting}[]
\NormalTok{mariokart }\SpecialCharTok{\%\textgreater{}\%} 
  \FunctionTok{select}\NormalTok{(duration, n\_bids, start\_pr,}
\NormalTok{         ship\_pr, total\_pr, }
\NormalTok{         seller\_rate, wheels) }\SpecialCharTok{\%\textgreater{}\%} 
  \FunctionTok{plot\_scatterplot}\NormalTok{(}\AttributeTok{by =} \StringTok{"total\_pr"}\NormalTok{)}
\end{Highlighting}
\end{Shaded}

}

\end{codelisting}%

\begin{figure}[H]

\centering{

\includegraphics[width=1\linewidth,height=\textheight,keepaspectratio]{040-verbildlichen_files/figure-pdf/fig-mario-scatter-1.pdf}

}

\caption{\label{fig-mario-scatter}Der Zusammenhang metrischer Variablen
mit Abschlusspreis}

\end{figure}%

Aha\ldots{} Was sagt uns das Bild? Hm. Es scheint einige Extremwerte zu
geben, die dafür sorgen, dass der Rest der Daten recht
zusammengequetscht auf dem Bild erscheint. Vielleicht sollten Sie solche
Extremwerte lieber entfernen? Sie entscheiden sich, nur Verkäufe mit
einem Abschlusspreis von weniger als 100 Dollar anzuschauen
(\texttt{total\_pr\ \textless{}\ 100}). Das Ergebnis ist in
Abbildung~\ref{fig-mario-scatter2} zu sehen.

\begin{Shaded}
\begin{Highlighting}[]
\NormalTok{mariokart\_no\_extreme }\OtherTok{\textless{}{-}}
\NormalTok{  mariokart }\SpecialCharTok{\%\textgreater{}\%} 
  \FunctionTok{filter}\NormalTok{(total\_pr }\SpecialCharTok{\textless{}} \DecValTok{100}\NormalTok{)}

\NormalTok{mariokart\_no\_extreme }\SpecialCharTok{\%\textgreater{}\%} 
  \FunctionTok{select}\NormalTok{(duration, n\_bids, start\_pr, }
\NormalTok{         ship\_pr, total\_pr, }
\NormalTok{         seller\_rate, wheels) }\SpecialCharTok{\%\textgreater{}\%} 
  \FunctionTok{plot\_scatterplot}\NormalTok{(}\AttributeTok{by =} \StringTok{"total\_pr"}\NormalTok{)}
\end{Highlighting}
\end{Shaded}

\begin{figure}[H]

\centering{

\includegraphics[width=1\linewidth,height=\textheight,keepaspectratio]{040-verbildlichen_files/figure-pdf/fig-mario-scatter2-1.pdf}

}

\caption{\label{fig-mario-scatter2}Der Zusammenhang metrischer Variablen
mit Abschlusspreis}

\end{figure}%

Ohne Extremwerte schält sich ein deutlicheres Bild
(Abbildung~\ref{fig-mario-scatter2}) hervor: Startpreis
(\texttt{start\_pr}) und Anzahl der Räder (\texttt{wheels}) scheinen am
stärksten mit dem Abschlusspreis zusammenzuhängen.

Das Argument \texttt{by\ =\ "total\_pr"} bei \texttt{plot\_scatterplot}
weist R an, als Y-Variable stets \texttt{total\_pr} zu verwenden. Alle
übrigen Variablen kommen jeweils einmal als X-Variable vor.\(\square\)

\begin{exercise}[Aufgabe Visualisieren Sie den Zusammenhang der beiden
metrischen Variablen \texttt{start\_pr} und
\texttt{total\_pr}.]\protect\hypertarget{exr-zsmnhang-metrisch}{}\label{exr-zsmnhang-metrisch}

Verwenden Sie den Datensatz ohne Extremwerte wie oben definiert.

\textbf{Lösung}

\begin{Shaded}
\begin{Highlighting}[]
\NormalTok{mariokart\_no\_extreme }\SpecialCharTok{|\textgreater{}} 
  \FunctionTok{select}\NormalTok{(start\_pr, total\_pr) }\SpecialCharTok{|\textgreater{}} 
  \FunctionTok{plot\_scatterplot}\NormalTok{(}\AttributeTok{by =} \StringTok{"total\_pr"}\NormalTok{)}
\end{Highlighting}
\end{Shaded}

\begin{center}
\includegraphics[width=1\linewidth,height=\textheight,keepaspectratio]{040-verbildlichen_files/figure-pdf/unnamed-chunk-28-1.pdf}
\end{center}

\end{exercise}

\section{Unterschiede verbildlichen}\label{unterschiede-verbildlichen}

\subsection{Unterschied: nominale
Variablen}\label{unterschied-nominale-variablen}

Gute Nachrichten: Für nominale Variablen bieten sich Balkendiagramme
sowohl zur Darstellung von Zusammenhängen als auch von Unterschieden an.
Genau genommen zeigt ja Abbildung~\ref{fig-zshg-nom1} (links) den
\emph{Unterschied} zwischen neuen und gebrauchten Spielen hinsichtlich
der Frage, ob Photos beiliegen. Und wie man in
Abbildung~\ref{fig-zshg-nom1} sieht, ist der Anteil der Spiele mit Foto
bei den neuen Spielen höher als bei gebrauchten Spielen.

\subsection{Unterschied: quantitative
Variablen}\label{unterschied-quantitative-variablen}

Eine typische Analysefrage ist, ob sich zwei Gruppen hinsichtlich einer
metrischen Zielvariablen deutlich unterscheiden. Genauer gesagt
untersucht man z.B. oft, ob sich die Mittelwerte der beiden Gruppen
zwischen der Zielvariablen deutlich unterscheiden. Das hört sich
abstrakt an? Am besten wir schauen uns einige Beispiele an, s.
Abbildung~\ref{fig-compare-groups1}.

\begin{figure}[H]

\begin{minipage}{0.50\linewidth}

\centering{

\includegraphics[width=1\linewidth,height=\textheight,keepaspectratio]{040-verbildlichen_files/figure-pdf/fig-compare-groups1-1.pdf}

}

\subcaption{\label{fig-compare-groups1-1}Histogramm pro Gruppe}

\end{minipage}%
%
\begin{minipage}{0.50\linewidth}

\centering{

\includegraphics[width=1\linewidth,height=\textheight,keepaspectratio]{040-verbildlichen_files/figure-pdf/fig-compare-groups1-2.pdf}

}

\subcaption{\label{fig-compare-groups1-2}Boxplot pro Gruppe}

\end{minipage}%

\caption{\label{fig-compare-groups1}Unterschiede zwischen zwei Gruppen:
Metrische Y-Variable, nominale X-Variable}

\end{figure}%

Das linke Teildiagramm von Abbildung~\ref{fig-compare-groups1} zeigt das
Histogramm von \texttt{total\_pr}, getrennt für neue und gebrauchte
Spiele, vgl. Abbildung~\ref{fig-de-hist-density}. Das rechte
Teildiagramm zeigt die gleichen Verteilungen, aber mit einer
vereinfachten, groberen Darstellungsfrom, den \emph{Boxplot}.\footnote{Übrigens:
  Freunde lassen Freunde nicht Balkendiagramme verwenden, um Mittelwerte
  darzustellen:
  \url{https://github.com/cxli233/FriendsDontLetFriends\#1-friends-dont-let-friends-make-bar-plots-for-means-separation}.}
Was ein \enquote{deutlicher} (\enquote{substanzieller},
\enquote{bedeutsamer}, \enquote{relevanter} oder \enquote{(inhaltlich)
signifikanter}) Zusammenhang ist, ist keine statistische, sondern
inhaltliche Frage, die man mit Sachverstand zum Forschungsgegenstand
beantworten muss.

\begin{definition}[Boxplot]\protect\hypertarget{def-boxplot}{}\label{def-boxplot}

Der Boxplot ist eine Vereinfachung bzw. eine Zusammenfassung eines
Histograms. Damit stellt der Boxplot auch eine Verteilung (einer
metrischen Variablen) dar.\(\square\)

\end{definition}

In Abbildung~\ref{fig-hist-to-box} sieht man die \enquote{Übersetzung}
von Histogramm (oben) zu einem Boxplot (unten). Ob der Boxplot
horizontal oder vertikal steht, ist Ihrem Geschmack überlassen.

\begin{figure}[H]

\centering{

\includegraphics[width=1\linewidth,height=\textheight,keepaspectratio]{040-verbildlichen_files/figure-pdf/fig-hist-to-box-1.pdf}

}

\caption{\label{fig-hist-to-box}Übersetzung eines Histogramms zu einem
Boxplot}

\end{figure}%

Schauen wir uns die \enquote{Anatomie} des Boxplots näher an:

\begin{enumerate}
\def\labelenumi{\arabic{enumi}.}
\tightlist
\item
  Der \emph{dicke Strich} in der Box zeigt den Median der Verteilung,
  vgl. Kapitel~\ref{sec-median}.
\item
  Die \emph{Enden der Box} zeigen das 1. Quartil (41) bzw. das 3.
  Quartil (54). Damit zeigt die Breite der Box die Streuung der
  Verteilung an, genauer gesagt die Streuung der inneren 50\% der
  Beobachtungen. Je breiter die Box, desto größer die Streuung. Die
  Breite der Box nennt man auch den \emph{Interquartilsabstand} (IQR).
\item
  Die \enquote{\emph{Antennen}} des Boxplots zeigen die Streuung in den
  kleinsten 25\% der Werte (linke Antenne) bzw. die Streuung der größten
  25\% der Werte (rechte Antennen). Je länger die Antenne, desto größer
  die Streuung.
\item
  Falls es aber \emph{Extremwerte} gibt, so sollten die lieber einzeln,
  separat, außerhalb der Antennen gezeigt werden. Daher ist die
  Antennenlänge auf die 1,5-fache Länge der Box beschränkt. Werte die
  außerhalb dieses Bereichs liegen (also mehr als das 1,5-fache der
  Boxlänge von Q3 entfernt sind) werden mittels eines Punktes
  dargestellt.
\item
  Liegt der Median-Strich in der Mitte der Box, so ist die Verteilung
  \emph{symmetrisch} (bezogen auf die inneren 50\% der Werte), liegt der
  Median-Strich nicht in der Mitte der Box, so ist die Verteilung nicht
  symmetrisch (d.h. sie ist \emph{schief}). Gleiches gilt für die
  Antennenlängen: Sind die Antennen gleich lang, so ist der äußere Teil
  der Verteilung symmetrisch, andernfalls schief.
\end{enumerate}

\begin{example}[]\protect\hypertarget{exm-Boxplots}{}\label{exm-Boxplots}

In einer vorherigen Analyse haben Sie den Zusammenhang von
Abschlusspreis und der Anzahl der Lenkräder untersucht. Jetzt möchten
Sie eine sehr ähnliche Fragestellung betrachten: Wie
\emph{unterscheiden} sich die Verkaufspreise je nach Anzahl der
beigelegten Lenkräder? Flink erstellen Sie dazu folgendes Diagramm,
Abbildung~\ref{fig-box-wheels1}, links. Es zeigt die Verteilung des
Abschlusspreises, aufgebrochen nach Anzahl Lenkräder
(\texttt{by\ =\ "wheels"}). \(\square\)

\end{example}

Aber ganz glücklich sind Sie mit dem Diagramm nicht: R hat die Variable
\texttt{wheels} komisch aufgeteilt. Es wäre eigentlich ganz einfach,
wenn R die Gruppen \texttt{0}, \texttt{1}, \texttt{2}, \texttt{3} und
\texttt{4} aufteilen würde. Aber schaut man sich die Y-Achse (im linken
Teildiagramm von Abbildung~\ref{fig-box-wheels1}) an, so erkennt man,
dass R \texttt{wheels} als stetige Zahl betrachtet und nicht in ganze
Zahlen gruppiert. Vielleicht so, dass in jeder Gruppe gleich viele Wert
sind?{]} Aber wir möchten jeden einzelnen Wert von \texttt{wheels} (0,
1, 2, 3, 4) als \emph{Gruppe} verstehen. Mit anderen Worten, wir möchten
\texttt{wheels} als nominale Variable definieren. Das kann man mit dem
Befehle \texttt{factor(wheels)} erreichen (verpackt in \texttt{mutate}),
s. Abbildung~\ref{fig-box-wheels1} rechts.

\begin{Shaded}
\begin{Highlighting}[]
\NormalTok{mariokart\_no\_extreme }\SpecialCharTok{\%\textgreater{}\%} 
  \FunctionTok{select}\NormalTok{(total\_pr, wheels) }\SpecialCharTok{\%\textgreater{}\%} 
  \FunctionTok{plot\_boxplot}\NormalTok{(}\AttributeTok{by =} \StringTok{"wheels"}\NormalTok{)}

\NormalTok{mariokart\_no\_extreme }\SpecialCharTok{\%\textgreater{}\%} 
  \FunctionTok{select}\NormalTok{(total\_pr, wheels) }\SpecialCharTok{\%\textgreater{}\%} 
  \FunctionTok{mutate}\NormalTok{(}\AttributeTok{wheels =} \FunctionTok{factor}\NormalTok{(wheels)) }\SpecialCharTok{\%\textgreater{}\%} 
  \FunctionTok{plot\_boxplot}\NormalTok{(}\AttributeTok{by =} \StringTok{"wheels"}\NormalTok{)}
\end{Highlighting}
\end{Shaded}

\begin{figure}[H]

\begin{minipage}{0.50\linewidth}

\centering{

\includegraphics[width=1\linewidth,height=\textheight,keepaspectratio]{040-verbildlichen_files/figure-pdf/fig-box-wheels1-1.pdf}

}

\subcaption{\label{fig-box-wheels1-1}wheels als metrische Variable}

\end{minipage}%
%
\begin{minipage}{0.50\linewidth}

\centering{

\includegraphics[width=1\linewidth,height=\textheight,keepaspectratio]{040-verbildlichen_files/figure-pdf/fig-box-wheels1-2.pdf}

}

\subcaption{\label{fig-box-wheels1-2}wheels als nominale Variable}

\end{minipage}%

\caption{\label{fig-box-wheels1}Abschlusspreis nach Anzahl von
beigelegten Lenkrädern}

\end{figure}%

Sie schließen aus dem Bild, dass Lenkräder und Preis (positiv)
zusammenhängen. Allerdings scheint es wenig Daten für
\texttt{wheels\ ==\ 4} zu geben. Das prüfen Sie nach:

\begin{Shaded}
\begin{Highlighting}[]
\NormalTok{mariokart\_no\_extreme }\SpecialCharTok{\%\textgreater{}\%} 
  \FunctionTok{count}\NormalTok{(wheels)}
\end{Highlighting}
\end{Shaded}

\begin{longtable}[]{@{}rr@{}}
\toprule\noalign{}
wheels & n \\
\midrule\noalign{}
\endhead
\bottomrule\noalign{}
\endlastfoot
0 & 36 \\
1 & 52 \\
2 & 50 \\
3 & 2 \\
4 & 1 \\
\end{longtable}

Tatsächlich gibt es (in \texttt{mariokart\_no\_extreme}) auch für 3
Lenkräder schon wenig Daten, so dass wir die Belastbarkeit dieses
Ergebnisses skeptisch betrachten sollten.

Übrigens bezeichnet Sie Ihre Chefin nur noch als \enquote{Datengott}.

\begin{exercise}[Visualisieren Sie den Unterschied im Verkaufspreis
zwischen gebrauchten und neuen
Spielen.]\protect\hypertarget{exr-diff-plot}{}\label{exr-diff-plot}

Es gibt mehrere Diagrammtypen, die sich anbieten; mehrere Lösungen sind
also mögliche.

\emph{Lösung}

\begin{Shaded}
\begin{Highlighting}[]
\NormalTok{mariokart\_no\_extreme }\SpecialCharTok{|\textgreater{}} 
  \FunctionTok{select}\NormalTok{(cond, total\_pr) }\SpecialCharTok{|\textgreater{}} 
  \FunctionTok{plot\_boxplot}\NormalTok{(}\AttributeTok{by =} \StringTok{"cond"}\NormalTok{)}
\end{Highlighting}
\end{Shaded}

\end{exercise}

\begin{exercise}[Verkaufspreis im
Vergleich]\protect\hypertarget{exr-diff-plot}{}\label{exr-diff-plot}

Visualisieren Sie den Unterschied im Verkaufspreis abhängig von
\texttt{ship\_pr}; betrachten Sie \texttt{ship\_pr} als ein
Gruppierungsvariable. Interpretieren Sie das Ergebnis.

\textbf{Lösung}

\begin{Shaded}
\begin{Highlighting}[]
\NormalTok{mariokart\_no\_extreme }\SpecialCharTok{|\textgreater{}} 
  \FunctionTok{select}\NormalTok{(ship\_pr, total\_pr) }\SpecialCharTok{|\textgreater{}} 
  \FunctionTok{plot\_boxplot}\NormalTok{(}\AttributeTok{by =} \StringTok{"ship\_pr"}\NormalTok{)}
\end{Highlighting}
\end{Shaded}

\begin{center}
\includegraphics[width=1\linewidth,height=\textheight,keepaspectratio]{040-verbildlichen_files/figure-pdf/unnamed-chunk-35-1.pdf}
\end{center}

\texttt{plot\_boxplot} gruppiert \emph{metrische} Variablen, wie
\texttt{ship\_pr} automatisch in fünf Gruppen (mit gleichen Ranges). Wir
müssen also nichts tun, um die metrische Variable \texttt{ship\_pr} in
eine Gruppierungsvariable (Faktorvariable) umzuwandeln.

Es sieht so aus, als würde der Median zwischen den Gruppen leicht
steigen, mit Ausnahme der mittleren Gruppe.

\end{exercise}

\section{So lügt man mit Statistik}\label{so-luxfcgt-man-mit-statistik}

Diagramme werden mitunter eingesetzt, um die Wahrheit
\enquote{aufzuhübschen}. Hier folgen einige gebräuchlichen
Täuschungsmanöver.

Achsen zu stauchen ist ein einfacher Trick, s. Abbildung~\ref{fig-lie1}.

\begin{figure}[H]

\begin{minipage}{0.45\linewidth}

\centering{

\includegraphics[width=1\linewidth,height=\textheight,keepaspectratio]{040-verbildlichen_files/figure-pdf/fig-lie1-1.pdf}

}

\subcaption{\label{fig-lie1-1}Oh nein, dramatischer Einbruch des
Umsatzes!}

\end{minipage}%
%
\begin{minipage}{0.10\linewidth}
~\end{minipage}%
%
\begin{minipage}{0.45\linewidth}

\centering{

\includegraphics[width=1\linewidth,height=\textheight,keepaspectratio]{040-verbildlichen_files/figure-pdf/fig-lie1-2.pdf}

}

\subcaption{\label{fig-lie1-2}Kaum der Rede wert, ist nur ein bisschen
Schwankung!}

\end{minipage}%

\caption{\label{fig-lie1}Stauchen der Y-Achse, um mit Statistik zu
lügen}

\end{figure}%

Natürlich kann man auch durch \enquote{Abschneiden} der Y-Achse einen
eindrucksvollen Effekt erzielen, s. Abbildung~\ref{fig-lie2}.

\begin{figure}[H]

\begin{minipage}{0.50\linewidth}

\centering{

\includegraphics[width=1\linewidth,height=\textheight,keepaspectratio]{040-verbildlichen_files/figure-pdf/fig-lie2-1.pdf}

}

\subcaption{\label{fig-lie2-1}Oh nein, dramatischer Einbruch des
Umsatzes!}

\end{minipage}%
%
\begin{minipage}{0.50\linewidth}

\centering{

\includegraphics[width=1\linewidth,height=\textheight,keepaspectratio]{040-verbildlichen_files/figure-pdf/fig-lie2-2.pdf}

}

\subcaption{\label{fig-lie2-2}Kaum der Rede wert, ist nur ein bisschen
Schwankung!}

\end{minipage}%

\caption{\label{fig-lie2}Abschneiden der Y-Achse, um mit Statistik zu
lügen}

\end{figure}%

Scheinkorrelationen als \enquote{echte}, also kausale Effekte zu
verkaufen, ist ein anderer Trick, den man immer mal wieder beobachten
kann. Ein Beispiel: Messerli (2012) berichtet von einem Zusammenhang von
Schokoladenkonsum und Anzahl von Nobelpreisen (Beobachtungseinheit:
Länder), s. Abbildung~\ref{fig-choc}. Das ist doch ganz klar: Schoki
futtern macht schlau und Nobelpreise! (?)

\begin{figure}[H]

\centering{

\includegraphics[width=0.75\linewidth,height=\textheight,keepaspectratio]{img/choc.jpeg}

}

\caption{\label{fig-choc}Schokolodenkonsum und Nobelpreise}

\end{figure}%

Leider ist hier von einer \emph{Scheinkorrelation} auszugehen: Auch wenn
die beiden Variablen \emph{Schokoladenkonsum} und \emph{Nobelpreise}
zusammenhängen, heißt das \emph{nicht}, dass die Variable die Ursache
und die andere die Wirkung sein muss. So könnte auch eine Drittvariable
im Hintergrund die gleichzeitige Ursache von Schokoladenkonsum und
Nobelpreise sein, etwa der \emph{allgemeine Entwicklungsstand} des
Landes: In höher entwickelten Ländern wird mehr Schokolade konsumiert
und es werden mehr Nobelpreise gewonnen im Vergleich zu Ländern mit
geringerem Entwicklungsstand.

\section{Praxisbezug}\label{praxisbezug-3}

Ein, wie ich finde schlagendes Beispiel zur Stärke von Datendiagrammen
ist Abbildung~\ref{fig-vaccine}. Das Diagramm zeigt die Häufigkeit von
Masern, vor und nach der Einführung der Impfung. Die Daten und die Idee
zur Visualisierung gehen auf van Panhuis et al. (2013) zurück.

\begin{figure}[H]

\centering{

\includegraphics[width=0.75\linewidth,height=\textheight,keepaspectratio]{img/vaccine.jpg}

}

\caption{\label{fig-vaccine}Häufigkeit von Masern und Impfung in den USA
(Moore, 2015)}

\end{figure}%

In der \enquote{freien Wildbahn} findet man häufig sog.
\enquote{Tortendiagramme}. Zwar sind sie beliebt, doch ist
\href{https://www.data-to-viz.com/caveat/pie.html}{von ihrer Verwendung
zumeist abzuraten}, denn bei Tortenstücken ist es schwer, die Größe zu
vergleichen.

\section{Vertiefung}\label{vertiefung-4}

\subsection{Schicke Diagramme}\label{schicke-diagramme}

Ein Teil der Diagramm dieses Kapitels wurden mit dem R-Paket
\href{https://rpkgs.datanovia.com/ggpubr/}{ggpubr} erstellt. Mit diesem
Paket lassen sich einfach ansprechende Datendiagramme erstellen.

\begin{Shaded}
\begin{Highlighting}[]
\FunctionTok{library}\NormalTok{(ggpubr)  }\CommentTok{\# einmalig instalieren nicht vergessen}
\NormalTok{mariokart }\SpecialCharTok{\%\textgreater{}\%} 
  \FunctionTok{filter}\NormalTok{(total\_pr }\SpecialCharTok{\textless{}} \DecValTok{100}\NormalTok{) }\SpecialCharTok{\%\textgreater{}\%} 
  \FunctionTok{ggboxplot}\NormalTok{(}\AttributeTok{x =} \StringTok{"cond"}\NormalTok{, }\AttributeTok{y =} \StringTok{"total\_pr"}\NormalTok{)}
\end{Highlighting}
\end{Shaded}

Möchte man Mittelwerte vergleichen, so sind Boxplots nicht ideal, da
diese ja nicht den Mittelwert, sondern den \emph{Median} herausstellen.
Eine Abhilfe (also eine Darstellung des Mittelwerts) schafft man (z.B.)
mit \texttt{ggpubr}, s. Abbildung~\ref{fig-comp-means-ggpubr}.

\begin{Shaded}
\begin{Highlighting}[]
\FunctionTok{ggviolin}\NormalTok{(mariokart\_no\_extreme, }
         \AttributeTok{x =} \StringTok{"cond"}\NormalTok{, }
         \AttributeTok{y =} \StringTok{"total\_pr"}\NormalTok{,}
         \AttributeTok{add =} \StringTok{"mean\_sd"}\NormalTok{) }
\end{Highlighting}
\end{Shaded}

\begin{figure}[H]

\centering{

\includegraphics[width=1\linewidth,height=\textheight,keepaspectratio]{040-verbildlichen_files/figure-pdf/fig-comp-means-ggpubr-1.pdf}

}

\caption{\label{fig-comp-means-ggpubr}Vergleich der Verteilungen zweier
Gruppen mit Mittelwert und Standardabweichung pro Gruppe hervorgehoben}

\end{figure}%

Ein \enquote{Violinenplot} hat die gleiche Aussage wie ein
Dichtediagramm: Je breiter die \enquote{Violine}, desto mehr
Beobachtungen gibt es an dieser Stelle.

Übrigens sind Modelle -- und Diagramme sind Modelle -- immer eine
Vereinfachung, lassen also Informationen weg. Manchmal auch wichtige
Informationen.

\subsection{Farbwahl}\label{sec-farbwahl}

Einige Überlegungen zur Farbwahl findet sich bei Wilke (2019), Kap. 4.
Die Farbpalette von Okabe und Ito ist (vgl. Ichihara et al., 2008)
empfehlenswert, da sie auch bei Schwarz-Weiß-Druck und bei Sehschwächen
die Farben noch recht gut unterscheiden lässt, s.
Abbildung~\ref{fig-okabe}.

\begin{Shaded}
\begin{Highlighting}[]
\NormalTok{mariokart }\SpecialCharTok{\%\textgreater{}\%} 
  \FunctionTok{filter}\NormalTok{(total\_pr }\SpecialCharTok{\textless{}} \DecValTok{100}\NormalTok{) }\SpecialCharTok{\%\textgreater{}\%} 
  \FunctionTok{ggboxplot}\NormalTok{(}\AttributeTok{x =} \StringTok{"cond"}\NormalTok{, }\AttributeTok{y =} \StringTok{"total\_pr"}\NormalTok{, }\AttributeTok{fill =} \StringTok{"cond"}\NormalTok{) }\SpecialCharTok{+}
  \FunctionTok{scale\_fill\_okabeito}\NormalTok{()}
\end{Highlighting}
\end{Shaded}

\begin{figure}[H]

\centering{

\includegraphics[width=1\linewidth,height=\textheight,keepaspectratio]{040-verbildlichen_files/figure-pdf/fig-okabe-1.pdf}

}

\caption{\label{fig-okabe}Die Farbskala von Okabe und Ito: Geeignet bei
Farbseh-Schwächen und für Schwarz-Weiß-Druck. Außerdem nett
anzuschauen.}

\end{figure}%

\section{Aufgaben}\label{aufgaben-3}

Die Webseite \href{https://datenwerk.netlify.app}{datenwerk.netlify.app}
stellt eine Reihe von einschlägigen Übungsaufgaben bereit. Sie können
die Suchfunktion der Webseite nutzen, um die Aufgaben mit den folgenden
Namen zu suchen:

\begin{enumerate}
\def\labelenumi{\arabic{enumi}.}
\tightlist
\item
  \href{https://sebastiansauer.github.io/Datenwerk/posts/boxhist/boxhist.html}{boxhist}
\item
  \href{https://sebastiansauer.github.io/Datenwerk/posts/max-corr1/max-corr1.html}{max-corr1}
\item
  \href{https://sebastiansauer.github.io/Datenwerk/posts/max-corr2/max-corr2.html}{max-corr2}
\item
  \href{https://sebastiansauer.github.io/Datenwerk/posts/histogramm-in-boxplot/histogramm-in-boxplot}{Histogramm-in-Boxplot}
\item
  \href{https://sebastiansauer.github.io/Datenwerk/posts/diamonds-histogramm-vergleich2/diamonds-histogramm-vergleich2}{Diamonds-Histogramm-Vergleich2}
\item
  \href{https://sebastiansauer.github.io/Datenwerk/posts/boxplot-aussagen/boxplot-aussagen}{Boxplot-Aussagen}
\item
  \href{https://sebastiansauer.github.io/Datenwerk/posts/boxplots-de1a/boxplots-de1a.html}{boxplots-de1a}
\item
  \href{https://sebastiansauer.github.io/Datenwerk/posts/movies-vis1/movies-vis1.html}{movies-vis1}
\item
  \href{https://sebastiansauer.github.io/Datenwerk/posts/movies-vis2/movies-vis2.html}{movies-vis2}
\item
  \href{https://sebastiansauer.github.io/Datenwerk/posts/vis-gapminder/vis-gapminder}{vis-gapminder}
\item
  \href{https://sebastiansauer.github.io/Datenwerk/posts/boxplots-de1a/boxplots-de1a}{boxplots-de1a}
\item
  \href{https://sebastiansauer.github.io/Datenwerk/posts/diamonds-histogramm-vergleich/diamonds-histogramm-vergleich}{diamonds-histogramm-vergleich}
\item
  \href{https://sebastiansauer.github.io/Datenwerk/posts/wozu-balkendiagramm/wozu-balkendiagramm}{wozu-balkendiagramm}
\item
  \href{https://sebastiansauer.github.io/Datenwerk/posts/diamonds-histogram/diamonds-histogram}{diamonds-histogram}
\item
  \href{https://sebastiansauer.github.io/Datenwerk/posts/n-vars-diagram/n-vars-diagram}{n-vars-diagram}
\end{enumerate}

Noch mehr Aufgaben zum Thema Datenvisualisierung finden Sie im Datenwerk
unter dem Tag
\href{https://sebastiansauer.github.io/Datenwerk/\#category=vis}{vis}.

\section{Literaturhinweise}\label{literaturhinweise-3}

Sowohl \texttt{ggpubr} als auch \texttt{DataExplorer} (und viele andere
R-Pakete) bauen auf dem R-Paket \texttt{ggplot2} auf. \texttt{ggplot2}
ist eines der am weitesten ausgearbeiteten Softwarepakete zur Erstellung
von Datendiagrammen. Das Buch zur Software (vom Autor von
\texttt{ggplot2}) ist empfehlenswert (Wickham, 2016). Eine neuere, gute
Einführung in Datenvisualisierung findet sich bei Wilke (2019). Beide
Bücher sind kostenfrei online lesbar.

Wilke (2019) gibt einen hervorragenden Überblick über praktische Aspekte
der Datenvisualisierung; gut geeignet, wenn man mit R arbeitet. In
ähnlicher Richtung geht Fisher \& Meyer (2018).

\chapter{Punktmodelle 1}\label{sec-punktmodelle1}

\section{Lernsteuerung}\label{lernsteuerung-4}

Abbildung~\ref{fig-ueberblick} zeigt den Standort dieses Kapitels im
Lernpfad und gibt damit einen Überblick über das Thema dieses Kapitels
im Kontext aller Kapitel.

\subsection{Lernziele}\label{lernziele-5}

\begin{itemize}
\tightlist
\item
  Sie können gängige Arten von Lagemaße definieren.
\item
  Sie können erläutern, inwiefern man ein Lagemaß als ein Modell
  hernehmen kann.
\item
  Sie können Lagemaße mit R berechnen.
\end{itemize}

\subsection{Benötigte R-Pakete}\label{benuxf6tigte-r-pakete-2}

In diesem Kapitel benötigen Sie folgende R-Pakete.

\begin{Shaded}
\begin{Highlighting}[]
\FunctionTok{library}\NormalTok{(tidyverse)}
\FunctionTok{library}\NormalTok{(easystats)}
\end{Highlighting}
\end{Shaded}

\[
\definecolor{ycol}{RGB}{230,159,0}
\definecolor{modelcol}{RGB}{86,180,233}
\definecolor{errorcol}{RGB}{0,158,115}
\definecolor{beta0col}{RGB}{213,94,0}
\definecolor{beta1col}{RGB}{0,114,178}
\definecolor{xcol}{RGB}{204,121,167}
\]

\subsection{Benötigte Daten}\label{benuxf6tigte-daten-3}

\begin{Shaded}
\begin{Highlighting}[]
\NormalTok{mariokart\_path }\OtherTok{\textless{}{-}} \FunctionTok{paste0}\NormalTok{(}
  \StringTok{"https://vincentarelbundock.github.io/Rdatasets/"}\NormalTok{,}
  \StringTok{"csv/openintro/mariokart.csv"}\NormalTok{)}

\NormalTok{mariokart }\OtherTok{\textless{}{-}} \FunctionTok{read.csv}\NormalTok{(mariokart\_path)}
\end{Highlighting}
\end{Shaded}

\section{Mittelwert als Modell}\label{sec-mw}

Der \enquote{klassische} Mittelwert (das arithmetisches Mittel) ist ein
prototypisches Beispiel für ein Modell in der Statistik.

\begin{exercise}[]\protect\hypertarget{exr-mw-md-mod}{}\label{exr-mw-md-mod}

Welche Vorstellung haben Sie, wenn Sie hören, dass der \enquote{typische
deutsche Mann} 1,80m groß ist (vgl. Roser et al., 2013)?

\begin{enumerate}
\def\labelenumi{\alph{enumi})}
\tightlist
\item
  Die Hälfte der Männer ist größer als 1,80 m, die andere Hälfte
  kleiner.
\item
  Das arithmetische Mittel der Männer beträgt 1,80 m.
\item
  Die meisten Männer sind 1,80 m groß.
\item
  Etwas anderes.
\item
  Keine Ahnung! \(\square\)
\end{enumerate}

\end{exercise}

\begin{exercise}[]\protect\hypertarget{exr-mw2}{}\label{exr-mw2}

Laut dem Statistischen Bundesamt (2023-003-27) beträgt der Wert der
mittleren Größe deutscher Frauen etwa 1,66 m, also 14 cm weniger als bei
Männern.\footnote{\url{https://en.wikipedia.org/wiki/Average_human_height_by_country}}
\(\square\)

\emph{Ist das viel?}

\begin{enumerate}
\def\labelenumi{\alph{enumi})}
\tightlist
\item
  ja
\item
  nein
\item
  kommt drauf an
\item
  weiß nicht \(\square\)
\end{enumerate}

\emph{Antwort}

Auf dieser Frage gibt es keine Antwort, zumindest nicht ohne weitere
Annahmen. So könnte man z.B. sagen, \enquote{mehr als 5 cm sind viel}.
So eine Entscheidung ist aber keine statistische Angelegenheit, sondern
eine inhaltliche.

\end{exercise}

\begin{example}[Beispiel zum
Mittelwert]\protect\hypertarget{exm-mw}{}\label{exm-mw}

Ein Statistikkurs besteht aus drei Studentinnen: Anna, Berta und Carla.
Sie haben gerade ihre Noten in der Klausur erfahren. Anna hat eine 1,
Berta eine 2 und Carla eine 3. Der Durchschnitt (das arithmetische
Mittel, \(\varnothing\), der Durchschnitt) beträgt: 2. \(\square\)

\end{example}

\begin{quote}
{\emoji{student}} Zu easy!
\end{quote}

\begin{quote}
{\emoji{teacher}} Schon gut! Chill mal. Wird gleich interessanter.
\end{quote}

Die Rechenregel zum Mittelwert lautet:

\begin{enumerate}
\def\labelenumi{\arabic{enumi}.}
\tightlist
\item
  Addiere alle Werte
\item
  Teile durch die Anzahl der Werte
\item
  Fertig. 😄
\end{enumerate}

Etwas abstrakter kann man Beispiel~\ref{exm-mw} in folgendem Schaubild
darstellen, s. Gleichung~\ref{eq-mw}.

\begin{equation}\phantomsection\label{eq-mw}{
\begin{array}{|c|} \hline \\ \\ \square \\ \hline \end{array} + \begin{array}{|c|} \hline \\ \square \\ \square \\ \hline \end{array} + \begin{array}{|c|} \hline \square \\ \square \\ \square \\ \hline \end{array} = 3 \cdot \begin{array}{|c|} \hline \\ \square \\ \square \\ \hline \end{array}
}\end{equation}

Der Nutzen des Mittelwerts liegt darin, dass er uns ein Bild gibt (ein
Modell ist!) für die \enquote{typische Note} im Statistikkurs, s.
Gleichung~\ref{eq-mw2}.

\begin{equation}\phantomsection\label{eq-mw2}{\begin{array}{|c|} \hline \\ \\ \square \\ \hline \end{array} + \begin{array}{|c|} \hline \\ \square \\ \square \\ \hline \end{array} + \begin{array}{|c|} \hline \square \\ \square \\ \square \\ \hline \end{array} \qquad \leftrightarrow  \qquad \underbrace{\begin{array}{|c|} \hline \\ \square \\ \square \\ \hline \end{array}}_{\text{"typischer Vertreter"}}}\end{equation}

\begin{tcolorbox}[enhanced jigsaw, bottomtitle=1mm, leftrule=.75mm, breakable, title=\textcolor{quarto-callout-important-color}{\faExclamation}\hspace{0.5em}{Wichtig}, bottomrule=.15mm, titlerule=0mm, left=2mm, opacityback=0, colframe=quarto-callout-important-color-frame, rightrule=.15mm, colback=white, coltitle=black, toprule=.15mm, toptitle=1mm, colbacktitle=quarto-callout-important-color!10!white, arc=.35mm, opacitybacktitle=0.6]

Der Nutzen des Mittelwerts liegt darin, dass er eine Datenreihe zu einen
\enquote{typischen Vertreter} zusammenfasst. Er ist typisch in dem
Sinne, als dass die Werte aller Merkmalsträger in gleichem Maße
einfließen. Er gibt uns eine (mögliche) Vorstellung (ein Modell!), wie
wir uns die Werte der Datenreihe vorstellen sollen.

\end{tcolorbox}

Eine nützliche Anschauung zum Mittelwert ist die Vorstellung des
Mittelwerts als eine ausbalancierte Wippe, s. Abbildung~\ref{fig-wippe}.

\begin{figure}[H]

\centering{

\includegraphics[width=0.7\linewidth,height=\textheight,keepaspectratio]{img/1280px-Seesaw_with_mean.svg.png}

}

\caption{\label{fig-wippe}Mittelwert als ausbalancierte Wippe mit
Mittelwert 3 (Maphry, 2009)}

\end{figure}%

In \enquote{Mathe-Sprech} bezeichnet man den Mittelwert häufig mit
\(\bar{x}\) und schreibt die Rechenregel so, s.
Gleichung~\ref{eq-mw-formel}.

\begin{equation}\phantomsection\label{eq-mw-formel}{\bar {x} :=\frac{1}{n} \sum_{i=1}^{n}{x_{i}}=\frac {x_{1}+x_{2}+\dotsb +x_{n}} {n}}\end{equation}

\begin{definition}[Mittelwert]\protect\hypertarget{def-mw}{}\label{def-mw}

Der Mittelwert (MW, mean) der Variablen \(X\) (präziser: das
arithmetische Mittel des Merkmal \(X\)) ist definiert als die Summe der
Elemente von \(X\) geteilt durch deren Anzahl, \(n\). Den Mittelwert von
\(X\) bezeichnet man auch mit \(\bar {x}\). \(\square\)

\end{definition}

\begin{example}[]\protect\hypertarget{exm-mw1}{}\label{exm-mw1}

Angenommen wir haben eine Reihe von Noten: 1,2,3. Der Mittelwert der
Noten beträgt dann 2: \(\bar{X} = \frac{1}{3}\sum (1+2+3) = 6/3 = 2\).
\(\square\)

\end{example}

Da der Mittelwert eine zentrale Rolle spielt in der Statistik, sollten
wir ihn uns noch etwas genauer anschauen. In s. Abbildung~\ref{fig-mw1}
sehen wir die Noten von (dieses Mal) vier Studentis. Die gestrichelte
horizontale Linie zeigt den Mittelwert der vier Noten. Die schwarzen
Punkte sind die Daten, in dem Fall die einzelnen Noten. Die vertikalen
Linien zeigen die Abweichungen der Noten zum Mittelwert.

Bezeichnen wir die Abweichung -- auch als \enquote{Fehler},
\enquote{Rest} oder \enquote{Residuum} bezeichnet -- der \(i\)-ten
Person mit \(\color{errorcol}{\text{e}_i}\) (\emph{e} wie engl.
\emph{error}, Fehler) und die \(i\)-te Note mit \(\color{ycol}{y_i}\),
so können wir mit Gleichung~\ref{eq-modell1} festhalten:

\begin{equation}\phantomsection\label{eq-modell1}{\color{ycol}{\text{y}_i} \color{black}{ = } \color{modelcol}{\;\bar{x}\;} + \color{errorcol}{\;\text{e}_i}}\end{equation}

Anders ausgedrückt (s. Gleichung~\ref{eq-modell2}):

\begin{equation}\phantomsection\label{eq-modell2}{\color{ycol}{\text{Daten}} \color{black}{ = } \color{modelcol}{\text{Modell}} + 
\color{errorcol}{\text{Rest}}}\end{equation}

Der Mittelwert ist hier unser Modell der Daten. Wie gesagt: Ein Modell
ist eine vereinfachte (zusammengefasste) Beschreibung einer Datenreihe.

Um Modelle darzustellen, wird in der Datenanalyse häufig folgende Art
von Modellgleichung verwendet, s. Gleichung~\ref{eq-sim-mean}.

\begin{equation}\phantomsection\label{eq-sim-mean}{\color{modelcol}{\hat{y}} \sim \color{xcol}{\text{ x}}}\end{equation}

Lies: \enquote{Der Modellwert \(\color{modelcol}{\hat{y}}\) ist eine
Funktion der Variable \(\color{xcol}{\text{x}}\)}. Der Kringel
\enquote{\textasciitilde{}} soll also hier heißen \enquote{\ldots{} ist
eine Funktion von \ldots{}}. Das \enquote{Kringel} oder die
\enquote{Welle} \textasciitilde{} nennt man auch \enquote{Tilde}.

Mit \(\color{modelcol}{\hat{y}}\) ist die vorhergesagte bzw. die
\emph{zu erklärende Variable} (synonym: AV, Output-Variable,
Zielvariable) gemeint. Das \enquote{Dach} über dem
\(\color{ycol}{\text{y}}\) bedeutet \enquote{vorhergesagter Y-Wert} oder
\enquote{Y-Wert laut dem Modell}. Der tatsächliche, beobachtete Wert
\(\color{ycol}{\text{y}}\) setzt sich zusammen aus dem Modellwert
\(\color{modelcol}{\text{m}}\) plus einem Fehler
\(\color{errorcol}{\text{e}}\), s. Gleichung~\ref{eq-modell3}.

\begin{equation}\phantomsection\label{eq-modell3}{\color{ycol}{y} \color{black}{ = } \color{modelcol}{\text{m}} + \color{errorcol}{\text{e}}}\end{equation}

Anstelle von \(\color{modelcol}{\text{m}}\) schreibt man auch
\(\color{modelcol}{\hat{y}}\) (\enquote{y-Dach}). In diesem Fall ist das
Modell einfach gleich dem Mittelwert (und nicht irgendeiner Funktion des
Mittelwerts), so dass wir mit Gleichung~\ref{eq-modell4} schreiben
können:

\begin{equation}\phantomsection\label{eq-modell4}{\color{ycol}{y}  \color{black}{ = } \color{modelcol}{\bar{x}} + \color{errorcol}{e}}\end{equation}

Die Zielvariable \(\color{ycol}{\text{y}}\) wird also durch ihren
eigenen Mittelwert erklärt, außer gehen wir von einem Fehler
\(\color{errorcol}e\) in unseren Modellvorhersagen aus. Nobody is
perfect. In späteren Kapiteln werden wir andere Variablen heranziehen,
um die Zielvariable zu erklären. Würden wir z.B. sagen wollen, dass wir
\(\color{ycol}{\text{y}}\) als Funktion einer Variable
\(\color{xcol}{X}\) erklären, so würden wir schreiben (s.
Gleichung~\ref{eq-modell5a}):

\begin{equation}\phantomsection\label{eq-modell5a}{\color{modelcol}{\bar{y}} \color{black}  { \sim } \color{xcol}{\text{ x}}}\end{equation}

Da wir im Moment aber keine andere Variablen bemühen, um
\(\color{ycol}{\text{y}}\) zu erklären, schreibt man mit
Gleichung~\ref{eq-modell5} auch:

\begin{equation}\phantomsection\label{eq-modell5}{\color{modelcol}{\bar{y}}\;\;  \color{black}{\sim \; 1}}\end{equation}

Diese Schreibweise sieht verwirrend aus. Die \(1\) soll aber nur zeigen,
dass wir keine andere Variable zur Erklärung von
\(\color{ycol}{\text{y}}\) verwenden, daher steht hier kein Buchstabe,
sondern eine einfache \(1\). Der mathematische Hintergrund liegt in der
Art, wie man Matrizen multipliziert.

\begin{example}[Noten, Mittelwert und
Abweichung]\protect\hypertarget{exm-noten}{}\label{exm-noten}

Vier Studentis -- Anna, Berta, Carl, Dani -- haben ihre
Statistik-Klausur zurückbekommen (Schluck). Die Noten sehen Sie in
Abbildung~\ref{fig-mw1}; gar nicht so schlecht ausgefallen. Außerdem ist
der Mittelwert (gestrichelte horizontale Linie) sowie die Abweichungen
der einzelnen Noten vom Mittelwert eingezeichnet.\(\square\)

\end{example}

Schauen Sie sich die Abweichungsbalken (Residuen, Fehler; häufig mit
\(e\) wie \emph{error} bezeichnet) in Abbildung~\ref{fig-mw1} einmal
genauer an.

\begin{figure}[H]

\centering{

\includegraphics[width=0.7\linewidth,height=\textheight,keepaspectratio]{050-zusammenfassen_files/figure-pdf/fig-mw1-1.pdf}

}

\caption{\label{fig-mw1}Der Mittelwert als horizontale (gestrichelte)
Linie. Die vertikalen Linien zeigen die Abweichungen der einzelnen Werte
zum Mittelwert. Die Abweichungen summieren sich zu Null auf.}

\end{figure}%

Jetzt stellen Sie sich vor, Sie würden die vom Mittelwert nach
\emph{oben} ragenden Balkenlängen aneinanderlegen (das sind die
gestrichelten. Sehen Sie das vor Ihrem geistigen Auge? Jetzt legen Sie
auch noch die Abweichungsbalken, die nach \emph{unten} ragen, aneinander
(die mit den durchgezogenen Linien). Wer viel Phantasie hat, erkennt
(sieht) jetzt, dass die Gesamtlänge der \enquote{Balken nach oben}
identisch ist zur Gesamtlänge der nach \enquote{unten ragenden Balken},
vgl. Abbildung~\ref{fig-wippe}.

Präziser ausgedrückt und ohne Ihre Phantasie zu strapazieren
(Gleichung~\ref{eq-summenull}):

\begin{equation}\phantomsection\label{eq-summenull}{\sum_{i=1}^n (x_i-\bar{x})=\sum_{i=1}^n x_i - \sum_{i=1}^n \bar{x} = n\cdot \bar{x} - n\cdot \bar{x}=0}\end{equation}

\begin{tcolorbox}[enhanced jigsaw, bottomtitle=1mm, leftrule=.75mm, breakable, title=\textcolor{quarto-callout-note-color}{\faInfo}\hspace{0.5em}{Hinweis}, bottomrule=.15mm, titlerule=0mm, left=2mm, opacityback=0, colframe=quarto-callout-note-color-frame, rightrule=.15mm, colback=white, coltitle=black, toprule=.15mm, toptitle=1mm, colbacktitle=quarto-callout-note-color!10!white, arc=.35mm, opacitybacktitle=0.6]

Die Summe der Abweichungen vom Mittelwert ist Null.

\end{tcolorbox}

\begin{exercise}[]\protect\hypertarget{exr-mw-wealth1}{}\label{exr-mw-wealth1}

Was schätzen Sie, wie hoch das mittlere Vermögen (arithmetisches Mittel)
der Haushalte in Deutschland in etwa ist (im Jahr 2021 auf Basis einer
Umfrage) (Bundesbank, 2023)?\footnote{316500€} \(\square\)

\begin{enumerate}
\def\labelenumi{\alph{enumi})}
\tightlist
\item
  50.000 Euro
\item
  100.000 Euro
\item
  150.000 Euro
\item
  200.000 Euro
\item
  300.000 Euro
\end{enumerate}

\end{exercise}

\begin{example}[Der wertvollste Fußballer der Welt in Ihrem
Hörsaal]\protect\hypertarget{exm-md}{}\label{exm-md}

Kommt der wertvollste Fußballspieler der Welt in Ihren Hörsaal, sagen
wir, es ist Kylian Mbappé (@ \textbf{transfermarkt2024?}). Sein
Jahreseinkommen (2023) liegt bei ca. 120 Millionen Euro (Arad, 2024).

\begin{quote}
{\emoji{supervillain}} Hey Leute, wie geht's denn so! Wie viel Kohle
verdient ihr eigentlich so?
\end{quote}

\begin{quote}
{\emoji{student}} Äh, wir studieren und verdienen fast nix!
\end{quote}

Die 100 Studis im Hörsaal schauen verdattert aus der Wäsche: Was ist das
für eine komische Frage!? Aber zumindest verteilt der Fußballspieler
Autogramme.

\end{example}

\begin{exercise}[Mittleres Einkommen im Hörsaal, mit Kylian
Mbappé]\protect\hypertarget{exr-elon}{}\label{exr-elon}

Schätzen Sie -- im Kopf -- das mittlere Vermögen im Hörsaal, gehen Sie
davon aus, dass alle der 100 Studentis jeweils 1000 Euro im Jahr
verdienen. \(\square\)

\end{exercise}

In R kann man das mittlere Einkommen (präziser: das arithmetische Mittel
des Einkommens) wie folgt berechnen, s. Listing~\ref{lst-einkommen}.
(Die Details der Syntax, z.B. der Befehl \texttt{rep()}, sind von
geringer Bedeutung.)

\begin{codelisting}

\caption{\label{lst-einkommen}Wir simulieren Einkommen von 100 Studis
plus Mbappé.}

\centering{

\begin{Shaded}
\begin{Highlighting}[]
\FunctionTok{set.seed}\NormalTok{(}\DecValTok{42}\NormalTok{)  }\CommentTok{\# Zufallszahlen festlegen, hier nicht so wichtig}
\NormalTok{einkommen\_studis }\OtherTok{\textless{}{-}} \FunctionTok{rep}\NormalTok{(}\AttributeTok{x =} \DecValTok{1000}\NormalTok{, }\AttributeTok{times =} \DecValTok{100}\NormalTok{)  }\CommentTok{\# "rep" wie "repeat": wiederhole 1000 USD 100 Mal}
\NormalTok{einkommen }\OtherTok{\textless{}{-}} \FunctionTok{c}\NormalTok{(einkommen\_studis, }\DecValTok{120}\SpecialCharTok{*}\FloatTok{1e6}\NormalTok{)  }\CommentTok{\# 100 Studis mit 1000, 1 Mbappé mit 120 Mio}
\NormalTok{einkommen\_mw }\OtherTok{\textless{}{-}} \FunctionTok{mean}\NormalTok{(einkommen)}
\NormalTok{einkommen\_mw}
\DocumentationTok{\#\# [1] 1189109}
\end{Highlighting}
\end{Shaded}

}

\end{codelisting}%

\begin{tcolorbox}[enhanced jigsaw, bottomtitle=1mm, leftrule=.75mm, breakable, title=\textcolor{quarto-callout-note-color}{\faInfo}\hspace{0.5em}{Hinweis}, bottomrule=.15mm, titlerule=0mm, left=2mm, opacityback=0, colframe=quarto-callout-note-color-frame, rightrule=.15mm, colback=white, coltitle=black, toprule=.15mm, toptitle=1mm, colbacktitle=quarto-callout-note-color!10!white, arc=.35mm, opacitybacktitle=0.6]

1 Million hat 6 Nuller hinter der führenden Eins: 1000000. In
Taschenrechner- oder Computerschreibweise: 1 Mio = \texttt{1e6}, das
\texttt{1e6} ist zu lesen als \enquote{1 Mal 10 hoch 6, also mit 6 im
\emph{E}xponenten}.

\end{tcolorbox}

Der Mittelwert im Hörsaal beträgt also 1,189,109 Euro, etwas mehr als
eine Million. Ist das ein gutes Modell für das \enquote{typische}
Vermögen im Hörsaal?

\subsection{Der Mittelwert als lineares
Modell}\label{der-mittelwert-als-lineares-modell}

Man kann den Mittelwert als Gerade einzeichnen, s.
Abbildung~\ref{fig-mw2}, bzw. als Gerade begreifen. Insofern kann man
vom Mittelwert auch als \emph{lineares Modell} sprechen.

\begin{definition}[Lineares
Modell]\protect\hypertarget{def-lm}{}\label{def-lm}

Ein lineares Modell verwendet eine Gerade als Modell der Daten. Es
erklärt die Daten anhand einer Geraden. \(\square\)

\end{definition}

\begin{figure}[H]

\begin{minipage}{0.50\linewidth}

\centering{

\includegraphics[width=0.7\linewidth,height=\textheight,keepaspectratio]{050-zusammenfassen_files/figure-pdf/fig-mw2-1.pdf}

}

\subcaption{\label{fig-mw2-1}Mit Extremwerten}

\end{minipage}%
%
\begin{minipage}{0.50\linewidth}

\centering{

\includegraphics[width=0.7\linewidth,height=\textheight,keepaspectratio]{050-zusammenfassen_files/figure-pdf/fig-mw2-2.pdf}

}

\subcaption{\label{fig-mw2-2}Ohne Extremwerte (\textless100 Euro)}

\end{minipage}%

\caption{\label{fig-mw2}Der mittlere Preis von Mariokart-Spielen als
horizontale Gerade eingezeichnet}

\end{figure}%

Abbildung~\ref{fig-mw2} zeigt den Mittelwert des Verkaufspreises der
Mariokart-Spiele (\texttt{total\_pr}), einmal mit (farbig markierten)
Extremwerte (a) bzw. einmal ohne Extremwerte (b).

\begin{definition}[Extremwert]\protect\hypertarget{def-extremwert}{}\label{def-extremwert}

Ein Extremwert (Ausreißer; \emph{outlier}) ist eine Beobachtung, deren
Wert deutlich vom Großteil der anderen Beobachtungen im Datensatz
abweicht, z.B. viel größer ist. \(\square\)

\end{definition}

Berechnen wir mal den Mittelwert von \texttt{einkommen} mit R mit dem
Befehl \texttt{lm}.

\begin{Shaded}
\begin{Highlighting}[]
\FunctionTok{lm}\NormalTok{(einkommen }\SpecialCharTok{\textasciitilde{}} \DecValTok{1}\NormalTok{)  }\CommentTok{\# lm wie "lineares Modell" oder engl. "linear modell"}
\DocumentationTok{\#\# }
\DocumentationTok{\#\# Call:}
\DocumentationTok{\#\# lm(formula = einkommen \textasciitilde{} 1)}
\DocumentationTok{\#\# }
\DocumentationTok{\#\# Coefficients:}
\DocumentationTok{\#\# (Intercept)  }
\DocumentationTok{\#\#     1189109}
\end{Highlighting}
\end{Shaded}

Der Befehl gibt als \emph{Koeffizient} einen Wert zurück und zwar den
Mittelwert von \texttt{einkommen}, vgl. auch
Listing~\ref{lst-einkommen}. Dieser Wert wird als Achsenabschnitt (engl.
\emph{intercept}) bezeichnet, das wird verständlich, wenn man z.B. in
Abbildung~\ref{fig-mw2} sieht, dass die Gerade (des Mittelwerts) genau
an diesem Punkt die Y-Achse schneidet. Die Syntax des Befehls
\texttt{lm()} sieht etwas merkwürdig aus. Ignorieren Sie das fürs Erste,
wir besprechen das später (\textbf{?@sec-gerade1}) ausführlich.
\texttt{lm} steht übrigens für \enquote{lineares Modell}.

\section{Median als Modell}\label{sec-median}

\begin{quote}
{\emoji{student}} Hey, der Mittelwert ist doch Quatsch! Das ist gar kein
typischer Wert für die Menschen im Hörsaal. Weder für den Mbappé, noch
für uns Studis!
\end{quote}

\begin{quote}
{\emoji{teacher}} Ja, da habt ihr Recht.
\end{quote}

\begin{quote}
{\emoji{soccer-ball}} Die Welt ist schon ungerecht!
\end{quote}

\begin{tcolorbox}[enhanced jigsaw, bottomtitle=1mm, leftrule=.75mm, breakable, title=\textcolor{quarto-callout-important-color}{\faExclamation}\hspace{0.5em}{Wichtig}, bottomrule=.15mm, titlerule=0mm, left=2mm, opacityback=0, colframe=quarto-callout-important-color-frame, rightrule=.15mm, colback=white, coltitle=black, toprule=.15mm, toptitle=1mm, colbacktitle=quarto-callout-important-color!10!white, arc=.35mm, opacitybacktitle=0.6]

Bei (sehr) schiefen Verteilungen (s. Abbildung~\ref{fig-mbappe}) ist der
Mittelwert (sehr) wenig aussagekräftig, da er nicht mehr
\enquote{typische} Werte für die Merkmalsträger beschreibt.

\end{tcolorbox}

Abbildung~\ref{fig-mbappe} stellt die Verteilung des Einkommens einer
mit \enquote{normal} skalierter Achse und einmal mit logarithmischer
X-Achse. Zur Erinnerung: 4.0+e07 bedeutet
\(4 \cdot 10^{07} = 40000000\), eine 4 gefolgt von 7 Nullern. Die
logarithmische X-Achse stellt den Unterschied von Mittelwert (MW) und
Median deutlicher heraus als die normale (additive) Achse.

\begin{figure}[H]

\begin{minipage}{\linewidth}

\centering{

\includegraphics[width=0.75\linewidth,height=\textheight,keepaspectratio]{050-zusammenfassen_files/figure-pdf/fig-mbappe-1.pdf}

}

\subcaption{\label{fig-mbappe-1}X-Achse in additiver Form}

\end{minipage}%
\newline
\begin{minipage}{\linewidth}

\centering{

\includegraphics[width=0.75\linewidth,height=\textheight,keepaspectratio]{050-zusammenfassen_files/figure-pdf/fig-mbappe-2.pdf}

}

\subcaption{\label{fig-mbappe-2}X-Achse in multiplikativer Form
(logarithmische Darstellung)}

\end{minipage}%

\caption{\label{fig-mbappe}Die Einkommensverteilung im Hörsaal}

\end{figure}%

Der Mittelwert ist Hörsaal ist nicht typisch für die Menschen im
Hörsaal: Weder für Mbappé, noch für die Studis. Genau genommen ist der
Mittelwert in diesem Fall ziemlich nutzlos.

\begin{tcolorbox}[enhanced jigsaw, bottomtitle=1mm, leftrule=.75mm, breakable, title=\textcolor{quarto-callout-important-color}{\faExclamation}\hspace{0.5em}{Wichtig}, bottomrule=.15mm, titlerule=0mm, left=2mm, opacityback=0, colframe=quarto-callout-important-color-frame, rightrule=.15mm, colback=white, coltitle=black, toprule=.15mm, toptitle=1mm, colbacktitle=quarto-callout-important-color!10!white, arc=.35mm, opacitybacktitle=0.6]

Der Mittelwert ist empfänglich für Extremwerte: Gibt es einen Extremwert
in einer Datenreihe, so spiegelt der Mittelwert stark diesen Wert wieder
und weniger die Mehrheit der gemäßigten Werte. Man sagt, der Mittelwert
ist nicht \emph{robust} (gegenüber Extremwerten).

\end{tcolorbox}

\begin{example}[Das Median-Einkommen einiger
Studentinnen]\protect\hypertarget{exm-med}{}\label{exm-med}

Fünf Studentinnen tauschen sich über ihr Einkommen aus, s.
Abbildung~\ref{fig-md1}, links. Es handelt sich um eine schiefe
Verteilung.

\begin{figure}[H]

\begin{minipage}{\linewidth}

\centering{

\includegraphics[width=0.7\linewidth,height=\textheight,keepaspectratio]{050-zusammenfassen_files/figure-pdf/fig-md1-1.pdf}

}

\subcaption{\label{fig-md1-1}Einkommen auf der Y-Achse}

\end{minipage}%
\newline
\begin{minipage}{\linewidth}

\centering{

\includegraphics[width=0.7\linewidth,height=\textheight,keepaspectratio]{050-zusammenfassen_files/figure-pdf/fig-md1-2.pdf}

}

\subcaption{\label{fig-md1-2}Einkommen auf der X-Achse}

\end{minipage}%

\caption{\label{fig-md1}Das Median-Einkommen einiger Studentinnen sowie
der Mittelwert (MW) ihres Einkommens}

\end{figure}%

Wir könnten jetzt behaupten, dass Carla das typische Einkommen (für
diese Datenreihe) aufweist, da es genauso viele Studentinnen gibt, die
mehr verdienen, wie solche, die weniger verdienen. \(\square\)

\end{example}

\begin{definition}[Median]\protect\hypertarget{def-median}{}\label{def-median}

Merkmalsausprägung, die bei (aufsteigend) sortierten Beobachtungen in
der Mitte liegt. \(\square\)

\end{definition}

\begin{exercise}[Alle mal
aufstehen]\protect\hypertarget{exr-aufstellen}{}\label{exr-aufstellen}

Auf Geheiß der Lehrkraft stehen jetzt alle Studis bitte auf und
sortieren sich der Größe nach im Raum, schön in einer Reihe aufgestellt.
Die Körpergröße der Person in der Mitte der Reihe, zu der also gleich
viele Personen zu links wie zu rechts stehen, das ist der Medien dieser
Datenreihe, vgl. Abbildung~\ref{fig-median-human}. \(\square\)

\end{exercise}

Der Median ist \emph{robust} (gegenüber) Extremwerten: Fügt man
Extremwerte zu einer Verteilung hinzu, ändert sich der Median zumeist
(deutlich) weniger als der Mittelwert.

Abbildung~\ref{fig-median-human} stellt den Median schematisch dar.

\begin{figure}[H]

\begin{minipage}{0.20\linewidth}

\includegraphics[width=0.1\linewidth,height=\textheight,keepaspectratio]{img/Human_Silhouette.svg.png}

\subcaption{\label{}1,60m}
\end{minipage}%
%
\begin{minipage}{0.20\linewidth}

\includegraphics[width=0.13\linewidth,height=\textheight,keepaspectratio]{img/Human_Silhouette.svg.png}

\subcaption{\label{}1,72m}
\end{minipage}%
%
\begin{minipage}{0.20\linewidth}

\includegraphics[width=0.15\linewidth,height=\textheight,keepaspectratio]{img/human-red.png}

\subcaption{\label{}1,79m: Median!}
\end{minipage}%
%
\begin{minipage}{0.20\linewidth}

\includegraphics[width=0.16\linewidth,height=\textheight,keepaspectratio]{img/Human_Silhouette.svg.png}

\subcaption{\label{}1,94}
\end{minipage}%
%
\begin{minipage}{0.20\linewidth}

\includegraphics[width=0.2\linewidth,height=\textheight,keepaspectratio]{img/Human_Silhouette.svg.png}

\subcaption{\label{}2,12m}
\end{minipage}%

\caption{\label{fig-median-human}Der Median als der Wert des
\enquote{mittleren} Objekts, wenn die Objekte aufsteigend sortiert sind.
Es gibt genauso viele Objekte mit kleinerem Wert als der Median wie
Objekte mit größerem Wert als der Median.}

\end{figure}%

Bei geradem \(n\) werden die beiden mittleren Werte betrachtet und das
arithmetische Mittel aus diesen beiden Werten gebildet.

\begin{example}[]\protect\hypertarget{exm-med2}{}\label{exm-med2}

Bei der Messreihe 1, 2, 3, 4, 5, 6, 8, 9 beträgt der Median
4.5.\(\square\)

\end{example}

\begin{exercise}[Emma wird
reich]\protect\hypertarget{exr-md2}{}\label{exr-md2}

Durch ein geniales Patent wird Emma steinreich. Ihr Einkommen erhöht
sich um das Hundertfache. Wie verändert sich der Median?\footnote{Er
  bleibt gleich, verändert sich also nicht: Der Median ist
  \emph{robust}, er verändert sich nicht oder kaum, wenn Extremwerte
  vorliegen.} \(\square\)

\end{exercise}

\begin{exercise}[Wer ist mehr \enquote{mittel}? Median oder
Mittelwert?]\protect\hypertarget{exr-mw-md}{}\label{exr-mw-md}

~

\begin{quote}
{\emoji{student}} Das arithmetische Mittel sollte Mittelwert heißen,
weil es die Mitte von zwei Messwerten widerspiegelt, also z.B. von 1 und
10 ist die Mitte 5,5 -- also genau beim Mittelwert!
\end{quote}

\begin{quote}
{\emoji{woman}} Moment! Der Median und nur der Median zeigt den
mittleren Messwert! Links und rechts sind gleich viele Messwerte, wenn
man die Werte der Größe nach sortiert. Also liegt der Median genau in
der Mitte!
\end{quote}

Nehmen Sie Stellung zu dieser Diskussion!\(\square\)

\end{exercise}

\begin{example}[Ein \enquote{mittlerer} Preis für
Mariokart]\protect\hypertarget{exm-md3}{}\label{exm-md3}

Der Mittelwert (das arithmetische Mittel) und der Median für das
Start-Gebot (\texttt{start\_pr)} von Mariokart-Spielen sind nicht
gleich, der Mittelwert ist höher als der Median.

\begin{Shaded}
\begin{Highlighting}[]
\NormalTok{mariokart }\OtherTok{\textless{}{-}} \FunctionTok{read.csv}\NormalTok{(mariokart\_path)  }

\NormalTok{mariokart }\SpecialCharTok{\%\textgreater{}\%} 
  \FunctionTok{summarise}\NormalTok{(}\AttributeTok{price\_mw =} \FunctionTok{mean}\NormalTok{(start\_pr),}
            \AttributeTok{price\_md =} \FunctionTok{median}\NormalTok{(start\_pr))}
\end{Highlighting}
\end{Shaded}

\begin{longtable}[]{@{}rr@{}}
\toprule\noalign{}
price\_mw & price\_md \\
\midrule\noalign{}
\endhead
\bottomrule\noalign{}
\endlastfoot
8.8 & 1 \\
\end{longtable}

Wie man sieht, ist der Mittelwert größer als der Median, s.
Abbildung~\ref{fig-mario-md}.

\begin{figure}[H]

\centering{

\includegraphics[width=0.7\linewidth,height=\textheight,keepaspectratio]{050-zusammenfassen_files/figure-pdf/fig-mario-md-1.pdf}

}

\caption{\label{fig-mario-md}Das Start-Gebot bei Mariokart-Spielen ist
schief verteilt: Median und Mittelwert sind unterschiedlich}

\end{figure}%

\end{example}

\begin{tcolorbox}[enhanced jigsaw, bottomtitle=1mm, leftrule=.75mm, breakable, title=\textcolor{quarto-callout-note-color}{\faInfo}\hspace{0.5em}{Hinweis}, bottomrule=.15mm, titlerule=0mm, left=2mm, opacityback=0, colframe=quarto-callout-note-color-frame, rightrule=.15mm, colback=white, coltitle=black, toprule=.15mm, toptitle=1mm, colbacktitle=quarto-callout-note-color!10!white, arc=.35mm, opacitybacktitle=0.6]

Klaffen Mittelwert und Median auseinander, so liegt eine schiefe
Verteilung vor. Ist der Mittelwert größer als der Median, so nennt man
die Verteilung rechtsschief. Bei schiefen Verteilungen ist der Median
dem Mittelwert als Modell für den \enquote{typischen Wert} vorzuziehen.

\end{tcolorbox}

\begin{exercise}[Mariokart ohne
Extremwerte]\protect\hypertarget{exr-mw-no-extrem}{}\label{exr-mw-no-extrem}

Im Datensatz \texttt{mariokart} gibt es einige wenige Spiele, die für
einen vergleichsweise hohen Preis verkauft wurden. Diese Extremwerte
verzerren den mittleren Verkaufspreis möglicherweise über die Gebühr.
\(\square\)

\textbf{Aufgabe} Entfernen Sie diese Werte und berechnen Sie dann
Mittelwert und Median erneut. Vergleichen Sie die Ergebnisse.

\textbf{Lösung}

\begin{Shaded}
\begin{Highlighting}[]
\NormalTok{mariokart\_no\_extreme }\OtherTok{\textless{}{-}} 
\NormalTok{mariokart }\SpecialCharTok{\%\textgreater{}\%} 
  \FunctionTok{filter}\NormalTok{(total\_pr }\SpecialCharTok{\textless{}} \DecValTok{100}\NormalTok{)}

\CommentTok{\# ohne Extremwerte:}
\NormalTok{mariokart\_no\_extreme }\SpecialCharTok{|\textgreater{}} 
  \FunctionTok{summarise}\NormalTok{(}\AttributeTok{total\_pr\_mittelwert =} \FunctionTok{mean}\NormalTok{(total\_pr),}
            \AttributeTok{total\_pr\_median =} \FunctionTok{median}\NormalTok{(total\_pr))}
\end{Highlighting}
\end{Shaded}

\begin{longtable}[]{@{}rr@{}}
\toprule\noalign{}
total\_pr\_mittelwert & total\_pr\_median \\
\midrule\noalign{}
\endhead
\bottomrule\noalign{}
\endlastfoot
47 & 46 \\
\end{longtable}

\begin{Shaded}
\begin{Highlighting}[]

\CommentTok{\# mit Extremwerten:}
\NormalTok{mariokart }\SpecialCharTok{|\textgreater{}} 
  \FunctionTok{summarise}\NormalTok{(}\AttributeTok{total\_pr\_mittelwert =} \FunctionTok{mean}\NormalTok{(total\_pr),}
            \AttributeTok{total\_pr\_median =} \FunctionTok{median}\NormalTok{(total\_pr))}
\end{Highlighting}
\end{Shaded}

\begin{longtable}[]{@{}rr@{}}
\toprule\noalign{}
total\_pr\_mittelwert & total\_pr\_median \\
\midrule\noalign{}
\endhead
\bottomrule\noalign{}
\endlastfoot
50 & 46 \\
\end{longtable}

\end{exercise}

\begin{exercise}[]\protect\hypertarget{exr-mw-wealthmd}{}\label{exr-mw-wealthmd}

Was schätzen Sie, wie hoch das \emph{mediane} Vermögen des Haushalte in
Deutschland im Jahr 2021 in etwa war (Bundesbank, 2023)?\footnote{ca.
  83600€}

\begin{enumerate}
\def\labelenumi{\alph{enumi})}
\tightlist
\item
  50.000 Euro
\item
  100.000 Euro
\item
  150.000 Euro
\item
  200.000 Euro
\item
  300.00 Euro\(\square\)
\end{enumerate}

\end{exercise}

\section{Quantile}\label{quantile}

Der Median teilt eine Verteilung in eine untere und ein obere Hälfte. Er
markiert sozusagen eine \enquote{50-Prozent-Marke} (der aufsteigend
sortierten Beobachtungen). Betrachten wir einmal nur alle Spiele, die
für weniger als 100 Euro verkauft wurden (\texttt{total\_pr}, finales
Verkaufsgebot), s. \textbf{?@fig-quantile-mario}. 50\% dieser Spiele
wurden für weniger als ca. 46 Euro verkauft und 50\% für mehr als 46
Euro. Der Median beträgt als 46 Euro.

Jetzt könnten wir nur die günstigere Hälfte betrachten und wieder nach
dem Median fragen (d.h. \texttt{total\_pr\ \textless{}\ 46}). Dieser
\enquote{Median der günstigeren Hälfte} grenzt damit das insgesamt
günstigste Viertel vom Rest der Verkaufsgebote ab. In unserem Datensatz
liegt dieser Wert bei ca. 41 Euro. Entsprechend kann man nach dem Wert
fragen, der das oberste Viertel vom Rest der Verkaufsgebote abtrennt.
Dieser Wert liegt bei ca. 54 Euro.

\begin{definition}[Quartile]\protect\hypertarget{def-quartile}{}\label{def-quartile}

Sortiert man die Daten aufsteigend, so nennt man den Wert, der das
Viertel mit den kleisten Wert vom Rest der Daten trennt das \emph{erste
Quartil} (Q1, 25\%). Den Median nennt man das \emph{zweite Quartil} (Q2,
50\%). Entsprechend heißt der Wert, der die drei Viertel kleinsten Werte
vom oberen Viertel abtrennt, das \emph{dritte Quartil} (Q3,
75\%).\(\square\)

\end{definition}

\begin{example}[Quartile des
Verkaufsgebot]\protect\hypertarget{exm-mario-qs}{}\label{exm-mario-qs}

\textbf{?@fig-quantile-mario} zeigt die Quartile für das
Verkaufsgebot.\(\square\)

\end{example}

Jetzt könnte man sagen, hey, warum nur in 25\%-Stücke die Verteilung
aufteilen? Warum nicht in 10\%-Schritten?

\begin{definition}[Dezile]\protect\hypertarget{def-dezile}{}\label{def-dezile}

Die neun Quantile \(p= 0.1, 0.2, \ldots, 1\), die die Verteilung in 10
gleiche Teile unterteilen, nennt man Dezile. \(\square\)

\end{definition}

Oder vielleicht in 1\%-Schritten oder in sonstigen Schnitten? Wo die
Quartile in 25\%-Schritten aufteilen, teilt in \emph{Quantil} in
\(p\)-Prozent-Schritten auf.

\begin{definition}[Quantile]\protect\hypertarget{def-quantile}{}\label{def-quantile}

Ein p-Quantil ist der Wert, der von \(p\) Prozent der Werte nicht
überschritten wird.\(\square\)

\end{definition}

\begin{tcolorbox}[enhanced jigsaw, bottomtitle=1mm, leftrule=.75mm, breakable, title=\textcolor{quarto-callout-note-color}{\faInfo}\hspace{0.5em}{Hinweis}, bottomrule=.15mm, titlerule=0mm, left=2mm, opacityback=0, colframe=quarto-callout-note-color-frame, rightrule=.15mm, colback=white, coltitle=black, toprule=.15mm, toptitle=1mm, colbacktitle=quarto-callout-note-color!10!white, arc=.35mm, opacitybacktitle=0.6]

Ein Quantil ist ein Oberbegriff für Quartile, Dezile, etc. \(\square\)

\end{tcolorbox}

\textbf{?@fig-quantile-qs1} zeigt das 1. (Q1), das 2. (Median) und das
3. Quartil für den Datensatz \texttt{mariokart2}.

\begin{figure}[H]

\begin{minipage}{0.50\linewidth}

\centering{

\includegraphics[width=0.7\linewidth,height=\textheight,keepaspectratio]{050-zusammenfassen_files/figure-pdf/fig-mario-qs1-1.pdf}

}

\subcaption{\label{fig-mario-qs1-1}Histogramm}

\end{minipage}%
%
\begin{minipage}{0.50\linewidth}

\centering{

\includegraphics[width=0.7\linewidth,height=\textheight,keepaspectratio]{050-zusammenfassen_files/figure-pdf/fig-mario-qs1-2.pdf}

}

\subcaption{\label{fig-mario-qs1-2}Dichtediagramm}

\end{minipage}%

\caption{\label{fig-mario-qs1}Q1, Q2 und Q3 für das Schlussgebot (nur
Spiele für weniger als 100 Euro)}

\end{figure}%

\emph{Quantile} kann man in R mit dem Befehl \texttt{quantile()}
berechnen:

\begin{Shaded}
\begin{Highlighting}[]
\NormalTok{mario\_quantile }\OtherTok{\textless{}{-}} 
\NormalTok{mariokart }\SpecialCharTok{\%\textgreater{}\%} 
  \FunctionTok{filter}\NormalTok{(total\_pr }\SpecialCharTok{\textless{}} \DecValTok{100}\NormalTok{) }\SpecialCharTok{\%\textgreater{}\%} 
  \FunctionTok{summarise}\NormalTok{(}\AttributeTok{q25 =} \FunctionTok{quantile}\NormalTok{(total\_pr, .}\DecValTok{25}\NormalTok{),}
            \AttributeTok{q50 =} \FunctionTok{quantile}\NormalTok{(total\_pr, .}\DecValTok{50}\NormalTok{),}
            \AttributeTok{q75 =} \FunctionTok{quantile}\NormalTok{(total\_pr, .}\DecValTok{75}\NormalTok{))}
\end{Highlighting}
\end{Shaded}

Abbildung~\ref{fig-quantile-mosaic} visualisiert verschiedene Quantile.
Man beachte, dass alle Regionen gleichgroße Flächen (d.h.
Wahrscheinlichkeitsmassen) aufweisen.

\begin{figure}[H]

\begin{minipage}{0.50\linewidth}

\centering{

\includegraphics[width=0.7\linewidth,height=\textheight,keepaspectratio]{050-zusammenfassen_files/figure-pdf/fig-quantile-mosaic-1.pdf}

}

\subcaption{\label{fig-quantile-mosaic-1}10\%-Schritte: Dezile}

\end{minipage}%
%
\begin{minipage}{0.50\linewidth}

\centering{

\includegraphics[width=0.7\linewidth,height=\textheight,keepaspectratio]{050-zusammenfassen_files/figure-pdf/fig-quantile-mosaic-2.pdf}

}

\subcaption{\label{fig-quantile-mosaic-2}1\%-Schritte: Perzentile}

\end{minipage}%

\caption{\label{fig-quantile-mosaic}Verschiedene Quantile visualisiert}

\end{figure}%

\section{Lagemaße}\label{sec-lage}

\begin{quote}
{\emoji{student}} Was ist der Oberbegriff für Median, Mittelwert und so
weiter?
\end{quote}

\begin{quote}
{\emoji{teacher}} Gute Frage! Wie würden Sie ihn nennen?
\end{quote}

\begin{definition}[Lagemaß]\protect\hypertarget{def-lage}{}\label{def-lage}

Ein \emph{Lagemaß} (synonym: Maß der zentralen Tendenz) für eine
Verteilung gibt einen Vorschlag, welchen Wert der Verteilung wir als
typisch, normal, erwartbar, repräsentativ oder \enquote{mittel} ansehen
sollten.\(\square\)

\end{definition}

\begin{example}[]\protect\hypertarget{exm-lagemaße}{}\label{exm-lagemaße}

Gebräuchliche Lagemaße sind:

\begin{itemize}
\tightlist
\item
  Mittelwert (arithmetisches Mittel)
\item
  Median
\item
  Quantile wie z.B. Quartile
\item
  Minimum (kleinster Wert)
\item
  Maximum (größter Wert)
\item
  Modus (häufigster Wert) \(\square\)
\end{itemize}

\end{example}

Berechnen wir Lagemaße für den Mariokart-Datensatz, s.
Listing~\ref{lst-mario-lage}. Es ist übrigens egal, wie sie die
Variablen benennen, die Sie berechnen: \texttt{mw} oder
\texttt{mittelwert} oder \texttt{mean} oder
\texttt{mein\_krasser\_variablenname} -- alles okay!

\begin{codelisting}

\caption{\label{lst-mario-lage}Syntax zur Berechnung von Lagemaßen}

\centering{

\begin{Shaded}
\begin{Highlighting}[]
\NormalTok{mariokart\_lagemaße\_total\_pr }\OtherTok{\textless{}{-}}
\NormalTok{  mariokart }\SpecialCharTok{\%\textgreater{}\%} 
  \FunctionTok{summarise}\NormalTok{(}\AttributeTok{mw =} \FunctionTok{mean}\NormalTok{(total\_pr),}
            \AttributeTok{md =} \FunctionTok{median}\NormalTok{(total\_pr),}
            \AttributeTok{q1 =} \FunctionTok{quantile}\NormalTok{(total\_pr, .}\DecValTok{25}\NormalTok{),}
            \AttributeTok{q2 =} \FunctionTok{quantile}\NormalTok{(total\_pr, .}\DecValTok{5}\NormalTok{),}
            \AttributeTok{q3 =} \FunctionTok{quantile}\NormalTok{(total\_pr, .}\DecValTok{75}\NormalTok{),}
            \AttributeTok{min =} \FunctionTok{min}\NormalTok{(total\_pr),}
            \AttributeTok{max =} \FunctionTok{max}\NormalTok{(total\_pr))}
\NormalTok{mariokart\_lagemaße\_total\_pr}
\end{Highlighting}
\end{Shaded}

\begin{longtable*}[]{@{}rrrrrrr@{}}
\toprule\noalign{}
mw & md & q1 & q2 & q3 & min & max \\
\midrule\noalign{}
\endhead
\bottomrule\noalign{}
\endlastfoot
50 & 46 & 41 & 46 & 54 & 29 & 327 \\
\end{longtable*}

}

\end{codelisting}%

\subsection{Gruppierte Lagemaße}\label{gruppierte-lagemauxdfe}

Häufig möchte man Statistiken wie Lagemaße für mehrere Teilgruppen --
z.B. Mittlere Körpergröße von Frauen vs.~Mittlere Körpergröße von Männer
-- berechnen und dann vergleichen. Die zugrundeliegende stehende
\emph{Forschungsfrage} könnte lauten:

\begin{quote}
Unterscheidet sich die mittlere Körpergröße von Frauen und Männern?
\end{quote}

Oder vielleicht:

\begin{quote}
Hat das Geschlecht einen Einfluss auf die Körpergröße?
\end{quote}

Anders ausgedrückt:

\begin{quote}
Körpergröße \(\color{ycol}{\text{y}}\) ist eine Funktion des Geschlechts
\(\color{xcol}{G}\).
\end{quote}

Die \emph{Modellformel} könnte also lauten:

\[\color{ycol}{y} \; \color{black}{ \sim } \; \color{xcol}{G}\]

Gruppierte Lagemaße lassen sich in R z.B. so berechnen, s.
Listing~\ref{lst-mario-lage-gruppiert}, also ähnlich wie in
Listing~\ref{lst-mario-lage}.

\begin{codelisting}

\caption{\label{lst-mario-lage-gruppiert}Gruppierte Lagemaße}

\centering{

\begin{Shaded}
\begin{Highlighting}[]
\NormalTok{mariokart\_lagemaße\_gruppiert }\OtherTok{\textless{}{-}}
\NormalTok{  mariokart }\SpecialCharTok{\%\textgreater{}\%} 
  \FunctionTok{group\_by}\NormalTok{(wheels) }\SpecialCharTok{\%\textgreater{}\%}  \CommentTok{\# neue Zeile, der Rest ist gleich!}
  \FunctionTok{summarise}\NormalTok{(}\AttributeTok{mw =} \FunctionTok{mean}\NormalTok{(total\_pr))}
\end{Highlighting}
\end{Shaded}

\begin{longtable*}[]{@{}rr@{}}
\toprule\noalign{}
wheels & mw \\
\midrule\noalign{}
\endhead
\bottomrule\noalign{}
\endlastfoot
0 & 41 \\
1 & 44 \\
2 & 61 \\
3 & 70 \\
4 & 65 \\
\end{longtable*}

}

\end{codelisting}%

Abbildung~\ref{fig-mw3} zeigt ein Beispiel für ungruppierte (links) bzw.
gruppierte (rechts) Mittelwerte; vgl. Abbildung~\ref{fig-mw2}. Wie man
in dem Diagramm sieht, kann das \emph{Residuum kleiner} werden bei einer
Gruppierung (im Vergleich zu einem ungruppierten, \enquote{globalen}
Mittelwert): Innerhalb der Gruppe ohne Lenkräder und innerhalb der
Gruppe mit 2 Lenkrädern sind die Abweichungen zu ihrem
Gruppen-Mittelwert relativ gering -- im Vergleich zu den Abweichungen
der Preise zum ungruppierten Mittelwert.

\begin{figure}[H]

\begin{minipage}{0.45\linewidth}

\centering{

\includegraphics[width=1\linewidth,height=\textheight,keepaspectratio]{050-zusammenfassen_files/figure-pdf/fig-mw3-1.pdf}

}

\subcaption{\label{fig-mw3-1}Mittelwert für Verkaufspreis (ungruppiert)}

\end{minipage}%
%
\begin{minipage}{0.10\linewidth}
~\end{minipage}%
%
\begin{minipage}{0.45\linewidth}

\centering{

\includegraphics[width=1\linewidth,height=\textheight,keepaspectratio]{050-zusammenfassen_files/figure-pdf/fig-mw3-2.pdf}

}

\subcaption{\label{fig-mw3-2}Mittelwert für Verkaufspreis gruppiert nach
Anzahl der Lenkräder}

\end{minipage}%

\caption{\label{fig-mw3}Der mittlere Preis von Mariokart-Spielen als
horizontale Gerade eingezeichnet}

\end{figure}%

\begin{definition}[Punktmodell]\protect\hypertarget{def-punktmodell}{}\label{def-punktmodell}

Ein Modell, welches für alle Beobachtungen ein und denselben Wert
annimmt (vorhersagt), heißt ein \emph{Punktmodell}. Anders gesagt fasst
ein Punktmodell eine Wertereihe (häufig ist das eine Tabellenspalte) zu
einer einzelnen Zahl zusammen, einem \enquote{Punkt} in diesem Sinne, s.
Gleichung~\ref{eq-desk}.\(\square\)

\end{definition}

\begin{equation}\phantomsection\label{eq-desk}{\begin{array}{|c|} \hline \\ \hline \\\\\\ \hline \end{array} \qquad \rightarrow \qquad \begin{array}{|c|} \hline \\ \hline  \hline \end{array}}\end{equation}

Mittelwert, Median und Quartile sind Beispiele für Punktmodelle: Sie
fassen eine Verteilung zu einem einzelnen Wert zusammen und geben uns
ein \enquote{Bild} der Daten, machen Sie uns verständlich - sie sind uns
ein Modell.

\section{Wie man mit Statistik
lügt}\label{wie-man-mit-statistik-luxfcgt-2}

Mit Statistik kann man vortrefflich lügen, heißt es. Woran liegt das?
Der Grund ist, dass die Statistik Freiheitsgrade lässt: Es gibt nicht
nur einen richtigen Weg, um eine statistische Analyse durchzuführen.
Viele Wege führen nach Rom (aber nicht alle). Um Manipulationsversuche
abzuwehren oder einfache Fehler und Unschärfen ohne böse Abwehr
aufzudecken, gibt es ein probates Gegenmittel: \emph{Transparenz}.

\phantomsection\label{callout-important}
Stellen Sie hohe Anforderung an die Transparenz einer statistischen
Analyse. Nur durch Nachprüfbarkeit können Sie sich von der
Stichhaltigkeit der Ergebnisse und deren Interpretation überzeugen.

Hier ist eine (nicht abschließende!) Checkliste, was Sie nachprüfen
sollten, um die Belastbarkeit einer Analyse sicherzustellen Wicherts et
al. (2016):

\begin{enumerate}
\def\labelenumi{\arabic{enumi}.}
\tightlist
\item
  Wurde die Art und die Zeitdauer der Datenerhebung vorab festgelegt und
  berichtet?
\item
  Wurden ausreichend Daten gesammelt (z.B. mind. 20 Beobachtungen pro
  Gruppe)?
\item
  Wurden alle untersuchten Variablen berichtet?
\item
  Wurden alle durchgeführten Interventionen berichtet?
\item
  Wurden Daten aus der Analyse entfernt? Wenn ja, gibt es eine
  (stichhaltige) Begründung?
\end{enumerate}

\section{Vertiefung}\label{vertiefung-5}

\begin{example}[Survival-Tipp]\protect\hypertarget{exm-survival1}{}\label{exm-survival1}

Eine Studentin aus dem dem Bachelorstudiengang \enquote{Angewandte
Medien- und Wirtschaftspsychologie} mit Schwerpunkt \emph{Data Science}
berichtet ihre \enquote{Survival-Tipps} für Statistik.

\begin{enumerate}
\def\labelenumi{\arabic{enumi}.}
\tightlist
\item
  Wenn man mal nicht weiterkommt, hilft es auch mal ein paar Tage
  Abstand von R und Statistik zu nehmen.
\item
  Es hilft, sich während des Semesters neue Begriffe und ihre Erklärung
  zusammenschreiben.
\item
  Gut ist auch, sich mit KommilitonInnen auszutauschen oder in höheren
  Semestern nach Tipps fragen.\(\square\)
\end{enumerate}

\end{example}

\begin{quote}
{\emoji{student}} Irgendwie kann ich mir R-Code so schlecht merken.
\end{quote}

\begin{quote}
{\emoji{teacher}} Frag doch mal ChatGPT, oder einen anderen Chatbot, da
bekommt man auch R-Code ausgegegeben.
\end{quote}

\begin{exercise}[Übungsfragen vom
Chat-Bot]\protect\hypertarget{exr-chatgpt}{}\label{exr-chatgpt}

Fragen Sie einen Chat-Bot wie ChatGPT nach Übungsaufgaben.

Sie können sich an folgenden Prompt orientieren. Empfehlenswert ist mit
verschiedenen Prompts zu experimentieren.

\begin{quote}
{\emoji{student}} Ich bin ein Student in einem Bachelor-Studiengang für
Psychologie. Gerade bereite ich mich auf die Klausur im Fach
\enquote{Grundlagen der Statistik} vor. Bitte schreibe mir Aufgaben, die
mir helfen, mich auf die Prüfung vorzubereiten. Die Fragen sollten
folgende Themen beinhalten: Maße der zentralen Tendenz, Grundlagen von
R, Skalenniveau (z.B. Nominalskala vs.~Intervallskala),
Verteilungsformen, Normalverteilungen, z-Werte. Bitte schreibe die
Aufgabe im Stil von Richtig-Falsch-Aufgaben. Schreibe ca. 10 Aufgaben.
\end{quote}

\(\square\)

\end{exercise}

\section{Aufgaben}\label{aufgaben-4}

Ein Teil der folgenden Aufgaben kann Stoff beinhalten, den Sie noch
nicht kennen, aber später kennenlernen. Ignorieren Sie daher
Aufgaben(teile) mit (noch) unbekannte Stoff.

Die Webseite \href{https://datenwerk.netlify.app}{datenwerk.netlify.app}
stellt eine Reihe von einschlägigen Übungsaufgaben bereit. Sie können
die Suchfunktion der Webseite nutzen, um die Aufgaben mit den folgenden
Namen zu suchen:

\begin{enumerate}
\def\labelenumi{\arabic{enumi}.}
\tightlist
\item
  \href{https://sebastiansauer.github.io/Datenwerk/posts/kennwert-robust/kennwert-robust}{Kennwert-robust}
\item
  \href{https://sebastiansauer.github.io/Datenwerk/posts/mw-berechnen/mw-berechnen.html}{mw-berechnen}
\item
  \href{https://sebastiansauer.github.io/Datenwerk/posts/mariokart-max2/mariokart-max2.html}{mariokart-max2}
\item
  \href{https://sebastiansauer.github.io/Datenwerk/posts/nasa01/nasa01.html}{nasa01}
\item
  \href{https://sebastiansauer.github.io/Datenwerk/posts/mariokart-mean1/mariokart-mean1.html}{mariokart-mean1}
\item
  \href{https://sebastiansauer.github.io/Datenwerk/posts/wrangle10/wrangle10.html}{wrangle10}
\item
  \href{https://sebastiansauer.github.io/Datenwerk/posts/summarise01/summarise01.html}{summarise01}
\item
  \href{https://sebastiansauer.github.io/Datenwerk/posts/mariokart-max1/mariokart-max1.html}{mariokart-max1}
\item
  \href{https://sebastiansauer.github.io/Datenwerk/posts/schiefe1/schiefe1}{Schiefe1}
\item
  \href{https://sebastiansauer.github.io/Datenwerk/posts/mariokart-mean2/mariokart-mean2.html}{mariokart-mean2}
\item
  \href{https://sebastiansauer.github.io/Datenwerk/posts/summarise03/summarise03.html}{summarise03}
\item
  \href{https://sebastiansauer.github.io/Datenwerk/posts/mariokart-mean4/mariokart-mean4.html}{mariokart-mean4}
\item
  \href{https://sebastiansauer.github.io/Datenwerk/posts/mariokart-mean3/mariokart-mean3.html}{mariokart-mean3}
\item
  \href{https://sebastiansauer.github.io/Datenwerk/posts/summarise02/summarise02.html}{summarise02}
\end{enumerate}

\begin{tcolorbox}[enhanced jigsaw, bottomtitle=1mm, leftrule=.75mm, breakable, title=\textcolor{quarto-callout-tip-color}{\faLightbulb}\hspace{0.5em}{Tipp}, bottomrule=.15mm, titlerule=0mm, left=2mm, opacityback=0, colframe=quarto-callout-tip-color-frame, rightrule=.15mm, colback=white, coltitle=black, toprule=.15mm, toptitle=1mm, colbacktitle=quarto-callout-tip-color!10!white, arc=.35mm, opacitybacktitle=0.6]

Schauen Sie sich auch mal auf
\href{https://datenwerk.netlify.app}{datenwerk.netlify.app} die Aufgaben
zu z.B. dem Tag
\href{https://sebastiansauer.github.io/Datenwerk/\#category=eda}{EDA}
an. \(\square\)

\end{tcolorbox}

\section{Literaturhinweise}\label{literaturhinweise-4}

Es gibt viele Lehrbücher zu den Grundlagen der Statistik; die Inhalte
dieses Kapitels gehören zu den Grundlagen der Statistik. Vielleicht ist
es am einfachsten, wenn Sie einfach in Ihrer Bibliothek des Vertrauens
nach einem typischen Lehrbuch schauen. Beispiel für Lehrbücher sind
Mittag \& Schüller (2020) oder Oestreich \& Romberg (2014); ein
Klassiker ist Bortz \& Schuster (2010). Ein Fokus auf R legt Sauer
(2019). Wer vor Englisch nicht zurückschreckt, ist mit Cetinkaya-Rundel
\& Hardin (2021) oder R. Poldrack (2022) gut beraten. Beide Bücher sind
online verfügbar. Tipp: Mit dem Browser einfach auf Deutsch übersetzen.

\chapter{Modellgüte}\label{modellguxfcte}

\section{Lernsteuerung}\label{lernsteuerung-5}

Abbildung~\ref{fig-ueberblick} zeigt den Standort dieses Kapitels im
Lernpfad und gibt damit einen Überblick über das Thema dieses Kapitels
im Kontext aller Kapitel.

\subsection{Lernziele}\label{lernziele-6}

\begin{itemize}
\tightlist
\item
  Sie kennen gängige Maße der Streuung einer Stichprobe und können diese
  definieren und mit Beispielen erläutern.
\item
  Sie können gängige Maße der Streuung einer Stichprobe mit R berechnen.
\item
  Sie können die Bedeutung von Streuung für die Güte eines Modells
  erläutern.
\end{itemize}

\subsection{Benötigte R-Pakete}\label{benuxf6tigte-r-pakete-3}

In diesem Kapitel benötigen Sie folgende R-Pakete.

\begin{Shaded}
\begin{Highlighting}[]
\FunctionTok{library}\NormalTok{(tidyverse)}
\FunctionTok{library}\NormalTok{(easystats)}
\FunctionTok{library}\NormalTok{(DataExplorer)}
\end{Highlighting}
\end{Shaded}

\subsection{Benötigte Daten}\label{benuxf6tigte-daten-4}

Listing~\ref{lst-mario-path} definiert den Pfad zum Datensatz
\texttt{mariokart} und importiert die zugehörige CSV-Datei in R, so dass
wir einen Tibble mit Namen \texttt{mariokart} erhalten.

\begin{codelisting}

\caption{\label{lst-mario-path}Pfad zum Datensatz \enquote*{mariokart}}

\centering{

\begin{Shaded}
\begin{Highlighting}[]
\NormalTok{mariokart\_path }\OtherTok{\textless{}{-}} \FunctionTok{paste0}\NormalTok{(}
  \StringTok{"https://vincentarelbundock.github.io/Rdatasets/"}\NormalTok{,}
  \StringTok{"csv/openintro/mariokart.csv"}\NormalTok{)}

\NormalTok{mariokart }\OtherTok{\textless{}{-}} \FunctionTok{read.csv}\NormalTok{(mariokart\_path)}
\end{Highlighting}
\end{Shaded}

}

\end{codelisting}%

\subsection{Zum Einstieg}\label{zum-einstieg-1}

\begin{exercise}[Freiwillige
vor!]\protect\hypertarget{exr-streuung-erkennen}{}\label{exr-streuung-erkennen}

Für diese kleine Live-Demonstration brauchen wir einige Freiwillige. Die
Lehrkraft teilt die Freiwilligen in zwei Gruppen, Gruppe
\emph{Gleich-Groß} und Gruppe \emph{Verschieden-Groß}. Erkennen Sie,
dass die \emph{Unterschiedlichkeit} der Größe in Gruppe
\emph{Gleich-Groß} gering ist, aber in Gruppe \emph{Verschieden-Groß}
hoch? \(\square\)

\end{exercise}

\section{Warum Sie die Streuung Ihrer Daten kennen
sollten}\label{warum-sie-die-streuung-ihrer-daten-kennen-sollten}

\subsection{Die Schlankheitspille von
Prof.~Weiss-Ois}\label{sec-weiss-ois}

Prof.~Weiss-Ois hat eine Erfindung gemacht, eine Schlankheitspille
(flaticon, 2024).

\begin{figure}[H]

\begin{minipage}{0.46\linewidth}

\includegraphics[width=0.25\linewidth,height=\textheight,keepaspectratio]{img/teacher.png}

\subcaption{\label{}\textbf{Was er sagt:} \enquote{Ich habe eine
Schlankheitspille entwickelt, die pro Einnahme das Gewicht im Schnitt um
1kg reduziert!}}
\end{minipage}%
%
\begin{minipage}{0.09\linewidth}
~\end{minipage}%
%
\begin{minipage}{0.46\linewidth}

\includegraphics[width=0.25\linewidth,height=\textheight,keepaspectratio]{img/teacher.png}

\subcaption{\label{}\textbf{Was er NICHT sagt:} \enquote{Allerdings
streuten die Werte der Gewichtsveränderung um 10kg um den Mittelwert
herum.}}
\end{minipage}%

\end{figure}%

Würden Sie die Pille von Prof.~I. Ch. Weiss-Ois nehmen?

\begin{enumerate}
\def\labelenumi{\alph{enumi})}
\tightlist
\item
  ja, ich zahle 1000 Euro!
\item
  ja
\item
  nein
\item
  Nur wenn ich 100 Euro bekomme
\item
  Okay, für 1000 Euro\(\square\)
\end{enumerate}

\begin{tcolorbox}[enhanced jigsaw, bottomtitle=1mm, leftrule=.75mm, breakable, title=\textcolor{quarto-callout-important-color}{\faExclamation}\hspace{0.5em}{Wichtig}, bottomrule=.15mm, titlerule=0mm, left=2mm, opacityback=0, colframe=quarto-callout-important-color-frame, rightrule=.15mm, colback=white, coltitle=black, toprule=.15mm, toptitle=1mm, colbacktitle=quarto-callout-important-color!10!white, arc=.35mm, opacitybacktitle=0.6]

Wie sehr die Werte eines Modells streuen, ist eine wichtige
Information.\(\square\)

\end{tcolorbox}

\subsection{Wie man seine Kuh über den Fluss
bringt}\label{wie-man-seine-kuh-uxfcber-den-fluss-bringt}

Treffen sich zwei Bauern, Fritz Furchenzieher und Karla Kartoffelsack.
Fritz will mit seiner Kuh einen Fluss überqueren, nur kann die Kuh nicht
schwimmen (ob es Fritz kann, ist nicht überliefert).

\begin{quote}
{\emoji{man-farmer}} (Fritz): Sag mal, Karla, ist der Fluss tief?
\end{quote}

\begin{quote}
{\emoji{woman-farmer}} (Karla): Nö, im Schnitt nur einen Meter.
\end{quote}

Also führt Fritz seine Kuh durch den Fluss, leider kam die Kuh nicht am
anderen Ufer an, im Floß ersoffen, s. Abbildung~\ref{fig-fluss-tief}.

\begin{figure}[H]

\centering{

\pandocbounded{\includegraphics[keepaspectratio]{img/fluss-tief.png}}

}

\caption{\label{fig-fluss-tief}Der Fluss ist im Schnitt nur einen Meter
tief, trotzdem ist die Kuh ersoffen.}

\end{figure}%

\begin{quote}
{\emoji{woman-farmer}} (Karla): Übrigens, Lagemaße sagen nicht alles,
Fritz.
\end{quote}

\begin{quote}
{\emoji{man-farmer}} (Fritz): Läuft die Kuh durch den Fluss, kann sie
schwimmen oder 's ist Schluss.
\end{quote}

\begin{tcolorbox}[enhanced jigsaw, bottomtitle=1mm, leftrule=.75mm, breakable, title=\textcolor{quarto-callout-important-color}{\faExclamation}\hspace{0.5em}{Wichtig}, bottomrule=.15mm, titlerule=0mm, left=2mm, opacityback=0, colframe=quarto-callout-important-color-frame, rightrule=.15mm, colback=white, coltitle=black, toprule=.15mm, toptitle=1mm, colbacktitle=quarto-callout-important-color!10!white, arc=.35mm, opacitybacktitle=0.6]

Die Streuung ihrer Daten zu kennen ist eine wesentliche Information.
\(\square\)

\end{tcolorbox}

\section{Woran erkennt man ein gutes
Modell?}\label{woran-erkennt-man-ein-gutes-modell}

Abbildung~\ref{fig-streuung} zeigt ein einfaches Modell (Mittelwert) mit
wenig Streuung (links) vs.~ein einfaches Modell mit viel Streuung
(rechts). Links ist die Streuung der Schlankheitspille
\emph{Dicktableitin} und rechts von der Schlankheitspille
\emph{Pfundafliptan} abgetragen. Die vertikalen grauen Balken in
Abbildung~\ref{fig-streuung} kennzeichnen den (absoluten) Abstand von
jeweils einem Datenpunkt zum Mittelwert (horizontale orange Linie). Je
länger die vertikalen \enquote*{Abstandsbalken} insgesamt, desto größer
die Streuung. Die X-Achse (\texttt{id}) reiht die Versuchspersonen auf.

\begin{figure}[H]

\centering{

\includegraphics[width=0.75\linewidth,height=\textheight,keepaspectratio]{060-modellguete_files/figure-pdf/fig-streuung-1.pdf}

}

\caption{\label{fig-streuung}Wenig (links) vs.~viel Streuung (rechts).}

\end{figure}%

Bei einem Modell mit \emph{wenig} Streuung liegen die tatsächlichen,
beobachtete Werte (\(y\)) nah an den Modellwerten (vorhergesagten
Werten, \(\hat{y}\)); die Abweichungen \(e = y - \hat{y}\) sind also
gering (der Modellfehler ist klein). Bei einem Modell mit \emph{viel}
Streuung ist der Modellfehler \(e\) (im Vergleich dazu) groß.

\begin{example}[Daten zur Schlankheitskur von
Prof.~Weiss-Ois]\protect\hypertarget{exm-weiss-ois}{}\label{exm-weiss-ois}

In Abbildung~\ref{fig-streuung} sind die Daten zu der
Gewichtsveränderung nach Einnahme von \enquote{Schlankheitspillen}
zweier verschiedener Präparate. Wie man sieht unterscheidet sich die
typische (vorhergesagte) Gewichtsveränderung zwischen den beiden
Präparaten kaum. Die Streuung allerdings schon. Links sieht man die
Gewichtsveränderungen nach Einnahme des Präparats \enquote{Dickableibtin
extra mild} (c) und rechts das Präparat von Prof.~Weiss-Ois
\enquote{Pfundafliptan Forte}. Welches Präparat würden Sie lieber
einnehmen?\(\square\)

\end{example}

\begin{tcolorbox}[enhanced jigsaw, bottomtitle=1mm, leftrule=.75mm, breakable, title=\textcolor{quarto-callout-important-color}{\faExclamation}\hspace{0.5em}{Wichtig}, bottomrule=.15mm, titlerule=0mm, left=2mm, opacityback=0, colframe=quarto-callout-important-color-frame, rightrule=.15mm, colback=white, coltitle=black, toprule=.15mm, toptitle=1mm, colbacktitle=quarto-callout-important-color!10!white, arc=.35mm, opacitybacktitle=0.6]

Wir wollen ein präzises Modell, also kurze Fehlerbalken: Das Modell soll
die Daten gut erklären, also wenig vom tatsächlichen Wert abweichen.
Jedes Modell sollte Informationen über die Präzision des Modellwerts
bzw. der Modellwerte (Vorhersagen) angeben. Ein Modell ohne Angaben der
Modellgüte, d.h. der Präzision der Schätzung des Modellwerts, ist wenig
nütze.\(\square\)

\end{tcolorbox}

\begin{quote}
{\emoji{student}} Ich frage mich, ob man so ein Modell nicht verbessern
kann?
\end{quote}

\begin{quote}
{\emoji{teacher}} Die Frage ist, was wir mit \enquote{verbessern}
meinen?
\end{quote}

\begin{quote}
{\emoji{student}} Naja, kürzere Fehlerbalken, ist doch klar!
\end{quote}

Im Beispiel von Marikoart: Da die Anzahl der Lenkräder mit dem
Verkaufsgebot zusammenhängt, könnte es vielleicht sein, dass wir die
Lenkräder-Anzahl da irgendwie nutzen könnten. Das sollten wir
ausprobieren. Abbildung~\ref{fig-fehler-red} zeigt, dass die
Fehlerbalken \emph{kürzer} werden, wenn wir ein (sinnvolles) komplexeres
Modell finden. Innerhalb jeder der beiden Gruppen (mit 2 Lenkrädern
vs.~mit 0 Lenkrädern) sind die Fehlerbalken jeweils im Durchschnitt
kürzer (rechtes Teildiagramm) als im Modell ohne Gruppierung (linkes
Teildiagramm). Aus Gründen der Übersichtlichkeit wurden nur Autos mit
Verkaufsgebot von weniger als 100 Euros berücksichtigt und nur Spiele
mit 0 oder mit 2 Lenkrädern.

\begin{figure}[H]

\begin{minipage}{0.50\linewidth}

\centering{

\includegraphics[width=0.7\linewidth,height=\textheight,keepaspectratio]{060-modellguete_files/figure-pdf/fig-fehler-red-1.pdf}

}

\subcaption{\label{fig-fehler-red-1}Einfaches Modell mit viel
Fehlerstreuung}

\end{minipage}%
%
\begin{minipage}{0.50\linewidth}

\centering{

\includegraphics[width=0.7\linewidth,height=\textheight,keepaspectratio]{060-modellguete_files/figure-pdf/fig-fehler-red-2.pdf}

}

\subcaption{\label{fig-fehler-red-2}Komplexres Modell mit wenig
Fehlerstreuung}

\end{minipage}%

\caption{\label{fig-fehler-red}Fehlerbalken in einem einfachen und
komplexeren Modell. Links: Fehlerbalken im einfachen Modell: Ein
Mittelwert; viel Streuung insgesamt - y \textasciitilde{} 1.
Fehlerbalken im komplexen Modell: Zwei Mittelwerte; weniger Streuung in
jeder Gruppe. Das erkennt man daran, dass die vertikalen, grauen
Abstandsbalken im Schnitt kürzer sind als im einfachen Modell (links). y
\textasciitilde{} G}

\end{figure}%

\begin{tcolorbox}[enhanced jigsaw, bottomtitle=1mm, leftrule=.75mm, breakable, title=\textcolor{quarto-callout-important-color}{\faExclamation}\hspace{0.5em}{Wichtig}, bottomrule=.15mm, titlerule=0mm, left=2mm, opacityback=0, colframe=quarto-callout-important-color-frame, rightrule=.15mm, colback=white, coltitle=black, toprule=.15mm, toptitle=1mm, colbacktitle=quarto-callout-important-color!10!white, arc=.35mm, opacitybacktitle=0.6]

Durch sinnvolle, komplexere Modelle sinkt die Fehlerstreuung eines
Modells.\(\square\)

\end{tcolorbox}

\section{Streuungsmaße}\label{sec-streuung}

\begin{definition}[Streuungsmaße]\protect\hypertarget{def-streuungsmaße}{}\label{def-streuungsmaße}

Ein Streuungsmaß quantifiziert die Variabilität (Unterschiedlichkeit,
Streuung) eines Merkmals. \(\square\)

\end{definition}

\begin{definition}[]\protect\hypertarget{def-range}{}\label{def-range}

Ein einfaches Streuungsmaß ist der \emph{Range} \(R\), definiert als
Abstand von größtem und kleinsten Wert eines Merkmals X:
\(R := X_{max} - X_{min}. \square\)

\end{definition}

\begin{example}[]\protect\hypertarget{exm-range}{}\label{exm-range}

Angenommen, wir haben einen Datensatz zum Merkmal \enquote{Alter} mit
den Werte 1, 23, 42, 100. Dann beträgt der Range: \(R = 100 - 1 = 99\).
Das bedeutet, dass die Werte des Merkmals über 99 Einheiten (Jahre in
diesem Fall) verteilt sind. \(\square\)

\end{example}

Dieses Mermals ist aber nicht robust (gegenüber Extremwerten) und sollte
daher nur mit Einschränkung verwendet werden.

\subsection{Der mittlere
Abweichungsbalken}\label{der-mittlere-abweichungsbalken}

\begin{quote}
{\emoji{student}} Wir müssen jetzt mal präziser werden! Wie können wir
die Streuung berechnen?
\end{quote}

\begin{quote}
{\emoji{teacher}} Gute Frage! Am einfachsten ist es, wenn wir die
mittlere Länge eines Abweichungsbalkens ausrechnen.
\end{quote}

Legen wir (gedanklich) alle Abweichungsbalken \(e\) aneinander und
teilen durch die Anzahl \(n\) der Balken, so erhalten wir wir den
\enquote{mittleren Abweichungsbalken}, den wir mit \(\bar{e}\)
bezeichnen könnten. Diesen Kennwert bezeichnet man als \emph{Mean
Absolute Error} (MAE) bzw. als \emph{Mittlere Absolutabweichung} (MAA).
Er ist so definiert, s. Gleichung~\ref{eq-mae}.

\begin{definition}[Mittlere
Absolutabweichung]\protect\hypertarget{def-mae}{}\label{def-mae}

Die Mittlere Absolutabweichung (MAA, MAE) ist definiert als die Summe
der Absolutwerte der Differenzen eines Messwerts zum Mittelwert, geteilt
durch die Anzahl der Messwerte. (Wenn man solche Sätze liest, fühlt sich
die Formel fast einfacher an.)

\begin{equation}\phantomsection\label{eq-mae}{{\displaystyle \mathrm {MAE} :={\frac {\sum _{i=1}^{n}\left|y_{i}-\bar{y}\right|}{n}}={\frac {\sum _{i=1}^{n}\left|e_{i}\right|}{n}}=\bar{e}. \square}}\end{equation}

\end{definition}

\begin{example}[]\protect\hypertarget{exm-mae}{}\label{exm-mae}

Abbildung~\ref{fig-mae} visualisiert ein einfaches Beispiel zum MAE.
Rechnen wir den MAE für das Beispiel von Abbildung~\ref{fig-mae} aus:

\(MAE = \frac{1 + |- 3| + 1 + 1}{4} = 6/4 = 1.5 \quad \square\)

\end{example}

\begin{figure}[H]

\centering{

\includegraphics[width=0.5\linewidth,height=\textheight,keepaspectratio]{060-modellguete_files/figure-pdf/fig-mae-1.pdf}

}

\caption{\label{fig-mae}Abweichungsbalken und der MAE}

\end{figure}%

Natürlich können wir R auch die Rechenarbeit überlassen.

\begin{quote}
{\emoji{robot}} Loving it!!
\end{quote}

Schauen Sie: Den Mittelwert (s. Abbildung~\ref{fig-mae}) kann man doch
mit Fug und Recht als ein \emph{lineares Modell}, eine Gerade,
betrachten, oder nicht? Schließlich erklären wir \(y\) anhand einer
Gerade (die parallel zur X-Achse ist).

In R gibt es einen Befehl für ein \emph{l}ineares \emph{M}odell, er
heißt \texttt{lm}.

Die Syntax von \texttt{lm()} lautet:

\texttt{lm(y\ \textasciitilde{}\ 1,\ data\ =\ meine\_daten)}.

In Worten:

\begin{quote}
Hey R, berechne mit ein lineares Modell zur Erklärung von Y. Aber
verwende keine andere Variable zur Erklärung von Y, sondern nimm den
Mittelwert von Y.
\end{quote}

\begin{Shaded}
\begin{Highlighting}[]
\NormalTok{lm1 }\OtherTok{\textless{}{-}} \FunctionTok{lm}\NormalTok{(y }\SpecialCharTok{\textasciitilde{}} \DecValTok{1}\NormalTok{, }\AttributeTok{data =}\NormalTok{ d)}
\end{Highlighting}
\end{Shaded}

Den MAE können wir uns jetzt so ausgeben lassen:

\begin{Shaded}
\begin{Highlighting}[]
\FunctionTok{mae}\NormalTok{(lm1)}
\DocumentationTok{\#\# [1] 1.5}
\end{Highlighting}
\end{Shaded}

\subsection{Der Interquartilsabstand}\label{der-interquartilsabstand}

Der Interquartilsabstand (IQA; engl. inter quartile range, IQR) ist ein
Streuungsmaß, das nicht auf dem Mittelwert aufbaut. Der IQR ist robuster
als z.B. der MAA oder die Varianz und die Standardabweichung.

\begin{definition}[Interquartilsabstand]\protect\hypertarget{def-iqr}{}\label{def-iqr}

Der Interquartilsabstand ist definiert als der die (absolute) Differenz
vom 3. Quartil und 1. Quartil: \(IQR := Q_3-Q_1. \; \square\)

\end{definition}

\begin{example}[IQR im
Hörsaal]\protect\hypertarget{exm-iqr}{}\label{exm-iqr}

In einem Statistikkurs betragen die Quartile der Körpergröße: Q1: 1.65m,
Q2 (Median): 1,70m, Q3: 1.75m. Der IQR beträgt dann:
\(IQR = Q_3-Q_1 = 1.75m - 1.65m = 0.10m\), d.h. 10 cm.\(\square\)

\end{example}

Abbildung~\ref{fig-iqr-mario} stellt den IQR (und einige Quantile) für
den Verkaufspreise von Mariokart-Spielen dar.

\begin{figure}[H]

\begin{minipage}{0.50\linewidth}

\centering{

\includegraphics[width=0.7\linewidth,height=\textheight,keepaspectratio]{060-modellguete_files/figure-pdf/fig-iqr-mario-1.pdf}

}

\subcaption{\label{fig-iqr-mario-1}Histogramm}

\end{minipage}%
%
\begin{minipage}{0.50\linewidth}

\centering{

\includegraphics[width=0.7\linewidth,height=\textheight,keepaspectratio]{060-modellguete_files/figure-pdf/fig-iqr-mario-2.pdf}

}

\subcaption{\label{fig-iqr-mario-2}Dichtediagramm}

\end{minipage}%

\caption{\label{fig-iqr-mario}IQR, Q1, Q2 und Q3 für das Schlussgebot
(nur Spiele für weniger als 100 Euro)}

\end{figure}%

\subsection{Streuungsmaße für
Normalverteilungen}\label{streuungsmauxdfe-fuxfcr-normalverteilungen}

Normalverteilungen sind recht häufig anzutreffen in der Praxis der
Datenanalyse. Daher lohnt es sich, zu überlegen, wie man diese
Verteilungen gut zusammenfasst. Man kann zeigen, dass eine
Normalverteilung sich komplett über ihren \emph{Mittelwert} sowie ihre
\emph{Standardabweichung} beschreiben lässt (Lyon, 2014). Außerdem gilt:
Sind Ihre Daten normalverteilt, dann sind die Abweichungen vom
Mittelwert auch normalverteilt. Denn wenn man eine Konstante zu einer
Verteilung addiert (bzw. subtrahiert), \enquote{verschiebt man den Berg}
ja nur zur Seite, ohne seine Form zu verändern, s.
Abbildung~\ref{fig-norm-dev}.

\begin{figure}[H]

\centering{

\includegraphics[width=0.75\linewidth,height=\textheight,keepaspectratio]{060-modellguete_files/figure-pdf/fig-norm-dev-1.pdf}

}

\caption{\label{fig-norm-dev}Die Abweichungen zum Mittelwert (MW) einer
normalverteilten Variable sind selber normalverteilt}

\end{figure}%

\begin{tcolorbox}[enhanced jigsaw, bottomtitle=1mm, leftrule=.75mm, breakable, title=\textcolor{quarto-callout-note-color}{\faInfo}\hspace{0.5em}{Hinweis}, bottomrule=.15mm, titlerule=0mm, left=2mm, opacityback=0, colframe=quarto-callout-note-color-frame, rightrule=.15mm, colback=white, coltitle=black, toprule=.15mm, toptitle=1mm, colbacktitle=quarto-callout-note-color!10!white, arc=.35mm, opacitybacktitle=0.6]

Hat man normalverteilte Variablen/Abweichungen/Residuen, so ist die
\emph{Standardabweichung} (engl. standard deviation, SD, \(\sigma, s\))
eine komfortable Maßeinheit der Streuung, denn damit lässt sich die
Streuung (Abweichung vom Mittelwert, Residuen) der Normalverteilung gut
beschreiben.\(\square\)

\end{tcolorbox}

\begin{quote}
{\emoji{student}} Aber wie berechnet man jetzt diese Standardabweichung?
\end{quote}

\begin{quote}
{\emoji{teacher}} Moment, noch ein kurzer Exkurs zur Varianz \ldots{}
\end{quote}

\begin{quote}
{\emoji{student}} (seufzt)
\end{quote}

\subsection{Varianz}\label{varianz}

\subsubsection{Intuition}\label{intuition}

\begin{tcolorbox}[enhanced jigsaw, bottomtitle=1mm, leftrule=.75mm, breakable, title=\textcolor{quarto-callout-note-color}{\faInfo}\hspace{0.5em}{Hinweis}, bottomrule=.15mm, titlerule=0mm, left=2mm, opacityback=0, colframe=quarto-callout-note-color-frame, rightrule=.15mm, colback=white, coltitle=black, toprule=.15mm, toptitle=1mm, colbacktitle=quarto-callout-note-color!10!white, arc=.35mm, opacitybacktitle=0.6]

Die Varianz einer Variable (z.B. Verkaufspreis von Mariokart) ist, grob
gesagt, der typische Abstand eines Verkaufspreis vom mittleren
Verkaufspreis.\(\square\)

\end{tcolorbox}

Illustration zur Varianz als ``mittlerer Quadratfehler

\begin{figure}

\begin{minipage}{0.60\linewidth}
Abbildung~\ref{fig-var} illustriert die Varianz als \enquote{mittlerer
Quadratfehler}:

\begin{enumerate}
\def\labelenumi{\arabic{enumi}.}
\tightlist
\item
  Man gehe von der Häufigkeitsverteilung der Daten aus.
\item
  Betrachtet man die Daten als Gewichte auf einer Wippe, so ist der
  Schwerpunkt der Wippe der Mittelwert.
\item
  Man bilde Quadrate für jeden Datenpunkt mit der Kantenlänge, die dem
  Abstand des Punktes zum Mittelwert entspricht.
\item
  Die Quadrate quetscht man jetzt wo nötig in rechteckige Formen (ohne
  dass sich die Fläche ändern darf) und verschiebt sie, bis sich alle
  Formen zu einem Rechteck mit Seitenlänge \(n\) und \(\sigma^2\)
  anordnen.
\end{enumerate}

\end{minipage}%
%
\begin{minipage}{0.40\linewidth}

\begin{figure}[H]

\centering{

\pandocbounded{\includegraphics[keepaspectratio]{img/Variance_visualisation.svg.png}}

}

\caption{\label{fig-var}Varianz (Cmglee, 2015)}

\end{figure}%

\end{minipage}%

\end{figure}%

Abbildung~\ref{fig-mse} visualisiert die Varianz für
Beispiel~\ref{exm-mae}.\footnote{Die Abweichungsquadrate wirken optisch
  nicht quadratisch, da die X-Achse breiter skaliert dargestellt ist als
  die Y-Achse. Trotzdem sind es Quadrate, nur nicht optisch, wenn Sie
  wissen, was ich meine\ldots{}}

Links sind die \emph{Abweichungsquadrate} dargestellt, rechts die
Varianz als \enquote{\emph{typisches Abweichungsquadrat}}.

\begin{tcolorbox}[enhanced jigsaw, bottomtitle=1mm, leftrule=.75mm, breakable, title=\textcolor{quarto-callout-note-color}{\faInfo}\hspace{0.5em}{Hinweis}, bottomrule=.15mm, titlerule=0mm, left=2mm, opacityback=0, colframe=quarto-callout-note-color-frame, rightrule=.15mm, colback=white, coltitle=black, toprule=.15mm, toptitle=1mm, colbacktitle=quarto-callout-note-color!10!white, arc=.35mm, opacitybacktitle=0.6]

Die Varianz ist also ein Maß, das die typische Abweichung der
Beobachtungen vom Mittelwert in eine Zahl fasst.\(\square\)

\end{tcolorbox}

\begin{figure}[H]

\begin{minipage}{0.50\linewidth}

\centering{

\includegraphics[width=0.7\linewidth,height=\textheight,keepaspectratio]{060-modellguete_files/figure-pdf/fig-mse-1.pdf}

}

\subcaption{\label{fig-mse-1}Quadrierte Fehlerbalken}

\end{minipage}%
%
\begin{minipage}{0.50\linewidth}

\centering{

\includegraphics[width=0.7\linewidth,height=\textheight,keepaspectratio]{060-modellguete_files/figure-pdf/fig-mse-2.pdf}

}

\subcaption{\label{fig-mse-2}Varianz als \enquote*{typischer}
Fehlerbalken}

\end{minipage}%

\caption{\label{fig-mse}Sinnbild zur Varianz als typischer Fehlerbalken}

\end{figure}%

\begin{example}[]\protect\hypertarget{exm-var}{}\label{exm-var}

Sie arbeiten immer noch bei einem Online-Auktionshaus und untersuchen
den Verkauf von Videospielen. Natürlich mit dem Ziel, dass Ihre Firma
mehr von dem Zeug verkaufen kann.

Dazu berechnen Sie die Streuung in den Verkaufspreisen, s.
Listing~\ref{lst-mario-streu}. \(\square\)

\end{example}

\begin{codelisting}

\caption{\label{lst-mario-streu}Berechnung der Streuung des
Verkaufpreises als Indikatoren für die Modellgüte des Mittelwerts.}

\centering{

\begin{Shaded}
\begin{Highlighting}[]
\NormalTok{mariokart\_no\_extreme }\OtherTok{\textless{}{-}}
\NormalTok{  mariokart }\SpecialCharTok{\%\textgreater{}\%}
  \FunctionTok{filter}\NormalTok{(total\_pr }\SpecialCharTok{\textless{}} \DecValTok{100}\NormalTok{)  }\CommentTok{\# ohne Extremwerte}

\NormalTok{m\_summ }\OtherTok{\textless{}{-}} 
\NormalTok{  mariokart\_no\_extreme }\SpecialCharTok{\%\textgreater{}\%} 
  \FunctionTok{summarise}\NormalTok{(}
    \AttributeTok{pr\_mw =} \FunctionTok{mean}\NormalTok{(total\_pr),}
    \AttributeTok{pr\_iqr =} \FunctionTok{IQR}\NormalTok{(total\_pr),}
    \AttributeTok{pr\_maa =} \FunctionTok{mean}\NormalTok{(}\FunctionTok{abs}\NormalTok{(total\_pr }\SpecialCharTok{{-}} \FunctionTok{mean}\NormalTok{(total\_pr))),}
    \AttributeTok{pr\_var =} \FunctionTok{var}\NormalTok{(total\_pr),}
    \AttributeTok{pr\_sd =} \FunctionTok{sd}\NormalTok{(total\_pr))}
\end{Highlighting}
\end{Shaded}

}

\end{codelisting}%

\begin{longtable}[]{@{}lrrrr@{}}
\toprule\noalign{}
pr\_mw & pr\_iqr & pr\_maa & pr\_var & pr\_sd \\
\midrule\noalign{}
\endhead
\bottomrule\noalign{}
\endlastfoot
47.43 & 12.99 & 7.20 & 83.06 & 9.11 \\
\end{longtable}

Statistiken sind ja schön \ldots{} aber Bilder sind auch gut, s.
Abbildung~\ref{fig-var}. Datendiagramme eignen sich gut, um (grob) die
Streuung einer Variable zu erfassen.

\begin{Shaded}
\begin{Highlighting}[]
\NormalTok{mariokart }\SpecialCharTok{\%\textgreater{}\%} 
\NormalTok{  mariokart }\SpecialCharTok{\%\textgreater{}\%} 
  \FunctionTok{select}\NormalTok{(total\_pr) }\SpecialCharTok{\%\textgreater{}\%} 
  \FunctionTok{filter}\NormalTok{(total\_pr }\SpecialCharTok{\textless{}} \DecValTok{100}\NormalTok{) }\SpecialCharTok{\%\textgreater{}\%}  \CommentTok{\# ohne Extremwerte}
  \FunctionTok{plot\_density}\NormalTok{()}
\end{Highlighting}
\end{Shaded}

\begin{figure}[H]

\begin{minipage}{0.50\linewidth}

\centering{

\includegraphics[width=0.7\linewidth,height=\textheight,keepaspectratio]{060-modellguete_files/figure-pdf/fig-var-1.pdf}

}

\subcaption{\label{fig-var-1}Dichtediagramm}

\end{minipage}%
%
\begin{minipage}{0.50\linewidth}

\centering{

\includegraphics[width=0.7\linewidth,height=\textheight,keepaspectratio]{060-modellguete_files/figure-pdf/fig-var-2.pdf}

}

\subcaption{\label{fig-var-2}Violindiagramm}

\end{minipage}%

\caption{\label{fig-var}Die Verteilung des Verkaufspreises von
Mariokart-Spielen mit MW±SD in roter Farbe}

\end{figure}%

Wer sich die Berechnung von Hand für \texttt{pr\_maa} sparen möchte (s.
Listing~\ref{lst-mario-streu}), kann die
\href{https://rdrr.io/cran/DescTools/man/MeanAD.html}{Funktion
\texttt{MeanAD} aus dem Paket \texttt{DescTools}} nutzen.

\subsubsection{Kochrezept für die
Varianz}\label{kochrezept-fuxfcr-die-varianz}

Um die Standardabweichung zu berechnen, berechnet man zunächst die
\emph{Varianz}, \(s^2\) abgekürzt. Hier ist ein \enquote{Kochrezept}
(Algorithmus) zur Berechnung der Varianz:

\begin{enumerate}
\def\labelenumi{\arabic{enumi}.}
\tightlist
\item
  Für alle Datenpunkte \(x_i\): Berechne die Abweichungen vom
  Mittelwert, \(\bar{x}\)
\item
  Quadriere diese Werte
\item
  Summiere dann auf
\item
  Teile durch die Anzahl \(N\) der Werte
\end{enumerate}

Als Formel ausgedrückt, lautet die Definition der Varianz einer
Stichprobe wie folgt, s. Gleichung~\ref{eq-var} (hier geht es um die
sog. unkorrigierte Stichprobenvarianz; um anhand einer Stichprobe die
Varianz der zugehörigen Population zu schätzen, teilt man nicht durch
\(N\), sondern durch \(N-1\)) .

\begin{equation}\phantomsection\label{eq-var}{{\displaystyle s^{2}:={\frac {1}{N}}\sum _{i=1}^{n}\left(y_{i}-{\bar {y}}\right)^{2}={\frac {1}{N}}\sum _{i=1}^{n}e_i^{2}.}}\end{equation}

\begin{definition}[Varianz]\protect\hypertarget{def-var}{}\label{def-var}

Die Varianz (\(s^2, \sigma^2\)) ist definiert als der Mittelwert der
quadrierten Abweichungen, \(e_i^2\), (vom Mittelwert).\(\square\)

\end{definition}

Die Varianz steht im engen Verhältnis zur Kovarianz, s.
\textbf{?@sec-cov}. Die Varianz kann auch verstehen als den
\emph{mittleren Quadratfehler} (Mean Squared Error, MSE) eines Modells,
s. Gleichung~\ref{eq-mse}.

\begin{equation}\phantomsection\label{eq-mse}{{\displaystyle MSE:={\frac {1}{N}}\sum _{i=1}^{N}\left(y_{i}-{\hat {y}}\right)^{2}.}}\end{equation}

Im Fall eines Punktmodells ist der Mittelwert der vorhergesagte Wert
eines Modells.

\subsection{Die Standardabweichung}\label{die-standardabweichung}

Kennt man die Varianz, so lässt sich die Standardabweichung einfach als
Quadratwurzel der Varianz berechnen.

\begin{definition}[Standardabweichung]\protect\hypertarget{def-sd}{}\label{def-sd}

Die Standardabweichung (SD, s, \(\sigma\)) ist definiert als die
Quadratwurzel der Varianz, s. Gleichung~\ref{eq-sd}.

\begin{equation}\phantomsection\label{eq-sd}{s := \sqrt{s^2} \square}\end{equation}

\end{definition}

Durch das Wurzelziehen besitzt die Standardabweichung wieder \emph{in
etwa} die gleiche Größenordnung wie die Daten (im Gegensatz zur Varianz,
die durch das Quadrieren sehr groß werden kann).

Aus einem Modellierungsblickwinkel kann man die SD definieren als die
Wurzel von MSE. Dann nennt man sie \emph{Root Mean Squared Error}
(RMSE): \(RMSE := \sqrt{MSE}\).

\begin{tcolorbox}[enhanced jigsaw, bottomtitle=1mm, leftrule=.75mm, breakable, title=\textcolor{quarto-callout-note-color}{\faInfo}\hspace{0.5em}{Hinweis}, bottomrule=.15mm, titlerule=0mm, left=2mm, opacityback=0, colframe=quarto-callout-note-color-frame, rightrule=.15mm, colback=white, coltitle=black, toprule=.15mm, toptitle=1mm, colbacktitle=quarto-callout-note-color!10!white, arc=.35mm, opacitybacktitle=0.6]

Die SD ist i.d.R. \emph{un}gleich zur MAE, aber (fast) gleich zur RMSE.
Entsprechend ist die Varianz (fast) gleich zur MSE.\(\square\)

\end{tcolorbox}

\begin{example}[]\protect\hypertarget{exm-sd-mario}{}\label{exm-sd-mario}

Sie arbeiten weiter an Ihrem Mariokart-Projekt. Da Sie heute keine Lust
auf viel Tippen haben, nutzen Sie das R-Paket \texttt{easystats} mit der
Funktion \texttt{describe\_distribution}, s.
Tabelle~\ref{tbl-describe-dist1}.

\begin{Shaded}
\begin{Highlighting}[]
\FunctionTok{library}\NormalTok{(easystats)}

\NormalTok{mariokart }\SpecialCharTok{\%\textgreater{}\%} 
  \FunctionTok{select}\NormalTok{(total\_pr) }\SpecialCharTok{\%\textgreater{}\%} 
  \FunctionTok{describe\_distribution}\NormalTok{()}
\end{Highlighting}
\end{Shaded}

\begin{longtable}[]{@{}lrrrr@{}}

\caption{\label{tbl-describe-dist1}Ausgabe der Funktion
\texttt{describe\_distribution} (Auszug)}

\tabularnewline

\toprule\noalign{}
Variable & Mean & SD & IQR & n \\
\midrule\noalign{}
\endhead
\bottomrule\noalign{}
\endlastfoot
total\_pr & 50 & 26 & 13 & 143 \\

\end{longtable}

\begin{quote}
{\emoji{student}} Ah! Das war einfach. Reicht auch mal für
heute.\(\square\)
\end{quote}

\end{example}

\begin{example}[]\protect\hypertarget{exm-gruppen-mw}{}\label{exm-gruppen-mw}

Ihr Job als Datenanalyst ist anstrengend, aber auch mitunter
interessant. So auch heute. Bevor Sie nach Hause gehen, möchten Sie noch
eine Sache anschauen. In einer früheren Analyse (s.
Abbildung~\ref{fig-fehler-red}) fanden Sie heraus, dass die Fehlerbalken
kürzer werden, wenn man ein geschickteres und komplexeres Modell findet.
Das wollen Sie natürlich prüfen. Sie überlegen: \enquote{Okay, ich will
ein einfaches Modell, in dem der Mittelwert das Modell des Verkaufspreis
sein soll.}

Das spezifizieren Sie so:

\begin{Shaded}
\begin{Highlighting}[]
\NormalTok{lm1 }\OtherTok{\textless{}{-}} \FunctionTok{lm}\NormalTok{(total\_pr }\SpecialCharTok{\textasciitilde{}} \DecValTok{1}\NormalTok{, }\AttributeTok{data =}\NormalTok{ mariokart)}
\FunctionTok{mae}\NormalTok{(lm1)}
\DocumentationTok{\#\# [1] 10}
\end{Highlighting}
\end{Shaded}

Im nächsten Schritt spezifizieren Sie ein Modell, in dem der
Verkaufpreis eine Funktion der Anzahl der Lenkräder ist (ähnlich wie in
Abbildung~\ref{fig-fehler-red}):

\begin{Shaded}
\begin{Highlighting}[]
\NormalTok{lm2 }\OtherTok{\textless{}{-}} \FunctionTok{lm}\NormalTok{(total\_pr }\SpecialCharTok{\textasciitilde{}}\NormalTok{ wheels, }\AttributeTok{data =}\NormalTok{ mariokart)}
\FunctionTok{mae}\NormalTok{(lm2)}
\DocumentationTok{\#\# [1] 7.4}
\end{Highlighting}
\end{Shaded}

Ah! Sehr schön, Sie haben mit \texttt{lm2} ein besseres Modell als
einfach nur den Mittelwert gefunden. Ab nach Hause!\(\square\)

\end{example}

\section{Streuung als Modellfehler}\label{streuung-als-modellfehler}

Wenn wir den Mittelwert als Punktmodell des Verkaufpreises auffassen, so
kann man die verschiedenen Kennwerte der Streuung als verschiedene
Kennwerte der Modellgüte auffassen.

Definieren wir zunächst als Punktmodell auf Errisch:

\begin{Shaded}
\begin{Highlighting}[]
\NormalTok{lm\_mario1 }\OtherTok{\textless{}{-}} \FunctionTok{lm}\NormalTok{(total\_pr }\SpecialCharTok{\textasciitilde{}} \DecValTok{1}\NormalTok{, }\AttributeTok{data =}\NormalTok{ mariokart)}
\end{Highlighting}
\end{Shaded}

Zur Erinnerung: Wir modellieren \texttt{total\_pr} ohne Prädiktoren,
sondern als Punktmodell, und zwar schätzen wir den Mittelwert mit den
Daten \texttt{mariokoart}.

Das (Meta-)Paket \texttt{easystats} bietet komfortable Befehle, um die
Modellgüte zu berechnen:

\begin{Shaded}
\begin{Highlighting}[]
\FunctionTok{mae}\NormalTok{(lm\_mario1)  }\CommentTok{\# Mean absolute error}
\DocumentationTok{\#\# [1] 10}
\FunctionTok{mse}\NormalTok{(lm\_mario1)  }\CommentTok{\# Mean squared error}
\DocumentationTok{\#\# [1] 655}
\FunctionTok{rmse}\NormalTok{(lm\_mario1)  }\CommentTok{\# Root mean squared error}
\DocumentationTok{\#\# [1] 26}
\end{Highlighting}
\end{Shaded}

\section{z-Transformation}\label{z-transformation}

Sie arbeiten immer noch als Datenknecht, Moment, \emph{Datenhecht} bei
dem Online-Auktionshaus. Heute untersuchen Sie die Frage, wie gut sich
die Verkaufspreise mit einer einzigen Zahl, dem mittleren Verkaufspreis,
beschreiben lassen. Einige widerspenstige Werte haben Sie dabei einfach
des Datensatzes verwiesen. Schon ist das Leben leichter, s.
\texttt{mariokart\_no\_extreme}.

\begin{Shaded}
\begin{Highlighting}[]
\NormalTok{mariokart\_no\_extreme }\OtherTok{\textless{}{-}} 
\NormalTok{  mariokart }\SpecialCharTok{\%\textgreater{}\%} 
  \FunctionTok{filter}\NormalTok{(total\_pr }\SpecialCharTok{\textless{}} \DecValTok{100}\NormalTok{)}
\end{Highlighting}
\end{Shaded}

Abbildung~\ref{fig-mariokart_no_extreme} (links) zeigt, dass es einige
Streuung um den Mittelwert herum gibt.
Abbildung~\ref{fig-mariokart_no_extreme} (rechts) zeigt die (um den
Mittelwert) \emph{zentrierten} Daten.

\begin{figure}[H]

\begin{minipage}{0.50\linewidth}

\centering{

\includegraphics[width=0.7\linewidth,height=\textheight,keepaspectratio]{060-modellguete_files/figure-pdf/fig-mariokart_no_extreme-1.pdf}

}

\subcaption{\label{fig-mariokart_no_extreme-1}Wie nah drängen sich die
Verkaufspreise um ihren Mittelwert?}

\end{minipage}%
%
\begin{minipage}{0.50\linewidth}

\centering{

\includegraphics[width=0.7\linewidth,height=\textheight,keepaspectratio]{060-modellguete_files/figure-pdf/fig-mariokart_no_extreme-2.pdf}

}

\subcaption{\label{fig-mariokart_no_extreme-2}Abweichungen vom
Mittelwert: zentrierte Daten}

\end{minipage}%

\caption{\label{fig-mariokart_no_extreme}Verteilung von
\texttt{mariokart\_no\_extreme}}

\end{figure}%

Tja, das ist doch etwas Streuung um den Mittelwert herum.

\begin{tcolorbox}[enhanced jigsaw, bottomtitle=1mm, leftrule=.75mm, breakable, title=\textcolor{quarto-callout-important-color}{\faExclamation}\hspace{0.5em}{Wichtig}, bottomrule=.15mm, titlerule=0mm, left=2mm, opacityback=0, colframe=quarto-callout-important-color-frame, rightrule=.15mm, colback=white, coltitle=black, toprule=.15mm, toptitle=1mm, colbacktitle=quarto-callout-important-color!10!white, arc=.35mm, opacitybacktitle=0.6]

Je weniger Streuung um den Mittelwert (ca. 47 Euro) herum, desto besser
eignet sich der Mittelwert als Modell für die Daten, bzw. desto höher
die Modellgüte.\(\square\)

\end{tcolorbox}

Ja, es ist \emph{etwas} Streuung, aber wie viel? Kann man das genau
angeben? Sie überlegen \ldots{} und überlegen. Da! Eine Idee!

Man könnte vielleicht angeben, wie viel Euro jedes Spiel vom Mittelwert
entfernt ist. Je größer diese Abweichung, desto schlechter die
Modellgüte! Also rechnen Sie diese Abweichung aus.

\begin{Shaded}
\begin{Highlighting}[]
\NormalTok{mariokart\_no\_extreme }\OtherTok{\textless{}{-}}
\NormalTok{  mariokart\_no\_extreme }\SpecialCharTok{\%\textgreater{}\%} 
  \FunctionTok{mutate}\NormalTok{(}\AttributeTok{abw =} \FloatTok{47.4} \SpecialCharTok{{-}}\NormalTok{ total\_pr)}
\end{Highlighting}
\end{Shaded}

Anders gesagt: Wir haben die Verkaufspreise \emph{zentriert.}

\begin{definition}[Zentrieren]\protect\hypertarget{def-zentrieren}{}\label{def-zentrieren}

Zentrieren bedeutet, von jedem Wert einer Verteilung \(X\) den
Mittelwert abzuziehen. Daher ist der neue Mittelwert (der zentrierten
Verteilung) gleich Null. \(\square\)

\end{definition}

Aber irgendwie sind Sie noch nicht am Ziel Ihrer Überlegungen: Woher
weiß man, ob 10 Euro oder 20 Euro \enquote{viel} Abweichung vom
Verkaufspreis ist? Man müsste die Abweichung eines Verkaufpreis zu
irgendetwas in Bezug setzen. Wieder! Ein Geistesblitz! Man könnte doch
die jeweilige Abweichung in Bezug setzen zur \emph{mittleren (absoluten)
Abweichung} (MAA)! Ein alternativer, ähnlicher Kennwert zur mittlerer
absolute Abweichung ist die SD. Sie haben gehört, dass die SD
gebräuchlicher ist als die MAA. Um sich als Checker zu präsentieren,
berechnen Sie also auch die SD; die beiden Koeffizienten sind ja
ähnlich.

Also: Wenn ein Spiel 10 Euro vom Mittelwert abweicht und die SD 10 Euro
betragen sollte, dann hätten wir eine \enquote{standardisierte}
(abgekürzt manchmal mit \texttt{std}) Abweichung von 1, weil 10/10=1.

Begeistert über Ihre Schlauheit machen Sie sich ans Werk.

\begin{Shaded}
\begin{Highlighting}[]
\NormalTok{mariokart\_no\_extreme }\OtherTok{\textless{}{-}}
\NormalTok{  mariokart\_no\_extreme }\SpecialCharTok{\%\textgreater{}\%} 
  \FunctionTok{mutate}\NormalTok{(}\AttributeTok{abw\_std =}\NormalTok{ abw }\SpecialCharTok{/} \FunctionTok{sd}\NormalTok{(abw),  }\CommentTok{\# std wie "standardisiert"}
         \AttributeTok{abw\_std2 =}\NormalTok{ abw }\SpecialCharTok{/} \FunctionTok{mean}\NormalTok{(}\FunctionTok{abs}\NormalTok{(abw)))  }
\end{Highlighting}
\end{Shaded}

Zufrieden betrachten Sie Ihr Werk, s. Abbildung~\ref{fig-z-transf}. In
Abbildung~\ref{fig-z-transf} sieht man oben die Rohwerte und unten die
transformierten Werte, die wir hier als \emph{standardisiert}
bezeichnen, da wir sie in Bezug zur \enquote{typischen Abweichung}, der
SD, gesetzt haben.

\begin{figure}[H]

\centering{

\includegraphics[width=1\linewidth,height=\textheight,keepaspectratio]{060-modellguete_files/figure-pdf/fig-z-transf-1.pdf}

}

\caption{\label{fig-z-transf}Standardisierung von Abweichungswerten bzw.
einer Verteilung; der vertikale Balken zeigt den Mittelwert}

\end{figure}%

Wir fassen die Schritte unserer Umrechnung (\enquote{Transformation})
zusammen wie in einem Kochrezept:

\begin{enumerate}
\def\labelenumi{\arabic{enumi}.}
\tightlist
\item
  Nimm die Verteilung der Verkaufspreise
\item
  Berechne die Abweichungen vom mittleren Verkaufspreis (Differenz
  Mittelwert und jeweiliger Verkaufspreis)
\item
  Teile die Abweichungen (Schritt 2) durch die SD
\end{enumerate}

Diese Art von Transformation bezeichnet man als \emph{z-Transformation}
und die resultierenden Werte als \emph{z-Werte}.

\begin{definition}[z-Werte]\protect\hypertarget{def-z-werte}{}\label{def-z-werte}

z-Werte sind das Resultat der z-Transformation. Für die Variable \(X\)
berechnet sich der z-Wert der \(i\)-ten Beobachtung so:
\(z_i := \frac{x_i - \bar{x}}{sd_x}.\square\)

\end{definition}

z-Werte sind nützlich, weil sie die \enquote{relative} Abweichung
einzelner Beobachtungen vom Mittelwert anzeigen.

Nach einer \emph{Faustregel} spricht man von extremen Abweichungen
(Extremwerten, Ausreißern), wenn \(z_i \ge 2.5\) (Shimizu, 2022).

\section{Fazit}\label{fazit-3}

Der \enquote{gesunde Menschenverstand} würde spontan den mittleren
Absolutabstand (MAA oder MAE) der Varianz (oder der Standardabweichung,
SD) vorziehen. Das ist vernünftig, denn die MAA ist anschaulicher und
damit nützlicher als die Varianz und die SD.

Warum sollte man überhaupt ein unanschauliches Maß wie die Varianz
verwenden? Wenn es nur um deskriptive Statistik geht, braucht man die
Varianz (oder die SD) nicht unbedingt. Gründe, warum Sie die Varianz
(bzw. SD) kennen und nutzen sollten, sind:

\begin{itemize}
\tightlist
\item
  Die SD ist sehr nützlich zur Beschreibung der Normalverteilung
\item
  Die Varianz wird häufig verwendet bzw. in Forschungsarbeiten
  berichtet, also müssen Sie die Varianz kennen.
\end{itemize}

Liegen Extremwerte vor, kann es vorteilhafter sein, den IQR vorzuziehen
gegenüber Mittelwert basierten Streuungsmaßen (MAA, Varianz, SD).

\section{Aufgaben}\label{aufgaben-5}

\subsection{Datenwerk}\label{datenwerk}

Die Webseite \href{https://datenwerk.netlify.app}{datenwerk.netlify.app}
stellt eine Reihe von einschlägigen Übungsaufgaben bereit. Sie können
die Suchfunktion der Webseite nutzen, um die Aufgaben mit den folgenden
Namen zu suchen:

\begin{itemize}
\tightlist
\item
  \href{https://sebastiansauer.github.io/Datenwerk/posts/mariokart-sd2/mariokart-sd2}{mariokart-sd2}
\item
  \href{https://sebastiansauer.github.io/Datenwerk/posts/mariokart-sd3/mariokart-sd3}{mariokart-sd3}
\item
  \href{https://sebastiansauer.github.io/Datenwerk/posts/kennwert-robust/kennwert-robust}{Kennwert-robust}
\item
  \href{https://sebastiansauer.github.io/Datenwerk/posts/summarise04/summarise04}{summarise04}
\item
  \href{https://sebastiansauer.github.io/Datenwerk/posts/summarise05/summarise05}{summarise05}
\item
  \href{https://sebastiansauer.github.io/Datenwerk/posts/vis-mariokart-variab/vis-mariokart-variab}{vis-mariokart-variab}
\item
  \href{https://sebastiansauer.github.io/Datenwerk/posts/sd-vergleich/sd-vergleich}{sd-vergleich}
\item
  \href{https://sebastiansauer.github.io/Datenwerk/posts/nasa01/nasa01}{nasa01}
\item
  \href{https://sebastiansauer.github.io/Datenwerk/posts/streuung-histogramm/streuung-histogramm}{Streuung-Histogramm}
\item
  \href{https://sebastiansauer.github.io/Datenwerk/posts/mariokart-sd1/mariokart-sd1}{mariokart-sd1}
\item
  \href{https://sebastiansauer.github.io/Datenwerk/posts/summarise06/summarise06}{summarise06}
\item
  \href{https://sebastiansauer.github.io/Datenwerk/posts/mariokart-desk01/mariokart-desk01}{mariokart-desk01}
\end{itemize}

\section{Literaturhinweise}\label{literaturhinweise-5}

Allen Downey (2023) stellt in seinem vergnüglich zu lesenden Buch eine
kurzweilige Einführung in die Statistik vor; auch Streuungsmaße haben
dabei einen Auftritt. Wer mehr \enquote{Lehrbuch-Feeling} sucht, wird
bei Cetinkaya-Rundel \& Hardin (2021) fündig (das Buch ist online frei
verfügbar). Es ist kein Geheimnis, dass Streuungsmaße keine ganz neuen
Themen in der Statistik sind. Aber hey, Oldie is Goldie, ohne
Streuungsmaße geht's nicht. Jedenfalls werden Sie in jedem
Statistik-Lehrbuch, dass Sie in der Bib (oder sonst wo) aus dem Regal
ziehen, fündig werden zu diesem Thema. Die Bücher unterscheiden sich
meist \enquote{nur} in ihrem Anspruch bzw. der didaktischen Aufmachung;
für alle ist da was dabei.

\phantomsection\label{refs}
\begin{CSLReferences}{1}{0}
\bibitem[\citeproctext]{ref-ainali_standard_2007}
Ainali. (2007). \emph{Standard Deviation Diagram Micro}.
\url{https://commons.wikimedia.org/w/index.php?curid=3141713}

\bibitem[\citeproctext]{ref-Anscombe_1973}
Anscombe, F. J. (1973). Graphs in Statistical Analysis. \emph{The
American Statistician}, \emph{27}(1), 17--21.

\bibitem[\citeproctext]{ref-arad2024}
Arad, C. (2024, Juni 5). \emph{Kylian Mbappe: Gehalt und Vermögen im
Überblick (2024)}. ftd.de.
\url{https://www.ftd.de/vermoegen/mbappe-gehalt-vermoegen/}

\bibitem[\citeproctext]{ref-berger_jobs_2019}
Berger, G. (2019, Dezember 10). \emph{The {Jobs} of {Tomorrow}:
{LinkedIn}'s 2020 {Emerging Jobs Report}}.
\url{https://www.linkedin.com/blog/member/career/the-jobs-of-tomorrow-linkedins-2020-emerging-jobs-report}

\bibitem[\citeproctext]{ref-bortz_statistik_2010}
Bortz, J., \& Schuster, C. (2010). \emph{Statistik Für {Human-} Und
{Sozialwissenschaftler}}. Springer.
\url{https://doi.org/10.1007/978-3-642-12770-0}

\bibitem[\citeproctext]{ref-bowne-anderson_what_2018}
Bowne-Anderson, H. (2018). What {Data Scientists Really Do}, {According}
to 35 {Data Scientists}. \emph{Harvard Business Review}.
\url{https://hbr.org/2018/08/what-data-scientists-really-do-according-to-35-data-scientists}

\bibitem[\citeproctext]{ref-broman_data_2018}
Broman, K. W., \& Woo, K. H. (2018). Data {Organization} in
{Spreadsheets}. \emph{The American Statistician}, \emph{72}(1), 2--10.
\url{https://doi.org/10.1080/00031305.2017.1375989}

\bibitem[\citeproctext]{ref-statistisches_bundesamt_korpermase_2023}
Bundesamt, S. (2023-003-272023-003-27). \emph{Körpermaße nach
Altersgruppen und Geschlecht}. Statistisches Bundesamt.
\url{https://www.destatis.de/DE/Themen/Gesellschaft-Umwelt/Gesundheit/Gesundheitszustand-Relevantes-Verhalten/Tabellen/liste-koerpermasse.html}

\bibitem[\citeproctext]{ref-deutsche_bundesbank_household_2023}
Bundesbank, D. (2023). \emph{Household Wealth and Finances in {Germany}:
{Results} of the 2021 Household Wealth Survey}. Deutsche Bundesbank.
\url{https://www.bundesbank.de/resource/blob/908924/3ef9d9a4eaeae8a8779ccec3ac464970/mL/2023-04-vermoegensbefragung-data.pdf}

\bibitem[\citeproctext]{ref-cetinkaya-rundel_introduction_2021}
Cetinkaya-Rundel, M., \& Hardin, J. (2021). \emph{Introduction to
{Modern Statistics}}. \url{https://openintro-ims.netlify.app/}

\bibitem[\citeproctext]{ref-cmglee_english_2015}
Cmglee. (2015). \emph{English: {Geometric} Visualisation of the Variance
of the Example Distribution (2, 4, 4, 4, 5, 5, 7, 9) on w:{Standard}
Deviation.}
\url{https://commons.wikimedia.org/w/index.php?curid=39472834}

\bibitem[\citeproctext]{ref-cohen_power_1992}
Cohen, J. (1992). A Power Primer. \emph{Psychological Bulletin},
\emph{112}(1), 155--159.

\bibitem[\citeproctext]{ref-downey_probably_2023}
Downey, A. (2023). \emph{Probably Overthinking It: How to Use Data to
Answer Questions, Avoid Statistical Traps, and Make Better Decisions}.
The University of Chicago Press.

\bibitem[\citeproctext]{ref-fisher_making_2018}
Fisher, D., \& Meyer, M. (2018). \emph{Making Data Visual: A Practical
Guide to Using Visualization for Insight} (First edition). O'Reilly.

\bibitem[\citeproctext]{ref-flaticon_professor_2024}
flaticon. (2024). \emph{Professor}.
\url{https://www.flaticon.com/de/kostenlose-icons/professor}

\bibitem[\citeproctext]{ref-world_economic_forum_future_2020}
Forum, W. E. (2020). \emph{The {Future} of {Jobs Report} 2020}. World
Economic Forum.
\url{https://www3.weforum.org/docs/WEF_Future_of_Jobs_2020.pdf}

\bibitem[\citeproctext]{ref-haug_smartphone_2015}
Haug, S., Castro, R. P., Kwon, M., Filler, A., Kowatsch, T., \& Schaub,
M. P. (2015). Smartphone Use and Smartphone Addiction among Young People
in {Switzerland}. \emph{Journal of Behavioral Addictions}, \emph{4}(4),
299--307. \url{https://doi.org/10.1556/2006.4.2015.037}

\bibitem[\citeproctext]{ref-RJ-2023-4-cran}
Hornik, K., Ligges, U., \& Zeileis, A. (2023). Changes on CRAN.
\emph{The R Journal}, \emph{15}, 295--296.

\bibitem[\citeproctext]{ref-horst_tidy_2023}
Horst, A. (2023). \emph{Tidy {Data}}. \url{https://allisonhorst.com/}

\bibitem[\citeproctext]{ref-horst_statistics_2024}
Horst, A. (2024). \emph{Statistics {Artwork}}.
\url{https://allisonhorst.com/}

\bibitem[\citeproctext]{ref-hou_dynamics_2015}
Hou, J., Walsh, P. P., \& Zhang, J. (2015). The Dynamics of {Human
Development Index}. \emph{The Social Science Journal}, \emph{52}(3),
331--347. \url{https://doi.org/10.1016/j.soscij.2014.07.003}

\bibitem[\citeproctext]{ref-ichihara_color_2008}
Ichihara, Y. G., Okabe, M., Iga, K., Tanaka, Y., Musha, K., \& Ito, K.
(2008). Color Universal Design: The Selection of Four Easily
Distinguishable Colors for All Color Vision Types. \emph{Color {Imaging
XIII}: {Processing}, {Hardcopy}, and {Applications}}, \emph{6807},
206--213. \url{https://doi.org/10.1117/12.765420}

\bibitem[\citeproctext]{ref-imgflip_imageflip_2024}
imgflip. (2024a). \emph{Imageflip {Meme}}. \url{https://imgflip.com}

\bibitem[\citeproctext]{ref-imgflip_yoda_2024}
imgflip. (2024b). \emph{Yoda {Meme}}. \url{https://imgflip.com}

\bibitem[\citeproctext]{ref-transparency_international_corruption_2017}
International, T. (2017, Januar 25). \emph{Corruption {Perceptions
Index} 2016}. Transparency.org.
\url{https://www.transparency.org/en/news/corruption-perceptions-index-2016}

\bibitem[\citeproctext]{ref-ismay_statistical_2020}
Ismay, C., \& Kim, A. Y.-S. (2020). \emph{Statistical Inference via Data
Science: A {ModernDive} into {R} and the {Tidyverse}}. CRC Press /
Taylor \& Francis Group. \url{https://moderndive.com/}

\bibitem[\citeproctext]{ref-kaplan_statistical_2009}
Kaplan, D. T. (2009). \emph{Statistical Modeling: A Fresh Approach}.
CreateSpace. \url{https://dtkaplan.github.io/SM2-bookdown/}

\bibitem[\citeproctext]{ref-kwon_smartphone_2013}
Kwon, M., Kim, D.-J., Cho, H., \& Yang, S. (2013). The Smartphone
Addiction Scale: Development and Validation of a Short Version for
Adolescents. \emph{PloS One}, \emph{8}(12), e83558.
\url{https://doi.org/10.1371/journal.pone.0083558}

\bibitem[\citeproctext]{ref-lieberoth_covidistress_2020}
Lieberoth, A., Rasmussen, J., Stoeckli, S., Tran, T., Ćepulić, D.-B.,
Han, H., Lin, S.-Y., Tuominen, J., Travaglino, G. A., \& Vestergren, S.
(2020). \emph{{COVIDiSTRESS} Global Survey}.
\url{https://doi.org/10.17605/OSF.IO/Z39US}

\bibitem[\citeproctext]{ref-lieberoth2022}
Lieberoth, A., Rasmussen, J., Stoeckli, S., Tran, T., Ćepulić, D.-B.,
Han, H., Lin, S.-Y., Tuominen, J., Travaglino, G., \& Vestergren, S.
(2022). \emph{{COVIDiSTRESS} Global Survey}.
\url{https://doi.org/10.17605/OSF.IO/Z39US}

\bibitem[\citeproctext]{ref-lovett_applying_2000}
Lovett, M. C., \& Greenhouse, J. B. (2000). Applying {Cognitive Theory}
to {Statistics Instruction}. \emph{The American Statistician},
\emph{54}(3), 196--206.
\url{https://doi.org/10.1080/00031305.2000.10474545}

\bibitem[\citeproctext]{ref-lyon_why_2014}
Lyon, A. (2014). Why Are {Normal Distributions Normal}? \emph{The
British Journal for the Philosophy of Science}, \emph{65}(3), 621--649.
\url{https://doi.org/10.1093/bjps/axs046}

\bibitem[\citeproctext]{ref-m72004}
M7. (2004). \emph{Savinelli's {Italian} Smoking Pipe}.
\url{https://commons.wikimedia.org/wiki/File:Pipa_savinelli.jpg}

\bibitem[\citeproctext]{ref-mackay_scientific_2000}
MacKay, R. J., \& Oldford, R. W. (2000). Scientific {Method},
{Statistical Method} and the {Speed} of {Light}. \emph{Statistical
Science}, \emph{15}(3), 254--278.
\url{https://doi.org/10.1214/ss/1009212817}

\bibitem[\citeproctext]{ref-maphry_seesaw_2009}
Maphry. (2009). \emph{Seesaw with Mean}.
\url{https://commons.wikimedia.org/w/index.php?curid=79390659}

\bibitem[\citeproctext]{ref-marksanglin_historical_2020}
Marks‐Anglin, A., \& Chen, Y. (2020). A Historical Review of Publication
Bias. \emph{Research Synthesis Methods}, \emph{11}(6), 725--742.
\url{https://doi.org/10.1002/jrsm.1452}

\bibitem[\citeproctext]{ref-messerli_chocolate_2012}
Messerli, F. H. (2012). Chocolate {Consumption}, {Cognitive Function},
and {Nobel Laureates}. \emph{New England Journal of Medicine},
\emph{367}(16), 1562--1564. \url{https://doi.org/10.1056/NEJMon1211064}

\bibitem[\citeproctext]{ref-mittag_statistik_2020}
Mittag, H.-J., \& Schüller, K. (2020). \emph{Statistik: Eine Einführung
mit interaktiven Elementen}. Springer.
\url{https://doi.org/10.1007/978-3-662-61912-4}

\bibitem[\citeproctext]{ref-moore_recreating_2015}
Moore, B. (2015, April 9). \emph{Recreating the Vaccination Heatmaps in
{R}}. Benomics.
\url{https://benjaminlmoore.wordpress.com/2015/04/09/recreating-the-vaccination-heatmaps-in-r/}

\bibitem[\citeproctext]{ref-mulukom_psychological_2020}
Mulukom, V. van, Muzzulini, B., Rutjens, B., Lissa, C. J. van, \&
Farias, M. (2020). \emph{Psychological Impact of {COVID-19} Pandemic}.
\url{https://doi.org/10.17605/OSF.IO/TSJNB}

\bibitem[\citeproctext]{ref-obels_analysis_2020}
Obels, P., Lakens, D., Coles, N. A., Gottfried, J., \& Green, S. A.
(2020). Analysis of {Open Data} and {Computational Reproducibility} in
{Registered Reports} in {Psychology}. \emph{Advances in Methods and
Practices in Psychological Science}, \emph{3}(2), 229--237.
\url{https://doi.org/10.1177/2515245920918872}

\bibitem[\citeproctext]{ref-oestreich_keine_2014}
Oestreich, M., \& Romberg, O. (2014). \emph{Keine Panik vor Statistik!:
Erfolg und Spaß im Horrorfach nichttechnischer Studiengänge}. Springer
Fachmedien Wiesbaden. \url{https://doi.org/10.1007/978-3-658-04605-7}

\bibitem[\citeproctext]{ref-plesser_reproducibility_2018}
Plesser, H. E. (2018). Reproducibility vs. {Replicability}: {A Brief
History} of a {Confused Terminology}. \emph{Frontiers in
Neuroinformatics}, \emph{11}, 76.
\url{https://doi.org/10.3389/fninf.2017.00076}

\bibitem[\citeproctext]{ref-poldrack_statistical_2022}
Poldrack, R. (2022). \emph{Statistical {Thinking} for the 21st
{Century}}.
\url{https://statsthinking21.github.io/statsthinking21-core-site/index.html}

\bibitem[\citeproctext]{ref-poldrack_statistical_2023}
Poldrack, R. A. (2023). \emph{Statistical Thinking: Analyzing Data in an
Uncertain World}. Princeton University Press.
\url{https://statsthinking21.github.io/statsthinking21-core-site/}

\bibitem[\citeproctext]{ref-owidhumanheight}
Roser, M., Appel, C., \& Ritchie, H. (2013). Human Height. \emph{Our
World in Data}. \url{https://ourworldindata.org/human-height}

\bibitem[\citeproctext]{ref-rothstein_publication_2014}
Rothstein, H. R. (2014). Publication {Bias}. In \emph{Wiley {StatsRef}:
{Statistics Reference Online}}. John Wiley \& Sons, Ltd.
\url{https://doi.org/10.1002/9781118445112.stat07071}

\bibitem[\citeproctext]{ref-sauer2017a}
Sauer, S. (2017). \emph{Dataset 'Predictors of Performance in Stats
Test'} {[}Data set{]}. Open Science Framework.
\url{https://doi.org/10.17605/OSF.IO/SJHUY}

\bibitem[\citeproctext]{ref-sauer_moderne_2019}
Sauer, S. (2019). \emph{Moderne Datenanalyse mit R: Daten einlesen,
aufbereiten, visualisieren und modellieren} (1. Auflage 2019). Springer.
\url{https://www.springer.com/de/book/9783658215866}

\bibitem[\citeproctext]{ref-scherer_seasonal_2019}
Scherer, C., Radchuk, V., Staubach, C., Müller, S., Blaum, N., Thulke,
H., \& Kramer‐Schadt, S. (2019). Seasonal Host Life‐history Processes
Fuel Disease Dynamics at Different Spatial Scales. \emph{Journal of
Animal Ecology}, \emph{88}(11), 1812--1824.
\url{https://doi.org/10.1111/1365-2656.13070}

\bibitem[\citeproctext]{ref-schwaiger_impact_2022}
Schwaiger, E., \& Tahir, R. (2022). The Impact of Nomophobia and
Smartphone Presence on Fluid Intelligence and Attention.
\emph{Cyberpsychology: Journal of Psychosocial Research on Cyberspace},
\emph{16}(1). \url{https://doi.org/10.5817/CP2022-1-5}

\bibitem[\citeproctext]{ref-shimizu2022}
Shimizu, Y. (2022). Multiple {Desirable Methods} in {Outlier Detection}
of {Univariate Data With R Source Codes}. \emph{Frontiers in
Psychology}, \emph{12}, 819854.
\url{https://doi.org/10.3389/fpsyg.2021.819854}

\bibitem[\citeproctext]{ref-simmons_false-positive_2011}
Simmons, J. P., Nelson, L. D., \& Simonsohn, U. (2011). False-{Positive
Psychology}: {Undisclosed Flexibility} in {Data Collection} and
{Analysis Allows Presenting Anything} as {Significant}.
\emph{Psychological Science}, \emph{22}(11), 1359--1366.
\url{https://doi.org/10.1177/0956797611417632}

\bibitem[\citeproctext]{ref-spurzem2017}
Spurzem, L. (2017). \emph{{VW} 1303 von {Wiking} in 1:87}.
\url{https://de.wikipedia.org/wiki/Modellautomobil\#/media/File:Wiking-Modell_VW_1303_(um_1975).JPG}

\bibitem[\citeproctext]{ref-stigler_seven_2016}
Stigler, S. M. (2016). \emph{The Seven Pillars of Statistical Wisdom}.
Harvard University Press.

\bibitem[\citeproctext]{ref-van_panhuis_contagious_2013}
van Panhuis, W. G., Grefenstette, J., Jung, S. Y., Chok, N. S., Cross,
A., Eng, H., Lee, B. Y., Zadorozhny, V., Brown, S., Cummings, D., \&
Burke, D. S. (2013). Contagious {Diseases} in the {United States} from
1888 to the {Present}. \emph{New England Journal of Medicine},
\emph{369}(22), 2152--2158. \url{https://doi.org/10.1056/NEJMms1215400}

\bibitem[\citeproctext]{ref-ward_brain_2017}
Ward, A. F., Duke, K., Gneezy, A., \& Bos, M. W. (2017). Brain {Drain}:
{The Mere Presence} of {One}'s {Own Smartphone Reduces Available
Cognitive Capacity}. \emph{Journal of the Association for Consumer
Research}, \emph{2}(2), 140--154. \url{https://doi.org/10.1086/691462}

\bibitem[\citeproctext]{ref-wicherts_degrees_2016}
Wicherts, J. M., Veldkamp, C. L. S., Augusteijn, H. E. M., Bakker, M.,
Aert, R. C. M. van, \& Assen, M. A. L. M. van. (2016). Degrees of
{Freedom} in {Planning}, {Running}, {Analyzing}, and {Reporting
Psychological Studies}: {A Checklist} to {Avoid} p-{Hacking}.
\emph{Frontiers in Psychology}, \emph{7}.
\url{https://doi.org/10.3389/fpsyg.2016.01832}

\bibitem[\citeproctext]{ref-wickham_ggplot2_2016}
Wickham, H. (2016). \emph{Ggplot2: Elegant Graphics for Data Analysis}
(Second edition). Springer.

\bibitem[\citeproctext]{ref-wickham_tidy-data-sinnbild_2023}
Wickham, H. (2023). \emph{Tidy-{Data-Sinnbild}}.
\url{https://r4ds.hadley.nz/data-tidy\#fig-tidy-structure}

\bibitem[\citeproctext]{ref-wickham_r_2018}
Wickham, H., \& Grolemund, G. (2018). \emph{R für Data Science: Daten
importieren, bereinigen, umformen, modellieren und visualisieren} (F.
Langenau, Übers.; 1. Auflage). O'Reilly.
\url{https://r4ds.had.co.nz/index.html}

\bibitem[\citeproctext]{ref-wilke_fundamentals_2019}
Wilke, C. (2019). \emph{Fundamentals of Data Visualization: A Primer on
Making Informative and Compelling Figures} (First edition). O'Reilly
Media. \url{https://clauswilke.com/dataviz/}

\bibitem[\citeproctext]{ref-wilke_wilkelabpracticalgg_2024}
Wilke, C. (2024). \emph{Wilkelab/Practicalgg}. Wilke Lab.
\url{https://github.com/wilkelab/practicalgg} (Original work published
2019)

\end{CSLReferences}


\backmatter


\end{document}
